\documentclass{article}
\usepackage[utf8]{inputenc}
\usepackage[russian]{babel}
\usepackage{amsmath, amssymb,latexsym, amsthm}


\begin{document}
\section{Обратный оператор}
\subsection{Единица. Обратный элемент}
...
$x, y \in A$ \\
$x * y = e$

\subsubsection{Опр}
Еcли x *  y = e, то x называется левым обратным к y \\
y называется правым обратным к x

\subsubsection{Опр}
xz = xz = r \\
x называется обратимый к z и обозначатеся $x = z^{-1}$

\subsubsection{Лемма}
Если $y, z \in A$ \\
$\exists x$  - левый обратимый и y - правый обратимый \\
тогда: \\
\begin{enumerate}
\item z - обратим
\item x = y = z
\end{enumerate}


\subsection{Обратная матрица}

$K^n_n$ - алгебра матриц
\subsubsection{Опр}
Единичной матрицой нахывается $E: \forall A \in K^n_n$ \\
$AE = EA = A$

\subsubsection{Опр}
Обратной матрице называется $A^{-1}: AA^{-1} = E$

\subsubsection{Теорема}
$\exists A^{-1} \Leftrightarrow det A \not = 0$

\subsubsection{Способы вычисления $A^{-1}$}
\begin{description}
\item Метод Гауса \\
  $[A \vline E] \sim [E \vline A^{-1}]$
\item Союзная матрица \\
  $] A : \tilde{a}^i_j = A^i_j = (-1)^{i+j}M^i_j$ - союзная матрица \\
  $A^{-1} = \frac{1}{det A}\tilde{A}^T$
\end{description}

\subsection{Обртаный оператор}
$\varphi : X \rightarrow X$
\subsubsection{Опр}
Обртаным к опреатору $\varphi$ называется оператор $\varphi^{-1}$: \\
$\varphi^{-1}\varphi = \varphi \varphi^{-1} = I$

\subsubsection{Теорема}
Оператор $\varphi$ обратим если $\exists$ базис в котором его матрица невырождена

\subsubsection{NB}
\begin{multline*}
\begin{align}
\tilde{A} = SAT \\
det \tilde{A} = det(SAT) = detS detA detY
\end{align}
\end{multline*}

\subsubsection{Опр}
Ядро $\varphi : Ker \varphi = \{x \ in X: \varphi x = 0\}$ 

\subsubsection{Лемма}
$Ket \varphi$ - ЛП

\subsubsection{Опр}
Образ $\varphi : \Im \varphi = \{y \ in Y : \exists x : \varphi(x) = y\}$

\subsubsection{Лемма}
$\Im \varphi$ - ЛП

\subsubsection{Теорема (о ядре и образе)}
$\ \varphi : x \rightarrow X$
$\Rightarrow dim Ker \varphi + dim \Im \varphi = dim X$


\subsubsection{Теорема}
$] \varphi : X \rightarrow X \Rightarrow \exists \varphi^{-1} \Leftrightarrow
dim \Im \varphi = dim X \Leftrightarrow dim Ker \varphi = 0$

\section{Внешняя степень ЛОп}

\subsubsection{Опр}
Определителем набора векторов $\{x_i\}_{i=1}^n$  называется число $det[x_1, x_2,
\dots, x_n]$ такое, что: $x_1 \wedge x_2 \wedge \dots \wedge x_n = det[x_1, x_2,
\dots, x_n]e_1 \wedge e_2 \wedge \dots \wedge e_n$

\subsubsection{Опр}
$] \varphi : x \rightarrow X$
Внешней степенью $\varphi^{\Lambda_p}$ опреатора $\varphi$ называется
отображение:
$\varphi^{\Lambda_p}(x_1 \wedge x_2 \wedge \dots \wedge \x_p) = \varphi(x_1)
\wedge \varphi(x_2) \wedge \dots \wedge \varphi({x_p})$

\subsubsection{Опр}
Определитель линейного оператора $\varphi$ \\
$det \varphi = det[\varphi(x_1) \wedge \varphi(x_2) \wedge \dots \wedge
\varphi({x_p})] = det A_\varphi e_1 \wedge e_2 \wedge \dots \wedge e_n$

\end{document}