\documentclass[12pt]{article}
\usepackage{prelude}

\begin{document}
\section{Монотонность. Экстремумы}
\begin{theorem}[Критерий монотонности]
$f \in C(\langle a, b \rangle)$ $f$ - диф. на $(a, b)$, тогда $f \uparrow \Leftrightarrow$ $f'(x) \ge 0, \forall x \in (a, b)$
\end{theorem}
\begin{proof}
  $\Rightarrow$ по определению производной \\
  $\Leftarrow x_1 > x_2$ по т. Лагранжа $\exists c: f(x_1) - f(x_2) = f(c)\cdot(x_1 - x_2) \ge 0$ 
\end{proof}
\begin{corollary}
  $f: \langle a, b \rangle \rightarrow \mathbb{R}$, тогда \\
  $f = const \Leftrightarrow f \in C(\langle a, b \rangle )$, дифф на $(a, b) f' = 0$
\end{corollary}
\begin{corollary}
  $f \in C(\langle a, b \rangle)$, дифф на $(a, b)$, тогда $f$ - строго возрастает $\Leftrightarrow$
  \begin{enumerate}
  \item $f' \ge 0$ на $(a, b)$
  \item $f' \not = 0$ ни на каком промежутке
  \end{enumerate}
\end{corollary}
\begin{proof}
  $\Rightarrow$ очев. \\
  $\Leftarrow$ по Лемме о возрастании в точке
\end{proof}
\begin{corollary}[Доказательство неравенств]
  $g, f \in C(\langle a, b \rangle)$, дифф. на $(a, b)$ \\
  $f(a) \le g(a), \forall x \in (a, b) f'(x) \le g'(x)$, тогда $\forall \alpha \in (a, b) f(\alpha) \le g(\alpha)$
\end{corollary}
\end{document}
