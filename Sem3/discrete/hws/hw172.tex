% Created 2020-12-15 Tue 14:32
% Intended LaTeX compiler: pdflatex
\documentclass{article}
       \usepackage[T1, T2A]{fontenc}
\usepackage[lutf8]{luainputenc}
\usepackage[russian, english]{babel}
\usepackage{minted}
\usepackage{graphicx}
\usepackage{longtable}
\usepackage{hyperref}
\usepackage{natbib}
\usepackage{amssymb}
\usepackage{amsmath}
\usepackage{grffile}
\usepackage{wrapfig}
\usepackage{rotating}
\usepackage{placeins}
\usepackage[normalem]{ulem}
\usepackage{amsmath}
\usepackage{textcomp}
\usepackage{capt-of}
       
       \usepackage{geometry}
       \geometry{a4paper,left=2.5cm,top=2cm,right=2.5cm,bottom=2cm,marginparsep=7pt, marginparwidth=.6in}
\author{Ilya Yaroshevskiy}
\date{\today}
\title{172-187}
\hypersetup{
 pdfauthor={Ilya Yaroshevskiy},
 pdftitle={172-187},
 pdfkeywords={},
 pdfsubject={},
 pdfcreator={Emacs 28.0.50 (Org mode 9.3)}, 
 pdflang={English}}
\begin{document}

\maketitle
\tableofcontents


\section*{172}
\label{sec:org7d0c58a}
\(M = \langle X, I \rangle\), 
$$X = \left\{{\begin{pmatrix} 1 \\ 0 \end{pmatrix}}
, \begin{pmatrix} 0 \\ 1 \end{pmatrix}, \begin{pmatrix} 1 \\ 1 \end{pmatrix}
, \begin{pmatrix} 2 \\ 2 \end{pmatrix}\right\}$$
Независимое множество - линейно независимое  \\
Базы: $$
\left\{\begin{pmatrix} 1 \\ 0 \end{pmatrix}, \begin{pmatrix} 0 \\ 1 \end{pmatrix}\right\}, 
\left\{\begin{pmatrix} 1 \\ 0 \end{pmatrix}, \begin{pmatrix} 1 \\ 1 \end{pmatrix}\right\},
\left\{\begin{pmatrix} 1 \\ 0 \end{pmatrix}, \begin{pmatrix} 2 \\ 2 \end{pmatrix}\right\},
\left\{\begin{pmatrix} 0 \\ 1 \end{pmatrix}, \begin{pmatrix} 1 \\ 1 \end{pmatrix}\right\},
\left\{\begin{pmatrix} 0 \\ 1 \end{pmatrix}, \begin{pmatrix} 2 \\ 2 \end{pmatrix}\right\}
$$
Цикл: \(\left\{\begin{pmatrix} 1 \\ 1 \end{pmatrix}, \begin{pmatrix} 2 \\ 2 \end{pmatrix}\right\}\)
\section*{174}
\label{sec:org257c497}
\(M = \langle X, I \rangle\) \\
\(M / x = \langle X\textbackslash x, \underbrace{\{ A \textbackslash x \mid A \in I,\ x \in A \}}_{I_1} \rangle\), \(\{x\} \in I\) \\
Пользуемся тем, что \(\forall A : x \not\in A\), выполняется \(A\cup x \in I \Leftrightarrow A \in I_1\)
\begin{enumerate}
\item \(\{x\} \in I \Rightarrow \{x\} \textbackslash x = \emptyset \in I_1\)
\item \(A \in I_1,\ B \subset A \Rightarrow A \cup x \in I,\ B\cup x \subset A\cup x \Rightarrow B\cup x \in I \Rightarrow B \in I_1\)
\item \(A, B \in I_1, |A| > |B| \Rightarrow A\cup x, B\cup x \in I,\ |A\cup x| > |B\cup x| \Rightarrow \\ \Rightarrow \exists y \in (A\cup x) \textbackslash (B\cup x) : B\cup \{x, y\} \in I \Rightarrow \exists y \in A \textbackslash B : B \cup y \in I_1\)
\end{enumerate}
\section*{175}
\label{sec:orgf69b2e4}
\(M_1 = \langle X, I_1 \rangle\), \(M_2 = \langle Y, I_2 \rangle\), \(X \cap Y = \emptyset\) \\
\(M = \langle X \cup Y, \underbrace{\{A \cup B \mid A \in I_1,\ B \in I_2\}}_{I} \rangle\) \\
Пользуемся тем что, если \(A \in I\), то \(\exists X_1 \in I_1, Y_1 \in I_2 : X_1 \cup Y_1 = A\) \\
и если \(X_1 \in I_1,\ Y_1 \in I_2\), то \(X_1 \cup Y_1 \in I\) \\
\begin{enumerate}
\item \(\emptyset \in I_1,\ \emptyset \in I_2 \Rightarrow \emptyset \cup \emptyset = \emptyset \in I\)
\item \(A \in I,\ B \subset A \Rightarrow \exists X_1 \in I_1, Y_1 \in I_2 : X_1 \cup Y_2 = A\) \\
\(\exists X_2 \subseteq X_1,\ Y_2 \subseteq Y_1 : X_2 \cup Y_2 = B \Rightarrow X_2 \in I_1,\ Y_2 \in I_2 \Rightarrow B \in I\)
\item \(A, B \in I,\ |A| > |B| \Rightarrow \exists X_1, X_2 \in I_1,\ Y_1,Y_2 \in I_2: A = X_1 \cup Y_1,\ B = X_2 \cup Y_2\)
\begin{enumerate}
\item \(|X_1| > |X_2|\):
\(\Rightarrow \exists x \in X_1 \textbackslash X_2: X_2 \cup x \in I_1 \Rightarrow (X_2 \cup x) \cup Y_2 = B \cup x \in I\)
\item \(|Y_1| > |Y_2|\) - аналогично
\end{enumerate}
\end{enumerate}

\section*{180}
\label{sec:orge1b25b9}
\(M = \langle X, I \rangle\) \\
\(M|_k = \langle X, \underbrace{\{A \mid A \in I, |A| \le k\}}_{I_1} \rangle\) \\
Пользуемся тем, что \(|A| \le k,\ A \in I \Leftrightarrow A \in I_1\)
\begin{enumerate}
\item \(|\emptyset| = 0 \le k \Rightarrow \emptyset \in I_1\)
\item \(A \in I_1,\ B \subset A \Rightarrow A \in I \Rightarrow B \in I, |B| < |A| \le k \Rightarrow B \in I_1\)
\item \(A, B \in I_1,\ |A| > |B| \Rightarrow A, B \in I \Rightarrow \exists x \in A \textbackslash B : B\cup x \in I,\ |B\cup x| \le |A| \le k \Rightarrow B\cup x \in I_1\)
\end{enumerate}
\section*{192}
\label{sec:org116ba12}
\begin{enumerate}
\item \(\langle A \rangle \subseteq \langle \langle A \rangle \rangle\)
\item \(x \in \langle \langle A \rangle \rangle \Rightarrow r(\langle A \rangle \cup x) = r(\langle A \rangle) = r(A)\) \\
Заметим, что \(A\cup x \subseteq \langle A \rangle \cup x\) \\
Тогда \(r(A \cup x ) \le r(\langle A \rangle \cup x) = r(A)\) \\
\(r(A \cup x) \le r(A) \Rightarrow r(A) = r(A \cup x) \Rightarrow x \in \langle A \rangle\) \\
Так как это верно для всех \(x\) из \(\langle \langle A \rangle \rangle\), то \(\langle \langle A \rangle \rangle \subseteq \langle A \rangle\)
\end{enumerate}
1, 2 \(\Rightarrow \langle \langle A \rangle \rangle = \langle A \rangle\)
\end{document}
