% Created 2020-12-29 Tue 19:05
% Intended LaTeX compiler: pdflatex
\documentclass{article}
       \usepackage[T1, T2A]{fontenc}
\usepackage[lutf8]{luainputenc}
\usepackage[russian, english]{babel}
\usepackage{minted}
\usepackage{graphicx}
\usepackage{longtable}
\usepackage{hyperref}
\usepackage{natbib}
\usepackage{amssymb}
\usepackage{amsmath}
\usepackage{grffile}
\usepackage{wrapfig}
\usepackage{rotating}
\usepackage{placeins}
\usepackage[normalem]{ulem}
\usepackage{amsmath}
\usepackage{textcomp}
\usepackage{capt-of}
       
       \usepackage{geometry}
       \geometry{a4paper,left=2.5cm,top=2cm,right=2.5cm,bottom=2cm,marginparsep=7pt, marginparwidth=.6in}
\author{Ilya Yaroshevskiy}
\date{\today}
\title{Матан КР 1}
\hypersetup{
 pdfauthor={Ilya Yaroshevskiy},
 pdftitle={Матан КР 1},
 pdfkeywords={},
 pdfsubject={},
 pdfcreator={Emacs 28.0.50 (Org mode 9.3)}, 
 pdflang={English}}
\begin{document}

\maketitle
\tableofcontents


\section{Пределы}
\label{sec:org0e850b7}
\(\displaystyle{\lim_{(x, y) \to (a, b)}f(x, y)} = L\), \(\forall U(L)\ \exists w((a, b))\ \forall x \in w\ f(x) \in U(L)\) \\
\begin{itemize}
\item Повторный \(\displaystyle{\lim_{y \to b}\lim_{x \to a}f \overset{?}{=} L}\)
\item По направлению \(\displaystyle{\lim_{r \to 0}f(a + r\cos\varphi, b + r\sin\varphi)}\)
\item Вдоль кривой \(\displaystyle{\lim_{t \to t_0}}f(x(t), y(t))\)
\item Двойной \(\displaystyle{{\lim_\substack{x \to a \\ y \to b}}}f(x,y)\), \(\forall U(L)\ \exists U(a), V(b)\ \forall x \in \dot{U}(a)\ y \in \dot{V}(b)\ f(x, y) \in U(L)\)
\end{itemize}
\section{Дифференцирование}
\label{sec:orgfb36047}
\(df(x, y) = f'_xdx + f'_ydy\) \\
\(u(x, y) = f(g(x, y), h(x, y))\) \\
\(du = f'_Idg + f'_{II}dh\)
\section{Тейлор}
\label{sec:org827773c}
\(f(a + h) = f(a) + df(a, h) + \frac{d^2f(a, g)}{2!} + \dots + \frac{d^nf(a, h)}{n!} + d^{n + 1}f(a + \Theta h, h) * \frac{1}{(n + 1)!}\) \\
\textbf{Че такое \(h\)?} Вместо \(dx\) ставим \(h_1\), вместо \(dy\) \(h_2\)
\section{Замена переменных}
\label{sec:orgf4414ba}
\(z'_x = xz'_y\) \\
\(x = x(u ,v)\) \\
\(y = y(u ,v)\) \\
Пересчитать \\
\(\tilde{z}(u, v) = z(x(u ,v), y(u, v))\) \\
\(\tilde{z}'_u = z'_x x'_u + z'_y y'_u\) \\
\(\tilde{z}'_v = z'_x x'_v + z'_y y'_v\) \\
Из этой системы находим \(z'_x,\ z'_y\) \\
Обратная задача: \\
Можно свести к предыдущей, выразив \(x\) и \(y\) \\
\(z'_x \rightarrow ?\) \\
\(u = u(x, y)\) \\
\(v = v(x, y)\) \\
\(z(x, y) = \tilde{z}(u(x, y), v(x, y))\) \\
\(z'_x = \tilde{z}'_u u'_x + \tilde{z}'_v v'_x\) - Ответ(но кривой) \\
Другая задача \\
\(z'_x\) \\
\(x = x(u, v, w)\) \\
\(y = y(u, v, w)\) \\
\(z = \tilde{z}(u, v, w)\) \\
\(w\) - функция \\
\(z(x(u, v, w), y(u, v, w)) = \tilde{z}(u, v, w)\) \\
\(z'_x(x'_u + x'_w w'_u) + z'_y(y'_u + y'_w w'_u) = \tilde{z}'_u + \tilde{z}'_w w'_u\) \\
\(z'_x(x'_v + x'_w w'_v) + z'_y(y'_v + y'_w w'_v) = \tilde{z}'_v + \tilde{z}'_w w'_v\) \\
Из системы находим \(z'_x\)
\section{Эктремумы}
\label{sec:orgee2dcf3}
\href{http://mathprofi.ru/extremumy\_funkcij\_dvuh\_i\_treh\_peremennyh.html}{Mathprofi}
\subsection{Функция вида \(z = f(x, y)\)}
\label{sec:org3dd096c}
\subsubsection{Необходимое условие экстремума}
\label{sec:org1212596}
Точка \(M_0\) - подозрительная(критическая, стационарная) на эктремум
если \(f'_x(M_0) = 0, f'_y(M_0) = 0\)
Находим: 
\begin{gather*}
A = f''_{xx}(M_0) \\
B = f''_{xy}(M_0) \\
C = f''_{yy}(M_0) \\
\end{gather*}

\begin{enumerate}
\item \(AC - B^2 > 0\) - экстремум
\begin{enumerate}
\item \(A > 0\) - минимум
\item \(A < 0\) - максимум
\end{enumerate}
\item \(AC - B^2 < 0\) - нет экстремума
\item \(AC - B^2 = 0\) - доп. исследование
\end{enumerate}
\section{Геометрия}
\label{sec:org30dce11}
Демидович стр. 360
\begin{enumerate}
\item Параметрическое
\(x = x(t)\) \\
\(y = y(t)\) \\
\(z = z(t)\) \\
\(\begin{pmatrix} x(t) \\ y(t) \\ z(t) \end{pmatrix} = \begin{pmatrix} x(t_0) \\ y(t_0) \\ z(t_0) \end{pmatrix} + \begin{pmatrix} x'(t_0) \\ y'(t_0) \\ z'(t_0) \end{pmatrix}(t - t_0)\) \\
\(\begin{pmatrix} x' \\ y' \\ z'\end{pmatrix}\) - направление касательной \\
\(\frac{x - x(t_0)}{x'(t_0)} = \frac{y - y(t_0)}{y'(y_0)} = \frac{z - z(t_0)}{z'(t_0)}\) - уравнение касательной прямой \\
\(x'(t_0)(x - x(t_0)) + y'(t_0)(y - y(t_0)) + z'(t_0)(z - z(t_0)) = 0\) - нормальная плоскость
\item Поверхность
\(x = x(u, v)\) \\
\(y = y(u, v)\) \\
\(z = z(u, v)\) \\
\(\begin{pmatrix} x(u, v) \\ y(u, v) \\ z(u, v) \end{pmatrix} = \begin{pmatrix} x(u_0, v_0) \\ y(u_0, v_0) \\ z(u_0, v_0) \end{pmatrix} + \begin{pmatrix} x'_u \\ y'_u \\ z'_u \end{pmatrix}(u - u_0) + \begin{pmatrix} x'_v \\ y'_v \\ z'_v \end{pmatrix}(v - v_0)\) \\
\(\vec{n} = \begin{pmatrix} y'_uz'_v - z'_uy'_v \\ z'_ux'_v - x'_uz'_v \\ x'_uy'_v - x'_vy'_u \end{pmatrix}\) - вектор нормали \\
\(n_1(x - x(u_0, v_0)) + n_2(y - y(u_0, v_0)) + n_3(z - z(u_0, v_0)) = 0\) - касательная плоскость
\item \(F(x, y, z) = 0\) \\
\(F'_x(x - x_0) + F'_y(y - y_0) + F'_z(z - z_0) = 0\) \\
\(\vec{n} = \begin{pmatrix} F'_x & F'_y & F'_z \end{pmatrix}\) - Вектор нормали
\item \begin{cases}
F(x, y, z) = 0 \\
G(x, y, z) = 0
\end{cases}

\begin{cases}
\(F'_x(x - x_0) + F'_y(y - y_0) + F'_z(z - z_0) = 0\) \\
\(G'_x(x - x_0) + G'_y(y - y_0) + G'_z(z - z_0) = 0\) \\
\end{cases} - это уранение прямой
\end{enumerate}
\end{document}
