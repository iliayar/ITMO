% Created 2020-12-24 Thu 16:06
% Intended LaTeX compiler: pdflatex
\documentclass{article}
       \usepackage[T1, T2A]{fontenc}
\usepackage[lutf8]{luainputenc}
\usepackage[russian, english]{babel}
\usepackage{minted}
\usepackage{graphicx}
\usepackage{longtable}
\usepackage{hyperref}
\usepackage{natbib}
\usepackage{amssymb}
\usepackage{amsmath}
\usepackage{grffile}
\usepackage{wrapfig}
\usepackage{rotating}
\usepackage{placeins}
\usepackage[normalem]{ulem}
\usepackage{amsmath}
\usepackage{textcomp}
\usepackage{capt-of}
       
       \usepackage{geometry}
       \geometry{a4paper,left=2.5cm,top=2cm,right=2.5cm,bottom=2cm,marginparsep=7pt, marginparwidth=.6in}
\author{Ilya Yaroshevskiy}
\date{\today}
\title{For test}
\hypersetup{
 pdfauthor={Ilya Yaroshevskiy},
 pdftitle={For test},
 pdfkeywords={},
 pdfsubject={},
 pdfcreator={Emacs 28.0.50 (Org mode 9.3)}, 
 pdflang={English}}
\begin{document}

\maketitle
\tableofcontents

\begin{enumerate}
\item задача
\item разделяющиеся переменные
\item линейное
\item Бернулли
\item УПД
\item понижение порядка
\item понижение порядка
\item метод последовательных приближений Пикара
\end{enumerate}

\section{Задание 1}
\label{sec:org5b748f8}
Фезека
\section{Задание 2}
\label{sec:orgf007b50}
Уравнения приводящиеся в вид
\[ f(x)dx = g(y)dy \]
\[ \Downarrow \]
\[ \int f(x)dx = \int g(y)dy \]
\[ \Downarrow \]
\[ F(x) = G(y) \]
\section{Задание 3}
\label{sec:orga4d9313}
Линейные уравнения первого порядка
\begin{gather*}
y' + a(x)y = f(x) \\
u(x) = \exp{\int{a(x)dx}} \\
y = \frac{\int{u(x)f(x)dx} + C}{u(x)}
\end{gather*}
\section{Задание 4}
\label{sec:orgc067e88}
Уравнение вида \(y' + a(x)y = b(x)y^m\) \\
При \(m = 0\) - линейное уравнение \\
    \(m = 1\) - уравнение с разделяющими переменными \\
Иначе сводится к линейному подстановкой \(z = y^{1 - m}\) в уравнение 
\(z' + (1 - m)a(x)z = (1 - m)b(x)\) 
\section{Задание 5}
\label{sec:orgc2c5772}
Уравнения вида
\[ P(x, y)dx + Q(x, y)dy = 0 \]
уравнение в полных диффернциалах если \(\exists u(x, y): du(x, y) = P(x, y)dx + Q(x, y)dy\) \\
\textbf{И} \(Q'_x = P'_y\) \\
Тогда решение \(u(x, y) = C\)

\subsection{Алгоритм}
\label{sec:org16989da}
\begin{enumerate}
\item Запишем систему \\
\begin{cases}
\case u'_x = P(x, y) \\
\case u'_y = Q(x, y) \\
\end{cases}
\item Проинтегрируем первое уравнение по \(x\), считая \(y\) константой,
так-же примем С за \(\varphi(y)\) \\
\[ u(x, y) = \int P(x, y)dx + \varphi(y) \]
\item Продифиренцируем полученное \(u(x, y)\) по \(y\) и подставим во второе уравнение \\
\[ u'_y(x,y) = \left(\int P(x, y)dx \right)'_y + \varphi'_y(y) = Q(x, y)\]
\item Проинтегрируем полученное уранение и найдем \(\varphi(y)\)
\[ \varphi(y) = \int\left[Q(x, y) - \left(\int P(x, y)dx \right)'_y\right]dy \]
\item Подставим \(\varphi(y)\) в \(u(x, y)\), получим итоговое решение \\
\[ u(x, y) = C \]
\end{enumerate}

\subsection{Нахождение интегрируещего множителя}
\label{sec:org7387701}
Если \(\frac{\partial Q}{\partial x} \not= \frac{\partial P}{\partial y}\) \\
Находим такое \(\varphi(x, y)\), что:
\[ Q\varphi'_x - P\varphi'_y = \varphi(P'_y - Q'_x) \]

\begin{enumerate}
\item Если \(\varphi(x, y) = \varphi(x)\), то найдем его из уравнения
\[ \frac{1}{\varphi}\frac{d\varphi}{dx} = \frac{1}{Q}(P'_y - Q'_x) \]
\item Если \(\varphi(x, y) = \varphi(y)\), то найдем его из уравнения
\[ \frac{1}{\varphi}\frac{d\varphi}{dy} = -\frac{1}{P}(P'_y - Q'_x) \]
\item Если \(\varphi\) зависит от обоих переменных, 
тогда если \(\exists z: \varphi(z) = \varphi(x, y)\), то найдем \(\varphi\) из уравнения
\[ \frac{1}{\varphi}\frac{d\varphi}{dz} = \frac{P'_y - Q'_x}{Qz'_x - Pz'_y} \]
\end{enumerate}

\section{Задание 6}
\label{sec:orge503b69}
Уравнение вида
\[ F(x, y, y', y'') = 0 \]

\begin{enumerate}
\item \(y'' = f(x)\) \\
Возьмем новую функцию \(p(x): y' = p(x)\) \\
Тогда решим \(p' = f(x)\), затем решим \(y' = p(x)\)
\item \(y'' = f(y)\) \\
Возьмем новую функцию \(p(y): y' = p(y)\) \\
Тогда решим \(\frac{dp}{dy}p = f(y)\), затем решим \(y' = p(y)\)
\item \(y'' = f(y')\) \\
Возьмем новую функцию \(p(x): y' = p(x)\) \\
Тогда решим \(p' = f(p)\), затем решим \(y' = p(x)\)
\item \(y'' = f(x, y')\) \\
Возьмем новую функцию \(p(x): y' = p(x)\) \\
Тогда решим \(p' = f(x, p)\), затем решим \(y' = p(x)\)
\item \(y'' = f(y, y')\) \\
Возьмем новую функцию \(p(y): y' = p(y)\) \\
Тогда решим \(\frac{dp}{dy}p = f(y, p)\), затем решим \(y' = p(y)\)
\item \(F(x, y, y', y'')\) - одннородная функция аргументов \(y, y', y''\) \\
\(F(x, ky, ky', ky'') = k^mF(x, y, y', y'') \Rightarrow\) однородная \\
Используем подстановку \(y = e^{\int z dx}\) \\
Находим \(z\), затем находим \(y(x) = C_2e^{\int z dx}\)
\item \(F(x, y, y', y'')\) - точная производная \\
Если найдем \(\Phi(x, y, y'): F(x, y, y', y'') = \Phi'_x(x, y, y')\),
то решение: \(\Phi(x, y, y') = C\)
\end{enumerate}

\section{Задание 8}
\label{sec:org8a81292}
Метод Пикара \\
Дано \(x_0\), \(y' = f(x, y)\), \(y_0 = y(x_0)\) \\
\[ y_n(x) = y_0 + \int_{x_0}^xf(\xi, y_{k-1}(\xi))d\xi \]

\begin{itemize}
\item \textbf{Пример} \\

\(t_0 = 0\)
\begin{cases}
\case \dot{x} = x - y \\
\case \dot{y} = tx
\end{cases}

\emph{Прим.} \(\dot{u}\) = \(u'_t\)

\begin{gather*}
x(0)=0, y(0)=0 \\
x_n = x_0 + \int_0^t(x_{n-1}(\xi) - y_{n-1}(\xi))d\xi \\
y_n = y_0 + \int_0^t(\xi x_{n - 1}(\xi))d\xi \\
\\
x_1 = 1 + t \\
y_1 = \frac{t^2}{2} \\
x_2 = 1 + t + \frac{t^2}{2} - \frac{t^3}{6} \\
y_2 = \frac{t^2}{2} + \frac{t^3}{6}
\end{gather*}

\item \textbf{Пример}
\[ y'' - y'\sin{x} - x^2 = 0 \]
\[ y(0) = 1, y'(0) = 0 \]

\begin{cases}
\case y' = z \\
\case z' = z\sin{x} + x^2
\end{cases}

\begin{gather*}
y_0 = 1, z_0 = 0 \\
y_n = y_0 + \int_0^xz_{n - 1}(\xi)d\xi \\
z_n = z_0 + \int_0^x(z_{n-1}(\xi)\sin{\xi} + \xi^2)d\xi
\\
z_1 = \frac{x^3}{3} \\
y_2 = 1 + \frac{x^4}{12}
\end{gather*}
\end{itemize}

\section{Полезные техники}
\label{sec:org58e9f1c}
\subsection{Линейное однородное уравнение n-го порядка}
\label{sec:org85040b1}
Имеем уравнение: \(y^{(n)}(x) + a_1y^{(n - 1)}(x) + \dots + a_ny(x) = 0\) \\
Решим такое уравнение: \(\lambda^n + a_1\lambda^{n - 1} + \dots + a_{n-1}\lambda + a_n = 0\) \\
\begin{enumerate}
\item Все корни различные \\
Тогда решение: \(y(x) = C_1e^{\lambda_1x} + \dots + C_ne^{\lambda_nx}\)
\item Есть кратные корни \\
Есть n корней \\
Различные корни: \(\lambda_1, \dots, \lambda_m\) \\
Степени корней: \(k_1, \dots, k_m\) \\
Тогда решение: 
\[ y(x) = C_1e^{\lambda_1x} + C_2xe^{\lambda_1x} + \dots + C_{k_1}x^{k_1 - 1}e^{\lambda_1x} + \dots + C_{n - k_m + 1}e^{\lambda_mx} + C_{n - k_m + 2}xe^{\lambda_mx} + \dots + C_{n}x^{k_m - 1}e^{\lambda_mx}\]
\end{enumerate}
\end{document}
