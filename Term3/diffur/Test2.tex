% Created 2020-12-24 Thu 15:32
% Intended LaTeX compiler: pdflatex
\documentclass{article}
       \usepackage[T1, T2A]{fontenc}
\usepackage[lutf8]{luainputenc}
\usepackage[russian, english]{babel}
\usepackage{minted}
\usepackage{graphicx}
\usepackage{longtable}
\usepackage{hyperref}
\usepackage{natbib}
\usepackage{amssymb}
\usepackage{amsmath}
\usepackage{grffile}
\usepackage{wrapfig}
\usepackage{rotating}
\usepackage{placeins}
\usepackage[normalem]{ulem}
\usepackage{amsmath}
\usepackage{textcomp}
\usepackage{capt-of}
       
       \usepackage{geometry}
       \geometry{a4paper,left=2.5cm,top=2cm,right=2.5cm,bottom=2cm,marginparsep=7pt, marginparwidth=.6in}
\author{Ilya Yaroshevskiy}
\date{\today}
\title{Test 2}
\hypersetup{
 pdfauthor={Ilya Yaroshevskiy},
 pdftitle={Test 2},
 pdfkeywords={},
 pdfsubject={},
 pdfcreator={Emacs 28.0.50 (Org mode 9.3)}, 
 pdflang={English}}
\begin{document}

\maketitle
\tableofcontents


\section{Неоднородные \(n\) -го порядка}
\label{sec:org0e46468}
\href{http://www.math24.ru/\%D0\%BD\%D0\%B5\%D0\%BE\%D0\%B4\%D0\%BD\%D0\%BE\%D1\%80\%D0\%BE\%D0\%B4\%D0\%BD\%D1\%8B\%D0\%B5-\%D0\%B4\%D0\%B8\%D1\%84\%D1\%84\%D0\%B5\%D1\%80\%D0\%B5\%D0\%BD\%D1\%86\%D0\%B8\%D0\%B0\%D0\%BB\%D1\%8C\%D0\%BD\%D1\%8B\%D0\%B5-\%D1\%83\%D1\%80\%D0\%B0\%D0\%B2\%D0\%BD\%D0\%B5\%D0\%BD\%D0\%B8\%D1\%8F-n-\%D0\%B3\%D0\%BE-\%D0\%BF\%D0\%BE\%D1\%80\%D1\%8F\%D0\%B4\%D0\%BA\%D0\%B0-\%D1\%81-\%D0\%BF\%D0\%BE\%D1\%81\%D1\%82\%D0\%BE\%D1\%8F\%D0\%BD\%D0\%BD\%D1\%8B\%D0\%BC\%D0\%B8-\%D0\%BA\%D0\%BE\%D1\%8D\%D1\%84\%D1\%84\%D0\%B8\%D1\%86\%D0\%B8\%D0\%B5\%D0\%BD\%D1\%82\%D0\%B0\%D0\%BC\%D0\%B8.html}{Читать больше}

Уравнения вида \(y^{(n)} + a_1y^{(n - 1)} + \dots + a_{n - 2}y' + a_{n - 1}y = f(x)\) \\
Его решение будет \(y = y_0 + y_1\), где \(y_0\) - решение
соответствующего одноролного, а \(y_1\) - частное(хуй его че значит) решение неоднородного \\
\emph{Алгоритм}
\begin{enumerate}
\item Находим \(y_0 = C_1Y_1 + C_2Y_2 + \dots + C_nY_n\)
\item Принимаем \(C_m\) за функцию от \(x\). Решаем систему \\
\begin{cases}
C_1'Y_1 + C_2'Y_2 + \dots + C_n'Y_n = 0 \\
C_1'Y_1' + C_2'Y_2' + \dots + C_n'Y_n' = 0 \\
\vdots \\
C_1'Y_1^{(n-1)} + C_2'Y_2^{(n-1)} + \dots + C_n'Y_n^{(n-1)} = f(x) \\
\end{cases}
\end{enumerate}
Елси правая часть предсавляет собой \(P_n(x)e^{\alpha x}\) или \((P_n(x)\sin(\beta x) + Q_m(x)\cos(\beta x))e^{\alpha x}\) \\
Тогда ищем решение, например,  так:
\begin{enumerate}
\item Для \(f(x) = P_n(x)e^{\alpha x}\ y_1 = x^sA_n(x)e^{\alpha x}\)
, где \(A_n(x)\) - многочлен той-же степени что и \(P_n(x)\), коээффиценты ищем из кравнения \(y_1 = f(x)\) \\
Если \(\alpha\) совпадает с каким либо корнем \(y_0\), то \(s\) равно кратности этого корня, иначе \(s = 0\)
\item Для \(f(x) = (P_n\sin(\beta x) + Q_m\cos(\beta x))e^{\alpha x}\), \(y_1 = x^s(A_n\sin(\beta x) + B_m\cos(\beta x))e^{\alpha x}\) \\
Остальное аналогично предыдущему пункту
\end{enumerate}

\section{Неодноролные 2-го порядка}
\label{sec:orga5cda0e}
\href{http://www.math24.ru/\%D0\%BD\%D0\%B5\%D0\%BE\%D0\%B4\%D0\%BD\%D0\%BE\%D1\%80\%D0\%BE\%D0\%B4\%D0\%BD\%D1\%8B\%D0\%B5-\%D0\%B4\%D0\%B8\%D1\%84\%D1\%84\%D0\%B5\%D1\%80\%D0\%B5\%D0\%BD\%D1\%86\%D0\%B8\%D0\%B0\%D0\%BB\%D1\%8C\%D0\%BD\%D1\%8B\%D0\%B5-\%D1\%83\%D1\%80\%D0\%B0\%D0\%B2\%D0\%BD\%D0\%B5\%D0\%BD\%D0\%B8\%D1\%8F-\%D0\%B2\%D1\%82\%D0\%BE\%D1\%80\%D0\%BE\%D0\%B3\%D0\%BE-\%D0\%BF\%D0\%BE\%D1\%80\%D1\%8F\%D0\%B4\%D0\%BA\%D0\%B0-\%D1\%81-\%D0\%BF\%D0\%BE\%D1\%81\%D1\%82\%D0\%BE\%D1\%8F\%D0\%BD\%D0\%BD\%D1\%8B\%D0\%BC\%D0\%B8-\%D0\%BA\%D0\%BE\%D1\%8D\%D1\%84\%D1\%84\%D0\%B8\%D1\%86\%D0\%B8\%D0\%B5\%D0\%BD\%D1\%82\%D0\%B0\%D0\%BC\%D0\%B8.html}{Читать больше}

\section{Системы}
\label{sec:org8369600}
\href{http://www.math24.ru/\%D0\%BC\%D0\%B5\%D1\%82\%D0\%BE\%D0\%B4-\%D1\%81\%D0\%BE\%D0\%B1\%D1\%81\%D1\%82\%D0\%B2\%D0\%B5\%D0\%BD\%D0\%BD\%D1\%8B\%D1\%85-\%D0\%B7\%D0\%BD\%D0\%B0\%D1\%87\%D0\%B5\%D0\%BD\%D0\%B8\%D0\%B9-\%D0\%B8-\%D1\%81\%D0\%BE\%D0\%B1\%D1\%81\%D1\%82\%D0\%B2\%D0\%B5\%D0\%BD\%D0\%BD\%D1\%8B\%D1\%85-\%D0\%B2\%D0\%B5\%D0\%BA\%D1\%82\%D0\%BE\%D1\%80\%D0\%BE\%D0\%B2.html}{Читать больше}

\section{Устойчивость}
\label{sec:orgfcc2560}
\begin{cases}
\dot{x} = ax + by \\
\dot{y} = cx + dy
\end{cases}

\begin{pmatrix}
a & b \\
c & d
\end{pmatrix}

\(\Re \lambda_1 > 0\ or\ \Re \lambda 2 > 0\) - неустойчивое


\(v(t, x, y),\ v > 0,\ x, y \ne 0,\ v(0) = 0\), тогда:
\begin{itemize}
\item \(\frac{dv}{dt} \le 0\) - устойчивость
\item \(\frac{dv}{dt} < 0\) - ассимптотически устойчивость
\end{itemize}

\href{http://www.math24.ru/\%D0\%BF\%D0\%BE\%D0\%BB\%D0\%BE\%D0\%B6\%D0\%B5\%D0\%BD\%D0\%B8\%D1\%8F-\%D1\%80\%D0\%B0\%D0\%B2\%D0\%BD\%D0\%BE\%D0\%B2\%D0\%B5\%D1\%81\%D0\%B8\%D1\%8F-\%D0\%BB\%D0\%B8\%D0\%BD\%D0\%B5\%D0\%B9\%D0\%BD\%D1\%8B\%D1\%85-\%D0\%B0\%D0\%B2\%D1\%82\%D0\%BE\%D0\%BD\%D0\%BE\%D0\%BC\%D0\%BD\%D1\%8B\%D1\%85-\%D1\%81\%D0\%B8\%D1\%81\%D1\%82\%D0\%B5\%D0\%BC.html}{Читать больше} \\
\href{http://www.math24.ru/\%D0\%BE\%D1\%81\%D0\%BD\%D0\%BE\%D0\%B2\%D0\%BD\%D1\%8B\%D0\%B5-\%D0\%BF\%D0\%BE\%D0\%BD\%D1\%8F\%D1\%82\%D0\%B8\%D1\%8F-\%D1\%82\%D0\%B5\%D0\%BE\%D1\%80\%D0\%B8\%D0\%B8-\%D1\%83\%D1\%81\%D1\%82\%D0\%BE\%D0\%B9\%D1\%87\%D0\%B8\%D0\%B2\%D0\%BE\%D1\%81\%D1\%82\%D0\%B8.html\#:\~:text=\%D0\%A1\%D1\%82\%D1\%80\%D0\%BE\%D0\%B3\%D0\%BE\%D0\%B5\%20\%D0\%BE\%D0\%BF\%D1\%80\%D0\%B5\%D0\%B4\%D0\%B5\%D0\%BB\%D0\%B5\%D0\%BD\%D0\%B8\%D0\%B5\%20\%D1\%83\%D1\%81\%D1\%82\%D0\%BE\%D0\%B9\%D1\%87\%D0\%B8\%D0\%B2\%D0\%BE\%D1\%81\%D1\%82\%D0\%B8\%20\%D0\%B2\%20\%D1\%82\%D0\%B5\%D1\%80\%D0\%BC\%D0\%B8\%D0\%BD\%D0\%B0\%D1\%85,1892\%20\%D0\%B3\%D0\%BE\%D0\%B4\%D1\%83\%20\%D1\%80\%D1\%83\%D1\%81\%D1\%81\%D0\%BA\%D0\%B8\%D0\%BC\%20\%D0\%BC\%D0\%B0\%D1\%82\%D0\%B5\%D0\%BC\%D0\%B0\%D1\%82\%D0\%B8\%D0\%BA\%D0\%BE\%D0\%BC\%20\%D0\%90.\&text=\%D0\%92\%20\%D1\%81\%D0\%BB\%D1\%83\%D1\%87\%D0\%B0\%D0\%B5\%20n\%3D2\%20\%D1\%83\%D1\%81\%D1\%82\%D0\%BE\%D0\%B9\%D1\%87\%D0\%B8\%D0\%B2\%D0\%BE\%D1\%81\%D1\%82\%D1\%8C,\%E2\%89\%A50\%20(\%D1\%80\%D0\%B8\%D1\%81\%D1\%83\%D0\%BD\%D0\%BE\%D0\%BA\%201).}{Еще больше}
\end{document}
