% Created 2020-12-03 Thu 08:22
% Intended LaTeX compiler: pdflatex
\documentclass{article}
       \usepackage[T1, T2A]{fontenc}
\usepackage[lutf8]{luainputenc}
\usepackage[russian, english]{babel}
\usepackage{minted}
\usepackage{graphicx}
\usepackage{longtable}
\usepackage{hyperref}
\usepackage{natbib}
\usepackage{amssymb}
\usepackage{amsmath}
\usepackage{grffile}
\usepackage{wrapfig}
\usepackage{rotating}
\usepackage{placeins}
\usepackage[normalem]{ulem}
\usepackage{amsmath}
\usepackage{textcomp}
\usepackage{capt-of}
       
       \usepackage{geometry}
       \geometry{a4paper,left=2.5cm,top=2cm,right=2.5cm,bottom=2cm,marginparsep=7pt, marginparwidth=.6in}
\author{Ilya Yaroshevskiy}
\date{\today}
\title{Home work 11}
\hypersetup{
 pdfauthor={Ilya Yaroshevskiy},
 pdftitle={Home work 11},
 pdfkeywords={},
 pdfsubject={},
 pdfcreator={Emacs 28.0.50 (Org mode 9.3)}, 
 pdflang={English}}
\begin{document}

\maketitle
\tableofcontents


\section{}
\label{sec:org75e097c}
\(\overrightarrow{F} = \alpha^2rm\)

\begin{gather*}
m\overrightarrow{a} = \overrightarrow{F} \\
ma_x = -\alpha^2rm\cos{\varphi} \\
ma_y = -\alpha^2rm\sin{\varphi} \\
\cos{\varphi} = \frac{x}{r} \\
\sin{\varphi} = \frac{y}{r} \\
a_x = -\alpha^2x \\
a_y = -\alpha^2y  \\
\ddot{x} + \alpha^2x = 0 \\
\Downarrow \\
\lambda_{1,2} = \pm i\alpha \\
x = C_1\sin{\alpha t} + C_2\cos{\alpha t} \\
\end{gather*}
Аналогично \(y = C_3\sin{\alpha t} + C_4\cos{\alpha t}\) \\
Найдем коэффиценты: 
\begin{gather*}
\begin{cases}
C_2 = x_0 \\
\alpha C_1 = V_{x_0}
\end{cases} \Rightarrow
x = \frac{V_{x_0}}{\alpha}\sin{\alpha t} + x_0\cos{\alpha t} \\
\begin{cases}
C_4 = y_0 \\
\alpha C_3 = V_{y_0}
\end{cases} \Rightarrow
y = \frac{V_{y_0}}{\alpha}\sin{\alpha t} + y_0\cos{\alpha t} \\
\end{gather*}

\section{}
\label{sec:org4b338d7}
\begin{align*}
\begin{cases}
3\ddot{x_1} = - \alpha^2x_1 + \alpha^2(x_2 - x_1) \\
2\ddot{x_2} = - \alpha^2(x_2 - x_1)
\end{cases}
\end{align*}
решим уравнение вида \(\ddot{X} = KX\), где \(X = \begin{bmatrix} x_1 \\ x_2 \end{bmatrix}\), 
\(K = \begin{bmatrix}-\frac{2}{3}\alpha^2 & \frac{1}{3}\alpha^2 \\ \frac{1}{2}\alpha^2 & -\frac{1}{2}\alpha^2 \end{bmatrix}\) \\
Найдем собмтвенные значения матрицы \(K\), они равны \(\omega_1 = \alpha,\ \omega_2 = \frac{\alpha}{\sqrt{3}}\), 
а также соответствующие собственные вектора \(H_1, H_2\), получим \(H1 = \begin{bmatrix} 1 \\ -1 \end{bmatrix}\),
\(H_2 = \begin{bmatrix} 1 \\ 1 \end{bmatrix}\) \\
Теперь найдем решение в виде \(X = A_1 H_1\cos\omega_1 t + A_2 H_2 \cos\omega_2 t\) \\
\(A_1, A_2\) будут зависеть от начального полложения тел и равны \(A_1 = \frac{x_{10} - x_{20}}{2}, A_2 = \frac{x_{10} + x_{20}}{2}\) \\
\begin{align*}
\begin{cases}
x_1 = \frac{x_{10} + x_{20}}{2}\cos{\alpha t} + \frac{x_{10} - x_{20}}{2}\cos{\frac{\alpha}{\sqrt{3}} t} \\
x_2 = \frac{x_{10} + x_{20}}{2}\cos{\alpha t} - \frac{x_{10} - x_{20}}{2}\cos{\frac{\alpha}{\sqrt{3}} t} \\
\end{cases}
\end{align*}
\section{}
\label{sec:org97c29f7}
\section{}
\label{sec:org648e7de}
\(I_C,\ I_L,\ I_R\) - соответсвенно токи через конденсатор, катушку, резистор \\
По правилу Киргофа: \(I_L + I_C - I_R = 0\) \\
А также \(\begin{cases}U_R + U_L = E \\ U_R + U_C = E \end{cases}\) \\
То есть \(\begin{cases}I_R R + \frac{1}{C}\int I_C dt - E = 0\\ I_R R + L \frac{dI_L}{dt} - E = 0 \end{cases}\) \\
Будем находить \(I_R\) из уравнения \(I_R R + L I_L' = V\sin \omega t\), принимая \(I_R = A\sin(\omega t - \varphi)\) \\
\begin{gather*}
I_C' = C(E'' - I_R''R) \\
I_L' = I_R' + C(I_R''R - E'') \\
I_R R + L(I_R' + C(I_R''R - E'')) = E \\
I_R R + I_R'L + I_R''LRC = E + LCE'' \\
RA\sin(\omega t - \varphi) + LA\omega \cos(\omega t - \varphi) - LCRA\omega^2\sin(\omega t - \varphi) = V\sin(\omega t) - LCV\omega^2\sin(\omega t)
\end{gather*}
Найдем коэффиценты при \(\cos(\omega t)\) и \(\sin(\omega t)\) в обеих частях, получим: \\
\begin{align*}
\begin{cases}
RA\cos \varphi + LA\omega\sin \varphi - LCRA\omega^2\cos \varphi = V - LCV\omega^2 \\
-RA\sin \varphi + LA\omega\cos \varphi + LCRA\omega^2\sin \varphi = 0
\end{cases}
\end{align*}
Отсюда: 
\begin{align*}
\begin{cases}
A = \frac{V - LCV\omega^2}{R\cos \varphi + L\omega\sin \varphi - LCR\omega^2\cos \varphi} \\
\tg\varphi = \frac{L\omega}{R(1 - LC\omega^2)}
\end{cases}
\end{align*}
\begin{gather*}
\cos\varphi = \frac{R(1 - LC\omega^2)}{\sqrt{L^2\omega^2 + R^2(1-LC\omega^2)^2}} \\
A = \frac{V(1-LC\omega^2)}{\sqrt{L^2\omega^2 + R^2(1 - LC\omega^2)^2}}
\end{gather*}
\(\frac{dA}{dw} = 0 \Rightarrow \omega^2 = -\frac{1}{LC}\)
\section{}
\label{sec:orgd889739}
\end{document}
