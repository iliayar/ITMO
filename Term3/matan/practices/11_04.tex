% Created 2020-11-04 Wed 17:49
% Intended LaTeX compiler: pdflatex
\documentclass{article}
       \usepackage[T1, T2A]{fontenc}
\usepackage[lutf8]{luainputenc}
\usepackage[russian, english]{babel}
\usepackage{minted}
\usepackage{graphicx}
\usepackage{longtable}
\usepackage{hyperref}
\usepackage{natbib}
\usepackage{amssymb}
\usepackage{amsmath}
\usepackage{grffile}
\usepackage{wrapfig}
\usepackage{rotating}
\usepackage{placeins}
\usepackage[normalem]{ulem}
\usepackage{amsmath}
\usepackage{textcomp}
\usepackage{tikz}
\usepackage{capt-of}
       
       \usepackage{geometry}
       \geometry{a4paper,left=2.5cm,top=2cm,right=2.5cm,bottom=2cm,marginparsep=7pt, marginparwidth=.6in}
\author{Ilya Yaroshevskiy}
\date{\today}
\title{11.04 Практика}
\hypersetup{
 pdfauthor={Ilya Yaroshevskiy},
 pdftitle={11.04 Практика},
 pdfkeywords={},
 pdfsubject={},
 pdfcreator={Emacs 27.1 (Org mode 9.3)}, 
 pdflang={English}}
\begin{document}

\maketitle
\tableofcontents


\section{Занятие}
\label{sec:orgde3c081}
\(f(x, y, z) \righarrow\) экстремум \\
\(g(x, y, z) = 0\) \\
\[ F := f + \lambda g \]

\begin{cases}
\case F_x' = 0 \\
\case F_y' = 0 \\
\case F_z' = 0 \\
\case g = 0 \\
\end{cases}

Решение: неизвестные: \(x, y, z, \lambda\) \\
\(x_0, y_0, z_0, \lambda_0\) \\
Нужно ответить на вопрос: \\
\(d^2F > 0\) или \(< 0\) или бывает и такой и такой \\
если мы подставмм в него \(dx, dy, dz\) \\
связанные соотношением \(dg(x_0, y_0, z_0)\)

\subsection{3661}
\label{sec:org96855ae}
\[ u = x^2 + y^2 + z^2 \]
\[ \frac{x^2}{a^2} + \frac{y^2}{b^2} + \frac{z^2}{c^2} \]
\[ a > b > c > 0 \]

\begin{center}
\includegraphics[scale=0.7]{/home/iliayar/Pictures/screenshots/2020-11-04-173648_235x126_scrot.png}
\end{center}

\[ F = x^2 + y ^2 + z^2 + \lambda(\frac{x^2}{a^2} + \frac{y^2}{b^2} + \frac{z^2}{c^2} - 1)\]

\begin{cases}
\case 2x(1 + \frac{\lambda}{a^2}) = 0 \\
\case 2y(1 + \frac{\lambda}{b^2}) = 0 \\
\case 2z(1 + \frac{\lambda}{c^2}) = 0 \\
\case \frac{x^2}{a^2} + \frac{y^2}{b^2} + \frac{z^2}{c^2} = 1
\end{cases}

\(x = \pm a, y = z = 0\ \lambda = -a^2\) \\
\(y = \pm b, x = z = 0\ \lambda = -b^2\) \\
\(z = \pm c, x = y = 0\ \lambda = -c^2\) \\

Изучаем подозрительные точки: \\
(ref:1) \[ d^2F = 0 \cdot dx^2 + (1 - \frac{a^2}{b^2})2dy^2 + (1 - \frac{a^2}{c^2})2dz^2 \]

???, что \(d^2F \le 0\) но, возможно иметт место вырожденности

\emph{Примеание:} \\
\(> 0\) min, \(< 0\) max, \(> 0, < 0\) нет экстремума \\
\(> 0\), вырожд. - недостаточно информации
\end{document}
