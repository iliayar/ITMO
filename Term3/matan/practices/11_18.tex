% Created 2020-12-21 Mon 14:37
% Intended LaTeX compiler: pdflatex
\documentclass{article}
       \usepackage[T1, T2A]{fontenc}
\usepackage[lutf8]{luainputenc}
\usepackage[russian, english]{babel}
\usepackage{minted}
\usepackage{graphicx}
\usepackage{longtable}
\usepackage{hyperref}
\usepackage{natbib}
\usepackage{amssymb}
\usepackage{amsmath}
\usepackage{grffile}
\usepackage{wrapfig}
\usepackage{rotating}
\usepackage{placeins}
\usepackage[normalem]{ulem}
\usepackage{amsmath}
\usepackage{textcomp}
\usepackage{capt-of}
       
       \usepackage{geometry}
       \geometry{a4paper,left=2.5cm,top=2cm,right=2.5cm,bottom=2cm,marginparsep=7pt, marginparwidth=.6in}
\author{Ilya Yaroshevskiy}
\date{\today}
\title{Схрдимотсь полследовательности функций}
\hypersetup{
 pdfauthor={Ilya Yaroshevskiy},
 pdftitle={Схрдимотсь полследовательности функций},
 pdfkeywords={},
 pdfsubject={},
 pdfcreator={Emacs 28.0.50 (Org mode 9.3)}, 
 pdflang={English}}
\begin{document}

\maketitle
\tableofcontents


\section{Сходимость посл. фунцкий}
\label{sec:orgb274afa}
\(f_n \rightarrow f\) на \(E \Leftrightarrow \rho(f_n, f) \rightarrow 0\)  \\
\(\rho(f_n, f) = \sup_{x \in E}|f_n(x) - f(x)|\) \\
\emph{Пример} \textbf{2749} \\
\(f_n(x) = \frac{1}{x + n}\ E=(0, +\infty)\) \\
Есть ли равномерную сходимость \\
\emph{Напоминание}: Равномерная сходимость \(F_n \rightarrow f \Rightarrow \forall x \in E f_n(x) \rightarrow f(x)\)
\begin{enumerate}
\item Ищем \(f\) \\
\end{enumerate}
фиксируем x: \(\lim\limits_{n \to +\infty} f_n(x) = \lim\limits_{n \to +\infty} \frac{1}{x + n} = 0\), т.е \(f(x) = 0\) \\
\begin{enumerate}
\item Проверяем равномерную сходимость \\
\end{enumerate}
\(\rho(f_n, f) = \sup\limits_{x \in (0, +\infty)}\left|\frac{1}{x + n} - 0\right| = \sup\limits_{x\in(0, +\infty)}\frac{1}{x + n} = \frac{1}{n} \xrightarrow[n \to +\infty]{} 0\)
Равномерно сходится

\section{Занятие}
\label{sec:orgf0e461e}
\textbf{2756а} \\
\(f_n(x) = arctg x \xrightarrow[n \to +\infty]{} \frac{\pi}{2}\) \\
\(\sup\limits_{x \in (0, +\infty)} \left|f_n(x) - \frac{\pi}{2}\right| = sup \left| arcrg x - \frac{\pi}{2}\right| \ge \lim\limits_{x \to 0} \left| arctg(nx) - \frac{\pi}{2} = \frac{\pi}{2} \right|\) \textlnot{}\(\to\) 0\\
\(\ge \left| arctg(nx) - \frac{\pi}{2} \right|\Big|_{x = \frac{1}{n}} = \frac{\pi}{4} \not\to 0\) \\
\textbf{2756б} \\
\(f_n(x) = x arctg(nx) \to \frac{n}{2}x\) \\
\(\sup\limits_{n \in (0, +\infty)}\left|x(arctg(nx) - \frac{\pi}{4})\right|\)
\end{document}
