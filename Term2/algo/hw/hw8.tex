\documentclass[12pt]{article}
\usepackage[T1, T2A]{fontenc}
\usepackage[lutf8]{luainputenc}
\usepackage[russian]{babel}
\usepackage{listings}
                    
\begin{document}
\section{Задача 8.5}
Заведем массив расстояний \verb|dist|, в \verb|dist[v, k]| хранится расcтояние от
вершины \verb$v$ до $2^{k}$-го предка.
Предподсчитаем его вместе с двоичными подъемами тамким образом: \\
\begin{verb}
dist[v, k]= dist[v, k-1] + dist[jmp[v, k-1], k-1]
\end{verb}
и \verb|dist[v, 0]| расcтояние от вершины \verb|v| до родителя. \\
Будем искать ближайшего предка $v$ растояние до которого не меньше $l$ так: \\
Если \verb|k = 0|, то мы находимся в вершине, чей предок и есть искомая вершина. \\
Иначе если \verb|dist[v, k] >= l|, то запускаем поиск от \verb|v, k - 1, l|
\\
Иначе запускаем поиск от \verb|jmp[v, k], k - 1, l - dist[v, k]| \\
\begin{lstlisting}[language=Python]
search(v, k, l):
  if(k == 0):
    return jmp[v, k];
  if(dist[v, k] >= l):
    return search(v, k-1, l);
  return search(jmp[v, k], k - 1, l - dist[v, k]);
\end{lstlisting}
Т.к. с каждым шагом \verb|k| уменьшается, то время работы $O(k_{0})$, где
$k_{0}$ - начальное, такое максимальное, что $2^{k_{0}}$ не больше количества вершин на пути до корня. Тогда видно,
что время работы $O(\log n)$
\end{document}
