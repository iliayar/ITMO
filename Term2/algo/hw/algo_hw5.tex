% Drawing a graph using the PG 3.0 graphdrawing library
% Author: Mark Wibrow
\documentclass[12pt]{article}
\usepackage{tikz}
\usepackage[T1, T2A]{fontenc}
\usepackage[lutf8]{luainputenc}
\usepackage[russian]{babel}

\usetikzlibrary{graphdrawing}
\usetikzlibrary{graphs}
\usegdlibrary{trees}

\tikzset {
  delete/.style = {fill=red},
  rot/.style = {fill=yellow},
  ok/.style = {fill=green}
}
% \tikzset{fontscale/.style = {font=0.01}
% }
\begin{document}
\section{Задача 5.12}
Удаляем вершину 4
\begin{center}
\begin{tikzpicture}[every node/.style={circle, draw}]
  \graph [binary tree layout, grow=down, fresh nodes, level distance=0.2in, sibling distance=0.2in]
  {
    5 -- { 3 -- {2 -- {1,}, 4 [delete]}, 8 -- {7 -- {6,}, 10 -- {9, 11 -- {,12}}}
    } 
  };
\end{tikzpicture}
\end{center}

Поворачиваем ребро 3-2 \\
\begin{center}
\begin{tikzpicture}[>=stealth, every node/.style={circle, draw, minimum size=0.2cm}]
  \graph [binary tree layout, grow=down, fresh nodes, level distance=0.2in, sibling distance=0.2in]
  {
    5 -- { 3[rot] -- {2 -- {1,},}, 8 -- {7 -- {6,}, 10 -- {9, 11 -- {,12}}}
    } 
  };
\end{tikzpicture}
\end{center}
\begin{center}
\begin{tikzpicture}[>=stealth, every node/.style={circle, draw, minimum size=0.2cm}]
  \graph [binary tree layout, grow=down, fresh nodes, level distance=0.2in, sibling distance=0.2in]
  {
    5 -- { 2[ok] -- {1, 3}, 8 -- {7 -- {6,}, 10 -- {9, 11 -- {,12}}}
    } 
  };
\end{tikzpicture}
\end{center}

Поворачиваем ребро 5-8
\begin{center}
\begin{tikzpicture}[>=stealth, every node/.style={circle, draw, minimum size=0.2cm}]
  \graph [binary tree layout, grow=down, fresh nodes, level distance=0.2in, sibling distance=0.2in]
  {
    5[rot] -- { 2 -- {1, 3}, 8 -- {7 -- {6,}, 10 -- {9, 11 -- {,12}}}
    } 
  };
\end{tikzpicture}
\end{center}
\begin{center}
\begin{tikzpicture}[>=stealth, every node/.style={circle, draw, minimum size=0.2cm}]
  \graph [binary tree layout, grow=down, fresh nodes, level distance=0.2in, sibling distance=0.2in]
  {
    8[ok] -- {5 -- {2 -- {1,3}, 7 -- 6}, 10 -- {9, 11 -- {,12}}
    }
  };
\end{tikzpicture}
\end{center}
\end{document}

