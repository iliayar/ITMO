% Created 2021-03-04 Thu 00:09
% Intended LaTeX compiler: pdflatex
\documentclass[english]{article}
\usepackage[T1, T2A]{fontenc}
\usepackage[lutf8]{luainputenc}
\usepackage[english, russian]{babel}
\usepackage{minted}
\usepackage{graphicx}
\usepackage{longtable}
\usepackage{hyperref}
\usepackage{xcolor}
\usepackage{natbib}
\usepackage{amssymb}
\usepackage{amsmath}
\usepackage{caption}
\usepackage{mathtools}
\usepackage{amsthm}
\usepackage{tikz}
\usepackage{grffile}
\usepackage{extarrows}
\usepackage{wrapfig}
\usepackage{rotating}
\usepackage{placeins}
\usepackage[normalem]{ulem}
\usepackage{amsmath}
\usepackage{textcomp}
\usepackage{capt-of}

\usepackage{geometry}
\geometry{a4paper,left=2.5cm,top=2cm,right=2.5cm,bottom=2cm,marginparsep=7pt, marginparwidth=.6in}

 \usepackage{hyperref}
 \hypersetup{
     colorlinks=true,
     linkcolor=blue,
     filecolor=orange,
     citecolor=black,      
     urlcolor=cyan,
     }

\usetikzlibrary{decorations.markings}
\usetikzlibrary{cd}
\usetikzlibrary{patterns}

\newcommand\addtag{\refstepcounter{equation}\tag{\theequation}}
\newcommand{\eqrefoffset}[1]{\addtocounter{equation}{-#1}(\arabic{equation}\addtocounter{equation}{#1})}


\newcommand{\R}{\mathbb{R}}
\renewcommand{\C}{\mathbb{C}}
\newcommand{\N}{\mathbb{N}}
\newcommand{\rank}{\text{rank}}
\newcommand{\const}{\text{const}}
\newcommand{\grad}{\text{grad}}

\theoremstyle{plain}
\newtheorem{axiom}{Аксиома}
\newtheorem{lemma}{Лемма}
\newtheorem{manuallemmainner}{Лемма}
\newenvironment{manuallemma}[1]{%
  \renewcommand\themanuallemmainner{#1}%
  \manuallemmainner
}{\endmanuallemmainner}

\theoremstyle{remark}
\newtheorem*{remark}{Примечание}
\newtheorem*{solution}{Решение}
\newtheorem{corollary}{Следствие}[theorem]
\newtheorem*{examp}{Пример}
\newtheorem*{observation}{Наблюдение}

\theoremstyle{definition}
\newtheorem{task}{Задача}
\newtheorem{theorem}{Теорема}[section]
\newtheorem*{definition}{Определение}
\newtheorem*{symb}{Обозначение}
\newtheorem{manualtheoreminner}{Теорема}
\newenvironment{manualtheorem}[1]{%
  \renewcommand\themanualtheoreminner{#1}%
  \manualtheoreminner
}{\endmanualtheoreminner}
\captionsetup{justification=centering,margin=2cm}
\newenvironment{colored}[1]{\color{#1}}{}

\tikzset{->-/.style={decoration={
  markings,
  mark=at position .5 with {\arrow{>}}},postaction={decorate}}}
\makeatletter
\newcommand*{\relrelbarsep}{.386ex}
\newcommand*{\relrelbar}{%
  \mathrel{%
    \mathpalette\@relrelbar\relrelbarsep
  }%
}
\newcommand*{\@relrelbar}[2]{%
  \raise#2\hbox to 0pt{$\m@th#1\relbar$\hss}%
  \lower#2\hbox{$\m@th#1\relbar$}%
}
\providecommand*{\rightrightarrowsfill@}{%
  \arrowfill@\relrelbar\relrelbar\rightrightarrows
}
\providecommand*{\leftleftarrowsfill@}{%
  \arrowfill@\leftleftarrows\relrelbar\relrelbar
}
\providecommand*{\xrightrightarrows}[2][]{%
  \ext@arrow 0359\rightrightarrowsfill@{#1}{#2}%
}
\providecommand*{\xleftleftarrows}[2][]{%
  \ext@arrow 3095\leftleftarrowsfill@{#1}{#2}%
}
\makeatother
\author{Ilya Yaroshevskiy}
\date{\today}
\title{ДЗ}
\hypersetup{
 pdfauthor={Ilya Yaroshevskiy},
 pdftitle={ДЗ},
 pdfkeywords={},
 pdfsubject={},
 pdfcreator={Emacs 28.0.50 (Org mode )}, 
 pdflang={English}}
\begin{document}

\maketitle
\tableofcontents

\usetikzlibrary{decorations.pathmorphing}


\section*{Задание 3.9}
\label{sec:org7121ed2}
\emph{Дан неориентированный граф. Найти максимальное число вершинно непересекающихся путей
из \(s\) в \(t\)}.

\noindent\rule{\textwidth}{0.5pt}
Сделаем из неориентированного графа ориентированный, построив для
каждого неор. ребра \((u, v)\), ор. ребра \((u, v)\) и \((v, u)\) с
пропускными способностями равными \(1\).

Теперь перестроим граф следующим образом: каждую вершину \(u\), кроме \(s\) и \(t\), заменим
на две \(u_1\) и \(u_2\). \(\forall v: (v, u) \in E\) добавим ребро
\((v, u_1)\), аналогично \(\forall w: (u, w) \in E\) добавим ребро
\((u_2, w)\). Добавим ребро \((u_1, u_2)\) с пропускной способностью
равной \(1\).

\begin{center}
\begin{tikzpicture}
\node[draw=black] (A) at (0, 0) [circle] {\(u\)};
\draw[decorate,decoration=snake,->] (-1.5, 0) -- (A);
\draw[decorate,decoration=snake,->] (A) -- (1.5, 0);
\node at (0, -1) {\(\Downarrow\)};
\node[draw=black] (B) at (-1, -2) [circle] {\(u_1\)};
\node[draw=black] (C) at (1, -2) [circle] {\(u_2\)};
\draw[->] (B) -- node[above] {\(1\)} (C);
\draw[decorate,decoration=snake,->] (-2.4, -2) -- (B);
\draw[decorate,decoration=snake,->] (C) -- (2.4, -2);
\end{tikzpicture}
\end{center}

Рассмотрим вершинно непересекающиеся пути \(s \leadsto t\) в исходном графе. Заметим
что в новом графе соответсвующие пути также будут вершинно непересекающимися,
значит по каждому такому пути можно пустить поток величины
\(1\). Тогда будет выполняться неравенство \(k \le F_{\max}\), где \(k\) ---
число вершинно непересекающихся путей, \(F_{\max}\) --- максимальный поток в новом графе.

Рассмотрим максимальный поток, в котором поток по каждому ребру
целочисленный(задача 3.3).
\begin{enumerate}
\item \label{3_9_1}'Запустим' dfs из начальной вершины до конечной: будем
ходить по ребрам, поток по которым равен \(1\).
\item \label{3_9_2}Заметим, что для всех
ребер \((u_1, u_2)\) поток по ним не больше \(1\), значит суммарный
входящий в \(u_1\) и выходящий из \(u_2\) потоки также не превышают \(1\).
\begin{center}
\begin{tikzpicture}
\node[draw=black] (A) at (-1, 0) [circle] {\(u_1\)};
\node[draw=black] (B) at (1, 0) [circle] {\(u_2\)};
\draw[->] (-2.5, 1) -- node[above] {\(\color{red}1\color{black}/1\)} (A);
\draw[->] (-2.5, 0) -- node[above] {\(\color{red}0\color{black}/1\)} (A);
\draw[->] (-2.5, -1) -- node[above] {\(\color{red}0\color{black}/1\)} (A);
\draw[->] (A) -- node[above] {\(\color{red}1\color{black}/1\)} (B);
\draw[->] (B) -- node[above] {\(\color{red}1\color{black}/1\)} (2.5, 0);
\draw[->] (B) -- node[above] {\(\color{red}0\color{black}/1\)} (2.5, 1);
\draw[->] (B) -- node[above] {\(\color{red}0\color{black}/1\)} (2.5, -1);
\end{tikzpicture}
\end{center}
\end{enumerate}

С учетом п. \ref{3_9_2}, различные пути найденные в п. \ref{3_9_1} будут вершинно непересекающиеся и
их количество будет равно суммарному потоку: \(k \le k_{\max} \le F_{\max},\ k = F_{\max} \Rightarrow k_{\max} = F_{\max}\)

Осталось показать что найденные пути будут существовать в исходном
графе. Действительно, если путь проходит через вершину \(u_1\), то он
проходит и через вершину \(u_2\). Значит можно сжать эти две вершину в
одну, которая в свою очередь однозначно определяется в исходном графе.

Получаем что максимальное количество вершинно непересекающихся путей в
неориентированном графе равно максимальному потоку в графе,
построенному выше указаным образом.
\end{document}
