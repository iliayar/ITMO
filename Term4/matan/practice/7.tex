% Created 2021-04-07 Wed 23:05
% Intended LaTeX compiler: pdflatex

  \documentclass[english]{article}
  \usepackage[T1, T2A]{fontenc}
\usepackage[lutf8]{luainputenc}
\usepackage[english, russian]{babel}
\usepackage{minted}
\usepackage{graphicx}
\usepackage{longtable}
\usepackage{hyperref}
\usepackage{xcolor}
\usepackage{natbib}
\usepackage{amssymb}
\usepackage{stmaryrd}
\usepackage{amsmath}
\usepackage{caption}
\usepackage{mathtools}
\usepackage{amsthm}
\usepackage{tikz}
\usepackage{grffile}
\usepackage{extarrows}
\usepackage{wrapfig}
\usepackage{algorithm}
\usepackage{algorithmic}
\usepackage{lipsum}
\usepackage{rotating}
\usepackage{placeins}
\usepackage[normalem]{ulem}
\usepackage{amsmath}
\usepackage{textcomp}
\usepackage{capt-of}
  
  \usepackage{geometry}
  \geometry{a4paper,left=2.5cm,top=2cm,right=2.5cm,bottom=2cm,marginparsep=7pt, marginparwidth=.6in}
   \usepackage{hyperref}
 \hypersetup{
     colorlinks=true,
     linkcolor=blue,
     filecolor=orange,
     citecolor=black,      
     urlcolor=cyan,
     }

\usetikzlibrary{decorations.markings}
\usetikzlibrary{cd}
\usetikzlibrary{patterns}
\usetikzlibrary{automata, arrows}

\newcommand\addtag{\refstepcounter{equation}\tag{\theequation}}
\newcommand{\eqrefoffset}[1]{\addtocounter{equation}{-#1}(\arabic{equation}\addtocounter{equation}{#1})}


\newcommand{\R}{\mathbb{R}}
\renewcommand{\C}{\mathbb{C}}
\newcommand{\N}{\mathbb{N}}
\newcommand{\A}{\mathfrak{A}}
\newcommand{\rank}{\mathop{\rm rank}\nolimits}
\newcommand{\const}{\var{const}}
\newcommand{\grad}{\mathop{\rm grad}\nolimits}

\newcommand{\todo}{{\color{red}\fbox{\text{Доделать}}}}
\newcommand{\fixme}{{\color{red}\fbox{\text{Исправить}}}}

\newcounter{propertycnt}
\setcounter{propertycnt}{1}
\newcommand{\beginproperty}{\setcounter{propertycnt}{1}}

\theoremstyle{plain}
\newtheorem{propertyinner}{Свойство}
\newenvironment{property}{
  \renewcommand\thepropertyinner{\arabic{propertycnt}}
  \propertyinner
}{\endpropertyinner\stepcounter{propertycnt}}
\newtheorem{axiom}{Аксиома}
\newtheorem{lemma}{Лемма}
\newtheorem{manuallemmainner}{Лемма}
\newenvironment{manuallemma}[1]{%
  \renewcommand\themanuallemmainner{#1}%
  \manuallemmainner
}{\endmanuallemmainner}

\theoremstyle{remark}
\newtheorem*{remark}{Примечание}
\newtheorem*{solution}{Решение}
\newtheorem{corollary}{Следствие}[theorem]
\newtheorem*{examp}{Пример}
\newtheorem*{observation}{Наблюдение}

\theoremstyle{definition}
\newtheorem{task}{Задача}
\newtheorem{theorem}{Теорема}[section]
\newtheorem*{definition}{Определение}
\newtheorem*{symb}{Обозначение}
\newtheorem{manualtheoreminner}{Теорема}
\newenvironment{manualtheorem}[1]{%
  \renewcommand\themanualtheoreminner{#1}%
  \manualtheoreminner
}{\endmanualtheoreminner}
\captionsetup{justification=centering,margin=2cm}
\newenvironment{colored}[1]{\color{#1}}{}

\tikzset{->-/.style={decoration={
  markings,
  mark=at position .5 with {\arrow{>}}},postaction={decorate}}}
\makeatletter
\newcommand*{\relrelbarsep}{.386ex}
\newcommand*{\relrelbar}{%
  \mathrel{%
    \mathpalette\@relrelbar\relrelbarsep
  }%
}
\newcommand*{\@relrelbar}[2]{%
  \raise#2\hbox to 0pt{$\m@th#1\relbar$\hss}%
  \lower#2\hbox{$\m@th#1\relbar$}%
}
\providecommand*{\rightrightarrowsfill@}{%
  \arrowfill@\relrelbar\relrelbar\rightrightarrows
}
\providecommand*{\leftleftarrowsfill@}{%
  \arrowfill@\leftleftarrows\relrelbar\relrelbar
}
\providecommand*{\xrightrightarrows}[2][]{%
  \ext@arrow 0359\rightrightarrowsfill@{#1}{#2}%
}
\providecommand*{\xleftleftarrows}[2][]{%
  \ext@arrow 3095\leftleftarrowsfill@{#1}{#2}%
}
\makeatother

\newenvironment{rualgo}[1][]
  {\begin{algorithm}[#1]
     \selectlanguage{russian}%
     \floatname{algorithm}{Алгоритм}%
     \renewcommand{\algorithmicif}{{\color{red}\textbf{если}}}%
     \renewcommand{\algorithmicthen}{{\color{red}\textbf{тогда}}}%
     \renewcommand{\algorithmicelse}{{\color{red}\textbf{иначе}}}%
     \renewcommand{\algorithmicend}{{\color{red}\textbf{конец}}}%
     \renewcommand{\algorithmicfor}{{\color{red}\textbf{для}}}%
     \renewcommand{\algorithmicto}{{\color{red}\textbf{до}}}%
     \renewcommand{\algorithmicdo}{{\color{red}\textbf{делать}}}%
     \renewcommand{\algorithmicwhile}{{\color{red}\textbf{пока}}}%
     \renewcommand{\algorithmicrepeat}{{\color{red}\textbf{повторять}}}%
     \renewcommand{\algorithmicuntil}{{\color{red}\textbf{до тех пор пока}}}%
     \renewcommand{\algorithmicloop}{{\color{red}\textbf{повторять}}}%
     \renewcommand{\algorithmicnot}{{\color{blue}\textbf{не}}}%
     \renewcommand{\algorithmicand}{{\color{blue}\textbf{и}}}%
     \renewcommand{\algorithmicor}{{\color{blue}\textbf{или}}}%
     \renewcommand{\algorithmicrequire}{{\color{blue}\textbf{Предусловие}}}%
     \renewcommand{\algorithmicrensure}{{\color{blue}\textbf{Постусловие}}}%
     \renewcommand{\algorithmicrtrue}{{\color{blue}\textbf{истинна}}}%
     \renewcommand{\algorithmicrfalse}{{\color{blue}\textbf{ложь}}}%
     % Set other language requirements
  }
  {\end{algorithm}}
\author{Ilya Yaroshevskiy}
\date{\today}
\title{Практика 7}
\hypersetup{
 pdfauthor={Ilya Yaroshevskiy},
 pdftitle={Практика 7},
 pdfkeywords={},
 pdfsubject={},
 pdfcreator={Emacs 28.0.50 (Org mode 9.4.4)}, 
 pdflang={English}}
\begin{document}

\maketitle
\tableofcontents

\begin{task}
\[ \int_\alpha^\beta d\varphi \int_\gamma^\delta d\psi \dots \]
\end{task}
\begin{solution}
\[ x = r_0 \cos\varphi\cos\psi \]
\[ y = r_0 \sin\varphi\cos\psi \]
\[ z = r_0 \sin\psi \]
\[ \begin{vmatrix}
i & -r_0 \cos \varphi \cos \psi & - r \cos\varphi\sin\psi \\
j & r_0 \cos\varphi \cos \psi & - r\sin\varphi\sin\psi \\
k & 0 & r\cos\varphi
\end{vmatrix} = r^2 \begin{pmatrix}
\cos \varphi \cos^2 \psi \\
-\sin\varphi\cos^2\psi \\
\cos\psi\sin\psi
\end{pmatrix} \]
\[ r^2 \int^\beta_\alpha d\varphi \int_\gamma^\delta d\psi \cos\psi \]
\end{solution}

\begin{task}
\[ \iint\limits_{x\ge0,y\ge0} \frac{dx\,dy}{(1 + x^2 + y^3)^p} \]
\end{task}
\begin{solution}
\[ \left[\begin{array}{l}
x = r \cos\varphi \\
y = r^{\frac{2}{3}}\sin^{\frac{2}{3}}\varphi
\end{array}\right. \]
\[ I = \int\limits_0^{\frac{\pi}{2}}\int\limits_0^{ + \infty} \frac{\frac{2}{3} r^{\frac{2}{3}} \sin^{-\frac{1}{3}}\varphi}{(1 + r^2)^p}  \]
\end{solution}
\section{Тройной интеграл}
\label{sec:org09b7bf6}
Есть трехмерная фигура \(\Omega\)
\[ \iiint\limits_\Omega f\,dx\,dy\,dz \addtag\label{int_1} \]
\begin{description}
\item[{\(3 = 1 + 2\)}] \[ \ref{int_1} = \int_a^b dx \iint_{\Omega_x} f \,dy\,dz \]
\item[{\(3 = 2 + 1\)}] \[ \ref{int_1} = \iint_{\Omega_{xy}} dx\,dy \int_{z(x, y)}^{z_2(x, y)}f\,dz \]
\end{description}
\[ \iiint x y^2 z^3 \addtag\label{int_2} \]
\begin{itemize}
\item \(z = xy\)
\item \(y = x\)
\item \(x = 1\)
\item \(z = 0\)
\end{itemize}


\begin{enumerate}
\item \[ \ref{int_2} = \int\limits_0^1 dx \int\limits_0^x dy \int\limits_0^{xy} x y^2 z^3 dz \]
\item \[ \ref{int_2} = \int dy \int_y^1 dx \int_0^{xy} f dz\]
\item \[ \ref{int_2} = \int_0^1 dz \int_{\sqrt{z}}^1 dx \int_{\frac{z}{x}}^x f\,dy \]
\end{enumerate}
\begin{task}
\[ \int\limits_0^1 dx \int\limits_0^{1 - x}dy\int\limits_0^{x + y} f\, dz \]
\end{task}
\begin{solution}
\[ = \int\limits_0^1 dz \int\limits_z^1 dx \int\limits_0^{z - x} f\, dy + \int_0^z dx \int_{z - x}^{1 - x} f\, dy \]
\end{solution}

\begin{task}
\[ \sum_{k = 0}^N\sum_{l = 0}^k\sum_{m = 0}^l \frac{1}{m^3} \]
\[ \int\limits_0^a d\xi \int\limits_0^\xi d\eta \int\limits_0^\tau f(\zeta) d\zeta \]
\end{task}
\begin{solution}
\[ 0 \le m \le l \le k \le N \]
\[ 0 \le \zeta \le \eta \le \xi \le a \]
\[ = \int\limits_0^1 d\zeta \int\limits_\zeta^a \int\limits_\eta^a f(\zeta) d\xi = \int_0^a f(\zeta) \frac{(a - \zeta)2}{2} d\zeta \]
\end{solution}

\begin{task}
\[ \iiint \sqrt{x^2 + y^2 + z^2}\,dx\,dy\,dz \]
\[ x^2 + y^2 + z^2 = z \]
\end{task}
\begin{solution}
\[ \begin{cases}
x = r \cos\phi \cos\psi \\
\vdots
\end{cases} \]
\[ J = r^2\cos\psi \]
\end{solution}
\end{document}
