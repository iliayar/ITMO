% Created 2021-03-01 Mon 13:05
% Intended LaTeX compiler: pdflatex
\documentclass[english]{article}
\usepackage[T1, T2A]{fontenc}
\usepackage[lutf8]{luainputenc}
\usepackage[english, russian]{babel}
\usepackage{minted}
\usepackage{graphicx}
\usepackage{longtable}
\usepackage{hyperref}
\usepackage{xcolor}
\usepackage{natbib}
\usepackage{amssymb}
\usepackage{amsmath}
\usepackage{caption}
\usepackage{mathtools}
\usepackage{amsthm}
\usepackage{tikz}
\usepackage{grffile}
\usepackage{extarrows}
\usepackage{wrapfig}
\usepackage{rotating}
\usepackage{placeins}
\usepackage[normalem]{ulem}
\usepackage{amsmath}
\usepackage{textcomp}
\usepackage{capt-of}

\usepackage{geometry}
\geometry{a4paper,left=2.5cm,top=2cm,right=2.5cm,bottom=2cm,marginparsep=7pt, marginparwidth=.6in}

 \usepackage{hyperref}
 \hypersetup{
     colorlinks=true,
     linkcolor=blue,
     filecolor=orange,
     citecolor=black,      
     urlcolor=cyan,
     }

\usetikzlibrary{decorations.markings}
\usetikzlibrary{cd}
\usetikzlibrary{patterns}

\newcommand\addtag{\refstepcounter{equation}\tag{\theequation}}
\newcommand{\eqrefoffset}[1]{\addtocounter{equation}{-#1}(\arabic{equation}\addtocounter{equation}{#1})}


\newcommand{\R}{\mathbb{R}}
\renewcommand{\C}{\mathbb{C}}
\newcommand{\N}{\mathbb{N}}
\newcommand{\rank}{\text{rank}}
\newcommand{\const}{\text{const}}
\newcommand{\grad}{\text{grad}}

\theoremstyle{plain}
\newtheorem{axiom}{Аксиома}
\newtheorem{lemma}{Лемма}
\newtheorem{manuallemmainner}{Лемма}
\newenvironment{manuallemma}[1]{%
  \renewcommand\themanuallemmainner{#1}%
  \manuallemmainner
}{\endmanuallemmainner}

\theoremstyle{remark}
\newtheorem*{remark}{Примечание}
\newtheorem*{solution}{Решение}
\newtheorem{corollary}{Следствие}[theorem]
\newtheorem*{examp}{Пример}
\newtheorem*{observation}{Наблюдение}

\theoremstyle{definition}
\newtheorem{task}{Задача}
\newtheorem{theorem}{Теорема}[section]
\newtheorem*{definition}{Определение}
\newtheorem*{symb}{Обозначение}
\newtheorem{manualtheoreminner}{Теорема}
\newenvironment{manualtheorem}[1]{%
  \renewcommand\themanualtheoreminner{#1}%
  \manualtheoreminner
}{\endmanualtheoreminner}
\captionsetup{justification=centering,margin=2cm}
\newenvironment{colored}[1]{\color{#1}}{}

\tikzset{->-/.style={decoration={
  markings,
  mark=at position .5 with {\arrow{>}}},postaction={decorate}}}
\makeatletter
\newcommand*{\relrelbarsep}{.386ex}
\newcommand*{\relrelbar}{%
  \mathrel{%
    \mathpalette\@relrelbar\relrelbarsep
  }%
}
\newcommand*{\@relrelbar}[2]{%
  \raise#2\hbox to 0pt{$\m@th#1\relbar$\hss}%
  \lower#2\hbox{$\m@th#1\relbar$}%
}
\providecommand*{\rightrightarrowsfill@}{%
  \arrowfill@\relrelbar\relrelbar\rightrightarrows
}
\providecommand*{\leftleftarrowsfill@}{%
  \arrowfill@\leftleftarrows\relrelbar\relrelbar
}
\providecommand*{\xrightrightarrows}[2][]{%
  \ext@arrow 0359\rightrightarrowsfill@{#1}{#2}%
}
\providecommand*{\xleftleftarrows}[2][]{%
  \ext@arrow 3095\leftleftarrowsfill@{#1}{#2}%
}
\makeatother
\author{Ilya Yaroshevskiy}
\date{\today}
\title{Лекция 4}
\hypersetup{
 pdfauthor={Ilya Yaroshevskiy},
 pdftitle={Лекция 4},
 pdfkeywords={},
 pdfsubject={},
 pdfcreator={Emacs 28.0.50 (Org mode )}, 
 pdflang={English}}
\begin{document}

\maketitle
\tableofcontents

\newcommand{\X}{\mathcal{X}}
\newcommand{\A}{\mathfrak{A}}
\newcommand{\B}{\mathfrak{B}}

\begin{theorem}[об абсолютной непрерывности ингтерала]
\-
\begin{itemize}
\item \((X, \A, \mu)\)
\item \(f: X \to \overline{\R}\) --- суммируема
\end{itemize}
\uline{Тогда} \(\forall \varepsilon > 0\ \exists \delta > 0\ \forall E\) --- измеримым, \(\mu E < \delta\quad |\int_E f| < \varepsilon\)
\end{theorem}
\begin{corollary}
\-
\begin{itemize}
\item \(f\) --- суммируемая
\item \(\mu E \to 0\)
\end{itemize}
\uline{Тогда} \(\int_{E_n} f \to 0\)
\end{corollary}
\begin{proof}
Возьмем множества \(X_m := X(|f| \ge n)\), очевидно что \(X_n \supset X_{n + 1} \supset \dots\), а также \(\mu(\bigcap X_n) = 0\) \\
Утвержение: \(\forall \varepsilon\ \exists n_\varepsilon\quad \int_{X_{n_\varepsilon}}|f| < \frac{\varepsilon}{2}\) ---
это свойство непрерывности сверху меры \(A \mapsto \int_A |F| d\mu\) \\
Пусть \(\delta:=\frac{\varepsilon}{2n_\varepsilon}\), тода при \(\mu E < \delta\)
\[ \left|\int_E f\right| \le\int_{E_nX_{n_\varepsilon}}|f| + \int_{E_nX_{n_\varepsilon}}^C \le \int_{X_{n_\varepsilon}} |f| + \int_{E_nX_{n_\varepsilon}} n_\varepsilon < \frac{\varepsilon}{2} + \mu E\cdot n_\varepsilon \le \varepsilon \]
\end{proof}
Правда ли что:
\[ f_n \xRightarrow[\mu]{} f\quad \forall \varepsilon > 0\ \mu X(|f_n - f| > \varepsilon) \to 0\]
\[ \int_X|f_n - f| d\mu \to 0 \]
эквивалентны.
\begin{description}
\item[{\((\Rightarrow)\)}] \textbf{Нет}. \((X, \A, \mu) = (\R, \mathfrak{M}, \lambda)\) \\
\(f_n = \frac{1}{nx}\ f_n \xRightarrow[\lambda]{} 0\) \\
\(\int|f_n - f| = + \infty\) --- при всех \(n\)
\item[{\((\Leftarrow)\)}] \textbf{Да}. \[\mu \underbrace{X(|f_n - f| > \varepsilon)}_{X_n} = \int_{X_n} 1 \le \int_{X_n} \frac{|f_n - 1|}{\varepsilon} = \frac{1}{\varepsilon}\int_{X_n}|f_n - f| \le \frac{1}{\varepsilon}\int_X|f_n - f| \xrightarrow[n\to +\infty]{} 0\]
\end{description}
\begin{theorem}[Лебега]
\-
\begin{itemize}
\item \((X, \A, \mu)\)
\item \(f_n, f\) --- измеримые, почти везде конечные
\item \(f_n \xRightarrow[\mu]{} f\)
\item \(\exists g\) --- \textbf{суммируемая мажоранта}:
\begin{enumerate}
\item \label{lebega_1} \(\forall n\ |f_n| \le g\) почти везде
\item \(g\) --- усммируемая везде
\end{enumerate}
\end{itemize}
\uline{Тогда} \(f_n, f\) --- суммируемые и \(\int_X |f_n - f|d\mu\xrightarrow[n \to + \infty]{}0\), и 'тем более' \(\int_X f_n d\mu \to \int_X f d\mu\)
\end{theorem}
\begin{proof}
\(f_n\) --- суммируема в силу \ref{lebega_1}, \(f\) --- суммируема по следствию т. Рисса: \(|f| \le g\) почти везде \\
'тем более' = \(\left|\int_X f_n - \int_X f \right| \le \int_X |f_n - f| \to 0\)
\begin{enumerate}
\item \label{lebega_2} \(\mu X < + \infty\) фиксируем \(\varepsilon\ X_n = X(|f_n - f| > \varepsilon)\) \\
\(f_n \to f\), т.е. \(\mu X_n \to 0\)
\[ |f_n - f| \le |f_n| + |f| \le 2g \]
\[ \int_X|f_n - f| = \int_{X_n}+\int_{X_n^C} \le \int_{X_n} 2g + \int_{X_n^C} \varepsilon d\mu < \varepsilon + \varepsilon \mu X\]
По следствию т. об абсолютной непрерывности: \(\int_{X_n} 2g \xrightarrow[n \to + \infty]{} 0\)
\item \(\mu X = + \infty\) \\
Проверим утверждение: \(\forall \varepsilon > 0\ \exists A \subset X\) --- измеримое, \(\mu A\) --- конечная: \(\int_{X\setminus A} g < \varepsilon\)
\[ \int_X g = \sup \{\int g_n,\ 0\le g_n\le g,\ g_n\text{ --- ступенчатая}\} \]
\[ A := \{x:\ g_n(x) > 0\} \]
--- при достаточно больших \(n\)
\[\color{blue} 0 \le \int_X g - \int_X g_n = \int_A g- g_n + \int_{X\setminus A}g < \varepsilon \]
Фиксируем \(\varepsilon > 0\)
\[ \int_X |f_n - f| d\mu = \int_A + \int_{X\setminus A} \le \int_A |f_n -f| + \int_{X\setminus A}2g \]
По \ref{lebega_2} \(\int_A|f_n - f| \xrightarrow[n \to + \infty]{} 0\ \int_{X\setminus A}2g < 2\varepsilon\) \\
т.е. при больших \(n\) \(\int_x |f_n -f|d\mu < 2\varepsilon\)
\end{enumerate}
\end{proof}

\begin{theorem}[Лебега]
\-
\begin{itemize}
\item \((X, \A, \mu)\)
\item \(f_n, f\) --- измеримые, почти везде конечные
\item \(f_n \to f\) почти везде
\item \(\exists g\) --- \textbf{суммируемая мажоранта}:
\begin{enumerate}
\item \label{lebega_1} \(\forall n\ |f_n| \le g\) почти везде
\item \(g\) --- усммируемая везде
\end{enumerate}
\end{itemize}
\uline{Тогда} \(f_n, f\) --- суммируемые и \(\int_X |f_n - f|d\mu\xrightarrow[n \to + \infty]{}0\), и 'тем более' \(\int_X f_n d\mu \to \int_X f d\mu\)
\end{theorem}
\begin{proof}
\[ h_n := \sup(|f_n - f|,\ |f_{n + 1} - f|,\ \dots) \]
\begin{itemize}
\item \(0 \le h_n \le 2g\)
\item \(h_n\) --- монотонна убывает
\item \(\lim h_n = \overline{\lim}|f_n - f| = 0\) почти везде
\end{itemize}
\(2h - h_n \ge 0\) --- эта последовательность возрастает, \(2g - h_n \to 2g\) почти везде
\[ \int_X 2g - h_n \to \int_X 2g \Rightarrow \int_X h_n \to 0 \]
\[ \int_X|f_n -f| \le \int_X h_n \to 0 \]
\end{proof}
\begin{examp}
\[ \int_0^{ + \infty} t^{x - 1}e^{-t} dt \]
\[ \lim_{x \to x_0} \int_0^{ + \infty} t^{x - 1} e^{-t} dt \overset{?}{=} \int_0^{ + \infty}t^{x_0 - 1}e^{-t} dt\]
\textbf{Да}. \(t^{x - 1} e^{-t} \xrightarrow[x \to x_0]{} t^{x_0 - 1}e^{-t}\) при всех \(t>0\) \\
Суммируемая мажоранта: \(|t^{x - 1}e^{-t}| \le \underbrace{t^{\alpha - 1}e^{-t}}_\text{сумм.}\), \(0 < \alpha < x_0\)
\end{examp}

\begin{theorem}[Фату]
\begin{itemize}
\item \((X, \A, \mu)\)
\item \(f_n \ge 0\) --- измеримая
\item \(f_n \to f\) почти везде
\item \(\exist c > 0\ \forall n\ \int_X f_n \le c\)
\end{itemize}
\uline{Тогда} \(\int_X f \le c\) \\
\end{theorem}
\begin{remark}
Здесь не требуется чтобы \(\int_X f_n \to \int_X f\), это может быть не выполнено
\end{remark}
\begin{proof}
\[ g_n := \inf(f_n,\ f_{n + 1},\ \dots) \]
\[ 0 \le g_n \le g_{n + 1}\ \lim g_n = \underline{\lim} f_n = f\text{ почти везде} \]
\[ \int_X g_n \le \int_X f_n \le c \]
\[ \int_X g_n \to \int_X f \Rightarrow \int_X f \le c \]
\end{proof}
\begin{corollary}
\-
\begin{itemize}
\item \(f_n, f \ge 0\) --- измеримые, почти везде конечные
\item \(f_n \Rightarrow f\)
\item \(\exists c >0\ \forall n \int_X f_n \le c\)
\end{itemize}
\uline{Тогда} \(\int_X f \le c\)
\end{corollary}
\begin{proof}
\[ f_n \Rightarrow f \Rightarrow \exists n_k\ f_{n_k} \to f\text{ почти везде} \]
\end{proof}

\begin{corollary}
\-
\begin{itemize}
\item \(f_n \ge 0\) --- измеримые
\end{itemize}
\uline{Тогда} \[ \int_X \underline{\lim}f_n \le \underline{\lim}\int_X f_n \]
\end{corollary}
\begin{proof}
Как в теореме: \[ \int_X g_n \le \int_X f_n \]
Выберем \(n_k\): \[ \int_X f_{n_k} \xrightarrow[n \to + \infty]{} \underline{\lim}\int_X f_n \]
\color{red}Zzz..\color{black}
\end{proof}
\section{Плотность одной меры по отношению к другой}
\label{sec:orgf1887be}
\subsection{Замена перменных в интеграле}
\label{sec:org060c9f2}
\begin{itemize}
\item \((X, \A, mu)\)
\item \((Y, \B, \cdot)\)
\item \(\Phi: X \to Y\)
\end{itemize}


\begin{itemize}
\item Пусть \(\Phi\) --- измеримо в следующем смысле:
\[ \Phi^{-1}(\B) \subset \A \]
\end{itemize}

\noindentДля \(E \in \B\) положим \(\nu(E) = \mu \Phi^{-1}(E)\) \\
Тогда \(\nu\) --- мера:
\[ \nu(\bigsqcuo E_n) = \mu(\Phi^{-1}(\bigsqcup E)n) = \mu(\bigsqcup\Phi^{-1}(E_n)) = \sum \mu \Phi^{-1} (E_n) = \sum \nu E_n\]
Мера \(\nu\) называется образом \(\mu\) при отображении \(\Phi\) и
\[ \nu E = \int_{\Phi^{-1}(E)} 1 d\mu \]
\begin{remark}
\-
\begin{itemize}
\item \(f: Y \to \overline{\R}\) --- измерима относительно \(\B\)
\end{itemize}
Тогда \(f\circ \Phi\) --- измерима относитльно \(\A\ (f\circ \Phi: X\to\overline{\R})\)
\[ X(f(\Phi(x)) < a) = \Phi^{-1}(\underbrace{Y(f < a)}_{\in \B}) \in \A \]
\end{remark}

\begin{definition}
\-
\begin{itemize}
\item \(\omega: X\to\overline{\R}\) --- измерима(на \(X\) относительно \(\A\))
\item \(\omega \ge 0\)
\end{itemize}
\[ \forall B \in \B\ \nu(B) = \int_{\Phi^{-1}(B)}\omega(x)d\mu(x) \]
--- \textbf{взвешенный образ меры} \(\mu\) при отображении \(\Phi\), \(\omega\) --- \textbf{вес}
\end{definition}
\begin{theorem}
\-
\begin{itemize}
\item \((X, \A, \mu)\)
\item \((Y, \B, \nu)\)
\item \(\Phi: X \to Y\)
\item \(\nu\) --- взвешенный образ меры \(\mu\) при отображении \(\Phi\) с весом \(\omega\)
\item \(\omega \ge 0\) --- измерима на \(X\)
\end{itemize}
\uline{Тогда} \(\forall f\) --- измеримые на \(Y\) относительно \(\B\), \(f \ge 0\) \(f\circ \Phi\) --- измеримая на \(X\) относительно \(\A\) и
\[ \int_Y f(y) d\nu(y) = \int_X f(\Phi(x))\cdot\omega(x)\d\mu(x) \label{weight_1}\addtag \]
То же верно для суммируемых \(f\)
\end{theorem}
\begin{proof}
\(f\circ \Phi\) --- измеримая \\
\begin{enumerate}
\item Пусть \(f = \X_B, B \in \B\)
\[ f\circ \Phi(x) = f(\Phi(x)) = \left[\begin{array}{ll} 1 & ,\Phi(x) \in B \\ 0 & ,\Phi(x) \not\in B\end{array}\right. =\X_{\Phi^{-1}(B)} \]
Тогда \ref{weight_1}:
\[ \nu B \overset{?}{=} \int_X \X_{\Phi^{-1}(B)}\cdot\omegad\mu = \int_{\Phi^{-1}(B)}\omegad\mu  \]
--- это определение \(\nu\)
\item \(f\) --- ступенчатая. \ref{weight_1} следует из линейности интеграла
\item \(f \ge 0\) --- измеримая: таким образом ??? измеримая функция ступенчатая + т. Леви
\[ 0 \le h_1 \le h_2 \le \dots,\ h_i\text{ --- ступенчатая}\ h_i \le f\ h_i \to f \]
\[ \int_Y h_i d\nu = \int_X h_i \circ \Phi\cdot\omega d\mu \xrightarrow[i \to \infty]{} \]
\item \(f\) --- измеримая \(\Rightarrow\) для \(|f|\) выполнено \ref{weight_1} \(\Rightarrow\) \(|f|\) и \(|f\circ \Phi|\cdot \omega\) \\
\color{red}Что-то про \(f_+\) \color{black}
\end{enumerate}
\end{proof}
\begin{corollary}
В условиях теоремы:
\begin{itemize}
\item \(B \in \B\)
\item \(f\) --- суммируемая на \(B\)
\end{itemize}
\uline{Тогда} \[ \int_B f d\nu = \int_{\Phi^{-1}(B)}f(\Phi(x))\omega(x)d\mu\]
\end{corollary}
\begin{proof}
В теорему подствить \(f \leftrightarrow f\cdot\X_{B}\)
\end{proof}
\begin{remark}
Частный случай.
\begin{itemize}
\item \(X = Y\)
\item \(\A = \B\)
\item \(\Phi = \text{Id}\)
\item \(\nu(B) = \int_B\omega(x)d\mu\), \(\omega \ge 0\) --- измеримая
\end{itemize}
В этой ситуации \(\omega\) --- плотность(меры \(\nu\) относительно меры \(\mu\)) и тогда по теореме:
\[ \int_X f d\nu = \int_X f(x)\omega(x)d\mu \]
\end{remark}
\end{document}
