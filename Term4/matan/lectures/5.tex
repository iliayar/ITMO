% Created 2021-03-15 Mon 12:55
% Intended LaTeX compiler: pdflatex
\documentclass[english]{article}
\usepackage[T1, T2A]{fontenc}
\usepackage[lutf8]{luainputenc}
\usepackage[english, russian]{babel}
\usepackage{minted}
\usepackage{graphicx}
\usepackage{longtable}
\usepackage{hyperref}
\usepackage{xcolor}
\usepackage{natbib}
\usepackage{amssymb}
\usepackage{stmaryrd}
\usepackage{amsmath}
\usepackage{caption}
\usepackage{mathtools}
\usepackage{amsthm}
\usepackage{tikz}
\usepackage{grffile}
\usepackage{extarrows}
\usepackage{wrapfig}
\usepackage{rotating}
\usepackage{placeins}
\usepackage[normalem]{ulem}
\usepackage{amsmath}
\usepackage{textcomp}
\usepackage{capt-of}

\usepackage{geometry}
\geometry{a4paper,left=2.5cm,top=2cm,right=2.5cm,bottom=2cm,marginparsep=7pt, marginparwidth=.6in}

 \usepackage{hyperref}
 \hypersetup{
     colorlinks=true,
     linkcolor=blue,
     filecolor=orange,
     citecolor=black,      
     urlcolor=cyan,
     }

\usetikzlibrary{decorations.markings}
\usetikzlibrary{cd}
\usetikzlibrary{patterns}
\usetikzlibrary{automata, arrows}

\newcommand\addtag{\refstepcounter{equation}\tag{\theequation}}
\newcommand{\eqrefoffset}[1]{\addtocounter{equation}{-#1}(\arabic{equation}\addtocounter{equation}{#1})}


\newcommand{\R}{\mathbb{R}}
\renewcommand{\C}{\mathbb{C}}
\newcommand{\N}{\mathbb{N}}
\newcommand{\rank}{\text{rank}}
\newcommand{\const}{\text{const}}
\newcommand{\grad}{\text{grad}}

\theoremstyle{plain}
\newtheorem{axiom}{Аксиома}
\newtheorem{lemma}{Лемма}
\newtheorem{manuallemmainner}{Лемма}
\newenvironment{manuallemma}[1]{%
  \renewcommand\themanuallemmainner{#1}%
  \manuallemmainner
}{\endmanuallemmainner}

\theoremstyle{remark}
\newtheorem*{remark}{Примечание}
\newtheorem*{solution}{Решение}
\newtheorem{corollary}{Следствие}[theorem]
\newtheorem*{examp}{Пример}
\newtheorem*{observation}{Наблюдение}

\theoremstyle{definition}
\newtheorem{task}{Задача}
\newtheorem{theorem}{Теорема}[section]
\newtheorem*{definition}{Определение}
\newtheorem*{symb}{Обозначение}
\newtheorem{manualtheoreminner}{Теорема}
\newenvironment{manualtheorem}[1]{%
  \renewcommand\themanualtheoreminner{#1}%
  \manualtheoreminner
}{\endmanualtheoreminner}
\captionsetup{justification=centering,margin=2cm}
\newenvironment{colored}[1]{\color{#1}}{}

\tikzset{->-/.style={decoration={
  markings,
  mark=at position .5 with {\arrow{>}}},postaction={decorate}}}
\makeatletter
\newcommand*{\relrelbarsep}{.386ex}
\newcommand*{\relrelbar}{%
  \mathrel{%
    \mathpalette\@relrelbar\relrelbarsep
  }%
}
\newcommand*{\@relrelbar}[2]{%
  \raise#2\hbox to 0pt{$\m@th#1\relbar$\hss}%
  \lower#2\hbox{$\m@th#1\relbar$}%
}
\providecommand*{\rightrightarrowsfill@}{%
  \arrowfill@\relrelbar\relrelbar\rightrightarrows
}
\providecommand*{\leftleftarrowsfill@}{%
  \arrowfill@\leftleftarrows\relrelbar\relrelbar
}
\providecommand*{\xrightrightarrows}[2][]{%
  \ext@arrow 0359\rightrightarrowsfill@{#1}{#2}%
}
\providecommand*{\xleftleftarrows}[2][]{%
  \ext@arrow 3095\leftleftarrowsfill@{#1}{#2}%
}
\makeatother
\author{Ilya Yaroshevskiy}
\date{\today}
\title{Лекция 5}
\hypersetup{
 pdfauthor={Ilya Yaroshevskiy},
 pdftitle={Лекция 5},
 pdfkeywords={},
 pdfsubject={},
 pdfcreator={Emacs 28.0.50 (Org mode )}, 
 pdflang={English}}
\begin{document}

\maketitle
\tableofcontents

\newcommand{\X}{\mathcal{X}}
\newcommand{\A}{\mathfrak{A}}
\newcommand{\B}{\mathfrak{B}}
\newcommand{\M}{\mathfrak{M}}

\section{Плотности}
\label{sec:org01917b7}

\noindentПлотность \((X, \A, \mu)\) и \(\nu: \A \to \overline{\R}\) --- мера \\
Плотность  меры \(\nu\) онсительно \(\mu\) --- это функция \(\omega: X \to \overline{\R}\) \\
\(\forall B \in A\quad \nu B = \int_B \omega d\mu\)

\begin{theorem}[критерий плотности]
\-
\begin{itemize}
\item \((X, \A, \mu),\ \nu\) --- еще одна мера
\item \(\omega: X \to \overline{\R},\ \omega \ge 0\) --- измеримая
\end{itemize}
\uline{Тогда} \(\omega\) --- плотность \(\nu\) отнсительно \(\mu\) \(\Leftrightarrow\)
\[ \forall A \in \A\ \mu A \cdot \inf_A \omega \le \nu(A) \le \mu A \cdot \sup_A \omega \]
\end{theorem}
\begin{examp}[нет плотности]
\-
\begin{itemize}
\item \(X = \R\)
\item \(\A = \M'\)
\item \(\mu = \lambda_1\)
\item \(\nu\) --- одноточечная мера \(\nu(A) = \left[\begin{array}{ll} 1 & ,\text{если } 0 \in A \\ 0 & ,\text{иначе}\end{array}\right.\) \\
считаем \(\nu: \A \to \R\)
\end{itemize}
\end{examp}

\begin{theorem}[Необходимое условие существования плотности]
\(\mu A = 0 \Rightarrow \nu A = 0\)
\end{theorem}
\begin{theorem}[теорема Радона-Никодина]
Это так-же достаточное условие
\end{theorem}

\begin{proof}[Доказательство критерия плотности]
\begin{description}
\item[{\((\Rightarrow)\)}] очевидно
\item[{\((\Leftarrow)\)}] Не умаляя общности \(\omega > 0: e = X(\omega = 0)\) \\
\(\nu(e) = \int_e \omega d\mu = 0\) \\
Для случая когда \(A \cup e = \emptyset\) все только лучше \\
Фиксируем \(q \in (0, 1)\) \\
\(A_j = A(q^j \le \omega < q^{j - 1}), j \in \mathbb{Z}\) \\
\begin{center}
\begin{tikzpicture}
\draw[->] (-2, 0) -- (2, 0);
\node at (-1.9, 0) (A) [below] {\(0\)};
\node at (-1.4, 0) (B) [below] {\(q^2\)};
\node at (-0.9, 0) (C) [below] {\(q\)};
\node at (-0.2, 0) (D) [below] {\(1 = q^0\)};
\node at (0.6, 0) (E) [above] {\(q^{-1}\)};
\node at (1.5, 0) (F) [above] {\(q^{-2}\)};
\end{tikzpicture}
\end{center}
\[ A = \bigsqcup_{j \in \mathbb{Z}} A_j \]
\[ \mu A_j \cdot q^{j} \le \nu A_j \le \mu A_i \cdot q^{j - 1} \addtag\label{5_1_neq}\]
\[ \mu A_j \cdot q^j \le \int_{A_j} \omega d\mu \le \mu A_j \cdot q^{j - 1} \addtag\label{5_2_neq} \]
Тогда
\[ q \cdot \int_A \omega d\mu \le q \cdot \sum \int_{A_j} \le \sum q^j \mu A_j \le \sum \nu A_j \le \frac{1}{q} \sum q^j \mu A_j \le \frac{1}{q} \sum \int_{A_j} = \frac{1}{q} \int_A \omega d\mu \ \]
то есть:
\[ q \int_A \omega d\mi \le \nu A \le \frac{1}{q} \int_A \omega d\mu \]
и \(q \to 1 - 0\)
\end{description}
\end{proof}
\begin{lemma}
\-
\begin{itemize}
\item \(f, g\) --- суммируемые
\item \((X, \A, \mu)\)
\item \(\forall A \in \A\)
\item \(\int_A f = \int_A g\)
\end{itemize}
\uline{Тогда} \(f = g\) почти везде
\end{lemma}
\begin{proof}
\(g := f - g\) \\
Дано \(\forall A \int_A h = 0\) \\
Доказать \(h = 0\) почти везде \\
\begin{itemize}
\item \(A_{+} := X(h \ge 0)\)
\item \(A_{-} := X(h < 0)\)
\end{itemize}
\(X = A_+ \sqcup A_-\)
\[ \int_{A_+} |h| = \int_{A_+} h = 0 \]
\[ \int_{A_-} |h| = -\int_{A_-} h = 0 \]
тогда \[ \int_X |h| = 0 \]
\(\Rightarrow h = 0\) почти везде
\end{proof}
\begin{remark}
\(\mathcal{L}(X)\) --- линейное пространство отображений \(l_A : f \mapsto \int_A f d\mu\) --- линейный функционал \\
Таким образом множество функционалов \(\{l_A, A \in \A\}\) --- разделяет точки \\
\(\forall f, g \in \mathcal{L}(X) \exists A l_A(f) \neq l_A(g)\)
\end{remark}
\section{Мера лебега}
\label{sec:orgac2c08a}
\begin{lemma}[о мере образа малых кубических ячеек]
\-
\begin{itemize}
\item \(O \subset \R^m\) --- открытое
\item \(a \in O\)
\item \(\Phi: O \to \R^m\)
\item \(\Phi \in C^1\)
\end{itemize}
Пусть \(c > |\det\Phi'(a)| \neq 0\) \\
\uline{Тогда} \(\exists \delta > 0\ \forall\) куба \(Q \subset B(a, \delta),\ a\in Q\) \\
выполняется неравенство \(\lambda \Phi(Q) < c \cdot \lambda Q\)
\end{lemma}
\begin{remark}
Здесь можно считать что кубы замкнутые
\end{remark}
\begin{proof}
\(L := \Phi'(a)\) --- обратимо \\
\[ \Phi(x) = \Phi(a) + L\cdot(x - a) + o(x - a)\quad x \to a \]
\[ \underbrace{a + L^{-1}(\Phi(x) - \Phi(a))}_{\Psi(x)} = x + o(x - a) \]
\[ \forall \varepsilon > 0 \exists \text{ шар} B_\varepsilon(a)\ \forall x \in B_\varepsilon(A)\ |\Psi(x) - x| < \frac{\varepsilon}{\sqrt{m}} |x - a| \]
Пусть \(Q \subset B_\varepsilon(a) a \in Q\) --- куб со стороной \(h\). При \(x \in Q:\ |x - a| \le \sqrt{m}h\)
\[ |\Psi(x) - x| \le \frac{\varepsilon}{\sqrt{m}}|x - a| \le \varepsilon h \]
Тогда \(\Psi(Q) \subset\) Куб со стороной \((1 + 2\varepsilon)h\): при \(x, y \in Q\)
\[ |\Psi(x)_i - \Psi(y)_i| \le |\Psi(x)_i - x_i| + |x_i - y_i| + |\Psi(y)_i - y_i| \le |\Psi(x) - x| + h + |\Psi(y) - y| \le (1 + 2\varepsilon)h\]
\[ \lambda(\Psi(Q)) \le (1 + 2\varepsilon)^m \cdot \lambda Q  \]
\(\Psi\) и \(\Phi\) отличаются только сдвигом и линейным отображением
\[ \lambda \Phi(Q) = |\det L| \cdot \lambda \Psi(Q) \le \underbrace{|\det L|\cdot(1 + 2\varepsilon)^m}_{\text{выбираем }\varepsilon\text{ чтобы } ... < c } \lambda Q \]
потом берем \(\delta = \text{радиус } B_\varepsilon(a)\)
\end{proof}
\begin{lemma}
\-
\begin{itemize}
\item \(O \subset \R^m\) --- открытое
\item \(f: O \to \R\) --- непрерывное
\item \(Q \subset \overline{Q} \subset O\) --- кубическая ячейка
\item \(A \subset Q\)
\end{itemize}
\uline{Тогда} \[ \inf_{\substack{G: A \subset G \\ G\text{ --- открытое } \subset O}}\left(\lambda(G)\sup_G f\right) = \lambda A\cdot \sup_A f\]
\label{orge0788ba}
\end{lemma}
\begin{theorem}
\-
\begin{itemize}
\item \(\Phi: O \subset \R^m \to \R^m\) --- диффеоморфизм
\end{itemize}
\uline{Тогда} \(\forall A \in \M^m, A \in O\)
\[ \lambda \Phi(A) = \int_A \left|\det \Phi'(x)\right| d\lambda(x) \]
\label{org5e12d67}
\end{theorem}
\begin{proof}
Обозначим якобиан \(J_\Phi(x) = |\det \Phi'(x)|\) \\
\(\nu A := \lambda \Phi(A)\) --- мера. Т.е. надо доказать: \(J_\Phi\) --- плотность \(\nu\) относительно \(\lambda\). Тогда достаточно проверить условие критерия плотности
\[ \inf_A J_\Phi \cdot \lambda A \le \nu A \le \sup_A J_\Phi \cdot \lambda A \addtag\label{5_3_neq}\]
Достаточно проверить только правое неравенство. левое --- это "правое для \(\Phi(A)\) и отображения \(\Phi^{-1}\)"
\[ \inf \frac{1}{|\det(\Phi')|}\cdot \lambda \Phi(A) \le \lambda A  \]
\begin{enumerate}
\item Проверяем второе неравенство \ref{5_3_neq} для случая когда \(A\) --- кубическая ячейка. \(A \subset \overline{A} \subset O\). От противного:
\[ \lambda Q \cdot \sup_Q J_\Phi < \nu(Q) \]
Возьмем \(C > \sup_Q J_\Phi:\ C \cdot \lambda Q < \nu(Q)\). Запускаем процесс половинного деления: \\
Режем \(Q\) на \(2^m\) более мелких кубических ячеек. Выберем "мелкую" ячейку \(Q_1 \subset Q:\ C\cdot \lambda Q_1 < \nu Q_1\). Опять делим на \(2^m\) частей, берем \(Q_2:\ \C\cdot\lambda Q_2 < \nu Q_2\) и так далее
\[ Q_1 \supset Q_2 \supset \dots\quad \forall n C\cdot \lambda Q_n < \nu Q_n \addtag\label{5_4_kubi}\]
\[ a \in \bigcap \overline{Q_i}\quad c > \sup_Q J_\Phi = \sup_{\overline{Q}} J_\Phi,\text{ в частности } c > |\det\Phi'(a)| \]
Получаем противоречие с леммой: с скол угодно малой окрестности \(a\) имеются кубы \(\overline{Q_n}\), где выполняется \ref{5_4_kubi}. \textbf{Противоречие}
\item Проверим второе неравенство \ref{5_3_neq} для открытых множеств \(A \subset O\) \\
Это очевидно \(A = \bigsqcup Q_j\), \(Q_j\) --- кубическая ячейка, \(Q_j \subset \overline{Q_j} \subset A\)
\[ \nu A = \sum \lambda Q_j \le \sum \mu Q_j \sup_{Q_j} J_\Phi \le \sup_A J_\Phi \sum \mu Q_j = \sup_A J_\Phi\cdot \lambda A \addtag\label{5_5_neq}\]
\item По \hyperref[orge0788ba]{лемме} второе неравенство \ref{5_3_neq} выполнено для всех измеримых \(A\)
\[ O = \bigsqcup Q_j\text{ --- куба } Q_j \subset \overline{Q_j} \subset O \]
\[ A = \bigsqcup \underbrace{A \cup Q_j}_{A_j}\quad A\subset G\text{ --- открытое} \]
\[ J A_j \le \nu G \le \sup_G J_\Phi \cdot \lambda G \Rightarrow \nu A_j \le \int_G(\sup J_\Phi \cdot \lambda G) = \sup_{A_j} f \cdot \lambda A_j\]
Аналогично \ref{5_5_neq} получаем \(\nu A \le \sup_A f\cdot \lambda A\)
\end{enumerate}
\end{proof}
\begin{theorem}
\-
\begin{itemize}
\item \(\Phi: O \subset \R^m \to \R^m\) --- дифференцируемое
\end{itemize}
\uline{Тогда} \(\forall f\) --- измеримых, \(\ge 0\), заданная на \(O' = \Phi(O)\)
\[ \int_{O'}f(y) d\lambda = \int_O f(\Phi(x)) \cdot J_\Phi \cdot d\lambda \]
, где \(J_\Phi(x) = |\det \Phi'(x)|\). То же верно для суммируемых функций \(f\)
\end{theorem}
\begin{proof}
Применяем теорему о взвешенном образе меры. \\
Дано:
\begin{itemize}
\item \((X, \A, \mu)\)
\item \((T, \B, \nu)\)
\item \(\Phi: X \to Y\) --- с сохранением измеримости
\item \(\Phi^{-1}(\B) \subset \A\)
\item \(\omega: Y \to \R,\ \ge 0\), измеримый
\item \(\nu\) --- взвешенный образ \(\mu\) с весом \(\omega\): \[\mu(B) = \int_{\Phi^{-1}(B)} \omega d\mu\]
\end{itemize}
Тогда \[ \int_B f d\nu = \int_{\Phi^{-1}(B)}f(\Phi(x)) \omega(x) d\mu \]
В нашем случае
\begin{itemize}
\item \(X = Y - \R^m\)
\item \(\A = \B = \M^m\)
\item \(\Phi\) --- диффеоморфизм
\item \(\mu = \lambda\)
\item \(\nu(A) = \lambda \Phi(A)\)
\end{itemize}
Под действием гладкого отображния \(\Phi\), \(\sigma\)-аглебра \(\M^m\) сохраняется \\
По \hyperref[org5e12d67]{теореме} \[\nu(B) = \int_{\Phi^{-1}(A)} J_\Phi d\lambda\]
т.е. \(\lambda\) --- взвешенный образ исходной меры Лебега по отношению к \(\Phi\)
\end{proof}
\begin{examp}
Полярные координаты в \(R^2\). \\
\(\left\{\begin{array}{l} x = r\cos\varphi \\ y = r\sin\varphi \end{array}\right., \Phi: \{(r, \varphi), r> 0, \varphi \in (0, 2\pi)\} \to \R^2\) \\
диффеоморфизм \[ \Phi = \begin{pmatrix} \cos \varphi & -r \sin\varphi \\ \sin \varphi & r \cos\varphi\end{pmatrix} \]
\[ \det \Phi' = r\quad J_\Phi = r \]
\[ \iint_\Omega f(x, y) = d\lambda_r = \iint_{\Phi^{-1}(\Omega)} f(r \cos\varphi, r\sin\varphi) r \underset{d \lambda_r(r, \varphi)}{d\lambda_r} \]
\end{examp}
\begin{examp}
Сферические координаты в \(R^3\)
\[ \begin{cases} x = r \cos\varphi\cos\psi \\ y = r \sin\varphi \cos\psi \\ z = r\sin\psi \end{cases} 
 \left[\begin{matrix} r > 0 \\ \varphi \in (0, 2\pi \\ \psi \in \left(-\frac{\pi}{2}, \frac{\pi}{2}\right) \end{matrix}\right. \]
\[ \Phi' = \begin{pmatrix} \cos \varphi \cos \psi & -r \sin\varphi \cos\psi & - r \cos\varphi \sin \psi \\ \sin \varphi \cos \psi & r\cos\varphi\cos\psi? & - r\sin\varphi \sin \psi \\ \sin \psi & 0 & r\cos\psi \end{pmatrix} \]
\[ \det \Phi' = r^2(\sin^2\psi \cos \psi + \cos^3\psi) = r^2\cos\psi = J_\Phi\]
--- для географических координат: \(r\) --- растояние от центра Земли, \(\psi\) --- угол к плоскости экватора
\end{examp}
\end{document}
