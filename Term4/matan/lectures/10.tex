% Created 2021-04-19 Mon 13:14
% Intended LaTeX compiler: pdflatex

\documentclass[english]{article}
\usepackage[T1, T2A]{fontenc}
\usepackage[lutf8]{luainputenc}
\usepackage[english, russian]{babel}
\usepackage{minted}
\usepackage{graphicx}
\usepackage{longtable}
\usepackage{hyperref}
\usepackage{xcolor}
\usepackage{natbib}
\usepackage{amssymb}
\usepackage{stmaryrd}
\usepackage{amsmath}
\usepackage{caption}
\usepackage{mathtools}
\usepackage{amsthm}
\usepackage{tikz}
\usepackage{grffile}
\usepackage{extarrows}
\usepackage{wrapfig}
\usepackage{algorithm}
\usepackage{algorithmic}
\usepackage{lipsum}
\usepackage{rotating}
\usepackage{placeins}
\usepackage[normalem]{ulem}
\usepackage{amsmath}
\usepackage{textcomp}
\usepackage{capt-of}

\usepackage{geometry}
\geometry{a4paper,left=2.5cm,top=2cm,right=2.5cm,bottom=2cm,marginparsep=7pt, marginparwidth=.6in}
 \usepackage{hyperref}
 \hypersetup{
     colorlinks=true,
     linkcolor=blue,
     filecolor=orange,
     citecolor=black,      
     urlcolor=cyan,
     }

\usetikzlibrary{decorations.markings}
\usetikzlibrary{cd}
\usetikzlibrary{patterns}
\usetikzlibrary{automata, arrows}

\newcommand\addtag{\refstepcounter{equation}\tag{\theequation}}
\newcommand{\eqrefoffset}[1]{\addtocounter{equation}{-#1}(\arabic{equation}\addtocounter{equation}{#1})}
\newcommand{\llb}{\llbracket}
\newcommand{\rrb}{\rrbracket}


\newcommand{\R}{\mathbb{R}}
\renewcommand{\C}{\mathbb{C}}
\newcommand{\N}{\mathbb{N}}
\newcommand{\A}{\mathfrak{A}}
\newcommand{\B}{\mathfrak{B}}
\newcommand{\rank}{\mathop{\rm rank}\nolimits}
\newcommand{\const}{\var{const}}
\newcommand{\grad}{\mathop{\rm grad}\nolimits}

\newcommand{\todo}{{\color{red}\fbox{\text{Доделать}}}}
\newcommand{\fixme}{{\color{red}\fbox{\text{Исправить}}}}

\newcounter{propertycnt}
\setcounter{propertycnt}{1}
\newcommand{\beginproperty}{\setcounter{propertycnt}{1}}

\theoremstyle{plain}
\newtheorem{propertyinner}{Свойство}
\newenvironment{property}{
  \renewcommand\thepropertyinner{\arabic{propertycnt}}
  \propertyinner
}{\endpropertyinner\stepcounter{propertycnt}}
\newtheorem{axiom}{Аксиома}
\newtheorem{lemma}{Лемма}
\newtheorem{manuallemmainner}{Лемма}
\newenvironment{manuallemma}[1]{%
  \renewcommand\themanuallemmainner{#1}%
  \manuallemmainner
}{\endmanuallemmainner}

\theoremstyle{remark}
\newtheorem*{remark}{Примечание}
\newtheorem*{solution}{Решение}
\newtheorem{corollary}{Следствие}[theorem]
\newtheorem*{examp}{Пример}
\newtheorem*{observation}{Наблюдение}

\theoremstyle{definition}
\newtheorem{task}{Задача}
\newtheorem{theorem}{Теорема}[section]
\newtheorem*{definition}{Определение}
\newtheorem*{symb}{Обозначение}
\newtheorem{manualtheoreminner}{Теорема}
\newenvironment{manualtheorem}[1]{%
  \renewcommand\themanualtheoreminner{#1}%
  \manualtheoreminner
}{\endmanualtheoreminner}
\captionsetup{justification=centering,margin=2cm}
\newenvironment{colored}[1]{\color{#1}}{}

\tikzset{->-/.style={decoration={
  markings,
  mark=at position .5 with {\arrow{>}}},postaction={decorate}}}
\makeatletter
\newcommand*{\relrelbarsep}{.386ex}
\newcommand*{\relrelbar}{%
  \mathrel{%
    \mathpalette\@relrelbar\relrelbarsep
  }%
}
\newcommand*{\@relrelbar}[2]{%
  \raise#2\hbox to 0pt{$\m@th#1\relbar$\hss}%
  \lower#2\hbox{$\m@th#1\relbar$}%
}
\providecommand*{\rightrightarrowsfill@}{%
  \arrowfill@\relrelbar\relrelbar\rightrightarrows
}
\providecommand*{\leftleftarrowsfill@}{%
  \arrowfill@\leftleftarrows\relrelbar\relrelbar
}
\providecommand*{\xrightrightarrows}[2][]{%
  \ext@arrow 0359\rightrightarrowsfill@{#1}{#2}%
}
\providecommand*{\xleftleftarrows}[2][]{%
  \ext@arrow 3095\leftleftarrowsfill@{#1}{#2}%
}
\makeatother

\newenvironment{rualgo}[1][]
  {\begin{algorithm}[#1]
     \selectlanguage{russian}%
     \floatname{algorithm}{Алгоритм}%
     \renewcommand{\algorithmicif}{{\color{red}\textbf{если}}}%
     \renewcommand{\algorithmicthen}{{\color{red}\textbf{тогда}}}%
     \renewcommand{\algorithmicelse}{{\color{red}\textbf{иначе}}}%
     \renewcommand{\algorithmicend}{{\color{red}\textbf{конец}}}%
     \renewcommand{\algorithmicfor}{{\color{red}\textbf{для}}}%
     \renewcommand{\algorithmicto}{{\color{red}\textbf{до}}}%
     \renewcommand{\algorithmicdo}{{\color{red}\textbf{делать}}}%
     \renewcommand{\algorithmicwhile}{{\color{red}\textbf{пока}}}%
     \renewcommand{\algorithmicrepeat}{{\color{red}\textbf{повторять}}}%
     \renewcommand{\algorithmicuntil}{{\color{red}\textbf{до тех пор пока}}}%
     \renewcommand{\algorithmicloop}{{\color{red}\textbf{повторять}}}%
     \renewcommand{\algorithmicnot}{{\color{blue}\textbf{не}}}%
     \renewcommand{\algorithmicand}{{\color{blue}\textbf{и}}}%
     \renewcommand{\algorithmicor}{{\color{blue}\textbf{или}}}%
     \renewcommand{\algorithmicrequire}{{\color{blue}\textbf{Ввод}}}%
     \renewcommand{\algorithmicensure}{{\color{blue}\textbf{Вывод}}}%
     \renewcommand{\algorithmicreturn}{{\color{red}\textbf{Вернуть}}}%
     \renewcommand{\algorithmicrtrue}{{\color{blue}\textbf{истинна}}}%
     \renewcommand{\algorithmicrfalse}{{\color{blue}\textbf{ложь}}}%
     % Set other language requirements
  }
  {\end{algorithm}}
\author{Ilya Yaroshevskiy}
\date{\today}
\title{Лекция 10}
\hypersetup{
 pdfauthor={Ilya Yaroshevskiy},
 pdftitle={Лекция 10},
 pdfkeywords={},
 pdfsubject={},
 pdfcreator={Emacs 28.0.50 (Org mode 9.4.4)}, 
 pdflang={English}}
\begin{document}

\maketitle
\tableofcontents

\newcommand{\rot}{\mathop{\rm rot}\nolimits}
\newcommand{\M}{\mathfrak{M}}

\begin{theorem}[Формула Остроградского]
\-
\begin{itemize}
\item \(V = \{(x, y, z) \big| (x, y) \in G \subset \R^2,\ f(x, y) \le z \le F(x, y)\}\)
\item \(G\) --- компактно
\item \(\partial G\) --- кусочно гладкая
\item \(f, F \in C^1\)
\item \(R\) в окрестности \(V\) \(\to\) \(\R\), \(\in C^1\)
\end{itemize}
Фиксируем внешнюю сторону поверхности \\
\uline{Тогда}
\[ \iiint_V \frac{\partial R}{\partial z}\,dx\,dy\,dz = \iint_{\partial V}R\,dx\,dy = \color{blue} \iint_{\partial V} 0\,dy\,dz + 0\,dz\,dx + R\,dx\,dy \color{black} \]
\end{theorem}
\begin{proof}
\[ \iiint_V \frac{\partial R}{\partial z} = \iint_G \,dx\,dy \int_{f(x, y)}^{F(x, y)} \frac{\partial R}{\partial z}\,dz = \iint_G R(x, y, F(x, y))\,dx\,dy - \iint_G R(x, y, f(x, y)) \,dx\,dy =  \]
\[ = \iint_{\Omega_F} R(x, y, z)\,dx\,dy + \iint_{\Omega_f} R\,dx\,dy + \underbrace{\iint_{\Omega} R \,dx\,dy}_0 \]
\end{proof}
\begin{corollary}[обобщение формула Остроградского]
\[ \iiint_V \frac{\partial P}{\partial x} + \frac{\partial Q}{\partial y} + \frac{\partial R}{\partial z} \, dx\,dy\,dz = \iint_{\partial V_\text{внеш.}} P\,dy\,dz + Q\,dz\,dx + R\,dx\,dy\]
\end{corollary}
\begin{definition}
\(V\) --- гладкое векторное поле. \textbf{Дивергенция}:
\[ \mathop{\rm div}V = \frac{\partial P}{\partial x} + \frac{\partial Q}{\partial y} + \frac{\partial R}{\partial z} \]
\end{definition}
\begin{remark}
\[ \mathop{\rm div} = \lim_{\varepsilon \to 0} \frac{1}{\frac{4}{3}\pi\varepsilon^3} \iiint_{B(a, \varepsilon)} \mathop{\rm div} V \,dx\,dy\,dz  = \lim_{\varepsilon \to 0} \frac{1}{\frac{4}{3}\pi\varepsilon^3} \iint_{S(a, \varepsilon)} \langle V, \overline{n}_0 \rangle ds\]
--- не зависит от координат
\end{remark}
\begin{corollary}
\-
\begin{itemize}
\item \(l \in \R^3\)
\item \(f \in C^1(\text{окр}(V))\)
\end{itemize}
\[ \iiint_V \frac{\partial f}{\partial l} \,dx\,dy\,dz = \iint_{\partial V} f\cdot \langle f, n_0 \rangle ds \]
\end{corollary}
\section{Формула Стокса}
\label{sec:org1db52af}
\[ \int_{\partial \Omega}P\,dx + Q\,dy + R\,dz = \iint_\Omega \langle \mathop{\em rot}(V), n_0 \rangle ds\]
\[ \mathop{\rm rot}V = \left(\frac{\partial R}{\partial y} - \frac{\partial Q}{\partial z}, \frac{\partial P}{\partial z} - \frac{\partial R}{\partial x}, \frac{\partial Q}{\partial x} - \frac{\partial P}{\partial y}\right) \]
--- ротор векторного поля (вихрь векторного поля)
\begin{examp}
\[ V(x, y, z) = (-y, x, 0) \]
\[ \rot V = (0, 0, 2) \]
\end{examp}

\begin{remark}
\(V = (P, Q, R)\) --- потенциально, \(\exists f\)
\[ V = \grad f = \left(\frac{\partial f}{\partial x}, \frac{\partial f}{\partial y}, \frac{\partial f}{\partial z}\right) \]
\end{remark}
\begin{theorem}
\-
\begin{itemize}
\item \(\Omega\) --- область
\end{itemize}
\uline{Тогда} \(V\) --- потенциально \(\Leftrightarrow\) \(\rot V = 0\)
\end{theorem}
\begin{definition}
Векторное поле \(A = (A_1, A_2, A_3)\) --- \textbf{соленоидально} в области \(\Omega \subset \R^2\), если \(\exists\) гладкое векторное поле \(B\) в \(\Omega\):
\[ A = \rot B \]
\(B\) --- называется \textbf{векторным потенциалом} \(A\)
\end{definition}
\begin{theorem}[Пуанкаре']
\-
\begin{itemize}
\item \(\Omega\) --- открытый паралеллепипед
\item \(A\) --- векторное поле в \(\Omega\), \(A \in C^1\)
\end{itemize}
\uline{Тогда} \(A\) --- соленоидально \(\Leftrightarrow\) \(\mathop{\rm div} A = 0\)
\end{theorem}
\begin{proof}
\-
\begin{description}
\item[{\((\Rightarrow)\)}] \(\mathop{\rm div} \rot B = 0\)
\item[{\((\Leftarrow)\)}] Дано: \[{A_1}_x' + {A_2}_y' + {A_3}_z' = 0 \addtag\label{10_1}\]. Найдем векторный потенциал \(B = (B_1, B_2, B_3)\), \(A = \rot B\). Путь \(B_3 \equiv 0\)
\[ \left.\begin{array}{l}
  {B_3}_y' - {B_2}_z' = A_1 \\
  {B_1}_z' - {B_3}_x' = A_2 \\
  {B_2}_x' - {B_1}_y' = A_3
  \end{array}\right\} \leadsto \begin{array}{rl}
  -{B_2}_z' = A_1 & (1)\\
  {B_1}_z'  = A_2 & (2) \\
  {B_2}_x' - {B_1}_y' = A_3 & (3)
  \end{array}\]
\begin{description}
\item[{\((1)\)}] \[ B_2 \coloneqq - \int_{z_0}^z A_1 dz + \varphi(x, y) \]
\item[{\((2)\)}] \[ B_1 \coloneqq \int_{z_0}^z A_2 dz \]
\item[{\((3)\)}] \[ -\int_{z_0}^z {A_1}'_x\,dz + \varphi'_x - \int_{z_0}^z {A_z}'_y\,dz = A_3 \xRightarrow[\ref{10_1}]{} \int_{z_0}^z {A_3}_z' dz + \varphi_x' = A_3 \]
\[ A_3(x, y, z) - A_3(x, y, z_0) + \varphi_x' = A_3(x, y, z) \Leftrightarrow \varphi_x' = A_3(x, y, z_0) \]
Отсюда найдем \(\varphi = \int_{x_0}^x A_3(x,y,z_0)\,dx\)
\end{description}
\end{description}
\end{proof}
\begin{remark}
\[ \int_{\partial \Omega} A_l\,dl = \int_{\partial \Omega} \langle A, l_0 \rangle \,dl = \iint_{\Omega} (\rot A)_n \,ds \]
\[ (\rot A)_n(a) = \lim_{\varepsilon \to 0} \frac{1}{\lambda(\Omega_\varepsilon)} \iint_{\Omega_\varepsilon} (\rot A)_n\,ds = \lim_{\varepsilon \to 0}\frac{1}{\lambda \Omega} \cdot \int_{\partial \Omega_\varepsilon} A_l\,dl \]
\end{remark}
\begin{lemma}[Урнсона]
\-
\begin{itemize}
\item \(X\) --- нормальное
\item \(F_0, F_1 \subset X\) --- замкнутые, \(F_0 \cap F_1 = \emptyset\)
\end{itemize}
\uline{Тогда} \(\exists f: X \to \R\) --- непрерывная, \(0 \le f \le 1\), \(f\big|_{F_0} = 0\), \(f\big|_{F_1} = 1\)
\end{lemma}
\begin{proof}
Перефразируем нормальность: Если \(\underset{\text{замк.}}{F} \subset \underset{\text{октр.}{G}}\), то \(\exists U(F)\) --- открытое: \[F \subset U(F) \subset \overline{U(F)} \subset G\].
\[ F \leftrightarrow F_0 \quad G \leftrightarrow (F_1)^C \quad F_0 \subset \underbrace{U(F_0)}_{G_0} \subset \underbrace{\overline{U(F_0)}}_{\overline{G_0}} \subset \underbrace{F_1^C}_{G_1} \]
Строим \(G_\frac{1}{2}\): \[ \overline{G_0} \subset \underbrace{U(\overline{G_0})}_{G_\frac{1}{2}} \subset \underbrace{\overline{U(\overline{G_0})}}_{\overline{G_\frac{1}{2}}} \subset G_1 \]
Строим \(G_\frac{1}{4}\), \(G_\frac{3}{4}\): \[ \overline{G_\frac{1}{2}} \subset \underbrace{U(\overline{G_\frac{1}{2}})}_{G_\frac{3}{4}} \subset \overline{U(\overline{G_\frac{1}{2}})} \subset G_1 \]
Таким образом для любого двоично рационального числа \(\alpha \in [0, 1]\) задется множество \(G_\alpha\)
\[ f(x) \coloneqq \inf \{\alpha\text{ --- двоично рациональное} \big| x \in G_\alpha\}\]
Проверим что: \(f\) --- непрерывно \(\Leftrightarrow\) \(f^{-1}(a, b)\) --- всегда открыто. Достаточно проверить:
\begin{enumerate}
\item \(\forall b\ f^{-1}(-\infty, b)\) --- открыто
\item \(\forall a\ f^{-1}(-\infty, a)\) --- замкнуто
\end{enumerate}
Покажем это:
\begin{enumerate}
\item \[f^{-1}(-\infty, b) = \bigcup_{\substack{q < b \\ q\text{ --- дв. рац.}}} G_q\text{ --- открыто}\]
\begin{description}
\item[{\((\supset)\)}] Очевидно: При \(x \in G_q\ f(x) \le q - b\)
\item[{\((\subset)\)}] \(f(x) = b_0 < b\) Возьмем \(q: b_0 < q < b\). Тогда \(x \in G_q\)
\end{description}
\item \(f^{-1}(-\infty, a] = \bigsqcap_{q > a} G_q = \bigcap_{q > a}\overline{G_q}\) --- замкнуто
\begin{description}
\item[{\((\supset)\)}] Тривиально
\item[{\((\subset)\)}] \(q, r\) --- двоично рациональные
\[ \bigsqcap_{\substack{q > a \\ \text{всех}}} G_q \supset \bigcap_{\substack{r > a \\ \text{некоторых}}} \overline{G_r} \supset \bigcap_{\substack{r > a \\ \text{всех}}} \overline{G_r} \]
\end{description}
\end{enumerate}
\end{proof}
\begin{theorem}
\-
\begin{itemize}
\item \((\R, \M, \lambda_\M)\)
\item \(E \subset \R^m\) --- измеримое
\end{itemize}
\uline{Тогда} в \(L^P(E, \lambda_\M)\) множество непрерывных финитных функция плотно
\end{theorem}
\begin{remark}
\(f\) --- финитная в \(\R^m\) = \(\exists\) шар \(B\ f = 0\) вне \(B\). \(f\) --- непрерывная финитная на \(E\) = \(\exists g \in C_0(\R^m)\ f = g\big|_E\)
\end{remark}
\begin{proof}
\todo
\end{proof}
\begin{remark}
В \(L^\infty(E, \lambda_\M)\) утверждение теоремы неверно. \(L^\infty(\R, \lambda)\) \(B\left(\chi_{[a, b]}, \frac{1}{2}\right)\) не содержит непрерывных функций
\[ \sup_\R|f - \chi_A| \ge \max(\lim_{x \to a + 0} |f(x) - \chi_A|, \lim_{x \to a - 0}|f(x) - \chi_A|) =  \]
\[ = \max(|f(a) - 1|, |f(a) - 0|) \ge \frac{1}{2} \]
\end{remark}
\begin{remark}
В \(L^p(E, \lambda_\M)\), \(p < +\infty\) плотны:
\begin{itemize}
\item Гладкие функции
\item Непрерывные функции
\item \todo
\end{itemize}
\end{remark}
\end{document}
