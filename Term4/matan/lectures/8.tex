% Created 2021-04-05 Mon 13:10
% Intended LaTeX compiler: pdflatex

\documentclass[english]{article}
\usepackage[T1, T2A]{fontenc}
\usepackage[lutf8]{luainputenc}
\usepackage[english, russian]{babel}
\usepackage{minted}
\usepackage{graphicx}
\usepackage{longtable}
\usepackage{hyperref}
\usepackage{xcolor}
\usepackage{natbib}
\usepackage{amssymb}
\usepackage{stmaryrd}
\usepackage{amsmath}
\usepackage{caption}
\usepackage{mathtools}
\usepackage{amsthm}
\usepackage{tikz}
\usepackage{grffile}
\usepackage{extarrows}
\usepackage{wrapfig}
\usepackage{rotating}
\usepackage{placeins}
\usepackage[normalem]{ulem}
\usepackage{amsmath}
\usepackage{textcomp}
\usepackage{capt-of}

\usepackage{geometry}
\geometry{a4paper,left=2.5cm,top=2cm,right=2.5cm,bottom=2cm,marginparsep=7pt, marginparwidth=.6in}
 \usepackage{hyperref}
 \hypersetup{
     colorlinks=true,
     linkcolor=blue,
     filecolor=orange,
     citecolor=black,      
     urlcolor=cyan,
     }

\usetikzlibrary{decorations.markings}
\usetikzlibrary{cd}
\usetikzlibrary{patterns}
\usetikzlibrary{automata, arrows}

\newcommand\addtag{\refstepcounter{equation}\tag{\theequation}}
\newcommand{\eqrefoffset}[1]{\addtocounter{equation}{-#1}(\arabic{equation}\addtocounter{equation}{#1})}


\newcommand{\R}{\mathbb{R}}
\renewcommand{\C}{\mathbb{C}}
\newcommand{\N}{\mathbb{N}}
\newcommand{\rank}{\text{rank}}
\newcommand{\const}{\text{const}}
\newcommand{\grad}{\text{grad}}

\newcommand{\todo}{{\color{red}\fbox{\text{Доделать}}}}
\newcommand{\fixme}{{\color{red}\fbox{\text{Исправить}}}}

\newcounter{propertycnt}
\setcounter{propertycnt}{1}
\newcommand{\beginproperty}{\setcounter{propertycnt}{1}}

\theoremstyle{plain}
\newtheorem{propertyinner}{Свойство}
\newenvironment{property}{
  \renewcommand\thepropertyinner{\arabic{propertycnt}}
  \propertyinner
}{\endpropertyinner\stepcounter{propertycnt}}
\newtheorem{axiom}{Аксиома}
\newtheorem{lemma}{Лемма}
\newtheorem{manuallemmainner}{Лемма}
\newenvironment{manuallemma}[1]{%
  \renewcommand\themanuallemmainner{#1}%
  \manuallemmainner
}{\endmanuallemmainner}

\theoremstyle{remark}
\newtheorem*{remark}{Примечание}
\newtheorem*{solution}{Решение}
\newtheorem{corollary}{Следствие}[theorem]
\newtheorem*{examp}{Пример}
\newtheorem*{observation}{Наблюдение}

\theoremstyle{definition}
\newtheorem{task}{Задача}
\newtheorem{theorem}{Теорема}[section]
\newtheorem*{definition}{Определение}
\newtheorem*{symb}{Обозначение}
\newtheorem{manualtheoreminner}{Теорема}
\newenvironment{manualtheorem}[1]{%
  \renewcommand\themanualtheoreminner{#1}%
  \manualtheoreminner
}{\endmanualtheoreminner}
\captionsetup{justification=centering,margin=2cm}
\newenvironment{colored}[1]{\color{#1}}{}

\tikzset{->-/.style={decoration={
  markings,
  mark=at position .5 with {\arrow{>}}},postaction={decorate}}}
\makeatletter
\newcommand*{\relrelbarsep}{.386ex}
\newcommand*{\relrelbar}{%
  \mathrel{%
    \mathpalette\@relrelbar\relrelbarsep
  }%
}
\newcommand*{\@relrelbar}[2]{%
  \raise#2\hbox to 0pt{$\m@th#1\relbar$\hss}%
  \lower#2\hbox{$\m@th#1\relbar$}%
}
\providecommand*{\rightrightarrowsfill@}{%
  \arrowfill@\relrelbar\relrelbar\rightrightarrows
}
\providecommand*{\leftleftarrowsfill@}{%
  \arrowfill@\leftleftarrows\relrelbar\relrelbar
}
\providecommand*{\xrightrightarrows}[2][]{%
  \ext@arrow 0359\rightrightarrowsfill@{#1}{#2}%
}
\providecommand*{\xleftleftarrows}[2][]{%
  \ext@arrow 3095\leftleftarrowsfill@{#1}{#2}%
}
\makeatother
\author{Ilya Yaroshevskiy}
\date{\today}
\title{Лекция 8}
\hypersetup{
 pdfauthor={Ilya Yaroshevskiy},
 pdftitle={Лекция 8},
 pdfkeywords={},
 pdfsubject={},
 pdfcreator={Emacs 28.0.50 (Org mode 9.4.4)}, 
 pdflang={English}}
\begin{document}

\maketitle
\tableofcontents

\newcommand{\A}{\mathfrak{A}}
\newcommand{\esssup}{\mathop{\rm ess\,sup}\limits}


\begin{itemize}
\item \(M\)
\item \(\Phi: O \subset \R^2 \to \R^3\)
\item \(f\)
\item \(f \citc \Phi\)
\end{itemize}
\[ \int_{M/E} d fs = \int_{O/\Phi^{-1}(E)} f \circ \Phi \cdot |\Phi'u\times\Phi'v|\,du\,dv \]
\begin{definition}
\(M \subset \R^3\) --- \textbf{кусочно гладкое двумерное многообразие}, если
\(M\) --- конечное объединение
\begin{itemize}
\item простых гладких двумерных многообразий \(M_i\)
\item гладких кривых
\item точек
\end{itemize}
\end{definition}
\begin{remark}
Просто так сферу параметризовать не можем, но можем разбить ее на две полусферы и окружность и считать отдельно для каждой из них.
\end{remark}
\begin{definition}
\(E \subset M\) --- измеримое, если измеримы все \(E \cap M_i\). \\
\[S(E) := \sum_i S(E\cap M_i) \]
\[ \int_E f ds := \sum_i \int_{E \cap M_i} f ds \]
\end{definition}
\section{Поверхностный интеграл II рода}
\label{sec:orga44f5fc}
\begin{itemize}
\item \(M\) --- простое гладкое двумерное многообразие в \(R^3\) --- поверхность
\end{itemize}
\begin{definition}
\textbf{Сторона поверхности} --- непрерывное семейство единичных нормалей к этой поверхности \\
\(M \subset \R^3 \quad W: M \to \R^3\) \\
\(\forall x\ W(x)\) --- нормаль к \(M\), \(|w(x)| = 1\), \(w(x) \perp \Phi'u,\Phi'_v\)
\end{definition}
\begin{remark}
Локльно каждая повехность --- двустороннее. В общем случае --- 1 или 2 стороны
\end{remark}
\begin{remark}
График функции \(z(x, y)\)
\[ \Phi: (x, y) \mapsto \begin{pmatrix} x \\ y \\ z(x, y) \end{pmatrix} \]
\[ \Phi'_x = \begin{pmatrix} 1 \\ 0 \\ z'_x \end{pmatrix}\quad\Phi'_y = \begin{pmatrix} 0 \\ 1 \\ z'_y \end{pmatrix} \]
--- касательные векторы
\[n := \Phi'_x \times \Phi'_y = \begin{pmatrix} -z'_x \\ -z'_y \\ 1 \end{pmatrix}\] --- нормаль
\[ n_0 = \pm \left(-\frac{z'_x}{\sqrt{1 + z'_x^2 + z'_y^2}}, -\frac{z'_y}{\sqrt{\dots}}, \frac{1}{\sqrt{\dots}}\right) \]
\end{remark}
\begin{remark}
Другой способ задания стороны поверхности
\begin{enumerate}
\item \(u, v\) --- касательные векторы \\
\(u \not\parallel v\), \((u, v)\) --- касательный реп\(\acute{\text{е}}\)р \\
Если задано непрерывное поле реперов, то они задают сторону \(n = u \times v\)(отнормировать)
\item Задана петля + указано непрерывное движение
\end{enumerate}
\end{remark}
\begin{definition}
\(M\) --- поверхность в \(\R^3\), \(n_0\) --- сторона, \(\gamma\) --- контур(петля) в \(M\) --- ориентированный. \\
Говорят, что сторона поверхности \(n_0\) согласована с ориентацией \(\gamma\): \((\gamma' \times N_\text{внутр.}) \parallel n_0\). Т.е. если ориентация \(\gamma\) задает сторону \(n_0\)
\end{definition}
\begin{definition}
\-
\begin{itemize}
\item \(M\) --- простое двумерное гладкое многообразие
\item \(n_0\) --- сторона \(M\)
\item \(F: M \to \R^3\) --- векторное поле(непрерывное)
\end{itemize}
\[ \int_M \langle F, n_0 \rangle \, ds \] --- \textbf{интеграл II рода} векторного поля \(F\) по поверхности \(M\)
\end{definition}
\begin{remark}
Смена стороны = смена знака
\end{remark}
\begin{remark}
Не зависит от параметра
\end{remark}
\begin{remark}
\(F = (P, Q, R)\) обозначается
\[ \iint_M P \, dy\,dz + Q\,dz\,dx + R\,dx\,dy \]
\end{remark}
\begin{remark}
\(\Phi, n = \Phi'_u \times \Phi'_v \leadsto n_0\)
\[ \int_M \langle F, n_0 \rangle = \int_O \left\langle F, \frac{\Phi'_u \times \Phi'_v}{|\Phi'_u\times\Phi'_v} \right\rangle |\Phi'_u\times\Phi'_v|\,du\,dv =  \]
\[ \int_O \underbrace{\langle F, \Phi'_u\times\Phi'_v \rangle}_\text{смешенное произведение} \,du\,dv \addtag\label{int_1_8} \]
\[ \Phi(u, v) = (x(u, v), y(u, v), z(u, v)) \]
\[ \langle F, \Phi'_u\times\Phi'_v \rangle = \det\begin{pmatrix}P & x'_u & x'_v \\ Q & y'_u & y'_v \\ R & z'_u & z'_v\end{pmatrix} \]
\[ \ref{int_1_8} = \int_O P\cdot\begin{vmatrix} y'_u & z'_v \\ z'_u & z'_v \end{vmatrix} + Q\cdot\begin{vmatrix}z'_u & z'_v \\ x'_u & x'_v\end{vmatrix} + R\cdot\begin{vmatrix}x'_u & x'_v \\ y'_u & y'_v\end{vmatrix}\,du\,dv \]
\end{remark}
\begin{examp}
График \(z(x, y)\) над областью \(G\) по верхней стороне
\[ \iint_{\Gamma_z} R\,dx\,dy = \iint_{\Gamma_z} 0\,dy\,dz + 0\,dz\,dy + R(x, y, z)\,dx\,dy \addtag\label{int_2_8} \]
\[ n_0 = \left(-\frac{z'_x}{\sqrt{1 + z'_x^2 + z'_y^2}}, -\frac{z'_y}{\sqrt{\dots}}, \frac{1}{\sqrt{\dots}}\right) \]
\[ \ref{int_2_8} = \iint_{\Gamma_z} R(x, y, z)\cdot \frac{1}{\sqrt{1 + z'_x^2 + z'_y^2}}\,ds = \iint_G R(x, y, z(x, y)) \,dx\,dy = \iint_G R\,dx\,dy \]
т.е. этот интеграл II рода равен интегралу по проекции
\end{examp}
\begin{corollary}
\-
\begin{itemize}
\item \(V \subset \R^3\)
\item \(M = \partial V\) --- гладкая двумерная поверхность
\item \(n_0\) --- внешняя нормаль
\end{itemize}
\[ \lambda_3 V = \iint_{\partial V} z\,dx\,dy = \frac{1}{3}\iint_{\partial V} x\,dy\,dz + y\,dz\,dx + z\,dx\,dy \]
\end{corollary}
\begin{corollary}
\(\Omega\) --- гладкая кривая в \(\R^2\), \(M\) (--- цилиндр над \(\Omega\)) \(=\Omega \times [z_0, z_1]\) \\
\uline{Тогда} (сторона \(M\) любая) \(\int_M R\,dx\,dy = 0\)
\end{corollary}
\section{Ряды Фурье}
\label{sec:org24825dc}
\subsection{Пространства \(L^p\)}
\label{sec:org5ccc7f1}
\begin{property}
\-
\begin{itemize}
\item \((X, \A, \mu)\)
\item \(f: X \to \C\) \\
\(x = f(x) = u(x) + iv(x)\) \\
\(u = \Re f,\ v = \Im f\) \\
\item \(f\) --- измеримая, если \(u\) и \(v\) --- измеримые
\item \(f\) --- суммируемая, \(u\) и \(v\) --- суммирумые
\item \(f\) --- суммируемая: \(\int_E f = \int_E u + \int_E v\)
\end{itemize}
\end{property}
\begin{property}[Неравенство Гёльдера]
\-
\begin{itemize}
\item \(p,q > 1\) \(\frac{1}{p} + \frac{1}{q} = 1\)
\item \((X, \A, \mu)\)
\item \(E\) --- измеримое
\item \(f, g: E \to \C\) --- измеримые
\end{itemize}
\uline{Тогда} \[ \int_E |fg| d\mu \le \left(\int_E |f|^p\right)^{\frac{1}{p}} \left(\int_E |g|^q\right)^{\frac{1}{q}} \]
\label{org37da624}
\end{property}
\begin{property}[Неравенство Минковского]
Те-же условия что и в \hyperref[org37da624]{Неравенстве Гельдера}
\[ \left(\int_E |f + g|^p\right)^{\frac{1}{p}} \le \left(\int_E |f|^p\right)^{\frac{1}{p}} + \left(\int_E |g|^p\right)^{\frac{1}{p}} \]
\end{property}
\begin{remark}
При \(p = 1\) неравенство тоже верно
\end{remark}
\begin{property}
\begin{definition}
\(L^p\), \(1 \le p \le +\infty\)
\begin{itemize}
\item \((X, \A, \mu)\)
\item \(E \subset X\) --- измеримое
\end{itemize}
\[ \mathcal{L}^p(E, \mu) := \left\{f: \text{ почти везде }E \to \R(\C) \Big| f\text{ --- измеримая}, \int_E |f|^p\,d\mu < +\infty\right\} \]
--- это линейное пространство (по неравенству Минковского) \\
\(f, g \in \mathcal{L}^p(E, \mu): f \sim g\quad f = h\text{ почти везде}\). \(\mathcal{L}^p/_N = L^p(E, \mu)\) --- линейной пространство. Задаем норму \(\Vert f \Vert_{L^p(E, \mu)} = \left(\int_E |f|^p\right)^{\frac{1}{p}}\)
\end{definition}
\end{property}

\begin{property}
\-
\begin{itemize}
\item \(L^\infty(E,\mu)\)
\item \((X, \A, \mu)\)
\item \(E\) --- измеримое
\item \(f\) --- почти везде \(E \to \overline{\R}\) --- измеримая
\end{itemize}
\[ \esssup_{x \in E} f = \inf \{A \in \overline{R}\Big| f \le A\text{ почти везде}\} \]
\end{property}
\beginproperties
\begin{property}
\(\esssup f \le \sup f\)
\end{property}
\begin{property}
\(f \le \esssup f\) почти везде
\end{property}
\begin{proof}
\(B = \esssup f\) \\
Тогда 
\end{proof}

\begin{property}
\(f\) --- сумм, \(\esssup_E |g| < +\infty\) \\
\uline{Тогда} \[ \left| \int_E fg \right| \le \esssup |g| \cdot \int_E |f| \]
\end{property}
\begin{proof}
\[ \left| \int_E fg \right| \le \int_E |fg| \le \int_E \esssup |g|\cdot|f| \]
\end{proof}
\begin{remark}
\(L^\infty(E, \mu) = \{f: \text{п.в. } E \to \overline{\R}(\overline{\C}),\text{ изм.}, \esssup |f| < +\infty \}/_\sim\). Эквивалентные функции отождествленны --- это нормированное пространство
\[ \Vert f \Vert_{L^\infty(E, \mu)} := \esssup_E |f| = \Vert f \Vert_\infty \]
\end{remark}
\begin{remark}
В новых обозначениях. Неравенство Гельдера:
\[ \Vert fg \Vert_1 \le \Vert f \Vert_p \cdot \Vert g \Vert_q \]
Здесь можно брать \(p = 1,\ q = +\infty\)
\end{remark}
\begin{remark}
\(f \in L^p\) \(\Rightarrow\) \(f\) --- почти везде конечны. \(1 \le p \le +\infty\) \(\Rightarrow\) можно считать \(f\) --- задана всюду на \(E\), и всюду конечна
\end{remark}
\end{document}
