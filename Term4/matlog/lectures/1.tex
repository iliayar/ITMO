% Created 2022-06-11 Sat 01:27
% Intended LaTeX compiler: pdflatex

\documentclass[english]{article}
\usepackage[T1, T2A]{fontenc}
\usepackage[lutf8]{luainputenc}
\usepackage[english, russian]{babel}
\usepackage{minted}
\usepackage{graphicx}
\usepackage{longtable}
\usepackage{hyperref}
\usepackage{xcolor}
\usepackage{natbib}
\usepackage{amssymb}
\usepackage{stmaryrd}
\usepackage{amsmath}
\usepackage{caption}
\usepackage{mathtools}
\usepackage{amsthm}
\usepackage{tikz}
\usepackage{fancyhdr}
\usepackage{lastpage}
\usepackage{titling}
\usepackage{grffile}
\usepackage{extarrows}
\usepackage{wrapfig}
\usepackage{algorithm}
\usepackage{algorithmic}
\usepackage{lipsum}
\usepackage{rotating}
\usepackage{placeins}
\usepackage[normalem]{ulem}
\usepackage{amsmath}
\usepackage{textcomp}
\usepackage{svg}
\usepackage{capt-of}

\usepackage{geometry}
\geometry{a4paper,left=2.5cm,top=2cm,right=2.5cm,bottom=2cm,marginparsep=7pt, marginparwidth=.6in}
\documentclass[12pt, a4paper]{article}

\usepackage{mathtools}
\usepackage{xltxtra}
\usepackage{libertine}
\usepackage{amsmath}
\usepackage{amsthm}
\usepackage{amsfonts}
\usepackage{amssymb}
\usepackage{enumitem}
\usepackage[left=2.3cm, right=2.3cm, top=2.7cm, bottom=2.7cm, bindingoffset=0cm]{geometry}
\usepackage{fancyhdr}

\pagestyle{fancy}
\lfoot{M3137y2019}
\rhead{\thepage}

\DeclareMathOperator*{\xor}{\oplus}
\DeclareMathOperator*{\R}{\mathbb{R}}
\DeclareMathOperator*{\Q}{\mathbb{Q}}
\DeclareMathOperator*{\C}{\mathbb{C}}
\DeclareMathOperator*{\Z}{\mathbb{Z}}
\DeclareMathOperator*{\N}{\mathbb{N}}

\DeclarePairedDelimiter{\ceil}{\lceil}{\rceil}

\setmainfont{Linux Libertine}

\theoremstyle{plain}
\newtheorem{theorem}{Теорема}
\newtheorem{axiom}{Аксиома}
\newtheorem{lemma}{Лемма}

\theoremstyle{remark}
\newtheorem*{remark}{Примечание}
\newtheorem*{consequence}{Следствие}
\newtheorem*{example}{Пример}

\theoremstyle{definition}
\newtheorem*{definition}{Определение}
\author{Ilya Yaroshevskiy}
\date{\today}
\title{Лекция 1}
\hypersetup{
	pdfauthor={Ilya Yaroshevskiy},
	pdftitle={Лекция 1},
	pdfkeywords={},
	pdfsubject={},
	pdfcreator={Emacs 28.1 (Org mode 9.5.3)},
	pdflang={English}}
\begin{document}

\maketitle
\tableofcontents


\section{Исчесление высказываний}
\label{sec:org29ec248}
\subsection{Язык}
\label{sec:org695cb85}
\begin{enumerate}
	\item Пропозициональные переменные \\
	      \(A'_i\) --- большая буква начала латинского алфавита
	\item Связки \\
	      \(\underbrace{\alpha}_\text{\color{green}метапеременная}, \beta\) --- высказывания \\
	      Тогда \((\alpha \to \beta),(\alpha \& \beta),(\alpha \vee \beta), (\neg \alpha)\) --- высказывания
\end{enumerate}
\subsection{Мета и предметные}
\label{sec:org00429d0}
\begin{itemize}
	\item \(\alpha, \beta, \gamma, \dots, \varphi, \psi, \dots\) --- метапеременные для выражений
	\item \(X, Y, Z\) --- метапеременные для предметных переменные
\end{itemize}
Метавыражение: \(\alpha \to \beta\) \\
Предметное выражение: \(A \to (A \to A)\) (заменили \(\alpha\) на \(A\), \(\beta\) на \((A \to A)\) )
\begin{examp}
	Черным --- предметные выражения, Синим --- метавыражения
	\[ (\color{blue}X \color{black}\to\color{blue} Y\color{black})\color{blue}[X \coloneqq A, Y \coloneqq B] \color{black} \equiv A \to B \]
	\[ (\color{blue}\alpha \color{black} \to (A \to \color{blue}X \color{black}))\color{blue}[\alpha \coloneqq A, X \coloneqq B] \equiv \color{black} A \to (A \to B) \]
	\[ (\color{blue}\alpha \color{black} \to (A \to \color{blue}X \color{black}))\color{blue}[\alpha \coloneqq (A \to P), X \coloneqq B] \equiv \color{black} (A \to P) \to (A \to B) \]
\end{examp}
\subsection{Сокращение записи}
\label{sec:orgc8e0dc1}
\begin{itemize}
	\item \(\vee, \&, \neg\) --- скобки слева направо(лево-ассоциативная)
	\item \(\to\) --- правоассоциативная
	\item Приоритет по возрастанию: \(\to, \vee, \&, \neg\)
\end{itemize}
\begin{examp}
	Расставление скобок
	\[ \left(A \to \left( \left(B \& C\right) \to D\right)\right) \]
	\[ \left(A \to \left(B \to C\right)\right) \]
\end{examp}
\subsection{Теория моделей}
\label{sec:org9810c92}
\begin{itemize}
	\item \(\mathcal{P}\) --- множество предметных переменных
	\item \(\llb\cdot\rrb: \mathcal{T} \to V\), где \(\mathcal{T}\) --- множество высказываний, \(V = \{\text{И}, \text{Л}\}\) --- множество истиностных значений
\end{itemize}



\begin{enumerate}
	\item \(\llb x \rrb: \mathcal{P} \to V\) --- задается при оценке \\
	      \(\llb \rrb^{A \coloneqq v_1, B \coloneqq v_2}\):
	      \begin{itemize}
		      \item \(\mathcal{P} = v_1\)
		      \item \(\mathcal{P} = v_2\)
	      \end{itemize}
	\item \(\llb \alpha \star \beta \rrb = \llb \alpha \rrb \color{blue}\underbrace{\star}_{\substack{\text{определенно} \\ \text{ественным образом}}}\color{black} \llb \beta \rrb\), где \(\star \in [\&, \vee, \neg, \to]\)
	      \begin{examp}
		      \[ \llb A \to A \rrb^{A \coloneqq \text{И}, B \coloneqq \text{Л}} = \llb A \rrb^{A \coloneqq \text{И}, B \coloneqq \text{Л}} \color{blue}\to\color{black}\llb A \rrb^{A \coloneqq \text{И}, B \coloneqq \text{Л}} = \color{blue} \text{И} \to \text{И} = \text{И} \]
		      Также можно записать так:
		      \[ \llb A \to A \rrb^{A \coloneqq \text{И}, B \coloneqq \text{Л}} = f_\to(\llb A \rrb^{A \coloneqq \text{И}, B \coloneqq \text{Л}}, \llb A \rrb^{A \coloneqq \text{И}, B \coloneqq \text{Л}}) = f_\to(\color{blue} \text{И}\color{black} , \color{blue}\text{И}\color{black}) \color{blue}= \text{И} \]
		      , где \(f_\to\) определена так:
		      \begin{center}
			      \begin{tabular}{ll|l}
				      \(a\) & \(b\) & \(f_\to\) \\
				      \hline
				      И     & И     & И         \\
				      И     & Л     & Л         \\
				      Л     & И     & И         \\
				      Л     & Л     & И         \\
			      \end{tabular}
		      \end{center}
	      \end{examp}
\end{enumerate}

\subsection{Теория доказательств}
\label{sec:orgdfac5b9}
\begin{definition}
	\textbf{Схема высказывания} --- строка соответсвующая определению высказывания, с:
	\begin{itemize}
		\item метапеременными \(\alpha, \beta, \dots\)
	\end{itemize}
\end{definition}
\begin{definition}
	Аксиома --- высказывания:
	\begin{enumerate}
		\item \(\alpha \to (\beta \to \alpha)\)
		\item \((\alpha \to \beta) \to (\alpha \to \beta \to \gamma) \to (\alpha \to \gamma)\)
		\item \(\alpha \to \beta \to \alpha \& \beta\)
		\item \(\alpha \& \beta \to \alpha\)
		\item \(\alpha \& \beta \to \beta\)
		\item \(\alpha \to \alpha \vee \beta\)
		\item \(\beta \to \alpha \vee \beta\)
		\item \((\alpha \to \gamma) \to (\beta \to \gamma) \to (\alpha \vee \beta \to \gamma)\)
		\item \((\alpha \to \beta) \to (\alpha \to \neg \beta) \to \neg \alpha\)
		\item \(\neg\neg \alpha \to \alpha\)
	\end{enumerate}
\end{definition}
\subsection{Правило Modus Ponens и доказательство}
\label{sec:orgf885186}
\begin{definition}
	\textbf{Доказательство} (вывод) --- последовательность высказываний \(\alpha_1, \dots, \alpha_n\), где \(\alpha_i\):
	\begin{itemize}
		\item аксиома
		\item существует \(k, l < i\), что \(\alpha_k = \alpha_l \to \alpha\) \\
		      \[ \frac{A,\ A \to B}{B} \]
	\end{itemize}
\end{definition}
\begin{examp}
	\(\vdash A \to A\)
	\begin{center}
		\begin{tabular}{r|ll}
			1 & \(A \to A\ \to A\)                                              & (схема аксиом 1) \\
			2 & \(A \to (A \to A) \to A\)                                       & (схема аксиом 1) \\
			3 & \((A \to (A \to A)) \to (A \to (A \to A) \to A) \to (A \to A)\) & (схема аксиом 2) \\
			4 & \((A \to (A \to A) \to A) \to (A \to A)\)                       & (M.P. 1 и 3)     \\
			5 & \(A \to A\)                                                     & (M.P. 2 и 4)     \\
		\end{tabular}
	\end{center}
\end{examp}
\begin{definition}
	Доказательством высказывания \(\beta\) --- список высказываний \(\alpha_1, \dots, \alpha_n\)
	\begin{itemize}
		\item \(\alpha_1, \dots, \alpha_n\) --- доказательство
		\item \(\alpha_n \equiv \beta\)
	\end{itemize}
\end{definition}
\end{document}
