% Created 2022-06-11 Sat 01:27
% Intended LaTeX compiler: pdflatex

\documentclass[english]{article}
\usepackage[T1, T2A]{fontenc}
\usepackage[lutf8]{luainputenc}
\usepackage[english, russian]{babel}
\usepackage{minted}
\usepackage{graphicx}
\usepackage{longtable}
\usepackage{hyperref}
\usepackage{xcolor}
\usepackage{natbib}
\usepackage{amssymb}
\usepackage{stmaryrd}
\usepackage{amsmath}
\usepackage{caption}
\usepackage{mathtools}
\usepackage{amsthm}
\usepackage{tikz}
\usepackage{fancyhdr}
\usepackage{lastpage}
\usepackage{titling}
\usepackage{grffile}
\usepackage{extarrows}
\usepackage{wrapfig}
\usepackage{algorithm}
\usepackage{algorithmic}
\usepackage{lipsum}
\usepackage{rotating}
\usepackage{placeins}
\usepackage[normalem]{ulem}
\usepackage{amsmath}
\usepackage{textcomp}
\usepackage{svg}
\usepackage{capt-of}

\usepackage{geometry}
\geometry{a4paper,left=2.5cm,top=2cm,right=2.5cm,bottom=2cm,marginparsep=7pt, marginparwidth=.6in}
 \usepackage{hyperref}
 \hypersetup{
     colorlinks=true,
     linkcolor=blue,
     filecolor=orange,
     citecolor=black,      
     urlcolor=cyan,
     }

\usetikzlibrary{decorations.markings}
\usetikzlibrary{cd}
\usetikzlibrary{patterns}
\usetikzlibrary{automata, arrows}

\newcommand\addtag{\refstepcounter{equation}\tag{\theequation}}
\newcommand{\eqrefoffset}[1]{\addtocounter{equation}{-#1}(\arabic{equation}\addtocounter{equation}{#1})}
\newcommand{\llb}{\llbracket}
\newcommand{\rrb}{\rrbracket}


\newcommand{\R}{\mathbb{R}}
\renewcommand{\C}{\mathbb{C}}
\newcommand{\N}{\mathbb{N}}
\newcommand{\A}{\mathfrak{A}}
\newcommand{\B}{\mathfrak{B}}
\newcommand{\rank}{\mathop{\rm rank}\nolimits}
\newcommand{\const}{\var{const}}
\newcommand{\grad}{\mathop{\rm grad}\nolimits}
\newcommand{\custombracket}[3]{\left#1 #3 \right#2}
\newcommand{\custombracketsame}[2]{\custombracket{#1}{#1}{#2}}
\newcommand{\pair}[1]{\custombracket{\langle}{\rangle}{#1}}
\newcommand{\eval}[1]{\custombracket{\llbracket}{\rrbracket}{#1}}

\newcommand{\todo}{{\color{red}\fbox{\text{Доделать}}}}
\newcommand{\fixme}{{\color{red}\fbox{\text{Исправить}}}}

\newcounter{propertycnt}
\setcounter{propertycnt}{1}
\newcommand{\beginproperty}{\setcounter{propertycnt}{1}}

\theoremstyle{plain}
\newtheorem{propertyinner}{\protect\propertyname}
\newenvironment{property}{
  \renewcommand\thepropertyinner{\arabic{propertycnt}}
  \propertyinner
}{\endpropertyinner\stepcounter{propertycnt}}
\newtheorem{axiom}{\protect\axiomname}
\newtheorem{lemma}{\protect\lemmaname}
\newtheorem*{statement}{\protect\statementname}
\newtheorem{manuallemmainner}{\protect\lemmaname}
\newenvironment{manuallemma}[1]{%
  \renewcommand\themanuallemmainner{#1}%
  \manuallemmainner
}{\endmanuallemmainner}

\theoremstyle{remark}
\newtheorem*{remark}{\protect\remarkname}
\newtheorem*{solution}{\protect\solutionname}
\newtheorem{corollary}{\protect\corollaryname}[theorem]
\newtheorem*{examp}{\protect\exampname}
\newtheorem*{observation}{\protect\observationname}

\theoremstyle{definition}
\newtheorem{task}{\protect\taskname}
\newtheorem{theorem}{\protect\theoremname}[section]
\newtheorem*{definition}{\protect\definitionname}
\newtheorem*{symb}{\protect\symbname}
\newtheorem{manualtheoreminner}{\protect\theoremname}
\newenvironment{manualtheorem}[1]{%
  \renewcommand\themanualtheoreminner{#1}%
  \manualtheoreminner
}{\endmanualtheoreminner}

\newtheoremstyle{colon}%
{}
{}
{}%bodyfont
{}%indent
{\bfseries}%headfont
{:}%head punctuation
{ }%space after head
{}
\theoremstyle{colon}
\newtheorem*{answer}{\protect\answername}

\newcommand{\propertyname}{}
\newcommand{\axiomname}{}
\newcommand{\lemmaname}{}
\newcommand{\statementname}{}
\newcommand{\remarkname}{}
\newcommand{\solutionname}{}
\newcommand{\corollaryname}{}
\newcommand{\exampname}{}
\newcommand{\observationname}{}
\newcommand{\taskname}{}
\newcommand{\theoremname}{}
\newcommand{\definitionname}{}
\newcommand{\symbname}{}
\newcommand{\answername}{}
\addto\captionsrussian{%
  \renewcommand{\propertyname}{Свойство}%
  \renewcommand{\axiomname}{Аксиома}%
  \renewcommand{\lemmaname}{Лемма}%
  \renewcommand{\statementname}{Утверждение}%
  \renewcommand{\remarkname}{Замечание}%
  \renewcommand{\solutionname}{Решение}%
  \renewcommand{\corollaryname}{Следствие}%
  \renewcommand{\exampname}{Пример}%
  \renewcommand{\observationname}{Наблюдение}%
  \renewcommand{\taskname}{Задача}%
  \renewcommand{\theoremname}{Теорема}%
  \renewcommand{\definitionname}{Определение}%
  \renewcommand{\symbname}{Обозначение}%
  \renewcommand{\answername}{Ответ}%
}
\addto\captionsenglish{%
  \renewcommand{\propertyname}{Property}%
  \renewcommand{\axiomname}{Axiom}%
  \renewcommand{\lemmaname}{Lemma}%
  \renewcommand{\statementname}{Statement}%
  \renewcommand{\remarkname}{Remark}%
  \renewcommand{\solutionname}{Solution}%
  \renewcommand{\corollaryname}{Corollary}%
  \renewcommand{\exampname}{Example}%
  \renewcommand{\observationname}{Observation}%
  \renewcommand{\taskname}{Task}%
  \renewcommand{\theoremname}{Theorem}%
  \renewcommand{\definitionname}{Definition}%
  \renewcommand{\answername}{Answer}%
}


\captionsetup{justification=centering,margin=2cm}
\newenvironment{colored}[1]{\color{#1}}{}

\tikzset{->-/.style={decoration={
  markings,
  mark=at position .5 with {\arrow{>}}},postaction={decorate}}}
\makeatletter
\newcommand*{\relrelbarsep}{.386ex}
\newcommand*{\relrelbar}{%
  \mathrel{%
    \mathpalette\@relrelbar\relrelbarsep
  }%
}
\newcommand*{\@relrelbar}[2]{%
  \raise#2\hbox to 0pt{$\m@th#1\relbar$\hss}%
  \lower#2\hbox{$\m@th#1\relbar$}%
}
\providecommand*{\rightrightarrowsfill@}{%
  \arrowfill@\relrelbar\relrelbar\rightrightarrows
}
\providecommand*{\leftleftarrowsfill@}{%
  \arrowfill@\leftleftarrows\relrelbar\relrelbar
}
\providecommand*{\xrightrightarrows}[2][]{%
  \ext@arrow 0359\rightrightarrowsfill@{#1}{#2}%
}
\providecommand*{\xleftleftarrows}[2][]{%
  \ext@arrow 3095\leftleftarrowsfill@{#1}{#2}%
}
\makeatother

\newenvironment{rualgo}[1][]
  {\begin{algorithm}[#1]
     \selectlanguage{russian}%
     \floatname{algorithm}{Алгоритм}%
     \renewcommand{\algorithmicif}{{\color{red}\textbf{если}}}%
     \renewcommand{\algorithmicthen}{{\color{red}\textbf{тогда}}}%
     \renewcommand{\algorithmicelse}{{\color{red}\textbf{иначе}}}%
     \renewcommand{\algorithmicend}{{\color{red}\textbf{конец}}}%
     \renewcommand{\algorithmicfor}{{\color{red}\textbf{для}}}%
     \renewcommand{\algorithmicto}{{\color{red}\textbf{до}}}%
     \renewcommand{\algorithmicdo}{{\color{red}\textbf{делать}}}%
     \renewcommand{\algorithmicwhile}{{\color{red}\textbf{пока}}}%
     \renewcommand{\algorithmicrepeat}{{\color{red}\textbf{повторять}}}%
     \renewcommand{\algorithmicuntil}{{\color{red}\textbf{до тех пор пока}}}%
     \renewcommand{\algorithmicloop}{{\color{red}\textbf{повторять}}}%
     \renewcommand{\algorithmicnot}{{\color{blue}\textbf{не}}}%
     \renewcommand{\algorithmicand}{{\color{blue}\textbf{и}}}%
     \renewcommand{\algorithmicor}{{\color{blue}\textbf{или}}}%
     \renewcommand{\algorithmicrequire}{{\color{blue}\textbf{Ввод}}}%
     \renewcommand{\algorithmicensure}{{\color{blue}\textbf{Вывод}}}%
     \renewcommand{\algorithmicreturn}{{\color{red}\textbf{Вернуть}}}%
     \renewcommand{\algorithmicrtrue}{{\color{blue}\textbf{истинна}}}%
     \renewcommand{\algorithmicrfalse}{{\color{blue}\textbf{ложь}}}%
     % Set other language requirements
  }
  {\end{algorithm}}

\newenvironment{enalgo}[1][]
  {\begin{algorithm}[#1]
     \selectlanguage{english,russian}%
     \floatname{algorithm}{Program}%
     \renewcommand{\algorithmicif}{{\color{red}\textbf{if}}}%
     \renewcommand{\algorithmicthen}{{\color{red}\textbf{then}}}%
     \renewcommand{\algorithmicelse}{{\color{red}\textbf{else}}}%
     \renewcommand{\algorithmicend}{{\color{red}\textbf{end}}}%
     \renewcommand{\algorithmicfor}{{\color{red}\textbf{for}}}%
     \renewcommand{\algorithmicto}{{\color{red}\textbf{to}}}%
     \renewcommand{\algorithmicdo}{{\color{red}\textbf{do}}}%
     \renewcommand{\algorithmicwhile}{{\color{red}\textbf{while}}}%
     \renewcommand{\algorithmicrepeat}{{\color{red}\textbf{repeat}}}%
     \renewcommand{\algorithmicuntil}{{\color{red}\textbf{until}}}%
     \renewcommand{\algorithmicloop}{{\color{red}\textbf{loop}}}%
     \renewcommand{\algorithmicnot}{{\color{blue}\textbf{not}}}%
     \renewcommand{\algorithmicand}{{\color{blue}\textbf{and}}}%
     \renewcommand{\algorithmicor}{{\color{blue}\textbf{or}}}%
     \renewcommand{\algorithmicrequire}{{\color{blue}\textbf{Input}}}%
     \renewcommand{\algorithmicensure}{{\color{blue}\textbf{Output}}}%
     \renewcommand{\algorithmicreturn}{{\color{red}\textbf{return}}}%
     \renewcommand{\algorithmicrtrue}{{\color{blue}\textbf{true}}}%
     \renewcommand{\algorithmicrfalse}{{\color{blue}\textbf{false}}}%
     % Set other language requirements
  }
  {\end{algorithm}}

\pagestyle{fancy}
\fancyhf{}
\fancyhead[R]{\thetitle}
\fancyfoot[L]{ITMO y2019}
\fancyfoot[C]{Page \thepage \hspace{1pt} of \pageref*{LastPage}}
\renewcommand{\footrulewidth}{0.4pt}

\author{Ilya Yaroshevskiy}
\date{\today}
\title{Лекция 3}
\hypersetup{
	pdfauthor={Ilya Yaroshevskiy},
	pdftitle={Лекция 3},
	pdfkeywords={},
	pdfsubject={},
	pdfcreator={Emacs 28.1 (Org mode 9.5.3)},
	pdflang={English}}
\begin{document}

\maketitle
\tableofcontents


\section{Правила вывода}
\label{sec:org27aa55d}
Сверху посылки, снизу заключения
\begin{itemize}
	\item Аксиома
	      \[ \frac{}{\Gamma, \varphi \vdash \varphi} \]
	\item Введение \(\to\)
	      \[ \frac{\Gamma, \varphi \vdash \psi}{\Gamma \vdash \varphi \to \psi} \]
	\item Удаление \(\to\)
	      \[ \frac{\Gamma \vdash \varphi \to \psi\quad \Gamma \vdash \varphi}{\Gamma \vdash \psi} \]
	\item Введение \(\&\)
	      \[ \frac{\Gamma \vdash \varphi \quad \Gamma \vdash \psi}{\Gamma \vdash \varphi \& \psi} \]
	\item Удаление \(\&\)
	      \[ \frac{\Gamma \vdash \varphi \& \psi}{\Gamma \vdash \varphi} \]
	      \[ \frac{\Gamma \vdash \varphi \& \psi}{\Gamma \vdash \psi} \]
	\item Введение \(\vee\)
	      \[ \frac{\Gamma \vdash \varphi}{\Gamma \vdash \varphi \vee \psi} \]
	      \[ \frac{\Gamma \vdash \psi}{\Gamma \vdash \varphi \vee \psi} \]
	\item Удалние \(\vee\)
	      \[ \frac{\Gamma, \varphi \vdash \rho \quad \Gamma, \psi \vdash \rho \quad \Gamma \vdash \varphi \vee \psi}{\Gamma \vdash \rho} \]
	\item Удаление \(\perp\)
	      \[ \frac{\Gamma \vdash \perp}{\Gamma \vdash \varphi} \]
\end{itemize}
\begin{examp}
	\[ \frac{\displaystyle\frac{}{A \vdash A}(\text{акс.})}{\vdash A \to A}(\text{вв. }\to) \]
\end{examp}
\begin{examp}
	Докажем \(\frac{}{\vdash A \& B \to B \& A}\)
	\[ \frac{\displaystyle\frac{\displaystyle\frac{\displaystyle\frac{}{A \& B \vdash A \& B}(\text{акс.})}{A\& B \vdash B}(\text{уд. } \&) \quad \frac{\displaystyle\frac{}{A \& B \vdash A \& B}(\text{акс.})}{A \& B \vdash A}(\text{уд. } \&)}{A\&B \vdash B \& A}(\text{вв. } \&)}{\vdash A \& B \to B & A}(\text{вв. }\to) \]
\end{examp}
\begin{definition}
	Фиксируем \(A\) \\
	Частичный порядок --- антисимметричное, транзитивное, рефлексивное отношение \\
	Линейный --- сравнимы любые 2 элемента \\
	\begin{itemize}
		\item \(a \le b \vee b \le a\)
		\item \textbf{Наименьший элемент} \(S\) --- такой \(k \in S\), что если \(x \in S\), то \(k \le x\)
		\item \textbf{Минимальный элемент} \(S\) --- такой \(k \in S\), что нет \(x \in S\), что \(x \le k\)
	\end{itemize}
	\label{org5d25b02}
\end{definition}
\begin{examp}
	\-
	\begin{center}
		\begin{tikzcd}
			\([9, 9 , 9]\) \arrow{d} \arrow{dr} &  & \([1, 2, 1]\) \arrow{dll} \arrow{dl} \arrow{d} \\
			\([1, 0 , 0]\) & \([0, 1, 0]\) & \([0, 0, 1]\)
		\end{tikzcd}
	\end{center}
	Нет наименьшего, но есть 3 минимальных. Стрелка из \(a\) в \(b\) обозначает \(b \le a\)
\end{examp}
\begin{definition}
	\-
	\begin{itemize}
		\item \textbf{Множество верхних граней} \(a\) и \(b\): \(\{x \big| a \le x \& b \le x\}\)
		\item \textbf{Множество нижних граней} \(a\) и \(b\): \(\{x \big| x \le a \& x \le b\}\)
	\end{itemize}
	\label{org05aaaed}
\end{definition}
\begin{definition}
	\-
	\begin{itemize}
		\item \textbf{\(a + b\)} --- нименьший элемент множества верхних граней
		\item \textbf{\(a \cdot b\)} --- наибольший элемент множества нижних граней
	\end{itemize}
	\label{orgb717294}
\end{definition}
\begin{definition}
	\textbf{Решетка} = \(\langle A, \le \rangle\) --- структура, где для каждых \(a, b\) есть как \(a + b\), так и \(a \cdot b\), \\
	т.е. \(a \in A, b \in B \implies a + b \in A\) и \(a \cdot b \in A\)
	\label{org98817ef}
\end{definition}
\begin{definition}
	\textbf{Дистрибутивная решетка} если всегда  \(a \cdot (b + c) = a \cdot b + a \cdot c\)
	\label{org4c1b85d}
\end{definition}
\begin{lemma}
	В дистрибутивной решетке \(a + b\cdot c = (a + b) \cdot(a + c)\)
	\label{org78b6809}
\end{lemma}
\begin{definition}
	\textbf{Псевдодополнение} \(a \to b = \text{наиб.}\{c \big| a \cdot c \le b\}\)
	\label{orgc09946b}
\end{definition}
\begin{definition}
	\textbf{Импликативная решетка} --- решетка, где для любых \(a, b\) есть \(a \to b\)
	\label{org922d0d4}
\end{definition}
\begin{definition}
	\textbf{0} --- наименьший элемент решетки, \textbf{1} --- наибольший элемент решетки
	\label{orge2e1c3d}
\end{definition}
\begin{definition}
	\textbf{Псевдобулева алгебра (алгебра Гейтинга)} --- импликативная решетка с \(0\)
	\label{orgb60cc47}
\end{definition}
\begin{definition}
	\textbf{Булева алгебра} --- псевдобулева алгебра, такая что \(a + (a \to 0) = 1\)
	\label{org8b1fe92}
\end{definition}
\begin{examp}
	\-
	\begin{center}
		\begin{tikzpicture}
			\node (A) at (0, 0) {\(1\)};
			\node (B) at (-1, -1) {\(a\)};
			\node (C) at (1, -1) {\(b\)};
			\node (D) at (0, -2) {\(0\)};
			\draw[->] (A) -- (B);
			\draw[->] (A) -- (C);
			\draw[->] (B) -- (D);
			\draw[->] (C) -- (D);
		\end{tikzpicture}
	\end{center}
	\begin{itemize}
		\item \(a \cdot 0 = 0\)
		\item \(1\cdot b = b\)
		\item \(a \cdot b = 0\)
		\item \(a + b = 1\)
		\item \(a \to b = \text{наиб.}\{x \big| a\cdot x \le b\} = b\) \\
		      \(\{x \big| a \cdot x \le \} = \{0, b\}\)
		\item \(a \to 1 = 1\)
		\item \(a \to 0 = 0\)
	\end{itemize}
	Можем представить в виде пары \(\langle x, y \rangle\)
	\begin{itemize}
		\item \(a = \langle 1, 0 \rangle\)
		\item \(b = \langle 0 , 1\rangle\)
		\item \(1 = \langle 1, 1 \rangle\)
		\item \(0 = \langle 0, 0 \rangle\)
	\end{itemize}
	\label{org9d15960}
\end{examp}
\begin{lemma}
	В импликативной решетке всегда есть \(1\).
	\label{orgb368e1b}
\end{lemma}
\begin{theorem}
	Любая алгебра Гейтинга --- модель ИИВ
	\label{org70fcbef}
\end{theorem}
\begin{theorem}
	Любая булева алгебра --- модель КИВ
	\label{org0965855}
\end{theorem}
\begin{definition}
	Рассмотрим множество \(X\) --- \textbf{носитель}. Рассмотрим \(\Omega \subseteq 2^X\) --- подмножество подмножеств \(X\) --- \textbf{топология}.
	\begin{enumerate}
		\item \(\bigcup X_i \in \Omega\), где \(X_i \in \Omega\)
		\item \(X_1 \cap \dots \cap X_n \in \Omega\), если \(X_i \in \Omega\)
		\item \(\emptyset, X \in \Omega\)
	\end{enumerate}
	\label{org963e84c}
\end{definition}
\begin{definition}
	\[ (X)^\circ = \text{наиб.}\{w \big| w \subseteq X, w\text{ --- откр.} \} \]
	\label{org81f1bd5}
\end{definition}
\begin{examp}
	Дискретная топология: \(\Omega = 2^X\) --- любое множество открыто. Тогда \(\langle \Omega, \le \rangle\) --- булева алгебра
	\label{orgd5b5287}
\end{examp}
\begin{theorem}
	\-
	\begin{itemize}
		\item \(a + b = a \cup b\)
		\item \(a \cdot b = a \cap b\)
		\item \(a \to b = \left((X \setminus a) \cup b\right)^\circ\)
		\item \(a \le b\) тогда и только тогда, когда \(a \subseteq b\)
	\end{itemize}
	\uline{Тогда} \(\langle \Omega, \le \rangle\) --- алгебра Гейтинга
	\label{orgb437329}
\end{theorem}
\begin{definition}
	\(X\) --- все формулы логики
	\begin{itemize}
		\item \(\alpha \le \beta\) --- это \(\alpha \vdash \beta\)
		\item \(\alpha \approx \beta\), если \(\alpha \vdash \beta\) и \(\beta \vdash \alpha\)
		\item \([\alpha]_\approx = \{\gamma \big| \gamma \approx \alpha\}\) --- класс эквивалентности
		\item \(X/_\approx = \{[\alpha]_\approx \big| \alpha \in X\}\)
	\end{itemize}
	\(\pair{X/_\approx, \le}\) --- алгебра Гейтинга
	\label{orgad7fdc5}
\end{definition}
\beginproperty
\begin{property}
	\(\langle X/_\approx, \le \rangle\) --- алгебра Линденбаума, где \(X, \approx\) --- из интуиционистской логики
	\label{org4ceb224}
\end{property}
\begin{theorem}
	Алгебра Гейтинга --- полная модель ИИВ
	\label{orga6b2edc}
\end{theorem}
\end{document}
