% Created 2022-06-11 Sat 01:27
% Intended LaTeX compiler: pdflatex

\documentclass[english]{article}
\usepackage[T1, T2A]{fontenc}
\usepackage[lutf8]{luainputenc}
\usepackage[english, russian]{babel}
\usepackage{minted}
\usepackage{graphicx}
\usepackage{longtable}
\usepackage{hyperref}
\usepackage{xcolor}
\usepackage{natbib}
\usepackage{amssymb}
\usepackage{stmaryrd}
\usepackage{amsmath}
\usepackage{caption}
\usepackage{mathtools}
\usepackage{amsthm}
\usepackage{tikz}
\usepackage{fancyhdr}
\usepackage{lastpage}
\usepackage{titling}
\usepackage{grffile}
\usepackage{extarrows}
\usepackage{wrapfig}
\usepackage{algorithm}
\usepackage{algorithmic}
\usepackage{lipsum}
\usepackage{rotating}
\usepackage{placeins}
\usepackage[normalem]{ulem}
\usepackage{amsmath}
\usepackage{textcomp}
\usepackage{svg}
\usepackage{capt-of}
\newcommand{\gedel}[1]{\custombracket{\ulcorner}{\urcorner}{#1}}

\usepackage{geometry}
\geometry{a4paper,left=2.5cm,top=2cm,right=2.5cm,bottom=2cm,marginparsep=7pt, marginparwidth=.6in}
 \usepackage{hyperref}
 \hypersetup{
     colorlinks=true,
     linkcolor=blue,
     filecolor=orange,
     citecolor=black,      
     urlcolor=cyan,
     }

\usetikzlibrary{decorations.markings}
\usetikzlibrary{cd}
\usetikzlibrary{patterns}
\usetikzlibrary{automata, arrows}

\newcommand\addtag{\refstepcounter{equation}\tag{\theequation}}
\newcommand{\eqrefoffset}[1]{\addtocounter{equation}{-#1}(\arabic{equation}\addtocounter{equation}{#1})}
\newcommand{\llb}{\llbracket}
\newcommand{\rrb}{\rrbracket}


\newcommand{\R}{\mathbb{R}}
\renewcommand{\C}{\mathbb{C}}
\newcommand{\N}{\mathbb{N}}
\newcommand{\A}{\mathfrak{A}}
\newcommand{\B}{\mathfrak{B}}
\newcommand{\rank}{\mathop{\rm rank}\nolimits}
\newcommand{\const}{\var{const}}
\newcommand{\grad}{\mathop{\rm grad}\nolimits}
\newcommand{\custombracket}[3]{\left#1 #3 \right#2}
\newcommand{\custombracketsame}[2]{\custombracket{#1}{#1}{#2}}
\newcommand{\pair}[1]{\custombracket{\langle}{\rangle}{#1}}
\newcommand{\eval}[1]{\custombracket{\llbracket}{\rrbracket}{#1}}

\newcommand{\todo}{{\color{red}\fbox{\text{Доделать}}}}
\newcommand{\fixme}{{\color{red}\fbox{\text{Исправить}}}}

\newcounter{propertycnt}
\setcounter{propertycnt}{1}
\newcommand{\beginproperty}{\setcounter{propertycnt}{1}}

\theoremstyle{plain}
\newtheorem{propertyinner}{\protect\propertyname}
\newenvironment{property}{
  \renewcommand\thepropertyinner{\arabic{propertycnt}}
  \propertyinner
}{\endpropertyinner\stepcounter{propertycnt}}
\newtheorem{axiom}{\protect\axiomname}
\newtheorem{lemma}{\protect\lemmaname}
\newtheorem*{statement}{\protect\statementname}
\newtheorem{manuallemmainner}{\protect\lemmaname}
\newenvironment{manuallemma}[1]{%
  \renewcommand\themanuallemmainner{#1}%
  \manuallemmainner
}{\endmanuallemmainner}

\theoremstyle{remark}
\newtheorem*{remark}{\protect\remarkname}
\newtheorem*{solution}{\protect\solutionname}
\newtheorem{corollary}{\protect\corollaryname}[theorem]
\newtheorem*{examp}{\protect\exampname}
\newtheorem*{observation}{\protect\observationname}

\theoremstyle{definition}
\newtheorem{task}{\protect\taskname}
\newtheorem{theorem}{\protect\theoremname}[section]
\newtheorem*{definition}{\protect\definitionname}
\newtheorem*{symb}{\protect\symbname}
\newtheorem{manualtheoreminner}{\protect\theoremname}
\newenvironment{manualtheorem}[1]{%
  \renewcommand\themanualtheoreminner{#1}%
  \manualtheoreminner
}{\endmanualtheoreminner}

\newtheoremstyle{colon}%
{}
{}
{}%bodyfont
{}%indent
{\bfseries}%headfont
{:}%head punctuation
{ }%space after head
{}
\theoremstyle{colon}
\newtheorem*{answer}{\protect\answername}

\newcommand{\propertyname}{}
\newcommand{\axiomname}{}
\newcommand{\lemmaname}{}
\newcommand{\statementname}{}
\newcommand{\remarkname}{}
\newcommand{\solutionname}{}
\newcommand{\corollaryname}{}
\newcommand{\exampname}{}
\newcommand{\observationname}{}
\newcommand{\taskname}{}
\newcommand{\theoremname}{}
\newcommand{\definitionname}{}
\newcommand{\symbname}{}
\newcommand{\answername}{}
\addto\captionsrussian{%
  \renewcommand{\propertyname}{Свойство}%
  \renewcommand{\axiomname}{Аксиома}%
  \renewcommand{\lemmaname}{Лемма}%
  \renewcommand{\statementname}{Утверждение}%
  \renewcommand{\remarkname}{Замечание}%
  \renewcommand{\solutionname}{Решение}%
  \renewcommand{\corollaryname}{Следствие}%
  \renewcommand{\exampname}{Пример}%
  \renewcommand{\observationname}{Наблюдение}%
  \renewcommand{\taskname}{Задача}%
  \renewcommand{\theoremname}{Теорема}%
  \renewcommand{\definitionname}{Определение}%
  \renewcommand{\symbname}{Обозначение}%
  \renewcommand{\answername}{Ответ}%
}
\addto\captionsenglish{%
  \renewcommand{\propertyname}{Property}%
  \renewcommand{\axiomname}{Axiom}%
  \renewcommand{\lemmaname}{Lemma}%
  \renewcommand{\statementname}{Statement}%
  \renewcommand{\remarkname}{Remark}%
  \renewcommand{\solutionname}{Solution}%
  \renewcommand{\corollaryname}{Corollary}%
  \renewcommand{\exampname}{Example}%
  \renewcommand{\observationname}{Observation}%
  \renewcommand{\taskname}{Task}%
  \renewcommand{\theoremname}{Theorem}%
  \renewcommand{\definitionname}{Definition}%
  \renewcommand{\answername}{Answer}%
}


\captionsetup{justification=centering,margin=2cm}
\newenvironment{colored}[1]{\color{#1}}{}

\tikzset{->-/.style={decoration={
  markings,
  mark=at position .5 with {\arrow{>}}},postaction={decorate}}}
\makeatletter
\newcommand*{\relrelbarsep}{.386ex}
\newcommand*{\relrelbar}{%
  \mathrel{%
    \mathpalette\@relrelbar\relrelbarsep
  }%
}
\newcommand*{\@relrelbar}[2]{%
  \raise#2\hbox to 0pt{$\m@th#1\relbar$\hss}%
  \lower#2\hbox{$\m@th#1\relbar$}%
}
\providecommand*{\rightrightarrowsfill@}{%
  \arrowfill@\relrelbar\relrelbar\rightrightarrows
}
\providecommand*{\leftleftarrowsfill@}{%
  \arrowfill@\leftleftarrows\relrelbar\relrelbar
}
\providecommand*{\xrightrightarrows}[2][]{%
  \ext@arrow 0359\rightrightarrowsfill@{#1}{#2}%
}
\providecommand*{\xleftleftarrows}[2][]{%
  \ext@arrow 3095\leftleftarrowsfill@{#1}{#2}%
}
\makeatother

\newenvironment{rualgo}[1][]
  {\begin{algorithm}[#1]
     \selectlanguage{russian}%
     \floatname{algorithm}{Алгоритм}%
     \renewcommand{\algorithmicif}{{\color{red}\textbf{если}}}%
     \renewcommand{\algorithmicthen}{{\color{red}\textbf{тогда}}}%
     \renewcommand{\algorithmicelse}{{\color{red}\textbf{иначе}}}%
     \renewcommand{\algorithmicend}{{\color{red}\textbf{конец}}}%
     \renewcommand{\algorithmicfor}{{\color{red}\textbf{для}}}%
     \renewcommand{\algorithmicto}{{\color{red}\textbf{до}}}%
     \renewcommand{\algorithmicdo}{{\color{red}\textbf{делать}}}%
     \renewcommand{\algorithmicwhile}{{\color{red}\textbf{пока}}}%
     \renewcommand{\algorithmicrepeat}{{\color{red}\textbf{повторять}}}%
     \renewcommand{\algorithmicuntil}{{\color{red}\textbf{до тех пор пока}}}%
     \renewcommand{\algorithmicloop}{{\color{red}\textbf{повторять}}}%
     \renewcommand{\algorithmicnot}{{\color{blue}\textbf{не}}}%
     \renewcommand{\algorithmicand}{{\color{blue}\textbf{и}}}%
     \renewcommand{\algorithmicor}{{\color{blue}\textbf{или}}}%
     \renewcommand{\algorithmicrequire}{{\color{blue}\textbf{Ввод}}}%
     \renewcommand{\algorithmicensure}{{\color{blue}\textbf{Вывод}}}%
     \renewcommand{\algorithmicreturn}{{\color{red}\textbf{Вернуть}}}%
     \renewcommand{\algorithmicrtrue}{{\color{blue}\textbf{истинна}}}%
     \renewcommand{\algorithmicrfalse}{{\color{blue}\textbf{ложь}}}%
     % Set other language requirements
  }
  {\end{algorithm}}

\newenvironment{enalgo}[1][]
  {\begin{algorithm}[#1]
     \selectlanguage{english,russian}%
     \floatname{algorithm}{Program}%
     \renewcommand{\algorithmicif}{{\color{red}\textbf{if}}}%
     \renewcommand{\algorithmicthen}{{\color{red}\textbf{then}}}%
     \renewcommand{\algorithmicelse}{{\color{red}\textbf{else}}}%
     \renewcommand{\algorithmicend}{{\color{red}\textbf{end}}}%
     \renewcommand{\algorithmicfor}{{\color{red}\textbf{for}}}%
     \renewcommand{\algorithmicto}{{\color{red}\textbf{to}}}%
     \renewcommand{\algorithmicdo}{{\color{red}\textbf{do}}}%
     \renewcommand{\algorithmicwhile}{{\color{red}\textbf{while}}}%
     \renewcommand{\algorithmicrepeat}{{\color{red}\textbf{repeat}}}%
     \renewcommand{\algorithmicuntil}{{\color{red}\textbf{until}}}%
     \renewcommand{\algorithmicloop}{{\color{red}\textbf{loop}}}%
     \renewcommand{\algorithmicnot}{{\color{blue}\textbf{not}}}%
     \renewcommand{\algorithmicand}{{\color{blue}\textbf{and}}}%
     \renewcommand{\algorithmicor}{{\color{blue}\textbf{or}}}%
     \renewcommand{\algorithmicrequire}{{\color{blue}\textbf{Input}}}%
     \renewcommand{\algorithmicensure}{{\color{blue}\textbf{Output}}}%
     \renewcommand{\algorithmicreturn}{{\color{red}\textbf{return}}}%
     \renewcommand{\algorithmicrtrue}{{\color{blue}\textbf{true}}}%
     \renewcommand{\algorithmicrfalse}{{\color{blue}\textbf{false}}}%
     % Set other language requirements
  }
  {\end{algorithm}}

\pagestyle{fancy}
\fancyhf{}
\fancyhead[R]{\thetitle}
\fancyfoot[L]{ITMO y2019}
\fancyfoot[C]{Page \thepage \hspace{1pt} of \pageref*{LastPage}}
\renewcommand{\footrulewidth}{0.4pt}

\author{Ilya Yaroshevskiy}
\date{\today}
\title{Лекция 11}
\hypersetup{
	pdfauthor={Ilya Yaroshevskiy},
	pdftitle={Лекция 11},
	pdfkeywords={},
	pdfsubject={},
	pdfcreator={Emacs 28.1 (Org mode 9.5.3)},
	pdflang={English}}
\begin{document}

\maketitle
\tableofcontents



\section{Геделева нумерация}
\label{sec:org09a84c6}
\begin{definition}
	\((\gedel{\bullet})\)
	\begin{center}
		\begin{tabular}{l|l}
			\(s\)       & \(\gedel{s}\)                 \\
			\hline
			\((\)       & \(3\)                         \\
			\hline
			\()\)       & \(5\)                         \\
			\hline
			\(,\)       & \(7\)                         \\
			\hline
			\(\&\)      & \(9\)                         \\
			\hline
			\(\vee\)    & \(11\)                        \\
			\hline
			\(\neg\)    & \(13\)                        \\
			\hline
			\(\to\)     & \(15\)                        \\
			\hline
			\(\forall\) & \(17\)                        \\
			\hline
			\(\exists\) & \(19\)                        \\
			\hline
			\(.\)       & \(21\)                        \\
			\hline
			\(f^n_k\)   & \(23 + 6\cdot 2^n \cdot 3^k\) \\
			\hline
			\(P^n_k\)   & \(25 + 6\cdot 2^n\cdot 3^k\)  \\
			\hline
			\(x_k\)     & \(27 + 6\cdot 2^k\)           \\
		\end{tabular}
	\end{center}
	Тогда известные функции будут:
	\begin{itemize}
		\item \((=) = P^2_0\)
		\item \((0) = f^0_0\)
		\item \((+) = f^2_0\)
		\item \((\cdot) = f^2_1\)
		\item \((') = f^1_0\)
	\end{itemize}
	\label{orge38736f}
\end{definition}
\begin{definition}
	\(\gedel{a_0a_1\dots a_{n - 1}} = 2^{\gedel{a_0}}\cdot 3^{\gedel{a_1}} \cdot \dots \cdot p_n^{\gedel{a_{n - 1}}}\)
	\label{org13c4631}
\end{definition}
\begin{definition}
	\(S_0\ S_1\ S_2 = 2^{\gedel{S_0}}\cdot 3^{\gedel{S_1}}\cdot\dots\cdot p_n^{\gedel{S_n}}\)
	\label{orgd571b4a}
\end{definition}
\begin{remark}
	\(p_i\) --- \(i\)-е простое (\(p_1 = 2\))
	\label{org5e34870}
\end{remark}
\begin{examp}
	\(\gedel{a = 0} = 2^{27 + 6}\cdot 3^{25 + 6\cdot 4}\cdot 5^{23 + 6}\)
\end{examp}
\begin{theorem}
	Рассмотрим функцию \(\mathop{\rm Proof}(x, p) = \begin{cases}
		1 & \text{если }p\text{ --- геделев номер доказательства }\chi \\
		0 & \text{иначе}
	\end{cases}\), Proof --- рекурсивна
\end{theorem}
\begin{theorem}
	Если функция представима в формальной арифметике, то она рекурсивна
	\label{orgcdcb4bd}
\end{theorem}
\begin{proof}
	\(f: \N^n \to \N\), т.е. существует формула \(\varphi\) с \(n + 1\) свободными переменными \(x_1, \dots, x_{n + 1}\). Если \(f(k_1, \dots, k_n) = k_{n + 1}\) \\
	\textbf{\uline{Ожидаем}} \(\vdash \varphi(\overline{k_1}, \dots, \overline{k_n}, \overline{k_{n + 1}})\), т.е. существует доказательство \(\delta\) --- последовательность \(\delta_1, \dots, \delta_t\)
	\[ \mathop{\rm Proof}(\gedel{\varphi{\overline{k_1}, \dots, \overline{k_{n + 1}}}}, \gedel{k}) = 1 \]
	\begin{array}{l}
		S\langle{\rm plog}_2,                                                                                                                           \\
		\quad M \langle S \langle {\rm Proof},                                                                                                          \\
		\quad\quad S\pair{{\rm Subst}_{n + 1}, \gedel{\varphi}, P^2_{n + 1}, P^3_{n + 1}, \dots, P^{n + 1}_{n + 1}, S\pair{{\rm plog}_2, P^1_{n + 2}}}, \\
		\quad\quad S \pair{{\rm plog}_3, P^1_{n + 1}}                                                                                                   \\
		\quad \rangle                                                                                                                                   \\
		\rangle
	\end{array} \\
\end{proof}
\begin{remark}
	\({\rm Subst}\) --- функция которая подставляет аргументы в формулу
	\label{org22b65b6}
\end{remark}
\begin{remark}
	\(\chi\) --- формула формальной арифметики
	\[ W_1(\gedel{\chi}, \gedel{p}) = \begin{cases} 0 & \text{если }p\text{ --- доказательство }\chi[x_0\coloneqq\overline{\gedel{\chi}}] \\ 1 & \text{иначе} \end{cases} \]
	Реализация \(W_1\) через Subst очевидна, тогда \(W_1\) представима в формальной арифметике формулой \(\omega_1\).
	\(\sigma(x) = \forall p. \neg \omega_1(x, p)\) --- ``самоприменение \(x\) недоказуемо``
	\[\vdash^? \sigma(\overline{\gedel{\sigma}})\]
	\label{orgab2858e}
\end{remark}
\begin{definition}
	\(\omega\)-непротиворечивость. Теория \(\omega\)-непротиворечива, если для любой формулы \(\varphi(x)\):
	\begin{itemize}
		\item если \(\vdash \varphi(\overline{0}), \vdash \varphi(\overline{1}), \dots\), то \(\not\vdash \exists x. \neg \varphi(x)\)
	\end{itemize}
	\label{org9c2b1b7}
\end{definition}
\begin{lemma}
	Если теория \(\omega\)-непротиворечива, то непротиворечива
	\label{orgc88b6cc}
\end{lemma}
\begin{proof}
	Рассмотрим \(\varphi(x) \coloneqq x = x\)
	\[ \vdash \overline{0} = \overline{0} \quad \vdash \overline{1} = \overline{1} \quad \dots\]
	Т.е. \(\not\vdash \exists x. x\neq x\)
\end{proof}
\begin{theorem}[Геделя о неполноте арифметики №1]
	\-
	\begin{enumerate}
		\item Если формальная арифметика непротиворечива, то \(\not\vdash \sigma(\overline{\gedel{\sigma}})\)
		\item Если формальная арифметика \(\omega\)-непротиворечива, то \(\not\vdash \neg \sigma(\overline{\gedel{\sigma}})\)
	\end{enumerate}
	\label{orgc93cc28}
\end{theorem}
\begin{proof}
	\-
	\begin{enumerate}
		\item Пусть \(\vdash \sigma(\overline{\gedel{\sigma}})\), т.е. существует \(p\) --- геделев номер доказательства
		      \[ \vdash \sigma(\overline{\gedel{\sigma}}) \quad \vdash \forall p. \neg \omega_1(\overline{\gedel{\sigma}}, p) \]
		      С другой стороны, \(W_1(\gedel{\sigma}, p) = 0\), т.е. \(\vdash \omega_1(\overline{\gedel{\sigma}}, p)\)
		\item Пусть \(\vdash \neg \sigma(\overline{\gedel{\sigma}})\)
		      \[ \vdash \exists p. \omega_1(\overline{\gedel{\sigma}}, p) \]
		      \[ \left.\begin{matrix}
				      \vdash \neg \omega_1(\overline{\gedel{\sigma}}, \overline{0}) \\
				      \vdash \neg \omega_1(\overline{\gedel{\sigma}}, \overline{1}) \\
				      \vdash \neg \omega_1(\overline{\gedel{\sigma}}, \overline{2}) \\
				      \vdots
			      \end{matrix}\right\} \text{ иначе} \vdash \sigma(\overline{\gedel{\sigma}})  \]
		      \[ \not\vdash \exists p. \omega_1(\overline{\gedel{\sigma}}, p) \]
	\end{enumerate}
\end{proof}
\begin{corollary}
	Формальная арифметика со стандартной интерпретацией неполна
	\label{org569d202}
\end{corollary}
\begin{proof}
	\todo
\end{proof}
\begin{theorem}[Геделя о неполноте арифметики №1 в форме Россера]
	\[ W_2(x, p) = \begin{cases} 0 & \text{если }p\text{ --- доказательство }\lnot x(\overline{\gedel{x}}) \\ 1 & \text{иначе} \end{cases} \]
	\(\omega_2\) --- формула соответствующая \(W_2\)
	\[ \rho(x) = \forall p. \omega_1(x, p) \to \exists q. q < p \& \omega_2(x, q) \]

	\begin{enumerate}
		\item Если формальная арифметика непротиворечива, то \(\not\vdash \rho(\overline{\gedel{\rho}})\)
		\item Если формальная арифметика непротиворечива, то \(\not\vdash \neg\rho(\overline{\gedel{\rho}})\)
	\end{enumerate}
	\label{org2ef13e8}
\end{theorem}
\todo
\begin{definition}
	\[{\rm Consis} \equiv \forall p. \neg \pi(\overline{\gedel{1 = 0}}, p)\]
	\(\pi\) --- формула соответствующая \(Proof(x, p)\), т.е. \(p\) --- доказательство \(x\)
	\label{orgb1710cc}
\end{definition}

\begin{theorem}[Геделя о неполноте арифметики №2]
	\[ \vdash {\rm Consis} \to \sigma(\overline{\gedel{\sigma}}) \]
	Т.е. если докажем, что если формальная арифметика непротиворечива, то автоматически \(\sigma(\overline{\gedel{\sigma}})\), т.е. ФА противоречива
	\label{orgdd8e44b}
\end{theorem}
\begin{proof}[Схема]
	Прочтем что написано в теореме: \(\sigma(\overline{\gedel{\sigma}})\) раскрывается в \(\forall p. \neg \omega_1(\overline{\gedel{\sigma}}, p)\), т.е. если формальная арифметика непротиворечива, то не существует \(p\), который доказывает \(\sigma(\overline{\gedel{\sigma}})\), а это в точности утверждение теоремы Геделя о неполноте №1. Т.е. эта теорема --- формализация теоремы Геделя о неполноте №1.
	\label{org5dff23a}
\end{proof}
\begin{corollary}
	Никакая теория, содержащая формальную арифметику, не может доказать свою непротиворечивость
	\label{org573ded7}
\end{corollary}
\end{document}
