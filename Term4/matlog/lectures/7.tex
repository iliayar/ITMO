% Created 2021-04-02 Fri 15:05
% Intended LaTeX compiler: pdflatex

\documentclass[english]{article}
\usepackage[T1, T2A]{fontenc}
\usepackage[lutf8]{luainputenc}
\usepackage[english, russian]{babel}
\usepackage{minted}
\usepackage{graphicx}
\usepackage{longtable}
\usepackage{hyperref}
\usepackage{xcolor}
\usepackage{natbib}
\usepackage{amssymb}
\usepackage{stmaryrd}
\usepackage{amsmath}
\usepackage{caption}
\usepackage{mathtools}
\usepackage{amsthm}
\usepackage{tikz}
\usepackage{grffile}
\usepackage{extarrows}
\usepackage{wrapfig}
\usepackage{rotating}
\usepackage{placeins}
\usepackage[normalem]{ulem}
\usepackage{amsmath}
\usepackage{textcomp}
\usepackage{capt-of}

\usepackage{geometry}
\geometry{a4paper,left=2.5cm,top=2cm,right=2.5cm,bottom=2cm,marginparsep=7pt, marginparwidth=.6in}
 \usepackage{hyperref}
 \hypersetup{
     colorlinks=true,
     linkcolor=blue,
     filecolor=orange,
     citecolor=black,      
     urlcolor=cyan,
     }

\usetikzlibrary{decorations.markings}
\usetikzlibrary{cd}
\usetikzlibrary{patterns}
\usetikzlibrary{automata, arrows}

\newcommand\addtag{\refstepcounter{equation}\tag{\theequation}}
\newcommand{\eqrefoffset}[1]{\addtocounter{equation}{-#1}(\arabic{equation}\addtocounter{equation}{#1})}


\newcommand{\R}{\mathbb{R}}
\renewcommand{\C}{\mathbb{C}}
\newcommand{\N}{\mathbb{N}}
\newcommand{\rank}{\text{rank}}
\newcommand{\const}{\text{const}}
\newcommand{\grad}{\text{grad}}

\newcommand{\todo}{{\color{red}\fbox{\text{Доделать}}}}
\newcommand{\fixme}{{\color{red}\fbox{\text{Исправить}}}}

\newcounter{propertycnt}
\setcounter{propertycnt}{1}
\newcommand{\beginproperty}{\setcounter{propertycnt}{1}}

\theoremstyle{plain}
\newtheorem{propertyinner}{Свойство}
\newenvironment{property}{
  \renewcommand\thepropertyinner{\arabic{propertycnt}}
  \propertyinner
}{\endpropertyinner\stepcounter{propertycnt}}
\newtheorem{axiom}{Аксиома}
\newtheorem{lemma}{Лемма}
\newtheorem{manuallemmainner}{Лемма}
\newenvironment{manuallemma}[1]{%
  \renewcommand\themanuallemmainner{#1}%
  \manuallemmainner
}{\endmanuallemmainner}

\theoremstyle{remark}
\newtheorem*{remark}{Примечание}
\newtheorem*{solution}{Решение}
\newtheorem{corollary}{Следствие}[theorem]
\newtheorem*{examp}{Пример}
\newtheorem*{observation}{Наблюдение}

\theoremstyle{definition}
\newtheorem{task}{Задача}
\newtheorem{theorem}{Теорема}[section]
\newtheorem*{definition}{Определение}
\newtheorem*{symb}{Обозначение}
\newtheorem{manualtheoreminner}{Теорема}
\newenvironment{manualtheorem}[1]{%
  \renewcommand\themanualtheoreminner{#1}%
  \manualtheoreminner
}{\endmanualtheoreminner}
\captionsetup{justification=centering,margin=2cm}
\newenvironment{colored}[1]{\color{#1}}{}

\tikzset{->-/.style={decoration={
  markings,
  mark=at position .5 with {\arrow{>}}},postaction={decorate}}}
\makeatletter
\newcommand*{\relrelbarsep}{.386ex}
\newcommand*{\relrelbar}{%
  \mathrel{%
    \mathpalette\@relrelbar\relrelbarsep
  }%
}
\newcommand*{\@relrelbar}[2]{%
  \raise#2\hbox to 0pt{$\m@th#1\relbar$\hss}%
  \lower#2\hbox{$\m@th#1\relbar$}%
}
\providecommand*{\rightrightarrowsfill@}{%
  \arrowfill@\relrelbar\relrelbar\rightrightarrows
}
\providecommand*{\leftleftarrowsfill@}{%
  \arrowfill@\leftleftarrows\relrelbar\relrelbar
}
\providecommand*{\xrightrightarrows}[2][]{%
  \ext@arrow 0359\rightrightarrowsfill@{#1}{#2}%
}
\providecommand*{\xleftleftarrows}[2][]{%
  \ext@arrow 3095\leftleftarrowsfill@{#1}{#2}%
}
\makeatother
\author{Ilya Yaroshevskiy}
\date{\today}
\title{Лекция 7}
\hypersetup{
 pdfauthor={Ilya Yaroshevskiy},
 pdftitle={Лекция 7},
 pdfkeywords={},
 pdfsubject={},
 pdfcreator={Emacs 28.0.50 (Org mode 9.4.4)}, 
 pdflang={English}}
\begin{document}

\maketitle
\tableofcontents

\begin{itemize}
\item \(\Gamma \vDash \alpha\) --- \(\alpha\) следует из \(\Gamma\) при всех оценках, что все \(\gamma \in \Gamma\quad \llbracket \gamma \rrbracket = \text{И}\), выполнено \(\llbracket \alpha \rrbracket = \text{И}\)
\item \(x = 0 \vdash \forall x. x = 0\)
\item \(x = 0 \not\vDash \forall x. x = 0\)
\end{itemize}
\begin{definition}[Условие для корректности]
Правила для кванторов по свободным перменным из \(\Gamma\) запрещены. \\
\uline{Тогда} \(\Gamma \vdash \alpha\) влечет \(\Gamma \vDash \alpha\)
\end{definition}
\section{Полнота исчесления предикатов}
\label{sec:org2dc912a}
\begin{definition}
\(\Gamma\) --- \textbf{непротиворечивое} множество формул, если \(\Gamma \not\vdash \alpha \& \neg \alpha\) ни при каком \(\alpha\)
\end{definition}
\begin{examp}
Непротиворечивые:
\begin{itemize}
\item \(\emptyset\)
\item \(A \vee \neg A\)
\end{itemize}
Противоречивые:
\begin{itemize}
\item \(A \& \neg A\)
\end{itemize}
\end{examp}
\begin{remark}
Непротиворечивое множество замкнутых(не имеющая сводных перменных) бескванторных формул
\end{remark}
\begin{examp}
\(\{A\}, \{0 = 0\}\)
\end{examp}
\begin{definition}
\textbf{Моделью} для непротиворечивого множества замкнутых бескванторных формул \(\Gamma\) --- такая модель, что каждая формула из \(\Gamma\) оценивается в И
\end{definition}
\begin{definition}
Полное непротиворечивое замкнутых бескванторных формул --- такое, что для каждой замкнутой бескванторной формулы \(\alpha\): либо \(\alpha \in \Gamma\), либо \(\neg \alpha \in \Gamma\)
\end{definition}
\begin{symb}
\textbf{з.б.} --- замкнутая бескванторная. \textbf{непр. мн} --- непротиворечивое множество
\end{symb}
\begin{theorem}
Если \(\Gamma\) --- непротиворечивое множество з.б. фомул и \(\alpha\) --- з.б.  формула. \\
То либо \(\Gamma \cup \{\alpha\}\), либо \(\Gamma \cup \{\neg \alpha\}\) --- непр. мн. з.б. формул
\end{theorem}
\begin{proof}
Пусть и \(\Gamma \cup \{\alpha\}\) и \(\Gamma \cup \{\neg \alpha\}\)\todo
\end{proof}
\begin{theorem}
Если \(\Gamma\) --- непр. мн. з.б. фомул, то можно построить \(\Delta\) --- полное непр. мн. з.б. формул. \(\Gamma \subseteq \Delta\) и в языке --- счетное количество формул
\end{theorem}
\(\varphi_1, \varphi_2, \varphi_3, \dots\) --- формулы з.б. \\
\begin{itemize}
\item \(\Gamma_0 = \Gamma\)
\item \(\Gamma_1 = \Gamma_0 \cup \{\varphi_1\}\) либо \(\Gamma_0 \cup \{\neg \varphi_1\}\) --- смотря что непротиворечивое
\item \(\Gamma_2 = \Gamma_1 \cup \{\varphi_2\}\) либо \(\Gamma_1 \cup \{\neg \varphi_2\}\)
\end{itemize}
\[ \Gamma^* = \bigcup_i \Gamma_i \]
\begin{property}
\(\Gamma^*\) --- полное
\end{property}
\begin{property}
\(\Gamma^*\) --- непрерывное
\end{property}
\begin{proof}
Пусть \(\Gamma^* \vdash \beta \& \neg \beta\) \\
Конечное доказательство \(\gamma_1, \dots \gamma_n\), часть из которых гипотезы: \(\gamma_1, \dots, \gamma_k\) \\
\(\gamma_i \in \Gamma_{R_i}\). Возьмем \(\Gamma_{\max{R_i}}\). Правда ли \(\Gamma_{\max{R_i}} \vdash B \& \neg B\)
\end{proof}
\begin{theorem}
Любое полное непротиворечивое множество замкнутых бескванторных формул \(\Gamma\) имеет модель, т.е. существует оценка \(\llbracket \rrbracket\): если \(\gamma \in \Gamma\), то \(\llbracket \gamma \rrbracket = \text{И}\)
\end{theorem}
\begin{proof}
\(D\) --- все записи из функциональных символов.
\begin{itemize}
\item \(\llbracket f_0^n \rrbracket\) --- константа \(\Rightarrow\) \(``f_0^n``\)
\item \(\llbracket f_k^m (\Theta_1, \dots, \Theta_k) \rrbracket\) \(\Rightarrow\) \(``f_k^m(`` + \llbracket \Theta_1 \rrbracket + ``,`` + \dots + ``,`` + \llbracket \Theta_k \rrbracket + ``)``\)
\item \(\llbracket P(\Theta_1, \dots, \Theta_n) \rrbracket = \begin{cases} \text{И} & P(\Theta_1, \dots, \Theta_n) \in \Gamma \\ \text{Л} & \text{иначе} \end{cases}\)
\item свободные переменные: \(\emptyset\)
\end{itemize}
Так построенные модель --- модель для \(\Gamma\). Индукция по количеству связок. \\
\uline{База} очев. \\
\uline{Переход} \(\alpha \& \beta\). При этом
\begin{enumerate}
\item Если \(\alpha, \beta \in \Gamma\) \(\llbracket \alpha \rrbracket = \text{И}\) и \(\llbracket \beta \rrbracket = \text{И}\) то \(\alpha \& \beta \in \Gamma\)
\item Если \(\alpha, \beta \not\in \Gamma\) \(\llbracket \alpha \rrbracket \neq \text{И}\) или \(\llbracket \beta \rrbracket \neq \text{И}\) то \(\alpha \& \beta \not\in \Gamma\)
\end{enumerate}
Аналогично для других операций
\end{proof}
\begin{theorem}[Геделя о полноте]
Если \(\Gamma\) --- полное неротиворечивое множество замкнутых(не бескванторных) фомул, то оно имеет модель
\end{theorem}
\begin{corollary}
Пусть \(\vDash \alpha\), тогда \(\vdash \alpha\)
\end{corollary}
\begin{proof}
Пусть \(\vDash \alpha\), но \(\not\vdash \alpha\). Значит \(\{\neg \alpha\}\) --- непротиворечивое множество замкнутых формул. Тогда \(\{\alpha\}\) или \(\{\neg \alpha\}\) --- непр. мн. з. ф. Пусть \(\{\alpha\}\) --- непр. мн. з.ф., а \(\{\neg \alpha\}\) --- противоречивое. При этом \(\neg \alpha \vdash \beta \& \neg \beta\), \(\neg \alpha \vdash \alpha\), \(\beta \& \neg \beta \vDash \alpha\). \(\neg \alpha \vdash \alpha\), \(\alpha \vdash \alpha\). Значит \(\vdash \alpha\)
\end{proof}
\begin{itemize}
\item \(\Gamma\) --- п.м.з.ф.
\item перестроим \(\Gamma\) в \(\Gamma^\triangle\) --- п.н.м. \textbf{б.} з. ф.
\item по теореме о существование модели: \(M^\triangle\) --- модель для \(F^\triangle\)
\item покажем, что \(M^\triangle\) --- модель для \(\Gamma\) --- \(M\)
\end{itemize}
\(\Gamma_0 = \Gamma\), где все формулы --- в предварительной нормальной форме
\begin{definition}
ПНФ --- формула, где \(\forall \exists \forall \dots(\tau)\), \(\tau\) --- формула без кванторов
\end{definition}
\begin{theorem}
Если \(\varphi\) --- формула, то существует \(\psi\) --- в п.ф., то \(\varphi \to \psi\) и \(\psi \to \varphi\)
\end{theorem}
\begin{proof}
\(\Gamma_0 \subseteq \Gamma_1 \subseteq \Gamma_1 \subseteq \dots \subseteq \Gamma^*\). \(\Gamma^* = \bigcup_i \Gamma_i\) \\
Переход: \(\Gamma_i \to \Gamma_{i + 1}\) \\
Рассмторим: \(\varphi_j \in \Gamma_i\)
\begin{enumerate}
\item \(\varphi_j\) без кванторов --- не трогаем
\item \(\varphi_j \equiv \forall x. \psi\) --- добавим все формулы вида \(\psi[x := \Theta]\), где \(\Theta\) -- терм, состоящий из \(f\): \(d_0^e, d_1^{e'} \dots , d_{i - 1}^{e^{'\dots'}}\)
\item \(\varphi_j \equiv \exists x. \psi\) --- добавим \(\psi[x:=d^j_i]\)
\end{enumerate}
\(\Gamma_{i + 1} = \Gamma_i \cup \{\text{все добавленные формулы}\}\) --- счетное количество
\end{proof}
\begin{theorem}
Если \(\Gamma_i\) --- непротиворечиво, то \(\Gamma_{i + 1}\) --- непротиворечиво
\end{theorem}
\begin{theorem}
\(\Gamma*\) --- непротиворечиво
\end{theorem}
\begin{corollary}
\(\Gamma^\triangle = \Gamma*\) без формул с \(\forall, \exists\)
\end{corollary}
\end{document}
