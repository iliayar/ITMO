% Created 2021-03-12 Fri 17:14
% Intended LaTeX compiler: pdflatex
\documentclass[english]{article}
\usepackage[T1, T2A]{fontenc}
\usepackage[lutf8]{luainputenc}
\usepackage[english, russian]{babel}
\usepackage{minted}
\usepackage{graphicx}
\usepackage{longtable}
\usepackage{hyperref}
\usepackage{xcolor}
\usepackage{natbib}
\usepackage{amssymb}
\usepackage{stmaryrd}
\usepackage{amsmath}
\usepackage{caption}
\usepackage{mathtools}
\usepackage{amsthm}
\usepackage{tikz}
\usepackage{grffile}
\usepackage{extarrows}
\usepackage{wrapfig}
\usepackage{rotating}
\usepackage{placeins}
\usepackage[normalem]{ulem}
\usepackage{amsmath}
\usepackage{textcomp}
\usepackage{capt-of}

\usepackage{geometry}
\geometry{a4paper,left=2.5cm,top=2cm,right=2.5cm,bottom=2cm,marginparsep=7pt, marginparwidth=.6in}

 \usepackage{hyperref}
 \hypersetup{
     colorlinks=true,
     linkcolor=blue,
     filecolor=orange,
     citecolor=black,      
     urlcolor=cyan,
     }

\usetikzlibrary{decorations.markings}
\usetikzlibrary{cd}
\usetikzlibrary{patterns}
\usetikzlibrary{automata, arrows}

\newcommand\addtag{\refstepcounter{equation}\tag{\theequation}}
\newcommand{\eqrefoffset}[1]{\addtocounter{equation}{-#1}(\arabic{equation}\addtocounter{equation}{#1})}


\newcommand{\R}{\mathbb{R}}
\renewcommand{\C}{\mathbb{C}}
\newcommand{\N}{\mathbb{N}}
\newcommand{\rank}{\text{rank}}
\newcommand{\const}{\text{const}}
\newcommand{\grad}{\text{grad}}

\theoremstyle{plain}
\newtheorem{axiom}{Аксиома}
\newtheorem{lemma}{Лемма}
\newtheorem{manuallemmainner}{Лемма}
\newenvironment{manuallemma}[1]{%
  \renewcommand\themanuallemmainner{#1}%
  \manuallemmainner
}{\endmanuallemmainner}

\theoremstyle{remark}
\newtheorem*{remark}{Примечание}
\newtheorem*{solution}{Решение}
\newtheorem{corollary}{Следствие}[theorem]
\newtheorem*{examp}{Пример}
\newtheorem*{observation}{Наблюдение}

\theoremstyle{definition}
\newtheorem{task}{Задача}
\newtheorem{theorem}{Теорема}[section]
\newtheorem*{definition}{Определение}
\newtheorem*{symb}{Обозначение}
\newtheorem{manualtheoreminner}{Теорема}
\newenvironment{manualtheorem}[1]{%
  \renewcommand\themanualtheoreminner{#1}%
  \manualtheoreminner
}{\endmanualtheoreminner}
\captionsetup{justification=centering,margin=2cm}
\newenvironment{colored}[1]{\color{#1}}{}

\tikzset{->-/.style={decoration={
  markings,
  mark=at position .5 with {\arrow{>}}},postaction={decorate}}}
\makeatletter
\newcommand*{\relrelbarsep}{.386ex}
\newcommand*{\relrelbar}{%
  \mathrel{%
    \mathpalette\@relrelbar\relrelbarsep
  }%
}
\newcommand*{\@relrelbar}[2]{%
  \raise#2\hbox to 0pt{$\m@th#1\relbar$\hss}%
  \lower#2\hbox{$\m@th#1\relbar$}%
}
\providecommand*{\rightrightarrowsfill@}{%
  \arrowfill@\relrelbar\relrelbar\rightrightarrows
}
\providecommand*{\leftleftarrowsfill@}{%
  \arrowfill@\leftleftarrows\relrelbar\relrelbar
}
\providecommand*{\xrightrightarrows}[2][]{%
  \ext@arrow 0359\rightrightarrowsfill@{#1}{#2}%
}
\providecommand*{\xleftleftarrows}[2][]{%
  \ext@arrow 3095\leftleftarrowsfill@{#1}{#2}%
}
\makeatother
\author{Ilya Yaroshevskiy}
\date{\today}
\title{Лекция 5}
\hypersetup{
 pdfauthor={Ilya Yaroshevskiy},
 pdftitle={Лекция 5},
 pdfkeywords={},
 pdfsubject={},
 pdfcreator={Emacs 28.0.50 (Org mode )}, 
 pdflang={English}}
\begin{document}

\maketitle
\tableofcontents


\section{Программы}
\label{sec:orgff34711}
программа(функция)
\begin{itemize}
\item \(P: \alpha \to \beta\) --- берет \(\alpha\), возвращает \(\beta\)
\item \(P\) --- доказательство, что из \(\alpha\) следует \(\beta\)
\begin{examp}
\-
\begin{minted}[frame=lines,linenos=true,mathescape]{haskell}
f a = a
\end{minted}
\(f: A \to A\) --- \(f\) доказывает что, из \(A\) следует \(A\)
\end{examp}
\end{itemize}

\begin{center}
\begin{tabular}{ll}
логическок исчесления & Типизированное \(\lambda\)-исчесление\\
\hline
логическая формула & тип\\
доказательство & значение\\
доказуемая формула & обитаемый тип(имеет хотя бы один экземпляр)\\
\(\to\) & функция\\
\& & упорядоченная пара\\
\(\vee\) & алг. тип(тип-сумма)\\
\end{tabular}
\end{center}
\begin{examp}
\(5\) доказывает Int
\end{examp}
\begin{examp}
Список:
\begin{minted}[frame=lines,linenos=true,mathescape]{pascal}
Type list = Record
   Nul: boolean;
   case Nul of
     True  : ;
     False : Next: ^list;
end;
\end{minted}
\begin{minted}[frame=lines,linenos=true,mathescape]{c}
struct list {
	*list next;
};
\end{minted}
Если \texttt{next == NULL} --- то конец
\end{examp}
\begin{examp}
Дерево:
\begin{minted}[frame=lines,linenos=true,mathescape]{c}
struct tree {
	tree* left;
	tree* right;
	int value;
};
\end{minted}
\end{examp}

\begin{definition}
\textbf{Отмеченное(дизъюнктное)} объединение множеств: \\
\begin{itemize}
\item \(A, B\) --- множества
\item \(A \sqcup B = \{\langle ``A``, a \rangle| a\in A\}\cup\{\langle ``B``, a \rangle | b \in B\}\)
\end{itemize}
Пусть \(S \in A \sqcup B\). Мы знаем откуда \(S\)
\end{definition}
\begin{minted}[frame=lines,linenos=true,mathescape]{haskell}
data List a = Nil | Cons a (List a)
example = Cons 1 (Cons 2 (Cons 3 Nil)) -- [1; 2; 3]
\end{minted}
\begin{minted}[frame=lines,linenos=true,mathescape]{c}
union {
	int a;
	char b;
};
\end{minted}
\begin{examp}
\[
\frac{\Gamma \vdash \overset{\text{Nil}}{\alpha} \to \gamma\quad \Gamma \vdash \overset{\text{Cons}}{\beta} \to \gamma\quad \Gamm \vdash \alpha \vee \beta}{\Gamma \vdash \underset{\text{int}}{\gamma}}
\]
\begin{minted}[frame=lines,linenos=true,mathescape]{ocaml}
let rec count l (* $\alpha + \beta$ *) =
match l with
   | Nil (* $\alpha$ *) -> 0 (* $\alpha \to \text{int}$ *)
   | Cons(hd, tl) (* $\beta$ *) -> 1 + count tl (* $\beta \to \text{int}$ *)
\end{minted}
\end{examp}
\subsection{Исчесление предикатов}
\label{sec:orgb89c9a6}
\begin{definition}
Язык исчисление предикатов
\begin{itemize}
\item логические выражения "предикаты"/"формулы"
\item предметные выражния "термы"
\end{itemize}
\(\Theta\) --- метаперменные для термов \\
Термы:
\begin{itemize}
\item Атомы:
\begin{itemize}
\item \(a, b, c, d, \dots\) --- предметные переменные
\item \(x, y, z\) --- метапеременные для предметных перменных
\end{itemize}
\item Функциональные Символы
\begin{itemize}
\item \(f, g, h\) --- Функциональные символы(метаперемнные)
\item \(f(\Theta_1, \dots \Theta_n)\) --- применение функциональных символов
\end{itemize}
\item Логические выражения: \\
\color{gray}Если \(n = 0\), будем писать \(f, g\) --- без скобок\color{black}
\begin{itemize}
\item \(P\) --- метаперменные для предикатных символов
\item \(A, B, C\) --- предикатный символ
\item \(P(\Theta_1, \dots, \Theta_n)\) --- применение предикатных символов
\item \(\&, \vee, \neg, \to\) --- Cвязки
\item \(\forall x.\varphi\) и \(\exists x.\varphi\) --- кванторы \\
\color{gray}"<квантор> <переменная>.<выражение>"\color{black} \\
\end{itemize}
\end{itemize}
\end{definition}
\subsubsection{Сокращение записи}
\label{sec:org05b5db4}
И.В + жадность \(\forall, \exists\) \\
Метавыражение:
\[ \forall x. \color{green}(\color{black}P(x) \& \color{green}(\color{black}\forall y. P(y) \color{green}))\color{black} \]
Квантор съедает все что дают, т.е. имеет минимальный приоритет. \\
Правильный вариант(настоящее выражние):
\[ \forall a. B(A) \& \forall b. B(b) \]
\subsection{Теория моделей}
\label{sec:orgdc0c456}
Оценка формулы в исчислении предикатов:
\begin{enumerate}
\item Фиксируем \(D\) --- предметное множетво
\item Кажодму \(f_i(x_1, \dots, x_n)\) сопоставим функцию \(D^n \to D\)
\item Каждому \(P_j(x_1, \dots, x_m)\) сопоставим функцию(предикат) \(D^2 \to V\)
\item Каждой \(x_i\) сопоставим элемент из \(D\)
\end{enumerate}
\begin{examp}
\[\forall x.\forall y.\ E(x, y)\]
Чтобы определить формулу сначала определим \(D = \N\) 
\[ E(x, y) = \begin{cases}\text{И} & ,x = y \\ \text{Л} &, x\neq y\end{cases} \]
\begin{itemize}
\item \(\llbracket x \rrbracket = f_{x_i}\)
\item \(\llbracket \alpha \star \beta \rrbracket\) --- смотри ИИВ
\item \(\llbracket P_i(\Theta_1, \dots , \Theta_n) \rrbracket = f_{P_i}(\llbracket \Theta_1 \rrbracket, \dots, \llbracket \Theta_n \rrbracket)\)
\item \(\llbracket f_j(\Theta_1 , \dots, \Theta_n ) \rrbracket = f_{f_j}(\llbracket \Theta_1 \rrbracket, \dots, \llbracket \Theta_n \rrbracket)\)
\item \[ \llbracket \forall x. \varphi \rrbracket = \begin{cases} \text{И} & , \text{если } \llbracket \varphi \rrbracket^{f_x = k} = \text{И}\text{ при всех } k \in D  \\ \text{Л} &,\text{иначе}\end{cases} \]
\item \[ \llbracket \exists x.\varphi \rrbracket = \begin{cases} \text{И} &, \text{если } \llbracket \varphi \rrbracket^{f_x = k} = \text{И при некотором } k \in D \\ \text{Л} &,\text{иначе}\]
\end{itemize}
\[ \llbracket \forall x.\forall y.E(x, y) \rrbracket = \text{Л} \]
т.к. \(\llbracket E(x, y) \rrbracket^{x:=1,\ y:=2} = \text{Л}\)
\end{examp}
\newcommand{\colorboxed}[2]{\,\color{#1}\fbox{\color{black}#2}\color{black}\,}

\begin{examp}
\[ \forall \colorboxed{green}{\varepsilon > \colorboxed{blue}{0}}\ \exists N\ \forall \colorboxed{green}{\colorboxed{blue}{n} > \colorboxed{blue}{N}}\quad \colorboxed{green}{\colorboxed{blue}{|a_n - a|} < \colorboxed{blue}{\varepsilon}} \]
Синим отмечены функциональные конструкции(термы), зеленым предикатные
\[ \forall \varepsilon. (\varepsilon > 0) \to \exists N. \forall n. (n > N) \to (|a_n - a| < \varepsilon) \]
Обозначим:
\begin{itemize}
\item \((>)(a, b) = G(a, b)\) --- предикат
\item \(|\bullet|(a) = m_|(a)\)
\item \((-)(a, b) = m_-(a, b)\)
\item \(0() = m_0\)
\item \(a_\bullet(n) = m_a(n)\)
\end{itemize}
\[ \forall e. \colorboxed{green}{G(\colorboxed{blue}{e}, \colorboxed{blue}{m_0})} \to \exists n_0.\forall n. \colorboxed{green}{G(n, n_0)}\to \colorboxed{green}{G\bigg(e, \colorboxed{blue}{m_|\Big(m_- \big(m_a(n), a\big)\Big)}\bigg)} \]
\end{examp}
\subsection{Теория доказательств}
\label{sec:org199681d}
Все аксимомы И.В + M.P.
\begin{description}
\item[{(cхема 11)}] \((\forall x. \varphi) \to \varphi[x:=\Theta]\)
\item[{(схема 12)}] \(\varphi[x:=\Theta]\to \exists x. \varphi\)
\end{description}
Если \(\Theta\) свободен для подстановки вместо \(x\) в \(\varphi\).
\begin{definition}
\textbf{Свободен для подстановки} --- никакое свободное вхождение \(x\) в \(\Theta\) не станет связанным
\end{definition}
\begin{examp}
\-
\begin{minted}[frame=lines,linenos=true,mathescape]{c}
int y;
int f(int x) {
	x = y;
}
\end{minted}
Заменим \texttt{y := x}. Код сломается, т.к. у нас нет свобод для подстановки
\end{examp}
\begin{description}
\item[{(Правило \(\forall\))}] \[\frac{\varphi \to \psi}{\varphi \to \forall x. \psi}\]
\item[{(Правило \(\exists\))}] \[ \frac{\psi \to \varphi}{\exists x.\psi \to \varphi} \]
\end{description}
В обоих правилах \(x\) не входит свободно в \(\varphi\)
\begin{examp}
\[ \frac{x = 5 \to x^2 = 25}{x = 5 \to \forall x. x^2 = 25} \]
Между \(x\) и \(x^2\) была связь, мы ее разрушили. Нарушено ограничение
\end{examp}
\begin{examp}
\[ \exists y. x = y \]
\[ \forall x. \exists y. x = y \to \exists y. y + 1 = y \]
Делаем замену \texttt{x := y+1}. Нарушено требование свобод для подстановки. \(y\) входит в область действия квантора \(\exists\) и поэтому свободная переменная \(x\) стала связанная.
\end{examp}
\end{document}
