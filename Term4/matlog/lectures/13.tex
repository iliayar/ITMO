% Created 2022-06-11 Sat 01:27
% Intended LaTeX compiler: pdflatex

\documentclass[english]{article}
\usepackage[T1, T2A]{fontenc}
\usepackage[lutf8]{luainputenc}
\usepackage[english, russian]{babel}
\usepackage{minted}
\usepackage{graphicx}
\usepackage{longtable}
\usepackage{hyperref}
\usepackage{xcolor}
\usepackage{natbib}
\usepackage{amssymb}
\usepackage{stmaryrd}
\usepackage{amsmath}
\usepackage{caption}
\usepackage{mathtools}
\usepackage{amsthm}
\usepackage{tikz}
\usepackage{fancyhdr}
\usepackage{lastpage}
\usepackage{titling}
\usepackage{grffile}
\usepackage{extarrows}
\usepackage{wrapfig}
\usepackage{algorithm}
\usepackage{algorithmic}
\usepackage{lipsum}
\usepackage{rotating}
\usepackage{placeins}
\usepackage[normalem]{ulem}
\usepackage{amsmath}
\usepackage{textcomp}
\usepackage{svg}
\usepackage{capt-of}
\newcommand{\gedel}[1]{\custombracket{\ulcorner}{\urcorner}{#1}}

\usepackage{geometry}
\geometry{a4paper,left=2.5cm,top=2cm,right=2.5cm,bottom=2cm,marginparsep=7pt, marginparwidth=.6in}
\documentclass[12pt, a4paper]{article}

\usepackage{mathtools}
\usepackage{xltxtra}
\usepackage{libertine}
\usepackage{amsmath}
\usepackage{amsthm}
\usepackage{amsfonts}
\usepackage{amssymb}
\usepackage{enumitem}
\usepackage[left=2.3cm, right=2.3cm, top=2.7cm, bottom=2.7cm, bindingoffset=0cm]{geometry}
\usepackage{fancyhdr}

\pagestyle{fancy}
\lfoot{M3137y2019}
\rhead{\thepage}

\DeclareMathOperator*{\xor}{\oplus}
\DeclareMathOperator*{\R}{\mathbb{R}}
\DeclareMathOperator*{\Q}{\mathbb{Q}}
\DeclareMathOperator*{\C}{\mathbb{C}}
\DeclareMathOperator*{\Z}{\mathbb{Z}}
\DeclareMathOperator*{\N}{\mathbb{N}}

\DeclarePairedDelimiter{\ceil}{\lceil}{\rceil}

\setmainfont{Linux Libertine}

\theoremstyle{plain}
\newtheorem{theorem}{Теорема}
\newtheorem{axiom}{Аксиома}
\newtheorem{lemma}{Лемма}

\theoremstyle{remark}
\newtheorem*{remark}{Примечание}
\newtheorem*{consequence}{Следствие}
\newtheorem*{example}{Пример}

\theoremstyle{definition}
\newtheorem*{definition}{Определение}
\author{Ilya Yaroshevskiy}
\date{\today}
\title{Лекция 13}
\hypersetup{
	pdfauthor={Ilya Yaroshevskiy},
	pdftitle={Лекция 13},
	pdfkeywords={},
	pdfsubject={},
	pdfcreator={Emacs 28.1 (Org mode 9.5.3)},
	pdflang={English}}
\begin{document}

\maketitle
\tableofcontents


\section{Аксиома выбора}
\label{sec:orga534af2}
\begin{axiom*}{\bf Аксиома 8.}
	\begin{itemize}
		\item На любом семействе непустых множеств \(\{A_S\}_{S \in \mathbb{S}}\) можно определить функцию \(f: \mathbb{S} \to \bigcup_{S}A_S\), которая по множеству возвращает его элемент
		\item Любое множество можно вполне упорядочить
		\item Для любой сюрьективной функции \(f: A \to B\), найдется частично обратная \(g: B \to A\), \(g(f(x)) = x\)
	\end{itemize}
\end{axiom*}
\begin{definition}
	\textbf{Дизъюнктное семейство множество} --- семейство непересекающихся множеств
	\[ D(y):\ \forall p.\forall q. p \in y \& q \in y \to p \cap q = \varnothing \]
\end{definition}
\begin{definition}
	\textbf{Прямое произведение} дизъюнктного множества
	\[ \bigtimes S = \{t \big| \forall p. p \in S \leftrightarrow \exists ! c. c \in p \& c \in t\} \]
\end{definition}
\begin{axiom*}{\bf Аксиома 8.}
	Если \(D(y)\& \forall t. t \in y \to t \neq \varnothing\), то \(\bigtimes y \neq \varnothing\)
	\label{orgf16798c}
\end{axiom*}
\begin{theorem}[Диаконеску]
	Рассморим \(ZF\)(аксиоматика Цермело-Френкеля) поверх ИИП. Если добавим аксиому выбора то \(\vdash \alpha \lor \lnot \alpha\)
	\label{org48256dc}
\end{theorem}
\section{Аксиома фундирования}
\label{sec:org42f0eff}
\begin{axiom*}{\bf Аксиома 9.}
	\[ \forall x. x = \varnothing \lor \exists y. y \in x \& y \cap x = \varnothing \]
	В каждом непустом множестве есть элемент, не пересекающийся с ним
\end{axiom*}
\section{Схема аксиом подстановки}
\label{sec:orgdae2d8e}
ZFC --- Zemelo-Frenkel Choice \\
\begin{axiom*}{\bf Аксиома 10.}
	\(S\) --- множество, \(f\) --- функция, то \(f(S)\) --- множество, т.е. существует формула \(\varphi(x, y)\):
	\[\forall x \in S. \exists ! y. \varphi(x, y)\]
\end{axiom*}
\begin{examp}
	\[ f(x) = \begin{cases} {x} & p(x) \\ \varnothing & \neg p(x) \end{cases} \]
	\[ \{x \in S | p(x)\} = \cup f(S) \]
\end{examp}
\section{Мощность множества}
\label{sec:org48f1cd7}
\begin{definition}
	\textbf{Равномощность} \(|a| = |b|\), если существует биекция \(a \to b\)
	\label{orgb146a76}
\end{definition}
\begin{definition}
	\textbf{Строго большая мощность} \(|a| > |b|\), если существует \(f: b \to a\) --- инъекция, но не биекция
	\label{orgf0ca17d}
\end{definition}
\begin{definition}
	\textbf{Кардинальное число} \(t\) --- ординал \(x\): для всех \(y \in x\): \(|y| \neq |x|\)
	\label{org344ccbd}
\end{definition}
\begin{definition}
	Мощность множества \(|x|\) --- такое кардинальное число \(t\), что \(|t| = |x|\)
	\label{org63a54bc}
\end{definition}
\begin{lemma}
	\(a, b\) --- кардиналы и \(|a| = |b|\), то \(a = b\)
\end{lemma}
\begin{remark}
	\-
	\begin{itemize}
		\item \(\overline{0}, \overline{1}, \overline{2}, \overline{3}, \dots\) --- конечные кардиналы
		\item \(\aleph_0 = |\omega|\)
		\item \(\aleph_1\) --- следующий кардинал за \(\aleph_0\)
	\end{itemize}
	\label{orgdcb9b59}
\end{remark}
\begin{theorem}[Кантора]
	Рассмотрим \(S\) --- множетво, \(\mathcal{P}(S)\) --- множество всех подмножеств \\
	\uline{Тогда} \(|\mathcal{P}(S)| > |S|\)
	\label{org5574f0a}
\end{theorem}
\begin{theorem}[Кантора-Бернштейна]
	Если \(a, b\) --- множества, \(f: a \to b\), \(g: b \to a\) --- инъективны \\
	\uline{Тогда} существует биекция \(a \to b\)
	\label{org3ec2ede}
\end{theorem}
\end{document}
