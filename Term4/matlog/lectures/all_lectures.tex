% Created 2021-04-20 Tue 00:26
% Intended LaTeX compiler: pdflatex

\documentclass[oneside]{book}
\usepackage[T1, T2A]{fontenc}
\usepackage[lutf8]{luainputenc}
\usepackage[english, russian]{babel}
\usepackage{minted}
\usepackage{graphicx}
\usepackage{longtable}
\usepackage{hyperref}
\usepackage{xcolor}
\usepackage{natbib}
\usepackage{amssymb}
\usepackage{stmaryrd}
\usepackage{amsmath}
\usepackage{caption}
\usepackage{mathtools}
\usepackage{amsthm}
\usepackage{tikz}
\usepackage{grffile}
\usepackage{extarrows}
\usepackage{wrapfig}
\usepackage{algorithm}
\usepackage{algorithmic}
\usepackage{lipsum}
\usepackage{rotating}
\usepackage{placeins}
\usepackage[normalem]{ulem}
\usepackage{amsmath}
\usepackage{textcomp}
\usepackage{capt-of}
\usepackage{stmaryrd}

\addto\captionsrussian{\renewcommand{\chaptername}{Лекция}}
 \usepackage{hyperref}
 \hypersetup{
     colorlinks=true,
     linkcolor=blue,
     filecolor=orange,
     citecolor=black,      
     urlcolor=cyan,
     }

\usetikzlibrary{decorations.markings}
\usetikzlibrary{cd}
\usetikzlibrary{patterns}
\usetikzlibrary{automata, arrows}

\newcommand\addtag{\refstepcounter{equation}\tag{\theequation}}
\newcommand{\eqrefoffset}[1]{\addtocounter{equation}{-#1}(\arabic{equation}\addtocounter{equation}{#1})}
\newcommand{\llb}{\llbracket}
\newcommand{\rrb}{\rrbracket}


\newcommand{\R}{\mathbb{R}}
\renewcommand{\C}{\mathbb{C}}
\newcommand{\N}{\mathbb{N}}
\newcommand{\A}{\mathfrak{A}}
\newcommand{\B}{\mathfrak{B}}
\newcommand{\rank}{\mathop{\rm rank}\nolimits}
\newcommand{\const}{\var{const}}
\newcommand{\grad}{\mathop{\rm grad}\nolimits}

\newcommand{\todo}{{\color{red}\fbox{\text{Доделать}}}}
\newcommand{\fixme}{{\color{red}\fbox{\text{Исправить}}}}

\newcounter{propertycnt}
\setcounter{propertycnt}{1}
\newcommand{\beginproperty}{\setcounter{propertycnt}{1}}

\theoremstyle{plain}
\newtheorem{propertyinner}{Свойство}
\newenvironment{property}{
  \renewcommand\thepropertyinner{\arabic{propertycnt}}
  \propertyinner
}{\endpropertyinner\stepcounter{propertycnt}}
\newtheorem{axiom}{Аксиома}
\newtheorem{lemma}{Лемма}
\newtheorem{manuallemmainner}{Лемма}
\newenvironment{manuallemma}[1]{%
  \renewcommand\themanuallemmainner{#1}%
  \manuallemmainner
}{\endmanuallemmainner}

\theoremstyle{remark}
\newtheorem*{remark}{Примечание}
\newtheorem*{solution}{Решение}
\newtheorem{corollary}{Следствие}[theorem]
\newtheorem*{examp}{Пример}
\newtheorem*{observation}{Наблюдение}

\theoremstyle{definition}
\newtheorem{task}{Задача}
\newtheorem{theorem}{Теорема}[section]
\newtheorem*{definition}{Определение}
\newtheorem*{symb}{Обозначение}
\newtheorem{manualtheoreminner}{Теорема}
\newenvironment{manualtheorem}[1]{%
  \renewcommand\themanualtheoreminner{#1}%
  \manualtheoreminner
}{\endmanualtheoreminner}
\captionsetup{justification=centering,margin=2cm}
\newenvironment{colored}[1]{\color{#1}}{}

\tikzset{->-/.style={decoration={
  markings,
  mark=at position .5 with {\arrow{>}}},postaction={decorate}}}
\makeatletter
\newcommand*{\relrelbarsep}{.386ex}
\newcommand*{\relrelbar}{%
  \mathrel{%
    \mathpalette\@relrelbar\relrelbarsep
  }%
}
\newcommand*{\@relrelbar}[2]{%
  \raise#2\hbox to 0pt{$\m@th#1\relbar$\hss}%
  \lower#2\hbox{$\m@th#1\relbar$}%
}
\providecommand*{\rightrightarrowsfill@}{%
  \arrowfill@\relrelbar\relrelbar\rightrightarrows
}
\providecommand*{\leftleftarrowsfill@}{%
  \arrowfill@\leftleftarrows\relrelbar\relrelbar
}
\providecommand*{\xrightrightarrows}[2][]{%
  \ext@arrow 0359\rightrightarrowsfill@{#1}{#2}%
}
\providecommand*{\xleftleftarrows}[2][]{%
  \ext@arrow 3095\leftleftarrowsfill@{#1}{#2}%
}
\makeatother

\newenvironment{rualgo}[1][]
  {\begin{algorithm}[#1]
     \selectlanguage{russian}%
     \floatname{algorithm}{Алгоритм}%
     \renewcommand{\algorithmicif}{{\color{red}\textbf{если}}}%
     \renewcommand{\algorithmicthen}{{\color{red}\textbf{тогда}}}%
     \renewcommand{\algorithmicelse}{{\color{red}\textbf{иначе}}}%
     \renewcommand{\algorithmicend}{{\color{red}\textbf{конец}}}%
     \renewcommand{\algorithmicfor}{{\color{red}\textbf{для}}}%
     \renewcommand{\algorithmicto}{{\color{red}\textbf{до}}}%
     \renewcommand{\algorithmicdo}{{\color{red}\textbf{делать}}}%
     \renewcommand{\algorithmicwhile}{{\color{red}\textbf{пока}}}%
     \renewcommand{\algorithmicrepeat}{{\color{red}\textbf{повторять}}}%
     \renewcommand{\algorithmicuntil}{{\color{red}\textbf{до тех пор пока}}}%
     \renewcommand{\algorithmicloop}{{\color{red}\textbf{повторять}}}%
     \renewcommand{\algorithmicnot}{{\color{blue}\textbf{не}}}%
     \renewcommand{\algorithmicand}{{\color{blue}\textbf{и}}}%
     \renewcommand{\algorithmicor}{{\color{blue}\textbf{или}}}%
     \renewcommand{\algorithmicrequire}{{\color{blue}\textbf{Ввод}}}%
     \renewcommand{\algorithmicensure}{{\color{blue}\textbf{Вывод}}}%
     \renewcommand{\algorithmicreturn}{{\color{red}\textbf{Вернуть}}}%
     \renewcommand{\algorithmicrtrue}{{\color{blue}\textbf{истинна}}}%
     \renewcommand{\algorithmicrfalse}{{\color{blue}\textbf{ложь}}}%
     % Set other language requirements
  }
  {\end{algorithm}}
\author{Ilya Yaroshevskiy}
\date{\today}
\title{Лекции по Математической логике 4 семестр}
\hypersetup{
 pdfauthor={Ilya Yaroshevskiy},
 pdftitle={Лекции по Математической логике 4 семестр},
 pdfkeywords={},
 pdfsubject={},
 pdfcreator={Emacs 28.0.50 (Org mode 9.4.4)}, 
 pdflang={English}}
\begin{document}

\maketitle
\tableofcontents


\chapter{}
\label{sec:org5a6a543}
\section{Исчесление высказываний}
\label{sec:org5a8e7d4}
\subsection{Язык}
\label{sec:org8250306}
\begin{enumerate}
\item Пропозициональные переменные \\
\(A'_i\) --- большая буква начала латинского алфавита
\item Связки \\
\(\underbrace{\alpha}_\text{\color{green}метапеременная}, \beta\) --- высказывания \\
Тогда \((\alpha \to \beta),(\alpha \& \beta),(\alpha \vee \beta), (\neg \alpha)\) --- высказывания
\end{enumerate}
\subsection{Мета и предметные}
\label{sec:org9ea26c6}
\begin{itemize}
\item \(\alpha, \beta, \gamma, \dots, \varphi, \psi, \dots\) --- метапеременные для выражений
\item \(X, Y, Z\) --- метапеременные для предметные переменные
\end{itemize}
Метавыражение: \(\alpha \to \beta\) \\
Предметное выражение: \(A \to (A \to A)\) (заменили \(\alpha\) на \(A\), \(\beta\) на \((A \to A)\) )
\begin{examp}
Черным --- предметные выражения, Синим --- метавыражения
\[ (\color{blue}X \color{black}\to\color{blue} Y\color{black})\color{blue}[X \coloneqq A, Y \coloneqq B] \color{black} \equiv A \to B \]
\[ (\color{blue}\alpha \color{black} \to (A \to \color{blue}X \color{black}))\color{blue}[\alpha \coloneqq A, X \coloneqq B] \equiv \color{black} A \to (A \to B) \]
\[ (\color{blue}\alpha \color{black} \to (A \to \color{blue}X \color{black}))\color{blue}[\alpha \coloneqq (A \to P), X \coloneqq B] \equiv \color{black} (A \to P) \to (A \to B) \]
\end{examp}
\subsection{Сокращение записи}
\label{sec:orge161fb3}
\begin{itemize}
\item \(\vee, \&, \neg\) --- скобки слева направо(лево-ассоциативная)
\item \(\to\) --- правоассоциативная
\item Приоритет по возрастанию: \(\to, \vee, \&, \neg\)
\end{itemize}
\begin{examp}
Расставление скобок
\[ \left(A \to \left( \left(B \& C\right) \to D\right)\right) \]
\[ \left(A \to \left(B \to C\right)\right) \]
\end{examp}
\subsection{Теория моделей}
\label{sec:org480a48c}
\begin{itemize}
\item \(\mathcal{P}\) --- множество предметных переменных
\item \(\llb\cdot\rrb: \mathcal{T} \to \V\), где \(\mathcal{T}\) --- множество высказываний, \(V = \{\text{И}, \text{Л}\}\) --- множество истиностных значений
\end{itemize}



\begin{enumerate}
\item \(\llb x \rrb: \mathcal{P} \to V\) --- задается при оценке \\
\(\llb \rrb^{A \coloneqq v_1, B \coloneqq v_2}\):
\begin{itemize}
\item \(\mathcal{P} = v_1\)
\item \(\mathcal{P} = v_2\)
\end{itemize}
\item \(\llb \alpha \star \beta \rrb = \llb \alpha \rrb \color{blue}\underbrace{\star}_{\substack{\text{определенно} \\ \text{ественным образом}}}\color{black} \llb \beta \rrb\), где \(\star \in [\&, \vee, \neg, \to]\)
\begin{examp}
\[ \llb A \to A \rrb^{A \coloneqq \text{И}, B \coloneqq \text{Л}} = \llb A \rrb^{A \coloneqq \text{И}, B \coloneqq \text{Л}} \color{blue}\to\color{black}\llb A \rrb^{A \coloneqq \text{И}, B \coloneqq \text{Л}} = \color{blue} \text{И} \to \text{И} = \text{И} \]
Также можно записать так:
\[ \llb A \to A \rrb^{A \coloneqq \text{И}, B \coloneqq \text{Л}} = f_\to(\llb A \rrb^{A \coloneqq \text{И}, B \coloneqq \text{Л}}, \llb A \rrb^{A \coloneqq \text{И}, B \coloneqq \text{Л}}) = f_\to(\color{blue} \text{И}\color{black} , \color{blue}\text{И}\color{black}) \color{blue}= \text{И} \]
, где \(f_\to\) определена так:
\begin{center}
\begin{tabular}{ll|l}
\(a\) & \(b\) & \(f_\to\)\\
\hline
И & И & И\\
И & Л & Л\\
Л & И & И\\
Л & Л & И\\
\end{tabular}
\end{center}
\end{examp}
\end{enumerate}

\subsection{Теория доказательств}
\label{sec:org6c69fa7}
\begin{definition}
\textbf{Схема высказывания} --- строка соответсвующая определению высказывания, с:
\begin{itemize}
\item метапеременными \(\alpha, \beta, \dots\)
\end{itemize}
\end{definition}
\begin{definition}
Аксиома --- высказывания:
\begin{enumerate}
\item \(\alpha \to (\beta \to \alpha)\)
\item \((\alpha \to \beta) \to (\alpha \to \beta \to \gamma) \to (\alpha \to \gamma)\)
\item \(\alpha \to \beta \to \alpha \& \beta\)
\item \(\alpha \& \beta \to \alpha\)
\item \(\alpha \& \beta \to \beta\)
\item \(\alpha \to \alpha \vee \beta\)
\item \(\beta \to \alpha \vee \beta\)
\item \((\alpha \to \gamma) \to (\beta \to \gamma) \to (\alpha \vee \beta \to \gamma)\)
\item \((\alpha \to \beta) \to (\alpha \to \neg \beta) \to \neg \alpha\)
\item \(\neg\neg \alpha \to \alpha\)
\end{enumerate}
\end{definition}
\#+begin\textsubscript{defintion} org
\subsection{Правило Modus Ponens и доказательство}
\label{sec:org5ac49b6}
\begin{definition}
\textbf{Доказательство} (вывод) --- последовательность высказываний \(\alpha_1, \dots, \alpha_n\), где \(\alpha_i\):
\begin{itemize}
\item аксиома
\item существует \(k, l < i\), что \(\alpha_k = \alpha_l \to \alpha\) \\
\[ \frac{A,\ A \to B}{B} \]
\end{itemize}
\end{definition}
\begin{examp}
\(\vdash A \to A\)
\begin{center}
\begin{tabular}{r|ll}
1 & \(A \to A\ \to A\) & (схема аксиом 1)\\
2 & \(A \to (A \to A) \to A\) & (схема аксиом 1)\\
3 & \((A \to (A \to A)) \to (A \to (A \to A) \to A) \to (A \to A)\) & (схема аксиом 2)\\
4 & \((A \to (A \to A) \to A) \to (A \to A)\) & (M.P. 1 и 3)\\
5 & \(A \to A\) & (M.P. 2 и 4)\\
\end{tabular}
\end{center}
\end{examp}
\begin{definition}
Доказательством высказывания \(\beta\) --- список высказываний \(\alpha_1, \dots, \alpha_n\)
\begin{itemize}
\item \(\alpha_1, \dots, \alpha_n\) --- доказательство
\item \(\alpha_n \equiv \beta\)
\end{itemize}
\end{definition}
\chapter{}
\label{sec:orgc2e0f12}
\begin{symb}
\(\Gamma, \Delta, \Sigma\) --- списки высказываний
\end{symb}
\begin{definition}
Следование: \(\Gamma \vDash \alpha\), если
\begin{itemize}
\item \(\Gamma = \gamma_1, \dots, \gamma_n\)
\item Всегда когда все \(\llb \gamma_i \rrb = \text{И}\), то \(\llb \alpha \rrb = \text{И}\)
\end{itemize}
\end{definition}
\begin{examp}
\(\vDash \alpha\) --- \(\alpha\) общезначимо
\end{examp}
\begin{definition}
\sout{Теория} Исчисление высказываний корректна, если при любом \(\alpha\) из \(\vdash \alpha\) следует \(\vDash \alpha\)
\end{definition}
\begin{definition}
Исчисление полно, если при любом \(\alpha\) из \(\vDash \alpha\) следует \(\vdash \alpha\)
\end{definition}
\begin{theorem}[о дедукции]
\(\Gamma, \alpha \vdash \beta\) \uline{тогда и только тогда, когда} \(\Gamma \vdash \alpha \to \beta\)
\end{theorem}
\begin{proof}
\-
\begin{description}
\item[{\((\Leftarrow)\)}] Пусть \(\Gamma \vdash \alpha \to \beta\). \\
Т.е. существует доказательство \(\delta_1, \dots, \delta_n\), где \(\delta_n = \alpha \to \beta\) \\
Построим новое доказательство: \(\delta_1, \dots, \delta_n, \alpha(\text{гипотеза}), \beta(\text{M.P.})\) \\
Эта новая последовательность --- доказательство \(\Gamma, \alpha \vdash \beta\)
\item[{\((\Rightarrow)\)}] Рассмотрим \(\delta_1, \dots, \delta_n\) --- доказательство \(\Gamma, \alpha \vdash \beta\)
\begin{center}
\begin{tabular}{ll}
\(\sigma_1\) & \(\alpha \to \delta_1\)\\
\(\vdots\) & \(\vdots\)\\
\(\sigma_n\) & \(\alpha \to \delta_n\)\\
\end{tabular}
\end{center}
Утвреждение: последовательность \(\sigma_1, \dots, \sigma_n\) можно дополнить до доказательства, т.е. каждый \(\sigma_i\) --- аксиома, гипотеза или получается по M.P. Докажем по индукции: \\
\uline{База}: \(n = 0\) \\
\uline{Переход}: пусть \(\sigma_0, \dots, \sigma_n\) --- доказательсво. тогда \(\sigma_{n + 1} = \alpha \to \delta_{n + 1}\) по трем вариантам:
\begin{enumerate}
\item \(\delta_{n + 1}\) --- аксиома или гипотеза \(\not\equiv \alpha\)
\item \(\delta_{n + 1} \equiv \alpha\)
\item \(\delta_k \equiv \delta_l \to \delta_{n + 1},\ k,l\le n\)
\end{enumerate}
Докажем каждый из трех вариантов
\begin{enumerate}
\item \-
\begin{center}
\begin{tabular}{l|ll}
(n + 0.2) & \(\delta_{n + 1}\) & (аксиома или гипотеза)\\
(n + 0.4) & \(\deta_{n + 1} \to \alpha \to \delta_{n + 1}\) & (сх. акс. 1)\\
(n + 1) & \(\alpha \to \delta_{n + 1}\) & (M.P. \(n + 0.2, n + 0.4\))\\
\end{tabular}
\end{center}
\item \((n + 0.2, n + 0.4, n+0.6, n+0.8, n+1)\) --- доказательтво \(\alpha \to \alpha\)
\item \-
\begin{center}
\begin{tabular}{lll}
\((k)\) & \(\alpha \to (\sigma_l \to \sigma_{n + 1})\) & \\
\((l)\) & \(\alpha \to \sigma_l\) & \\
\((n + 0.2)\) & \((\alpha \to \delta_l) \to (\alpha \to (\delta_l \to \delta_{n + 1})) \to (\alpha \to \delta_{n + 1})\) & (сх. 2)\\
\((n + 0.4)\) & \((\alpha \to \delta_l \to \delta_{n + 1}) \to (\alpha \to \delta_{n + 1})\) & (M.P. \(n + 0.2, l\))\\
\((n + 1)\) & \(\alpha \to \delta_{n + 1}\) & (M.P. \(n + 0.4, k\))\\
\end{tabular}
\end{center}
\end{enumerate}
\end{description}
\end{proof}
\begin{theorem}[о корректности]
Пусть \(\vdash \alpha\) \\
\uline{Тогда} \(\vDash \alpha\)
\end{theorem}
\begin{proof}
Индукция по длине доказательства: каждая \(\llb \delta_i \rrb = \text{И}\), если \(\delta_1, \dots, \delta_k\) --- доказательство \(\alpha\) \\
Пусть \(\llb \delta_1 \rrb = \text{И}, \dots, \llb \delta_n \rrb = \text{И}\). Тогда осн. \(\delta_{n + 1}\):
\begin{enumerate}
\item \(\delta_{n + 1}\) --- аксиома
\begin{enumerate}
\item \(\delta_{n + 1} \equiv \alpha \to \beta \to \alpha\) (Сущесвуют \(\alpha, \beta\), что) \\
Пусть \(\delta_{n + 1} = A \to B \to A\). Тогда \(\alpha \equiv A, \beta \equiv B\) \\
\(\llb \alpha \to \beta \to \alpha \rrb ^{\llb \alpha \rrb \coloneqq a, \llb \beta \rrb \coloneqq b} = \text{И}\)
\begin{center}
\begin{tabular}{ll|l|l}
\(a\) & \(b\) & \(\beta \to \alpha\) & \(\alpha \to \beta \to \alpha\)\\
\hline
Л & Л & И & И\\
Л & И & Л & И\\
И & Л & И & И\\
И & И & И & И\\
\end{tabular}
\end{center}
\end{enumerate}
\item \(\delta_{n + 1}\) --- M.P. \(\delta_k = \delta_l \to \delta_{n + 1}\) \\
Фиксируем оценку \(\llb \delta_k \rrb = \llb \delta_l \rrb = \text{И}\), тогда \(\llb \delta_l \to \delta_{n + 1} \rrb = \text{И}\)
\begin{center}
\begin{tabular}{lll}
\(\llb \delta_l \rrb\) & \(\llb \delta_{n + 1} \rrb\) & \(\llb \delta_k \rrb = \llb \delta_l \to \delta_{n + 1} \rrb\)\\
\hline
\sout{Л} & \sout{Л} & \sout{И}\\
\sout{Л} & \sout{И} & \sout{И}\\
\sout{И} & \sout{Л} & \sout{Л}\\
И & И & И\\
\end{tabular}
\end{center}
Т.е. \(\llb \delta_{n + 1} \rrb = \text{И}\)
\end{enumerate}
\end{proof}
\begin{theorem}[о полноте]
Пусть \(\vDash \alpha\), тогда \(\vdash \alpha\)
\end{theorem}
\begin{symb}
\[ [\beta]^\alpha \equiv \begin{cases}
\alpha & \llb \beta \rrb = \text{И} \\
\neg \alpha & \llb \beta \rrb = \text{Л} 
\end{cases}\]
\end{symb}
\begin{proof}
Фиксируем набор перменных из \(\alpha\): \(P_1, \dots, P_n\) \\
Рассмотрим \(\llb \alpha \rrb^{P_1 \coloneqq x_1, \dots P_n \coloneqq x_n} = \text{И}\).
Докажем, что \(\underbrace{[x_1]^{P_1},\dots,[x_n]^{P_n}}_\Delta \vdash [\alpha]^\alpha\). \\
\uline{Индукция} по длине формулы (по структуре) \\
\uline{База}: \(\alpha \equiv P_i\) \([P_i]^{P_i} \vdash [P_i]^{P_i}\) \\
\uline{Переход}: пусть \(\eta, \zeta\): \(\Delta \vdash [\eta]^\eta, \Delta \vdash [\zeta]^\zeta\). Покажем, что \(\Delta \vdash [\eta \star \zeta]^{\eta \star \zeta}\), где \(\star\) --- все свзяки \\
Используя \hyperref[org56ceff8]{лемму}: \(\vDash \alpha\), т.е. \([x_1]^{P_1},\dots,[x_n]^{P_n} \vdash [\alpha]^\alpha\). Но \(\llb \alpha \rrb = \text{И}\) при любой оценке, \\
т.е. \([x_1]^{P_1},\dots,[x_n]^{P_n} \vdash \alpha\) при всех \(x_i\) \\
\[ \left.\begin{matrix}
[x_1]^{P_1},\dots,[x_{n - 1}]^{P_{n - 1}}, P_n \vdash \alpha \\
[x_1]^{P_1},\dots,[x_{n - 1}]^{P_{n - 1}}, \neg P_n \vdash \alpha
\end{matrix}\right| \xRightarrow{\text{лемма}} [x_1]^{P_1},\dots,[x_{n - 1}]^{P_{n - 1}} \vdash \alpha\]
\end{proof}
\begin{lemma}
\-
\begin{itemize}
\item \(\Gamma, \eta \vdash \zeta\)
\item \(\Gamma, \neg \eta \dash \zeta\)
\end{itemize}
\uline{Тогда} \(\Gamma \vdash \zeta\)
\label{org56ceff8}
\end{lemma}
\begin{lemma}
\([x_1]^{P_1},\dots,[x_n]^{P_n} \vdash \alpha\), то \([x_1]^{P_1},\dots,[x_{n - 1}]^{P_{n- 1}} \vdash \alpha\)
\label{org60d070e}
\end{lemma}
\section{Интуиционистская логика}
\label{sec:orgb1befea}
\(A \vee B\) --- плохо
\begin{examp}
Докажем: существует \(a, b\), что \(a, b \in \R \setminus \mathbb{Q}\), но \(a^b \in \mathbb{Q}\) \\
Пусть \(a = b = \sqrt{2}\). Рассмотрим \(\sqrt{2}^{\sqrt{2}} \in \R \setminus \mathbb{Q}\)
\begin{itemize}
\item Если да, то ОК
\item Если нет, то возьмем \(a = \sqrt{2}^{\sqrt{2}}, b = \sqrt{2}\), \(a^b = (\sqrt{2}^{\sqrt{2}})^{\sqrt{2}} = \sqrt{2}^{2} = 2\)
\end{itemize}
\end{examp}
\begin{defintion}
ВНК-интерпретация. \(\alpha, \beta\)
\begin{itemize}
\item \(\alpha \& \beta\) --- есть \(\alpha, \beta\)
\item \(\alpha \vee \beta\) --- есть \(\alpha\) либо \(\beta\) и мы знаем какое
\item \(\alpha \to \beta\) --- есть способ перестроить \(\alpha\) в \(\beta\)
\item \(\perp\) --- конструкция без построения \(\neg \alpha \equiv \alpha \to \perp\)
\end{itemize}
\end{defintion}
\chapter{}
\label{sec:org2567763}
\section{Правила вывода}
\label{sec:org777a6e9}
Сверху посылки, снизу заключения
\begin{itemize}
\item Аксиома
\[ \frac{}{\Gamma, \varphi \vdash \varphi} \]
\item Введение \(\to\)
\[ \frac{\Gamma, \varphi \vdash \psi}{\Gamma \vdash \varphi \to \psi} \]
\item Удаление \(\to\)
\[ \frac{\Gamma \vdash \varphi \to \psi\quad \Gamma \vdash \varphi}{\Gamma \vdash \psi} \]
\item Введение \(\&\)
\[ \frac{\Gamma \vdash \varphi \quad \Gamma \vdash \psi}{\Gamma \vdash \varphi \& \psi} \]
\item Удаление \(\&\)
\[ \frac{\Gamma \vdash \varphi \& \psi}{\Gamma \vdash \varphi} \]
\[ \frac{\Gamma \vdash \varphi \& \psi}{\Gamma \vdash \psi} \]
\item Введение \(\vee\)
\[ \frac{\Gamma \vdash \varphi}{\Gamma \vdash \varphi \vee \psi} \]
\[ \frac{\Gamma \vdash \psi}{\Gamma \vdash \varphi \vee \psi} \]
\item Удалние \(\vee\)
\[ \frac{\Gamma, \varphi \vdash \rho \quad \Gamma, \psi \vdash \rho \quad \Gamma \vdash \varphi \vee \psi}{\Gamma \vdash \rho} \]
\item Удаление \(\perp\)
\[ \frac{\Gamma \vdash \perp}{\Gamma \vdash \varphi} \]
\end{itemize}
\begin{examp}
\[ \frac{\displaystyle\frac{}{A \vdash A}(\text{акс.})}{\vdash A \to A}(\text{вв. }\to) \]
\end{examp}
\begin{examp}
Докажем \(\frac{}{\vdash A \& B \to B \& A}\)
\[ \frac{\displaystyle\frac{\displaystyle\frac{\displaystyle\frac{}{A \& B \vdash A \& B}(\text{акс.})}{A\& B \vdash B}(\text{уд. } \&) \quad \frac{\displaystyle\frac{}{A \& B \vdash A \& B}(\text{акс.})}{A \& B \vdash A}(\text{уд. } \&)}{A\&B \vdash B \& A}(\text{вв. } \&)}{\vdash A \& B \to B & A}(\text{вв. }\to) \]
\end{examp}
\begin{definition}
Фиксируем \(A\) \\
Частичный порядок --- антисимметричное, транзитивное, рефлексивное отношение \\
Линейный --- сравнимы любые 2 элемента \\
\begin{itemize}
\item \(a \le b \vee b \le a\)
\item \textbf{Наименьший элемент} \(S\) --- такой \(k \in S\), что если \(x \in S\), то \(k \le x\)
\item \textbf{Минимальный элемент} \(S\) --- такой \(k \in S\), что нет \(x \in S\), что \(x \le k\)
\end{itemize}
\end{definition}
\begin{examp}
\-
\begin{center}
\begin{tikzcd}
\([9, 9 , 9]\) \arrow{d} \arrow{dr} &  & \([1, 2, 1]\) \arrow{dll} \arrow{dl} \arrow{d} \\
\([1, 0 , 0]\) & \([0, 1, 0]\) & \([0, 0, 1]\)
\end{tikzcd}
\end{center}
Нет наименьшего, но есть 3 минимальных. Стрелка из \(a\) в \(b\) обозначает \(b \le a\)
\end{examp}
\begin{definition}
\-
\begin{itemize}
\item \textbf{Множество верхних граней} \(a\) и \(b\): \(\{x \big| a \le x \& b \le x\}\)
\item \textbf{Множество нижних граней} \(a\) и \(b\): \(\{x \big| x \le a \& x \le b\}\)
\end{itemize}
\end{definition}
\begin{definition}
\-
\begin{itemize}
\item \textbf{\(a + b\)} --- нименьший элемент множества верхних граней
\item \textbf{\(a \cdot b\)} --- наибольший элемент множества нижних граней
\end{itemize}
\end{definition}
\begin{definition}
\textbf{Решетка} = \(\langle A, \le \rangle\) --- структура, где для каждых \(a, b\) есть как \(a + b\), так и \(a \cdot b\), \\
т.е. \(a \in A, b \in B \implies a + b \in A\) и \(a \cdot b \in A\)
\end{definition}
\begin{definition}
\textbf{Дистрибутивная решетка} если всегда  \(a \cdot (b + c) = a\codt b + a \cdot c\)
\end{definition}
\begin{lemma}
В дистрибутивной решетке \(a + b\cdot c = (a + b) \cdot(a + c)\)
\end{lemma}
\begin{definition}
\textbf{Псевдодополнение} \(a \to b = \text{наиб.}\{c \big| a \cdot c \le b\}\)
\end{definition}
\begin{definition}
\textbf{Импликативная решетка} --- решетка, где для любых \(a, b\) есть \(a \to b\)
\end{definition}
\begin{definition}
\textbf{0} --- наименьший элемент решетки, \textbf{1} --- наибольший элемент решетки
\end{definition}
\begin{definition}
\textbf{Псевдобулева алгебра (алгебра Гейтинга)} --- импликативная решетка с \(0\)
\end{definition}
\begin{definition}
\textbf{Булева алгебра} --- псевдобулева алгебра, такая что \(a + (a \to 0) = 1\)
\end{definition}
\begin{examp}
\-
\begin{center}
\begin{tikzpicture}
\node (A) at (0, 0) {\(1\)};
\node (B) at (-1, -1) {\(a\)};
\node (C) at (1, -1) {\(b\)};
\node (D) at (0, -2) {\(0\)};
\draw[->] (A) -- (B);
\draw[->] (A) -- (C);
\draw[->] (B) -- (D);
\draw[->] (C) -- (D);
\end{tikzpicture}
\end{center}
\begin{itemize}
\item \(a \cdot 0 = 0\)
\item \(1\cdot b = b\)
\item \(a \cdot b = 0\)
\item \(a + b = 1\)
\item \(a \to b = \text{наиб.}\{x \big| a\cdot x \le b\} = b\) \\
\(\{x \big| a \cdot x \le \} = \{0, b\}\)
\item \(a \to 1 = 1\)
\item \(a \to 0 = 0\)
\end{itemize}
Можем представить в виде пары \(\langle x, y \rangle\)
\begin{itemize}
\item \(a = \langle 1, 0 \rangle\)
\item \(b = \langle 0 , 1\rangle\)
\item \(1 = \langle 1, 1 \rangle\)
\item \(0 = \langle 0, 0 \rangle\)
\end{itemize}
\end{examp}
\begin{lemma}
В импликативной решетке всегда есть \(1\).
\end{lemma}
\begin{theorem}
Любая алгебра Гейтинга --- модель ИИВ
\end{theorem}
\begin{theorem}
Любая булева алгебра --- модель КИВ
\end{theorem}
\begin{defintion}
Рассмотрим множество \(X\) --- \textbf{носитель}. Рассмотрим \(\Omega \subseteq 2^X\) --- подмножество подмножеств \(X\) --- \textbf{топология}.
\begin{enumerate}
\item \(\bigcup X_i \in \Omega\), где \(X_i \in \Omega\)
\item \(X_1 \cap \dots \cap X_n \in \Omega\), если \(X_i \in \Omega\)
\item \(\emptyset, X \in \Omega\)
\end{enumerate}
\end{defintion}
\begin{theorem}
\-
\begin{itemize}
\item \(a + b = a \cup b\)
\item \(a \cdot b = a \cap b\)
\item \(a \to b = \left((X \setminus a) \cup b\right)^\circ\)
\item \(a \le b\) тогда и только тогда, когда \(a \subseteq b\)
\end{itemize}
\uline{Тогда} \(\langle \Omega, \le \rangle\) --- алгебра Гейтинга
\end{theorem}
\begin{definition}
\(X\) --- все формулы логики
\begin{itemize}
\item \(\alpha \le \beta\) --- это \(\alpha \vdash \beta\)
\item \(\alpha \approx \beta\), если \(\alpha \vdash \beta\) и \(\beta \vdash \alpha\)
\item \([\alpha]_\approx = \{\gamma \big| \gamma \approx \alpha\}\) --- класс эквивалентности
\end{itemize}
\end{definition}
\beginproperty
\begin{property}
\(\langle X/_\approx, \le \rangle\) --- алгебра Гейтинга, где \(X/_\approx = \{[\alpha]_\approx \big| \alpha \in X\}\)
\end{property}
\begin{theorem}
Алгебра гейтинга --- полная модель ИИВ
\end{theorem}
\chapter{5 марта}
\label{sec:org6375990}
\renewcommand{\P}{\mathcal{P}}
\newcommand{\A}{\mathcal{A}}
\newcommand{\L}{\mathcal{L}}
\newcommand{\B}{\mathcal{B}}


\begin{definition}
\textbf{Предпорядок} --- транзитивное, рефлексивнре
\end{definition}
\begin{definition}
\textbf{Отношение порядка} (частичный) --- антисимметричное, транзитивное, рефлексивное
\end{definition}
\begin{definition}
\textbf{Линейный порядок} --- порядок в котором \(a \preceq b\) или \(b \preceq a\)
\end{definition}
\begin{definition}
\textbf{Полный порядок} --- линейный, каждое подмножество имеет наименьший элемент. 
\end{definition}
\begin{examp}
\(\N\) --- вполне упорядоченное множество
\end{examp}
\begin{examp}
\(\R\) --- не вполне упорядоченной множество
\begin{itemize}
\item \((0, 1)\) не имееи наименььшего
\item \(\R\) не имеет наименьшего
\end{itemize}
\end{examp}
\section{Табличные модели}
\label{sec:org6188146}
\begin{definition}
Назовем модель \textbf{табличной} для ИИВ:
\begin{itemize}
\item \(V\) --- множество истинностных значений \\
\(f_\to,f_\&, f_V: V^2 \to V\), \(f_\neg: V \to V\) \\
Выделенные значения \(T \in V\) \\
\(\llbracketp+i\rrbracket \in V\) \(f_p : p_i \to V\)
\item \(p_i = f_\P(p_i)\) \\
\(\llbracket\alpha \star \beta\rrbracket = f_\star(\llbracket\alpha\rrbracket, \llbracket\beta\rrbracket)\) \\
\(\llbracket\neg \alpha\rrbracket = f_\neg(\llbracket\alpha\rrbracket)\)
\end{itemize}
\sout{Если \(\vdash \alpha\), то} \(\vDash \alpha\) означает, что \(\llbracket\alpha\rrbracket = T\), при любой \(f_\P\)
\end{definition}
\begin{definition}
Конечная модель: модель где \(V\) --- конечно
\end{definition}
\begin{theorem}
У ИИВ не существует полной табличной модели
\end{theorem}
\section{Модели Крипке}
\label{sec:org5e7a985}
\begin{center}
\begin{tikzpicture}
\node at (0,0) (A) {\( P = NP? \)};
\node at (2, 2) (B) {все банки лопнут, RSA сломают!!!};
\node at (2, -2) (C) {RSA устоит};
\draw[->] (A) -- node[above] {\(+\)} (B);
\draw[->] (A) -- node[below] {\(-\)} (C);
\end{tikzpicture}
\end{center}
\begin{defintion}
\-
\begin{enumerate}
\item \(W = \{W_i\}\) --- множество миров
\item частичный порядок(\(\succeq\))
\item отношение вынужденности: \(W_j \Vdash p_i\) \\
\((\Vdash)  \subseteq W \times \P\) \\
При этом, если \(W_j \Vdash p_i\) и \(W_j \preceq W_k\), то \(W_j \Vdash p\)
\end{enumerate}
\end{defintion}
\begin{definition}
\-
\begin{enumerate}
\item \(W_i \Vdash \alpha\) и \(W_i \Vdash \beta\), тогда (и только тогда) \(W_i \Vdash \alpha \& \beta\) \\
\item \(W_i \Vdash \alpha\) или \(W_i \Vdash \beta\), то \(W_i \Vdash \alpha \vee \beta\)
\item Пусть во всех \(W_i \preceq W_j\) всегда когда \(W_j \Vdash \alpha\) имеет место \(W_j \Vdash \beta\) \\
Тогда \(W_i \Vdash \alpha \to \beta\)
\item \(W_i \Vdash \neg \alpha\) --- \(\alpha\) не вынуждено нигде, начиная с \(W_i\):
\(W_i \preceq W_j\), то \(W_j \not\Vdash \alpha\)
\end{enumerate}
\end{definition}
\begin{theorem}
Если \(W_i \Vdash \alpha\) и \(W_i \preceq W_j\), то \(W_j \Vdash \alpha\)
\end{theorem}
\begin{definition}
Если \(W_i \Vdash \alpha\) при всех \(W_i \in W\), то \(\vDash \alpha\)
\end{definition}
\begin{theorem}
ИИВ корректна в модели Крипке
\end{theorem}
\begin{proof}
\begin{enumerate}
\item \(\langle W, \Omega \rangle\) --- топология, где \(\Omega = \{w \subseteq W | \text{если }W_i \in w,\ W_i \preceq W_j,\text{ то } W_j \in w\}\) \\
\item \(\{W_k | W_k \Vdash p_j\}\) --- открытое множество \\
Примем \(\llbracket p_j \rrbracket = \{W_k | W_k \Vdash p_j\}\) \\
Аналогично \(\llbracket \alpha \rrbracket = \{W_k | W_k \Vdash \alpha\}\)
\end{enumerate}
\end{proof}
\section{Доказательство нетабличности}
\label{sec:org29264a9}
Пусть существует конечная табличная модель \(|V| = n\)
\[ \varphi_n =  \bigvee_{\substack{1 \le i, j \le n + 1 \\ i \neq j}} (p_i \to p_j \&p_j \to p_i)\]
\begin{enumerate}
\item \(\not\vdash\varphi\)
\begin{center}
\begin{tikzpicture}
\node[anchor=west] at (0, 0) (A) {\(W_0\)};
\node[anchor=west] at (1, 2) (B) {\(W_1\)};
\node[anchor=west] at (1, 1) (C) {\(W_2\)};
\node[anchor=west] at (1, 0) (D) {\(\vdots\)};
\node[anchor=west] at (1, -1) (E) {\(W_{n + 1}\)};
\draw[->] (A) -- (B);
\draw[->] (A) -- (C);
\draw[->] (A) -- (E);
\node[anchor=west] at (2, 2) {\(p_1\)};
\node[anchor=west] at (2, 1) {\(p_2\)};
\node[anchor=west] at (2, -1) {\(p_{n + 1}\)};
\end{tikzpicture}
\end{center}
\[ W_1 \not\Vdash (p_i \to p_k)\&(p_k\to p_1),\ k\neq 1 \]
Значит \[ \not\vDash (p_i\to p_j)\&(p_j\to p_i) \]
\[ \not\vDash \bigvee (p_i\to p_j)\&(p_j\to p_i) \]
\[ \not\vdash\varphi_n \]
\item \(\vDash_V \varphi_n\): по признаку Дирихле найдутся \(i\neq j:\llbracket p_i \rrbracket = \llbracket p_j \rrbracket\) \\
\(\llbracket p_i \to p_j \rrbracket = \text{И}\) и \(\llbracket \varphi_n \rrbracket = \text{И}\) \\
Значит \(\vdash \varphi_n\) --- противоречие
\end{enumerate}
\begin{definition}
\textbf{Дизъюнктивность} ИИВ: \(\vdash \alpha \vee \beta\) влечет \(\vdash \alpha\) или \(\vdash \beta\)
\end{definition}
\begin{definition}
Гёделева алгебра --- алгебра Гейтинга, такая что из \(\alpha + \beta = 1\) следует что \(\alpha = 1\) или \(\beta = 1\) \\
\end{definition}
\begin{definition}
Пусть \(\A\) --- алгебра Гейтинга, тогда:
\begin{enumerate}
\item \(\Gamma(\A)\) \\
\begin{center}
\begin{tikzpicture}
\draw (-1, 0) circle[radius=0.5cm] node {\(\A\)};
\draw (1, 0) circle[radius=0.5cm] node {\(\A\)};
\node (0, 0) {\(\Rightarrow\)};
\draw (-1, 0.5) circle[radius=1pt,fill=black] node[above] {\(1\)};
\draw (1, 0.5) circle[radius=1pt,fill=black] node[above right] {\(\omega\)};
\draw (1, 1.5) circle[radius=1pt,fill=black] node[above] {\(1\)};
\draw (1, 1.5) -- (1, 0.5);
\end{tikzpicture}
\end{center}

Добавим новый элемент \(1_{\Gamma(\A)}\) перенеименуем \(1_\A\) в  \(\omega\)
\end{enumerate}
\end{definition}
\begin{theorem}
\-
\begin{itemize}
\item \(\Gamma(\A)\) --- алгебра Гейтинга
\item \(\Gamma(\A)\) --- Геделева
\end{itemize}
\end{theorem}
\begin{definition}
\textbf{Гомоморфизм} алгебр Гейтинга \\
\begin{itemize}
\item \(\varphi: \A \to \B\)
\item \(\varphi(a \star b) = \varphi(a)\star\varphi(b)\)
\item \(\varphi(1_\A) = 1_\B\)
\item \(\varphi(0_\A) = 0_\B\)
\end{itemize}
\end{definition}
\begin{theorem}
\(a \le b\), то \(\varphi(a) \le \varphi(b)\)
\end{theorem}
\begin{definition}
\-
\begin{itemize}
\item \(\alpha\) --- формула ИИВ
\item \(f, g\): оценки ИИВ
\item \(f\): ИИВ \(\to\) \(\A\)
\item \(g\): ИИВ \(\to\) \(\B\)
\end{itemize}
\(\varphi\) согласованы \(f, g\), если \(\varphi(f(\alpha)) = g(\alpha)\)
\end{definition}
\begin{theorem}
если \(\varphi: \A \to \B\) согласована с \(f, g\) и оценка \(\llbracket \alpha \rrbracket_g \neq 1_\B\), то \(\llbracket \alpha \rrbracket_f \neq 1_\A\)
\end{theorem}
\begin{theorem}
ИИВ дизъюнктивно
\end{theorem}
\begin{proof}
Рассмторим алгебру Линденбаума: \(\mathcal{L}\) \\
Рассмотрим \(\Gamma(\mathcal{L})\) \\
\begin{itemize}
\item \(\varphi: \Gamma(\mathcal{L}) \to \mathcal{L}\)
\end{itemize}
\[ \varphi(x) = \begin{cases}1_\mathcal{L} & ,\substack{x =\omega \\ x = 1_{\Gamma(\mathcal{L})}} \\ x & , \text{иначе}\end{cases} \] 
\(\varphi\) --- гомоморфизм \\
Пусть \(\vdash \alpha \vee \beta\), тогда \(\llbracket \alpha \vee \beta \rrbracket_{\Gamma(\mathcal{L})} = 1_{\Gamma(\mathcal{L})}\) \\
\(\llbracket \alpha + \beta \rrbracket = 1\), и т.к. \(\Gamma(\mathcal{L})\) --- Геделева то \(\llbracket \alpha \rrbracket = 1\) или \(\llbracket \beta \rrbracket = 1\) \\
Пусть \(\not \vdash \alpha\) и \(\not \vdash \beta\), тогда \(\varphi(\llbracket \alpha \rrbracket) \neq 1_\mathcal{L}\) и \(\varphi(\llbracket \beta \rrbracket) \neq 1_\mathcal{L}\), т.е. \(\llbracket \alpha \rrbracket_\mathcal{L} \neq 1_\mathcal{L}\) и \(\llbracket \beta \rrbracket_\mathcal{L} \neq 1_\mathcal{L}\), тогда \(\llbracket \alpha \rrbracket_{\Gamma(\mathcal{L})} \neq 1_{\Gamma(\mathcal{L})}\) и \(\llbracket \beta \rrbracket_{\Gamma(\mathcal{L})} \neq 1_{\Gamma(\mathcal{L})}\) \(\Rightarrow\) Противоречие
\end{proof}
\chapter{12 марта}
\label{sec:org0e77adb}
\section{Программы}
\label{sec:org5f9b1c4}
программа(функция)
\begin{itemize}
\item \(P: \alpha \to \beta\) --- берет \(\alpha\), возвращает \(\beta\)
\item \(P\) --- доказательство, что из \(\alpha\) следует \(\beta\)
\begin{examp}
\-
\begin{minted}[frame=lines,linenos=true,mathescape]{haskell}
f a = a
\end{minted}
\(f: A \to A\) --- \(f\) доказывает что, из \(A\) следует \(A\)
\end{examp}
\end{itemize}

\begin{center}
\begin{tabular}{ll}
логическок исчесления & Типизированное \(\lambda\)-исчесление\\
\hline
логическая формула & тип\\
доказательство & значение\\
доказуемая формула & обитаемый тип(имеет хотя бы один экземпляр)\\
\(\to\) & функция\\
\& & упорядоченная пара\\
\(\vee\) & алг. тип(тип-сумма)\\
\end{tabular}
\end{center}
\begin{examp}
\(5\) доказывает Int
\end{examp}
\begin{examp}
Список:
\begin{minted}[frame=lines,linenos=true,mathescape]{pascal}
Type list = Record
   Nul: boolean;
   case Nul of
     True  : ;
     False : Next: ^list;
end;
\end{minted}
\begin{minted}[frame=lines,linenos=true,mathescape]{c}
struct list {
	*list next;
};
\end{minted}
Если \texttt{next == NULL} --- то конец
\end{examp}
\begin{examp}
Дерево:
\begin{minted}[frame=lines,linenos=true,mathescape]{c}
struct tree {
	tree* left;
	tree* right;
	int value;
};
\end{minted}
\end{examp}

\begin{definition}
\textbf{Отмеченное(дизъюнктное)} объединение множеств: \\
\begin{itemize}
\item \(A, B\) --- множества
\item \(A \sqcup B = \{\langle ``A``, a \rangle| a\in A\}\cup\{\langle ``B``, a \rangle | b \in B\}\)
\end{itemize}
Пусть \(S \in A \sqcup B\). Мы знаем откуда \(S\)
\end{definition}
\begin{minted}[frame=lines,linenos=true,mathescape]{haskell}
data List a = Nil | Cons a (List a)
example = Cons 1 (Cons 2 (Cons 3 Nil)) -- [1; 2; 3]
\end{minted}
\begin{minted}[frame=lines,linenos=true,mathescape]{c}
union {
	int a;
	char b;
};
\end{minted}
\begin{examp}
\[
\frac{\Gamma \vdash \overset{\text{Nil}}{\alpha} \to \gamma\quad \Gamma \vdash \overset{\text{Cons}}{\beta} \to \gamma\quad \Gamm \vdash \alpha \vee \beta}{\Gamma \vdash \underset{\text{int}}{\gamma}}
\]
\begin{minted}[frame=lines,linenos=true,mathescape]{ocaml}
let rec count l (* $\alpha + \beta$ *) =
match l with
   | Nil (* $\alpha$ *) -> 0 (* $\alpha \to \text{int}$ *)
   | Cons(hd, tl) (* $\beta$ *) -> 1 + count tl (* $\beta \to \text{int}$ *)
\end{minted}
\end{examp}
\subsection{Исчесление предикатов}
\label{sec:org09257ad}
\begin{definition}
Язык исчисление предикатов
\begin{itemize}
\item логические выражения ``предикаты``/``формулы``
\item предметные выражния ``термы``
\end{itemize}
\(\Theta\) --- метаперменные для термов \\
Термы:
\begin{itemize}
\item Атомы:
\begin{itemize}
\item \(a, b, c, d, \dots\) --- предметные переменные
\item \(x, y, z\) --- метапеременные для предметных перменных
\end{itemize}
\item Функциональные Символы
\begin{itemize}
\item \(f, g, h\) --- Функциональные символы(метаперемнные)
\item \(f(\Theta_1, \dots \Theta_n)\) --- применение функциональных символов
\end{itemize}
\item Логические выражения: \\
\color{gray}Если \(n = 0\), будем писать \(f, g\) --- без скобок\color{black}
\begin{itemize}
\item \(P\) --- метаперменные для предикатных символов
\item \(A, B, C\) --- предикатный символ
\item \(P(\Theta_1, \dots, \Theta_n)\) --- применение предикатных символов
\item \(\&, \vee, \neg, \to\) --- Cвязки
\item \(\forall x.\varphi\) и \(\exists x.\varphi\) --- кванторы \\
\color{gray}``<квантор> <переменная>.<выражение>``\color{black} \\
\end{itemize}
\end{itemize}
\end{definition}
\begin{enumerate}
\item Сокращение записи
\label{sec:orgd26c63c}
И.В + жадность \(\forall, \exists\) \\
Метавыражение:
\[ \forall x. \color{green}(\color{black}P(x) \& \color{green}(\color{black}\forall y. P(y) \color{green}))\color{black} \]
Квантор съедает все что дают, т.е. имеет минимальный приоритет. \\
Правильный вариант(настоящее выражние):
\[ \forall a. B(A) \& \forall b. B(b) \]
\end{enumerate}
\subsection{Теория моделей}
\label{sec:org72a2680}
Оценка формулы в исчислении предикатов:
\begin{enumerate}
\item Фиксируем \(D\) --- предметное множетво
\item Кажодму \(f_i(x_1, \dots, x_n)\) сопоставим функцию \(D^n \to D\)
\item Каждому \(P_j(x_1, \dots, x_m)\) сопоставим функцию(предикат) \(D^2 \to V\)
\item Каждой \(x_i\) сопоставим элемент из \(D\)
\end{enumerate}
\begin{examp}
\[\forall x.\forall y.\ E(x, y)\]
Чтобы определить формулу сначала определим \(D = \N\) 
\[ E(x, y) = \begin{cases}\text{И} & ,x = y \\ \text{Л} &, x\neq y\end{cases} \]
\begin{itemize}
\item \(\llbracket x \rrbracket = f_{x_i}\)
\item \(\llbracket \alpha \star \beta \rrbracket\) --- смотри ИИВ
\item \(\llbracket P_i(\Theta_1, \dots , \Theta_n) \rrbracket = f_{P_i}(\llbracket \Theta_1 \rrbracket, \dots, \llbracket \Theta_n \rrbracket)\)
\item \(\llbracket f_j(\Theta_1 , \dots, \Theta_n ) \rrbracket = f_{f_j}(\llbracket \Theta_1 \rrbracket, \dots, \llbracket \Theta_n \rrbracket)\)
\item \[ \llbracket \forall x. \varphi \rrbracket = \begin{cases} \text{И} & , \text{если } \llbracket \varphi \rrbracket^{f_x = k} = \text{И}\text{ при всех } k \in D  \\ \text{Л} &,\text{иначе}\end{cases} \]
\item \[ \llbracket \exists x.\varphi \rrbracket = \begin{cases} \text{И} &, \text{если } \llbracket \varphi \rrbracket^{f_x = k} = \text{И при некотором } k \in D \\ \text{Л} &,\text{иначе} \end{cases} \]
\end{itemize}
\[ \llbracket \forall x.\forall y.E(x, y) \rrbracket = \text{Л} \]
т.к. \(\llbracket E(x, y) \rrbracket^{x:=1,\ y:=2} = \text{Л}\)
\end{examp}
\newcommand{\colorboxed}[2]{\,\color{#1}\fbox{\color{black}#2}\color{black}\,}

\begin{examp}
\[ \forall \colorboxed{green}{\varepsilon > \colorboxed{blue}{0}}\ \exists N\ \forall \colorboxed{green}{\colorboxed{blue}{n} > \colorboxed{blue}{N}}\quad \colorboxed{green}{\colorboxed{blue}{|a_n - a|} < \colorboxed{blue}{\varepsilon}} \]
Синим отмечены функциональные конструкции(термы), зеленым предикатные
\[ \forall \varepsilon. (\varepsilon > 0) \to \exists N. \forall n. (n > N) \to (|a_n - a| < \varepsilon) \]
Обозначим:
\begin{itemize}
\item \((>)(a, b) = G(a, b)\) --- предикат
\item \(|\bullet|(a) = m_|(a)\)
\item \((-)(a, b) = m_-(a, b)\)
\item \(0() = m_0\)
\item \(a_\bullet(n) = m_a(n)\)
\end{itemize}
\[ \forall e. \colorboxed{green}{G(\colorboxed{blue}{e}, \colorboxed{blue}{m_0})} \to \exists n_0.\forall n. \colorboxed{green}{G(n, n_0)}\to \colorboxed{green}{G\bigg(e, \colorboxed{blue}{m_|\Big(m_- \big(m_a(n), a\big)\Big)}\bigg)} \]
\end{examp}
\subsection{Теория доказательств}
\label{sec:org07da556}
Все аксимомы И.В + M.P.
\begin{description}
\item[{(cхема 11)}] \((\forall x. \varphi) \to \varphi[x:=\Theta]\)
\item[{(схема 12)}] \(\varphi[x:=\Theta]\to \exists x. \varphi\)
\end{description}
Если \(\Theta\) свободен для подстановки вместо \(x\) в \(\varphi\).
\begin{definition}
\textbf{Свободен для подстановки} --- никакое свободное вхождение \(x\) в \(\Theta\) не станет связанным
\end{definition}
\begin{examp}
\-
\begin{minted}[frame=lines,linenos=true,mathescape]{c}
int y;
int f(int x) {
	x = y;
}
\end{minted}
Заменим \texttt{y := x}. Код сломается, т.к. у нас нет свобод для подстановки
\end{examp}
\begin{description}
\item[{(Правило \(\forall\))}] \[\frac{\varphi \to \psi}{\varphi \to \forall x. \psi}\]
\item[{(Правило \(\exists\))}] \[ \frac{\psi \to \varphi}{\exists x.\psi \to \varphi} \]
\end{description}
В обоих правилах \(x\) не входит свободно в \(\varphi\)
\begin{examp}
\[ \frac{x = 5 \to x^2 = 25}{x = 5 \to \forall x. x^2 = 25} \]
Между \(x\) и \(x^2\) была связь, мы ее разрушили. Нарушено ограничение
\end{examp}
\begin{examp}
\[ \exists y. x = y \]
\[ \forall x. \exists y. x = y \to \exists y. y + 1 = y \]
Делаем замену \texttt{x := y+1}. Нарушено требование свобод для подстановки. \(y\) входит в область действия квантора \(\exists\) и поэтому свободная переменная \(x\) стала связанная.
\end{examp}
\chapter{19 марта}
\label{sec:orgefce2dd}
\section{Исчисление предикатов}
\label{sec:orgfe54ac3}
\subsection{Расставление скобок}
\label{sec:org96f45e5}
Кванторы имеют наименьший приоритет
\begin{examp}
\[ \forall x. A \& B \& \forll y. C \& D \vee \exists z. E \]
\[ (\forall x. (A \& B \& \forall y. (C \& D \vee \exists z. (E)))) \]
\end{examp}
Еще раз про правила только со скобками
\begin{enumerate}
\item \[ \frac{\varphi \to \psi}{(\exists. \varphi) \to \psi} \]
\item \[ \frac{\psi \to \varphi}{\psi \to (\forall x. \varphi)} \]
\end{enumerate}
\begin{examp}
\[ \frac{\varphi \to \psi}{\exists x.(\varphi \to \psi)} \]
--- можно доказать, но это не правило вывода для \(\exists\)
\end{examp}
\begin{definition}
\-
\(\alpha_1, \dots, \alpha_n\) --- доказательство
\begin{itemize}
\item если \(\alpha_i\) --- аксимома
\item либо существует \(j, k < i\), что \(\alpha_k = \alpha_j \to \alpha_i\)
\item либо существует \(\alpha_j:\ \alpha_j = \varphi \to \psi\) и \(\alpha_i = (\exists x. \varphi) \to \psi\) причем \(x\) не входит свободно в \(\psi\)
\item либо существует \(j: \alpha_j = \psi \to \varphi\) и \(\alpha_i = \psi \to \forall x. \varphi\) причем \(x\) не входит свободно в \(\psi\)
\end{itemize}
\end{definition}
\subsection{Вхождение}
\label{sec:org1cabf0c}
\begin{examp}
\[ (P(\underset{1}{x}) \vee Q(\underset{2}{x})) \to (R(\underset{3}{x}) \& (\underbrace{\forall \underset{4}{x}. P_1(\underset{5}{x})}_{\text{область }\forall\text{ по }x})) \]
1, 2, 3 --- свободные, 5 --- связанное, по пермененной 4
\end{examp}
\begin{examp}
\[ \underbrace{\forall x. \forall y. \forall x. \forall y. \forall x. P(x)}_{\text{область }\forall\text{ по }x} \]
Здесь \(x\) в \(P(x)\) связано. \(x\) не входит свободно в эту формулу, потому что нет свободных вхождений
\end{examp}
\begin{definition}
Переменная \(x\) входит свободно если существует свободное вхождение
\end{definition}
\begin{definition}
Вхождение свободно, если не связано
\end{definition}
Можно относится к свободно входящим перменным как с перменным из библиотеки, т.е. мы не имеем права их переименовывать
\begin{examp}
Некорректная формула
\begin{description}
\item[{\(\alpha_1\)}] \(x = 0 \to x = 0\)
\item[{\(\alpha_2\)}] \color{red}\((\exists x. x = 0) \to (x = 0)\) --- не доказано\color{black}
\item[{\(\alpha_2'\)}] \((\exists t. x = 0) \to (x = 0)\) --- (правило \(\exists\))
\end{description}
\end{examp}
\begin{examp}
\-
\begin{description}
\item[{\((n)\)}] \(x = 0 \to y = 0\) --- откуда то
\item[{\((n + 1)\)}] \((\exists x. x = 0) \to (y = 0)\) --- (правило \(\exists\))
\end{description}
\end{examp}
\subsection{Свободные подстановки}
\label{sec:org861578f}
\begin{definition}
\(\Theta\) свободен для подстановки вместо \(x\) в \(\varphi\), если никакая свободная перменная в \(\Theta\) не станет связанной в \(\varphi[x := \Theta]\)
\end{definition}
\begin{definition}
\(\varphi[x := \Theta]\) --- "Заменить все свободные вхождения x в \(\varphi\) на \(\Theta\)"
\end{definition}
\begin{examp}
\[ (\forall x. \forall y. \forall x. P(x))[x := y] \equiv \forall x. \forall y. \forall x. P(x) \]
\end{examp}
\begin{examp}
\[ P(x) \vee \forall x. P(x)\ [x := y] \equiv P(y) \vee \forall x. P(y) \]
\end{examp}
\begin{examp}
\[ (\forall y. x = y)\ [x := \underbrace{y}_{\equiv \Theta}] \equiv \forall y. \underset{1}{y} = y\]
\(FV(\Theta) = \{y\}\) --- свободные перменные в \(\Theta\). Вхождение \(y\) с номером 1 стало связанным
\end{examp}
\begin{examp}
\[ P(x) \& \forall y. x = y\ [x := y + z] \equiv P(y + z) \& \forall y. \underset{1}{y} + z = y \]
Здесь при подстановке вхождение \(y\) с номером 1 cтало связанным. \(x\) --- библиотечная функция, переименовали \(x\) во что-то другое.
\end{examp}
\subsection{Пример доказательства}
\label{sec:org2cd84ed}
\begin{lemma}
Пусть \(\vdash \alpha\). Тогда \(\vdash \forall x. \alpha\)
\end{lemma}
\begin{proof}
\-
\begin{enumerate}
\item Т.к. \(\vdash \alpha\), то существует \(\gamma_1, \dots, \gamma_2: \gamma_n = \alpha\)
\[ \begin{matrix}
   (1) & \gamma_1 &  \\
   \vdots & \vdots &  \\
   (n) & \gamma_n (\equiv \alpha) &  \\
   (n + 1) & A\& A \to A & \text{(акс)} \\
   (n + 2) & \alpha \to ((A \& A \to A) \to \alpha) & \text{(акс)} \\
   (n + 3) & (A \& A \to A) \to \alpha & (\text{M.P } n, n + 2) \\
   (n + 4) & (A \& A \to A) \to \forall x.\alpha & (\text{введение }\forall\ n + 3) \\
   (n + 5) & \forall x. \alpha & (\text{M.P. } n + 1, n + 4)
   \end{matrix} \]
\end{enumerate}
\end{proof}
\subsection{Теорема о дедукции}
\label{sec:orgdb5e819}
\begin{theorem}
Пусть задана \(\Gamma,\ \alpha,\beta\)
\begin{enumerate}
\item Если \(\Gamma, \alpha \vdash \beta\), то \(\Gamma \vdash \alpha \to \beta\), при условии, если \(b\) в доказательстве \(\Gamma, \alpha \to \beta\) не применялись правила для \(\forall, \exists\) по перменным, входяшим свободно в \(\alpha\)
\item Если \(\Gamma \vdash \alpha \to \beta\), то \(\Gamma, \alpha \vdash \beta\)
\end{enumerate}
\end{theorem}
\chapter{2 апреля}
\label{sec:org6002804}
\begin{itemize}
\item \(\Gamma \vDash \alpha\) --- \(\alpha\) следует из \(\Gamma\) при всех оценках, что все \(\gamma \in \Gamma\quad \llbracket \gamma \rrbracket = \text{И}\), выполнено \(\llbracket \alpha \rrbracket = \text{И}\)
\item \(x = 0 \vdash \forall x. x = 0\)
\item \(x = 0 \not\vDash \forall x. x = 0\)
\end{itemize}
\begin{definition}[Условие для корректности]
Правила для кванторов по свободным перменным из \(\Gamma\) запрещены. \\
\uline{Тогда} \(\Gamma \vdash \alpha\) влечет \(\Gamma \vDash \alpha\)
\end{definition}
\section{Полнота исчесления предикатов}
\label{sec:org989eafb}
\begin{definition}
\(\Gamma\) --- \textbf{непротиворечивое} множество формул, если \(\Gamma \not\vdash \alpha \& \neg \alpha\) ни при каком \(\alpha\)
\end{definition}
\begin{examp}
Непротиворечивые:
\begin{itemize}
\item \(\emptyset\)
\item \(A \vee \neg A\)
\end{itemize}
Противоречивые:
\begin{itemize}
\item \(A \& \neg A\)
\end{itemize}
\end{examp}
\begin{remark}
Непротиворечивое множество замкнутых(не имеющая сводных перменных) бескванторных формул
\end{remark}
\begin{examp}
\(\{A\}, \{0 = 0\}\)
\end{examp}
\begin{definition}
\textbf{Моделью} для непротиворечивого множества замкнутых бескванторных формул \(\Gamma\) --- такая модель, что каждая формула из \(\Gamma\) оценивается в И
\end{definition}
\begin{definition}
Полное непротиворечивое замкнутых бескванторных формул --- такое, что для каждой замкнутой бескванторной формулы \(\alpha\): либо \(\alpha \in \Gamma\), либо \(\neg \alpha \in \Gamma\)
\end{definition}
\begin{symb}
\textbf{з.б.} --- замкнутая бескванторная. \textbf{непр. мн} --- непротиворечивое множество
\end{symb}
\begin{theorem}
Если \(\Gamma\) --- непротиворечивое множество з.б. фомул и \(\alpha\) --- з.б.  формула. \\
То либо \(\Gamma \cup \{\alpha\}\), либо \(\Gamma \cup \{\neg \alpha\}\) --- непр. мн. з.б. формул
\end{theorem}
\begin{proof}
Пусть и \(\Gamma \cup \{\alpha\}\) и \(\Gamma \cup \{\neg \alpha\}\)\todo
\end{proof}
\begin{theorem}
Если \(\Gamma\) --- непр. мн. з.б. фомул, то можно построить \(\Delta\) --- полное непр. мн. з.б. формул. \(\Gamma \subseteq \Delta\) и в языке --- счетное количество формул
\end{theorem}
\(\varphi_1, \varphi_2, \varphi_3, \dots\) --- формулы з.б. \\
\begin{itemize}
\item \(\Gamma_0 = \Gamma\)
\item \(\Gamma_1 = \Gamma_0 \cup \{\varphi_1\}\) либо \(\Gamma_0 \cup \{\neg \varphi_1\}\) --- смотря что непротиворечивое
\item \(\Gamma_2 = \Gamma_1 \cup \{\varphi_2\}\) либо \(\Gamma_1 \cup \{\neg \varphi_2\}\)
\end{itemize}
\[ \Gamma^* = \bigcup_i \Gamma_i \]
\begin{property}
\(\Gamma^*\) --- полное
\end{property}
\begin{property}
\(\Gamma^*\) --- непрерывное
\end{property}
\begin{proof}
Пусть \(\Gamma^* \vdash \beta \& \neg \beta\) \\
Конечное доказательство \(\gamma_1, \dots \gamma_n\), часть из которых гипотезы: \(\gamma_1, \dots, \gamma_k\) \\
\(\gamma_i \in \Gamma_{R_i}\). Возьмем \(\Gamma_{\max{R_i}}\). Правда ли \(\Gamma_{\max{R_i}} \vdash B \& \neg B\)
\end{proof}
\begin{theorem}
Любое полное непротиворечивое множество замкнутых бескванторных формул \(\Gamma\) имеет модель, т.е. существует оценка \(\llbracket \rrbracket\): если \(\gamma \in \Gamma\), то \(\llbracket \gamma \rrbracket = \text{И}\)
\end{theorem}
\begin{proof}
\(D\) --- все записи из функциональных символов.
\begin{itemize}
\item \(\llbracket f_0^n \rrbracket\) --- константа \(\Rightarrow\) \(``f_0^n``\)
\item \(\llbracket f_k^m (\Theta_1, \dots, \Theta_k) \rrbracket\) \(\Rightarrow\) \(``f_k^m(`` + \llbracket \Theta_1 \rrbracket + ``,`` + \dots + ``,`` + \llbracket \Theta_k \rrbracket + ``)``\)
\item \(\llbracket P(\Theta_1, \dots, \Theta_n) \rrbracket = \begin{cases} \text{И} & P(\Theta_1, \dots, \Theta_n) \in \Gamma \\ \text{Л} & \text{иначе} \end{cases}\)
\item свободные переменные: \(\emptyset\)
\end{itemize}
Так построенные модель --- модель для \(\Gamma\). Индукция по количеству связок. \\
\uline{База} очев. \\
\uline{Переход} \(\alpha \& \beta\). При этом
\begin{enumerate}
\item Если \(\alpha, \beta \in \Gamma\) \(\llbracket \alpha \rrbracket = \text{И}\) и \(\llbracket \beta \rrbracket = \text{И}\) то \(\alpha \& \beta \in \Gamma\)
\item Если \(\alpha, \beta \not\in \Gamma\) \(\llbracket \alpha \rrbracket \neq \text{И}\) или \(\llbracket \beta \rrbracket \neq \text{И}\) то \(\alpha \& \beta \not\in \Gamma\)
\end{enumerate}
Аналогично для других операций
\end{proof}
\begin{theorem}[Геделя о полноте]
Если \(\Gamma\) --- полное неротиворечивое множество замкнутых(не бескванторных) фомул, то оно имеет модель
\end{theorem}
\begin{corollary}
Пусть \(\vDash \alpha\), тогда \(\vdash \alpha\)
\end{corollary}
\begin{proof}
Пусть \(\vDash \alpha\), но \(\not\vdash \alpha\). Значит \(\{\neg \alpha\}\) --- непротиворечивое множество замкнутых формул. Тогда \(\{\alpha\}\) или \(\{\neg \alpha\}\) --- непр. мн. з. ф. Пусть \(\{\alpha\}\) --- непр. мн. з.ф., а \(\{\neg \alpha\}\) --- противоречивое. При этом \(\neg \alpha \vdash \beta \& \neg \beta\), \(\neg \alpha \vdash \alpha\), \(\beta \& \neg \beta \vDash \alpha\). \(\neg \alpha \vdash \alpha\), \(\alpha \vdash \alpha\). Значит \(\vdash \alpha\)
\end{proof}
\begin{itemize}
\item \(\Gamma\) --- п.м.з.ф.
\item перестроим \(\Gamma\) в \(\Gamma^\triangle\) --- п.н.м. \textbf{б.} з. ф.
\item по теореме о существование модели: \(M^\triangle\) --- модель для \(F^\triangle\)
\item покажем, что \(M^\triangle\) --- модель для \(\Gamma\) --- \(M\)
\end{itemize}
\(\Gamma_0 = \Gamma\), где все формулы --- в предварительной нормальной форме
\begin{definition}
ПНФ --- формула, где \(\forall \exists \forall \dots(\tau)\), \(\tau\) --- формула без кванторов
\end{definition}
\begin{theorem}
Если \(\varphi\) --- формула, то существует \(\psi\) --- в п.ф., то \(\varphi \to \psi\) и \(\psi \to \varphi\)
\end{theorem}
\begin{proof}
\(\Gamma_0 \subseteq \Gamma_1 \subseteq \Gamma_1 \subseteq \dots \subseteq \Gamma^*\). \(\Gamma^* = \bigcup_i \Gamma_i\) \\
Переход: \(\Gamma_i \to \Gamma_{i + 1}\) \\
Рассмторим: \(\varphi_j \in \Gamma_i\) \\
Построим семейство ф.с. \(d^j_i\) --- новые перменные
\begin{enumerate}
\item \(\varphi_j\) без кванторов --- не трогаем
\item \(\varphi_j \equiv \forall x. \psi\) --- добавим все формулы вида \(\psi[x := \Theta]\), где \(\Theta\) -- терм, состоящий из \(f\): \(d_0^e, d_1^{e'} \dots , d_{i - 1}^{e^{'\dots'}}\)
\item \(\varphi_j \equiv \exists x. \psi\) --- добавим \(\psi[x:=d^j_i]\)
\end{enumerate}
\(\Gamma_{i + 1} = \Gamma_i \cup \{\text{все добавленные формулы}\}\) --- счетное количество
\end{proof}
\begin{theorem}
Если \(\Gamma_i\) --- непротиворечиво, то \(\Gamma_{i + 1}\) --- непротиворечиво
\end{theorem}
\begin{theorem}
\(\Gamma*\) --- непротиворечиво
\end{theorem}
\begin{corollary}
\(\Gamma^\triangle = \Gamma*\) без формул с \(\forall, \exists\)
\end{corollary}
\chapter{9 апреля}
\label{sec:org82d5d0c}
\section{Исчесление предиктов}
\label{sec:orgd0e3189}
\begin{theorem}[Геделя о полноте ИП]
У любого н.м.з.ф. (непротиворечивого множества замкнутых формул) ИП существует модель
\end{theorem}
\begin{theorem}
Если формула \(\phi\) --- замкнутая формула ИП \\
\uline{Тогда} найдется \(\psi\) --- замкнутая формула ИП, что \(\vdash \varphi \to \psi\) и \(\psi \to \varphi\). \(\psi\) --- с поверхностными кванторами
\end{theorem}
\begin{proof}
См. ДЗ
\end{proof}
\begin{remark}
Рассмотрим \(\Gamma\) --- н.м.з.ф. --- рассмотрим \(\Gamma'\) --- полное расширение \(\Gamma\). Пусть \(\varphi\) --- фомула из \(\Gamma'\), тогда найдется \(\psi in \Gamma'\), что \(\psi\) --- с поверхностными кванторами и \(\vdash \varphi \to \psi\), \(\vdash \psi \to \varphi\)
\end{remark}
\begin{proof}[Доказательство теоремы Геделя о полноте ИП]
Рассмотрим множество констант(нуль местных функциональных символов) --- \(d^i_j\). Построим \(\{\Gamma_j\}:\)
\[ \Gamma' = \Gamma_0 \subseteq \Gamma_1 \subseteq \Gamma2 \subseteq \dots \subseteq \Gamma_j \subseteq \dots \]
Переход \(\Gamma_j \Rightarrow \Gamma_{j + 1}\): рассмторим все формулы из \(\Gamma_j\): \(\{\gamma_1, \gamma_2, \gamma_3, \dots\}\)
\begin{enumerate}
\item \(\gamma_i\) ---  формула без кванторов --- оставим на месте
\item \(\gamma_i \equiv \forall x.\varphi\) --- добваим к \(\Gamma_{j + 1}\) все формулы вида \(\varphi[x:=\Theta]\), где \(\Theta\) --- составлен из всех ф.с. ИП и констант вида \(d_1^k,\dots,d^k_j\)
\item \(\gamma_i \equiv \exists x.\varphi\) --- добавим одну формулу --- \(\varphi[x:=d^i_{j + 1}]\)
\end{enumerate}


\begin{description}
\item[{\textbf{Утв. 1}}] \(Gamma_{i + 1}\) непр., если \(\Gamma_i\) --- непр. \\
Докажем от противного. \(\Gamma_{i + 1} \vdash \beta \& \neg \beta\)
\[ \Gamma_i, \gamma_1, \dots, \gamma_n \vdash \beta \& \neg \beta \quad \gamma_i \in \Gamma_{i + 1} \setminus \Gamma_i \]
\[ \Gamma_i \vdash \gamma_1 \to \gamma_2 \to \dots \to \gamma_n \to \beta \& \neg \beta \]
\(\gamma_i\) --- замкнутое \(\implies\) т. о дедукции. Докажем что \(\Gamma_i \vdash \beta \& \neg \beta\) по индукции.
\[ \Gamma_i \vdash \gamma \to \varepsilon \]
Покажем \(\Gamma_i \vdash \varepsilon\), т.е. \(\gamma\) получен из \(\forall x. \xi\) или \(\forall x. \xi\) \(\in \Gamma_i\)
\begin{description}
\item[{\textbf{\((\forall x. \xi)\)}}] Заметим, что \(\Gamma_i \vdash \forall x.\xi\)
\[ \begin{array}{ll}
    \vdots & \text{по условию} \\
    \gamma \to \varepsilon & \text{по построению }\Gamma_{i + 1} \\
    \forall x.\xi \to (\underbrace{\xi[x:=\Theta]}_\gamma) & \text{(акс. 11)} \\
    (\forall x.\xi) \to \varepsilon & \left|\begin{matrix} \eta \to \xi \\ \xi \to \kappa \end{matrix}\right. \implies \eta \to \kappa\\
    \forall x.\xi & \\
    \varepsilon & \text{(M.P.)}
    \end{array} \]
\item[{\textbf{\((\exists x. \xi)\)}}] \[ \Gamma_i \vdash \overbrace{\xi[x:=d^k_{i + 1}]}^\gamma \to \varepsilon \]
Заметим, что \(d^k_{i + 1}\) не входит в \(\varepsilon\). Заменим все \(d^k_{i + 1}\) в доказательстве на \(y\) --- новая перменная
\[ \Gamma_i \vdash \xi[x:=y] \to \varepsilon \]
\[ \begin{array}{ll}
    \exists y. \xi[x:=y] \to \varepsilon & \\
    (\exists x. \xi x) \to (\exists t. \xi[x:=y]) & \\
    (\exists x.\xi) \to \varepsilon & \\
    \exists x. \xi & \\
    \varespilon & 
    \end{array}\]
\[ \fixme \]
\end{description}
\item[{\textbf{Утв. 2}}] \(\Gamma^*\) --- непр. \(\Gamma_0 \vdash \gamma_1 \to \dots \to \gamma_n \to \beta \& \neg \beta\)
\[ \Gamma_{\max_i(0..n)} \vdash \beta \& \neg \beta \]
Значит \(\Gamma_\max\) --- противоречиво, \(\Gamma^\triangle = \Gamma^*\) без кванторов \\
\uline{Значит} у \(\Gamma^\triangle\) есть модель \(M\)
\item[{\textbf{Утв. 3}}] \(\gamma \in \Gamma'\), то \(\llb \gamma \rrb_M = \text{И}\) \\
Индукция по количеству кванторов в \(\gamma\). Рассмторим:
\begin{enumerate}
\item \(\gamma \equiv \forall x. \delta\) \\
\(\llb \forall x. \delta \rrb\), если \(\llb \delta \rrb^{x := \kappa} = \text{И}, \kappa \in D\). Рассмотри \(\llb \delta \rrb^{x := \kappa},\ k \in D\). \(\kappa\) содержит константы и ф-с, \(\kappa\) осмысленно \(\Gamma_p\). \(\delta\) добавлена на шаге \(q\). Рассмотрим шаг \(\Gamma_{\max(p, q)}\) \(\forall x. \delta: \Gamma_{\max(p, q) + 1}\) добавлена \(\delta[x:=\kappa]\). \(\delta[x:=\kappa]\) --- меньше на 1 квантор, \(\llb \delta[x:=k] \rrb = \text{И}\)
\item \(\gamma \equiv \exists x. \delta\) --- аналогично
\end{enumerate}
\end{description}
\end{proof}

\begin{theorem}
ИП неразрешимо
\end{theorem}
\begin{definition}
\textbf{Язык} --- множество слов. Язык \(\mathcal{L}\) разрешим, если существует \(A\) --- алгоритм, что по слову \(w\): \\
\(A(w)\) --- останавливается в `1`, если \(w \in \mathcal{L}\) и `0`, если \(w \not\in \mathcal{L}\)
\end{definition}
\begin{remark}
Проблема останова: не существует алгоритма, который по программе для машина Тьюринга ответит, остановится она или нет. \\
Пусть \(\mathcal{L}'\) --- язык всех останов программы для машины Тьюринга. \(\mathcal{L}'\) неразрешим
\end{remark}
\begin{remark}
\texttt{[a, b, c, d, e] = cons(a, cons(b, cons(c, cons(d, cons(e, nil)))))} \\
\(A\) --- алфавит ленты
\[ \left.\begin{array}{l}
S_x,\ x \in A \\
e \text{ --- } \text{nil}
\end{array}\right\} \text{ --- } 0\text{-местные функциональные символы}\]
\[ C(a, b) \text{ --- } 2\text{-местные функциональные символы} \]
\(b_s, s \in \mathcal{S}\) --- множество всех состояний, \(b_0\) --- начальное состояние.
\[ C(s_c, C(s_b, C(s_a, e))) \quad C(s_d, C(s_e, e)) \]
Заведем предикат, которых отвечает было ли состояние в процессе. Начальное состояние --- машина Тьюринга запущена на строке \(\alpha\):
\[ R(\alpha, e, b_0) \]
Переход:
\[ (s_x, b_s) \to (s_y, b_t, \leftrightarrow) \]
\[ (s_x, b_s) \to (s_y, b_t, \leftarrow) \]
Если пермещение законно, то можем построить для каждого такие правила:
\[ \forall z. \forall w. R(C(s_x, z), w, b_s) \to R(C(s_y, z), w, b_t) \]
\[ \dots  R(z, C(s_y, w), b_t)\]
Сделаем коньюнкцию вех эти правил: \(R(\dots)\&R(\dots)\&\dots\&R(\dots) \to \exists z. \exists . R(z, w, b_\triangle)\)
\fixme
\end{remark}
\begin{examp}
\-
\begin{enumerate}
\item \(R(C(s_k, e), e, b_0)\) --- доказуемо(мы так сказали)
Двинем голвку вправо:
\[ \forall x. \forall y. R(C(s_k, x), y, b_0) \to R(x, C(s_k, y), b_1) \]
\end{enumerate}
\end{examp}
\chapter{16 апреля}
\label{sec:org382db56}
\section{Теория первого порядка}
\label{sec:orgada62cb}
\begin{definition}
\textbf{Теория I порядка} --- Исчесление предикатов + нелогические функции + предикатные символы + нелогические (математические) аксиомы.
\end{definition}
\begin{definition}
Будем говорить, что \(N\) соответсвует \textbf{аксиоматике Пеано} если:
\begin{itemize}
\item задан \(('): N \to N\) --- инъективная функция (для разных элементов, разные значения)
\item задан \(0 \in N\): нет \(a \in N\), что \(a' = 0\)
\item если \(P(x)\) --- некоторое утверждение, зависящее от \(x \in N\), такое, что \(P(0)\) и всегда, когда \(P(x)\), также и \(P(x')\). Тогда \(P(x)\)
\end{itemize}
\end{definition}
\beginproperty
\begin{property}
\(0\) единственный
\label{org18b68ce}
\end{property}
\begin{proof}
\(P(x)=x = 0\) либо существует \(t:\ t' = x\)
\begin{itemize}
\item \(P(0): 0 = 0\)
\item \(P(x) \to P(x')\). Заметим, что \(x'\) --- не `ноль`
\end{itemize}
\(P(x)\) выполнено при всех \(x \in N\)
\label{org092f906}
\end{proof}
\begin{definition}
\[ a + b = \begin{cases}
a & b = 0 \\
(a + c)' & b = c'
\end{cases}\]
Можем определить это опираясь на \hyperref[org092f906]{доказательтво}
\end{definition}
\begin{definition}
\begin{itemize}
\item \(1 = 0'\)
\item \(2 = 0''\)
\item \(3 = 0'''\)
\item \(4 = 0''''\)
\item \(\dots\)
\end{itemize}
\end{definition}
\begin{task}
\(2 + 2 = 4\)
\end{task}
\begin{solution}
\[ 2 + 2 = 0'' + 0'' = (0'' + 0')' = ((0'' + 0)')' = ((0'')')' = 0'''' = 4 \]
\end{solution}
\begin{definition}
\[ a \cdot b = \begin{cases}
0 & b = 0 \\
(a \cdot c) + a & b = c'
\end{cases}\]
\end{definition}
\begin{definition}
\[ a^b = \begin{cases}
1 & b = 0 \\
(a^c)\cdot a & b = c'
\end{cases}\]
\end{definition}
\beginproperty
\begin{property}
\(a + 0 = 0 + a\)
\label{orgf73d7d9}
\end{property}
\begin{proof}
\(P(a) = (a + 0 = 0 + a)\) \\
\uline{База} \(P(0): 0 + 0 = 0 + 0\) \\
\uline{Переход} \(P(x) \to P(x')\)
\[ x + 0 = 0 + x \]
\[ x' + 0 \xlongequal{?} 0 + x' \]
\[ 0 + x' = (0 + x)' \quad\text{определение }+ \]
\[ (0 + x)' = (x + 0)' \quad\text{предположение} \]
\[ (x + 0)' = x' \quad\text{определение }+\]
\[ x' = x' + 0 \quad\text{определение }+ \]
\end{proof}
\begin{property}
\(a + b' = a' + b\)
\end{property}
\begin{proof}
\-
\begin{description}
\item[{\(b = 0\)}] \(a + 0' = a' + 0\)
\[ a' = (a + 0)' = a + 0' = a'+0 = a' \]
\item[{\(b = c'\)}] Есть: \(a + c' = a' + c\). Покажем: \(a + c'' = a' + c'\)
\[ (a + c')' = (a' + c)' = a' + c \]
\end{description}
\end{proof}
\begin{property}
\(a + b = b + a\)
\end{property}
\begin{proof}
\uline{База} \(b = 0\) --- \hyperref[orgf73d7d9]{свойство} \\
\uline{Переход} \(a + c'' = c'' + a\), если \(a + c' = c' + a\)
\[ a + c'' = (a + c')' = (c' + a)' = c' + a' = c'' + a\]
\end{proof}
\subsection{Формальная арифметика}
\label{sec:org4b14e91}
\begin{definition}
Исчесление предикатов:
\begin{itemize}
\item Функциональные символы:
\begin{itemize}
\item \(0\) --- 0-местный
\item \((')\) --- 1-местный
\item \((\cdot)\) --- 2-местный
\item \((+)\) --- 2-местный
\end{itemize}
\item \((=)\) --- 2-местный предикатный символ
\end{itemize}
Аксимомы:
\begin{enumerate}
\item \(a = b \to a' = b'\)
\item \(a = b \to a = c \to b = c\)
\item \(a' = b' \to a= b\)
\item \(\neg a' = 0\)
\item \(a + b' = (a + b)'\)
\item \(a + 0 = a\)
\item \(a\cdot 0 = 0\)
\item \(a\cdot b' = a\cdot b + a\)
\item Схема аксиом индукции:
\[ (\psi[x:=0])\&(\forall x. \psi \to (\psi[x:=x'])) \to \psi \]
\(x\) входит свободно в \(\psi\)
\end{enumerate}
\end{definition}
\beginproperty
\begin{property}
\[ ((a + 0 = a) \to (a + 0 = a) \to (a = a)) \]
\end{property}
\begin{proof}
\[ \forall a. \forall b. \forall c. a = b \to a = c \to b = c \]
\[ (\forall a. \forall b. \forall c. a = b \to a = c \to b = c) \to \forall b. \forall c. (a + 0 = b \to a + 0 = c \to b = c) \]
\[ \forall b. \forall c. a + 0 = b \to a + 0 = c\to b = c \]
\[ (\forall b. \forall c. a + 0 = b \to a + 0 = c \to b = c) \to \forall c.(a + 0 = a \to a + 0 = c \to a=c) \]
\[ \forall c. a + 0 = a \to a + 0 = c \to a = c \]
\[ (\forall c. a + 0 = a \to a + 0 = c \to a = c) \to a+0 = a \to a + 0 = a \to a= a \]
\[ a + 0  = a \to a + 0 = a \to a = a \]
\[ a + 0 = a \]
\[ a + 0 = a \to a = a \]
\[ a = a \]
\[ \forall b. \forall c. a = b \to a = c \to b = c \]
\[ (0 = 0 \to 0 = 0 \to 0 = 0) \]
\[ (\forall b. \forall c. a = b \to a = c\ to b = c) \to (0 = 0 \to 0 = 0 \to 0 = 0) \to \phi \]
\fixme
\end{proof}
\begin{definition}
\(\exists! x.\varphi(x) \equiv (\exists x. \varphi(x))\&\forall p.\forall q. \varphi(p)\&\varphi(q) \to p = q\) \\
Можно также записать \(\exists ! x.\neg \exists s. s' = x\) или \((\forall q.(\exists x. x' = q)\vee q= 0)\)
\end{definition}
\begin{definition}
\(a \le b\) --- сокращение для \(\exists n. a + n = b\)
\end{definition}
\begin{definition}
\[ \overline{n} = 0^{(n)}\]
\[ 0^{(n)} = \begin{cases}
0 & n = 0 \\
0^{(n - 1)'} & n > 0
\end{cases}\]
\end{definition}
\begin{definition}
\(W \subseteq \N_0^n\). \(W\) --- выразимое в формальной арифметике. отношение, если существует формула \(\omega\) со свободными переменными \(x_1,\dots,x_n\). Пусть \(k_1,\dots,k_n \in \N\)
\begin{itemize}
\item \((k_1,\dots,k_n) \in W\), тогда \(\vdash \omega[x_1:=\overline{k_1}, \dots, x_n := \overline{k_n}]\)
\item \((k_1,\dots,k_n) \not\in W\), тогда \(\vdash \neg \omega[x_1:=\overline{k_1},\dots,x_n:=\overline{k_n}]\)
\end{itemize}
\[ \omega[x_1:=\Theta_1,\dots,x_n:=\Theta_n] \equiv \omega(\Theta_1, \dots, \Theta_n) \]
\end{definition}
\begin{definition}
\(f: \N^n \to \N\) --- представим в формальной арифметике, если найдется \(\varphi\) --- фомула с \(n + 1\) свободными переменными \(k_1, \dots, k_{n + 1} \in \N\)
\begin{itemize}
\item \(f(k_1,\dots,k_n) = k_{n + 1}\), то \(\vdash \varphi(\overline{k_1},\dots,\overline{k_{n + 1}})\) \\
\item \(\vdash \exists! x.\varphi(\overline{k_1},\dots,\overline{k_n},x)\)
\end{itemize}
\end{definition}
\end{document}
