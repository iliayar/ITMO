% Created 2022-06-11 Sat 01:26
% Intended LaTeX compiler: pdflatex

\documentclass[english]{article}
\usepackage[T1, T2A]{fontenc}
\usepackage[lutf8]{luainputenc}
\usepackage[english, russian]{babel}
\usepackage{minted}
\usepackage{graphicx}
\usepackage{longtable}
\usepackage{hyperref}
\usepackage{xcolor}
\usepackage{natbib}
\usepackage{amssymb}
\usepackage{stmaryrd}
\usepackage{amsmath}
\usepackage{caption}
\usepackage{mathtools}
\usepackage{amsthm}
\usepackage{tikz}
\usepackage{fancyhdr}
\usepackage{lastpage}
\usepackage{titling}
\usepackage{grffile}
\usepackage{extarrows}
\usepackage{wrapfig}
\usepackage{algorithm}
\usepackage{algorithmic}
\usepackage{lipsum}
\usepackage{rotating}
\usepackage{placeins}
\usepackage[normalem]{ulem}
\usepackage{amsmath}
\usepackage{textcomp}
\usepackage{svg}
\usepackage{capt-of}

\usepackage{geometry}
\geometry{a4paper,left=2.5cm,top=2cm,right=2.5cm,bottom=2cm,marginparsep=7pt, marginparwidth=.6in}
 \usepackage{hyperref}
 \hypersetup{
     colorlinks=true,
     linkcolor=blue,
     filecolor=orange,
     citecolor=black,      
     urlcolor=cyan,
     }

\usetikzlibrary{decorations.markings}
\usetikzlibrary{cd}
\usetikzlibrary{patterns}
\usetikzlibrary{automata, arrows}

\newcommand\addtag{\refstepcounter{equation}\tag{\theequation}}
\newcommand{\eqrefoffset}[1]{\addtocounter{equation}{-#1}(\arabic{equation}\addtocounter{equation}{#1})}
\newcommand{\llb}{\llbracket}
\newcommand{\rrb}{\rrbracket}


\newcommand{\R}{\mathbb{R}}
\renewcommand{\C}{\mathbb{C}}
\newcommand{\N}{\mathbb{N}}
\newcommand{\A}{\mathfrak{A}}
\newcommand{\B}{\mathfrak{B}}
\newcommand{\rank}{\mathop{\rm rank}\nolimits}
\newcommand{\const}{\var{const}}
\newcommand{\grad}{\mathop{\rm grad}\nolimits}
\newcommand{\custombracket}[3]{\left#1 #3 \right#2}
\newcommand{\custombracketsame}[2]{\custombracket{#1}{#1}{#2}}
\newcommand{\pair}[1]{\custombracket{\langle}{\rangle}{#1}}
\newcommand{\eval}[1]{\custombracket{\llbracket}{\rrbracket}{#1}}

\newcommand{\todo}{{\color{red}\fbox{\text{Доделать}}}}
\newcommand{\fixme}{{\color{red}\fbox{\text{Исправить}}}}

\newcounter{propertycnt}
\setcounter{propertycnt}{1}
\newcommand{\beginproperty}{\setcounter{propertycnt}{1}}

\theoremstyle{plain}
\newtheorem{propertyinner}{\protect\propertyname}
\newenvironment{property}{
  \renewcommand\thepropertyinner{\arabic{propertycnt}}
  \propertyinner
}{\endpropertyinner\stepcounter{propertycnt}}
\newtheorem{axiom}{\protect\axiomname}
\newtheorem{lemma}{\protect\lemmaname}
\newtheorem*{statement}{\protect\statementname}
\newtheorem{manuallemmainner}{\protect\lemmaname}
\newenvironment{manuallemma}[1]{%
  \renewcommand\themanuallemmainner{#1}%
  \manuallemmainner
}{\endmanuallemmainner}

\theoremstyle{remark}
\newtheorem*{remark}{\protect\remarkname}
\newtheorem*{solution}{\protect\solutionname}
\newtheorem{corollary}{\protect\corollaryname}[theorem]
\newtheorem*{examp}{\protect\exampname}
\newtheorem*{observation}{\protect\observationname}

\theoremstyle{definition}
\newtheorem{task}{\protect\taskname}
\newtheorem{theorem}{\protect\theoremname}[section]
\newtheorem*{definition}{\protect\definitionname}
\newtheorem*{symb}{\protect\symbname}
\newtheorem{manualtheoreminner}{\protect\theoremname}
\newenvironment{manualtheorem}[1]{%
  \renewcommand\themanualtheoreminner{#1}%
  \manualtheoreminner
}{\endmanualtheoreminner}

\newtheoremstyle{colon}%
{}
{}
{}%bodyfont
{}%indent
{\bfseries}%headfont
{:}%head punctuation
{ }%space after head
{}
\theoremstyle{colon}
\newtheorem*{answer}{\protect\answername}

\newcommand{\propertyname}{}
\newcommand{\axiomname}{}
\newcommand{\lemmaname}{}
\newcommand{\statementname}{}
\newcommand{\remarkname}{}
\newcommand{\solutionname}{}
\newcommand{\corollaryname}{}
\newcommand{\exampname}{}
\newcommand{\observationname}{}
\newcommand{\taskname}{}
\newcommand{\theoremname}{}
\newcommand{\definitionname}{}
\newcommand{\symbname}{}
\newcommand{\answername}{}
\addto\captionsrussian{%
  \renewcommand{\propertyname}{Свойство}%
  \renewcommand{\axiomname}{Аксиома}%
  \renewcommand{\lemmaname}{Лемма}%
  \renewcommand{\statementname}{Утверждение}%
  \renewcommand{\remarkname}{Замечание}%
  \renewcommand{\solutionname}{Решение}%
  \renewcommand{\corollaryname}{Следствие}%
  \renewcommand{\exampname}{Пример}%
  \renewcommand{\observationname}{Наблюдение}%
  \renewcommand{\taskname}{Задача}%
  \renewcommand{\theoremname}{Теорема}%
  \renewcommand{\definitionname}{Определение}%
  \renewcommand{\symbname}{Обозначение}%
  \renewcommand{\answername}{Ответ}%
}
\addto\captionsenglish{%
  \renewcommand{\propertyname}{Property}%
  \renewcommand{\axiomname}{Axiom}%
  \renewcommand{\lemmaname}{Lemma}%
  \renewcommand{\statementname}{Statement}%
  \renewcommand{\remarkname}{Remark}%
  \renewcommand{\solutionname}{Solution}%
  \renewcommand{\corollaryname}{Corollary}%
  \renewcommand{\exampname}{Example}%
  \renewcommand{\observationname}{Observation}%
  \renewcommand{\taskname}{Task}%
  \renewcommand{\theoremname}{Theorem}%
  \renewcommand{\definitionname}{Definition}%
  \renewcommand{\answername}{Answer}%
}


\captionsetup{justification=centering,margin=2cm}
\newenvironment{colored}[1]{\color{#1}}{}

\tikzset{->-/.style={decoration={
  markings,
  mark=at position .5 with {\arrow{>}}},postaction={decorate}}}
\makeatletter
\newcommand*{\relrelbarsep}{.386ex}
\newcommand*{\relrelbar}{%
  \mathrel{%
    \mathpalette\@relrelbar\relrelbarsep
  }%
}
\newcommand*{\@relrelbar}[2]{%
  \raise#2\hbox to 0pt{$\m@th#1\relbar$\hss}%
  \lower#2\hbox{$\m@th#1\relbar$}%
}
\providecommand*{\rightrightarrowsfill@}{%
  \arrowfill@\relrelbar\relrelbar\rightrightarrows
}
\providecommand*{\leftleftarrowsfill@}{%
  \arrowfill@\leftleftarrows\relrelbar\relrelbar
}
\providecommand*{\xrightrightarrows}[2][]{%
  \ext@arrow 0359\rightrightarrowsfill@{#1}{#2}%
}
\providecommand*{\xleftleftarrows}[2][]{%
  \ext@arrow 3095\leftleftarrowsfill@{#1}{#2}%
}
\makeatother

\newenvironment{rualgo}[1][]
  {\begin{algorithm}[#1]
     \selectlanguage{russian}%
     \floatname{algorithm}{Алгоритм}%
     \renewcommand{\algorithmicif}{{\color{red}\textbf{если}}}%
     \renewcommand{\algorithmicthen}{{\color{red}\textbf{тогда}}}%
     \renewcommand{\algorithmicelse}{{\color{red}\textbf{иначе}}}%
     \renewcommand{\algorithmicend}{{\color{red}\textbf{конец}}}%
     \renewcommand{\algorithmicfor}{{\color{red}\textbf{для}}}%
     \renewcommand{\algorithmicto}{{\color{red}\textbf{до}}}%
     \renewcommand{\algorithmicdo}{{\color{red}\textbf{делать}}}%
     \renewcommand{\algorithmicwhile}{{\color{red}\textbf{пока}}}%
     \renewcommand{\algorithmicrepeat}{{\color{red}\textbf{повторять}}}%
     \renewcommand{\algorithmicuntil}{{\color{red}\textbf{до тех пор пока}}}%
     \renewcommand{\algorithmicloop}{{\color{red}\textbf{повторять}}}%
     \renewcommand{\algorithmicnot}{{\color{blue}\textbf{не}}}%
     \renewcommand{\algorithmicand}{{\color{blue}\textbf{и}}}%
     \renewcommand{\algorithmicor}{{\color{blue}\textbf{или}}}%
     \renewcommand{\algorithmicrequire}{{\color{blue}\textbf{Ввод}}}%
     \renewcommand{\algorithmicensure}{{\color{blue}\textbf{Вывод}}}%
     \renewcommand{\algorithmicreturn}{{\color{red}\textbf{Вернуть}}}%
     \renewcommand{\algorithmicrtrue}{{\color{blue}\textbf{истинна}}}%
     \renewcommand{\algorithmicrfalse}{{\color{blue}\textbf{ложь}}}%
     % Set other language requirements
  }
  {\end{algorithm}}

\newenvironment{enalgo}[1][]
  {\begin{algorithm}[#1]
     \selectlanguage{english,russian}%
     \floatname{algorithm}{Program}%
     \renewcommand{\algorithmicif}{{\color{red}\textbf{if}}}%
     \renewcommand{\algorithmicthen}{{\color{red}\textbf{then}}}%
     \renewcommand{\algorithmicelse}{{\color{red}\textbf{else}}}%
     \renewcommand{\algorithmicend}{{\color{red}\textbf{end}}}%
     \renewcommand{\algorithmicfor}{{\color{red}\textbf{for}}}%
     \renewcommand{\algorithmicto}{{\color{red}\textbf{to}}}%
     \renewcommand{\algorithmicdo}{{\color{red}\textbf{do}}}%
     \renewcommand{\algorithmicwhile}{{\color{red}\textbf{while}}}%
     \renewcommand{\algorithmicrepeat}{{\color{red}\textbf{repeat}}}%
     \renewcommand{\algorithmicuntil}{{\color{red}\textbf{until}}}%
     \renewcommand{\algorithmicloop}{{\color{red}\textbf{loop}}}%
     \renewcommand{\algorithmicnot}{{\color{blue}\textbf{not}}}%
     \renewcommand{\algorithmicand}{{\color{blue}\textbf{and}}}%
     \renewcommand{\algorithmicor}{{\color{blue}\textbf{or}}}%
     \renewcommand{\algorithmicrequire}{{\color{blue}\textbf{Input}}}%
     \renewcommand{\algorithmicensure}{{\color{blue}\textbf{Output}}}%
     \renewcommand{\algorithmicreturn}{{\color{red}\textbf{return}}}%
     \renewcommand{\algorithmicrtrue}{{\color{blue}\textbf{true}}}%
     \renewcommand{\algorithmicrfalse}{{\color{blue}\textbf{false}}}%
     % Set other language requirements
  }
  {\end{algorithm}}

\pagestyle{fancy}
\fancyhf{}
\fancyhead[R]{\thetitle}
\fancyfoot[L]{ITMO y2019}
\fancyfoot[C]{Page \thepage \hspace{1pt} of \pageref*{LastPage}}
\renewcommand{\footrulewidth}{0.4pt}

\author{Ilya Yaroshevskiy}
\date{\today}
\title{Коллоквиум 1}
\hypersetup{
	pdfauthor={Ilya Yaroshevskiy},
	pdftitle={Коллоквиум 1},
	pdfkeywords={},
	pdfsubject={},
	pdfcreator={Emacs 28.1 (Org mode 9.5.3)},
	pdflang={English}}
\begin{document}

\maketitle
\tableofcontents


\section{Топология}
\label{sec:org92a76be}
\subsection{Топологическое пространство, открытое и замкнутое множество}
\label{sec:orgcbeb70d}
\begin{definition}
	Рассмотрим множество \(X\) --- \textbf{носитель}. Рассмотрим \(\Omega \subseteq 2^X\) --- подмножество подмножеств \(X\) --- \textbf{топология}.
	\begin{enumerate}
		\item \(\bigcup X_i \in \Omega\), где \(X_i \in \Omega\)
		\item \(X_1 \cap \dots \cap X_n \in \Omega\), если \(X_i \in \Omega\)
		\item \(\emptyset, X \in \Omega\)
	\end{enumerate}
	\label{orgcbc984d}
\end{definition}
\subsection{Внутренность и замыкание множества}
\label{sec:orga8717b9}
\begin{definition}
	\[ (X)^\circ = \text{наиб.}\{w \big| w \subseteq X, w\text{ --- откр.} \} \]
	\label{org43706fc}
\end{definition}
\begin{definition}
	\textbf{Замыкание} X --- \(\overline{X} = \text{наим.}\{A \not\in \Omega \big| X \subseteq A \}\)
\end{definition}
\subsection{Топология стрелки}
\label{sec:org3f98477}
\begin{theorem}
	\-
	\begin{itemize}
		\item \(a + b = a \cup b\)
		\item \(a \cdot b = a \cap b\)
		\item \(a \to b = \left((X \setminus a) \cup b\right)^\circ\)
		\item \(a \le b\) тогда и только тогда, когда \(a \subseteq b\)
	\end{itemize}
	\uline{Тогда} \(\langle \Omega, \le \rangle\) --- алгебра Гейтинга
	\label{orgc32870c}
\end{theorem}
\subsection{Дискретная топология}
\label{sec:org5fbc0b1}
\begin{examp}
	Дискретная топология: \(\Omega = 2^X\) --- любое множество открыто. Тогда \(\langle \Omega, \le \rangle\) --- булева алгебра
	\label{orgf8fbc60}
\end{examp}
\subsection{Топология на частично упорядоченном множестве}
\label{sec:orga1db7bd}
Топология на частично упорядоченном множестве \(\langle \Omega, \le \rangle\) --- булева алгебра, где \(\Omega\) --- дискретная топология
\subsection{Индуцированная топология}
\label{sec:orgd24fc0a}
\begin{definition}
	\textbf{Индуцированная топология} на подпространстве \(\langle X, \Omega \rangle\) --- топлогия. Пусть \(Y \subset X\). Определим \(\Omega_Y\) --- семейство подмножеств \(Y\) так:
	\[ \Omega_Y = \{U \cap Y \big| U \in \Omega\} \]
	\(\Omega_Y\) --- индуцированная топология на подпространстве \(Y\).
\end{definition}
\subsection{Связность}
\label{sec:orgb7a16ed}
Будем говорить, что топологическое пространство \(\pair{X, \Omega}\) \textbf{связно}, если нет таких открытых множеств \(A\) и \(B\), что \(X = A \cup B\), но \(A \cap B = \varnothing\)
\section{Исчисление высказываний}
\label{sec:orgacf708f}
\subsection{Метапеременные, пропозициональные переменные, Высказывания}
\label{sec:org9848095}
\subsubsection{Язык}
\label{sec:org3f088ee}
\begin{enumerate}
	\item Пропозициональные переменные \\
	      \(A'_i\) --- большая буква начала латинского алфавита
	\item Связки \\
	      \(\underbrace{\alpha}_\text{\color{green}метапеременная}, \beta\) --- высказывания \\
	      Тогда \((\alpha \to \beta),(\alpha \& \beta),(\alpha \vee \beta), (\neg \alpha)\) --- высказывания
\end{enumerate}
\subsubsection{Мета и предметные}
\label{sec:org771f037}
\begin{itemize}
	\item \(\alpha, \beta, \gamma, \dots, \varphi, \psi, \dots\) --- метапеременные для выражений
	\item \(X, Y, Z\) --- метапеременные для предметных переменные
\end{itemize}
Метавыражение: \(\alpha \to \beta\) \\
Предметное выражение: \(A \to (A \to A)\) (заменили \(\alpha\) на \(A\), \(\beta\) на \((A \to A)\) )
\begin{examp}
	Черным --- предметные выражения, Синим --- метавыражения
	\[ (\color{blue}X \color{black}\to\color{blue} Y\color{black})\color{blue}[X \coloneqq A, Y \coloneqq B] \color{black} \equiv A \to B \]
	\[ (\color{blue}\alpha \color{black} \to (A \to \color{blue}X \color{black}))\color{blue}[\alpha \coloneqq A, X \coloneqq B] \equiv \color{black} A \to (A \to B) \]
	\[ (\color{blue}\alpha \color{black} \to (A \to \color{blue}X \color{black}))\color{blue}[\alpha \coloneqq (A \to P), X \coloneqq B] \equiv \color{black} (A \to P) \to (A \to B) \]
\end{examp}
\subsection{Схемы аксиом, доказуемость}
\label{sec:org707e3d8}
\subsubsection{Теория доказательств}
\label{sec:org3b47684}
\begin{definition}
	\textbf{Схема высказывания} --- строка соответсвующая определению высказывания, с:
	\begin{itemize}
		\item метапеременными \(\alpha, \beta, \dots\)
	\end{itemize}
\end{definition}
\begin{definition}
	Аксиома --- высказывания:
	\begin{enumerate}
		\item \(\alpha \to (\beta \to \alpha)\)
		\item \((\alpha \to \beta) \to (\alpha \to \beta \to \gamma) \to (\alpha \to \gamma)\)
		\item \(\alpha \to \beta \to \alpha \& \beta\)
		\item \(\alpha \& \beta \to \alpha\)
		\item \(\alpha \& \beta \to \beta\)
		\item \(\alpha \to \alpha \vee \beta\)
		\item \(\beta \to \alpha \vee \beta\)
		\item \((\alpha \to \gamma) \to (\beta \to \gamma) \to (\alpha \vee \beta \to \gamma)\)
		\item \((\alpha \to \beta) \to (\alpha \to \neg \beta) \to \neg \alpha\)
		\item \(\neg\neg \alpha \to \alpha\)
	\end{enumerate}
\end{definition}
\subsection{Правило Modus Ponens, доказательство, вывод из гипотез}
\label{sec:orgb81a163}
\subsubsection{Правило Modus Ponens и доказательство}
\label{sec:orge39ad3e}
\begin{definition}
	\textbf{Доказательство} (вывод) --- последовательность высказываний \(\alpha_1, \dots, \alpha_n\), где \(\alpha_i\):
	\begin{itemize}
		\item аксиома
		\item существует \(k, l < i\), что \(\alpha_k = \alpha_l \to \alpha\) \\
		      \[ \frac{A,\ A \to B}{B} \]
	\end{itemize}
\end{definition}
\begin{examp}
	\(\vdash A \to A\)
	\begin{center}
		\begin{tabular}{r|ll}
			1 & \(A \to A\ \to A\)                                              & (схема аксиом 1) \\
			2 & \(A \to (A \to A) \to A\)                                       & (схема аксиом 1) \\
			3 & \((A \to (A \to A)) \to (A \to (A \to A) \to A) \to (A \to A)\) & (схема аксиом 2) \\
			4 & \((A \to (A \to A) \to A) \to (A \to A)\)                       & (M.P. 1 и 3)     \\
			5 & \(A \to A\)                                                     & (M.P. 2 и 4)     \\
		\end{tabular}
	\end{center}
\end{examp}
\begin{definition}
	Доказательством высказывания \(\beta\) --- список высказываний \(\alpha_1, \dots, \alpha_n\)
	\begin{itemize}
		\item \(\alpha_1, \dots, \alpha_n\) --- доказательство
		\item \(\alpha_n \equiv \beta\)
	\end{itemize}
\end{definition}
\subsection{Множество истинностных значений, модель (оценка перменных), Оценка высказывания}
\label{sec:orgcca8a66}
\subsubsection{Теория моделей}
\label{sec:org4c11caf}
\begin{itemize}
	\item \(\mathcal{P}\) --- множество предметных переменных
	\item \(\llb\cdot\rrb: \mathcal{T} \to V\), где \(\mathcal{T}\) --- множество высказываний, \(V = \{\text{И}, \text{Л}\}\) --- множество истиностных значений
\end{itemize}



\begin{enumerate}
	\item \(\llb x \rrb: \mathcal{P} \to V\) --- задается при оценке \\
	      \(\llb \rrb^{A \coloneqq v_1, B \coloneqq v_2}\):
	      \begin{itemize}
		      \item \(\mathcal{P} = v_1\)
		      \item \(\mathcal{P} = v_2\)
	      \end{itemize}
	\item \(\llb \alpha \star \beta \rrb = \llb \alpha \rrb \color{blue}\underbrace{\star}_{\substack{\text{определенно} \\ \text{ественным образом}}}\color{black} \llb \beta \rrb\), где \(\star \in [\&, \vee, \neg, \to]\)
	      \begin{examp}
		      \[ \llb A \to A \rrb^{A \coloneqq \text{И}, B \coloneqq \text{Л}} = \llb A \rrb^{A \coloneqq \text{И}, B \coloneqq \text{Л}} \color{blue}\to\color{black}\llb A \rrb^{A \coloneqq \text{И}, B \coloneqq \text{Л}} = \color{blue} \text{И} \to \text{И} = \text{И} \]
		      Также можно записать так:
		      \[ \llb A \to A \rrb^{A \coloneqq \text{И}, B \coloneqq \text{Л}} = f_\to(\llb A \rrb^{A \coloneqq \text{И}, B \coloneqq \text{Л}}, \llb A \rrb^{A \coloneqq \text{И}, B \coloneqq \text{Л}}) = f_\to(\color{blue} \text{И}\color{black} , \color{blue}\text{И}\color{black}) \color{blue}= \text{И} \]
		      , где \(f_\to\) определена так:
		      \begin{center}
			      \begin{tabular}{ll|l}
				      \(a\) & \(b\) & \(f_\to\) \\
				      \hline
				      И     & И     & И         \\
				      И     & Л     & Л         \\
				      Л     & И     & И         \\
				      Л     & Л     & И         \\
			      \end{tabular}
		      \end{center}
	      \end{examp}
\end{enumerate}
\subsection{Общезначимость}
\label{sec:org30ad036}
\begin{examp}
	\(\vDash \alpha\) --- \(\alpha\) общезначимо
	\label{org9b671f9}
\end{examp}
\subsection{Выполнимость}
\label{sec:orgf305012}
Существует оценка, при которой высказывание истинно
\subsection{Невыполнимость}
\label{sec:org6eacce0}
Отрицание выполнимости
\subsection{Следование}
\label{sec:orgbf1b8f0}
\begin{definition}
	Следование: \(\Gamma \vDash \alpha\), если
	\begin{itemize}
		\item \(\Gamma = \gamma_1, \dots, \gamma_n\)
		\item Всегда когда все \(\llb \gamma_i \rrb = \text{И}\), то \(\llb \alpha \rrb = \text{И}\)
	\end{itemize}
	\label{orgca64659}
\end{definition}
\subsection{Корректность}
\label{sec:orge894b71}
\begin{definition}
	\sout{Теория} Исчисление высказываний корректна, если при любом \(\alpha\) из \(\vdash \alpha\) следует \(\vDash \alpha\)
	\label{org39813cb}
\end{definition}
\subsection{Полнота}
\label{sec:org461bd14}
\begin{definition}
	Исчисление полно, если при любом \(\alpha\) из \(\vDash \alpha\) следует \(\vdash \alpha\)
	\label{orgd1c3573}
\end{definition}
\subsection{Противоречивость}
\label{sec:orge3cc00e}
\begin{definition}
	Множество формул \(\Gamma\) \textbf{противоречиво}, если для некоторой
	формулы \(\alpha\) имеем \(\Gamma \vdash \alpha\) и \(\Gamma \vdash
	\neg \alpha\)
\end{definition}
\subsection{Теорема о дедукции}
\label{sec:org272d106}
\begin{theorem}[о дедукции]
	\(\Gamma, \alpha \vdash \beta\) \uline{тогда и только тогда, когда} \(\Gamma \vdash \alpha \to \beta\)
	\label{org377dfdc}
\end{theorem}
\subsection{Теорема о корректности}
\label{sec:org8a5933c}
\begin{theorem}[о корректности]
	Пусть \(\vdash \alpha\) \\
	\uline{Тогда} \(\vDash \alpha\)
	\label{org517bca4}
\end{theorem}
\subsection{Теорема о полноте ИВ}
\label{sec:orge624408}
\begin{theorem}[о полноте]
	Пусть \(\vDash \alpha\), тогда \(\vdash \alpha\)
	\label{orgdcace4c}
\end{theorem}
\section{Интуиционистское исчисление высказываний}
\label{sec:org847b2aa}
Отличается от ИВ 10-ой схемой аксиом: вместо \(\neg \neg \alpha \to \alpha\) стало \(\alpha \to \neg \alpha \to \beta\)
\subsection{Закон исключенного третьего}
\label{sec:orgec2d432}
\[ \vdash A \vee \neg A \]
\subsection{Закон снятия двойного отрицания}
\label{sec:org126b825}
\[ \vdash \neg\neg A \to A \]
\subsection{Закон Пирса}
\label{sec:orgd83b715}
\[ \vdash ((A \to B) \to A) \to A \]
\subsection{ВНК-интерпретация логических связок}
\label{sec:org598c121}
\subsubsection{Интуиционистская логика}
\label{sec:org79df238}
\(A \vee B\) --- плохо
\begin{examp}
	Докажем: существует \(a, b\), что \(a, b \in \R \setminus \mathbb{Q}\), но \(a^b \in \mathbb{Q}\) \\
	Пусть \(a = b = \sqrt{2}\). Рассмотрим \(\sqrt{2}^{\sqrt{2}} \in \R \setminus \mathbb{Q}\)
	\begin{itemize}
		\item Если нет, то ОК
		\item Если да, то возьмем \(a = \sqrt{2}^{\sqrt{2}}, b = \sqrt{2}\), \(a^b = (\sqrt{2}^{\sqrt{2}})^{\sqrt{2}} = \sqrt{2}^{2} = 2\)
	\end{itemize}
\end{examp}
\begin{defintion}
	ВНК-интерпретация. \(\alpha, \beta\)
	\begin{itemize}
		\item \(\alpha \& \beta\) --- есть \(\alpha, \beta\)
		\item \(\alpha \vee \beta\) --- есть \(\alpha\) либо \(\beta\) и мы знаем какое
		\item \(\alpha \to \beta\) --- есть способ перестроить \(\alpha\) в \(\beta\)
		\item \(\perp\) --- конструкция без построения \(\neg \alpha \equiv \alpha \to \perp\)
	\end{itemize}
\end{defintion}
\subsection{Теорема Гливенко}
\label{sec:orgdc24986}
\begin{theorem}
	Обозначим доказуемость высказывания \(\alpha\) в классической логике как \(\vdash_{\text{к}} \alpha\), а в интуицонистской как \(\vdash_{\text{и}}\). Оказывается возможным показать, что какое бы ни было \(\alpha\), если \(\vdash_{\text{к}} \alpha\), то \(\vdash_{\text{и}} \neg\neg \alpha\)
\end{theorem}
\subsection{Решетка}
\label{sec:orgd1f8726}
\begin{definition}
	Фиксируем \(A\) \\
	Частичный порядок --- антисимметричное, транзитивное, рефлексивное отношение \\
	Линейный --- сравнимы любые 2 элемента \\
	\begin{itemize}
		\item \(a \le b \vee b \le a\)
		\item \textbf{Наименьший элемент} \(S\) --- такой \(k \in S\), что если \(x \in S\), то \(k \le x\)
		\item \textbf{Минимальный элемент} \(S\) --- такой \(k \in S\), что нет \(x \in S\), что \(x \le k\)
	\end{itemize}
	\label{org823a1c9}
\end{definition}
\begin{definition}
	\-
	\begin{itemize}
		\item \textbf{Множество верхних граней} \(a\) и \(b\): \(\{x \big| a \le x \& b \le x\}\)
		\item \textbf{Множество нижних граней} \(a\) и \(b\): \(\{x \big| x \le a \& x \le b\}\)
	\end{itemize}
	\label{org92d8d5c}
\end{definition}
\begin{definition}
	\-
	\begin{itemize}
		\item \textbf{\(a + b\)} --- нименьший элемент множества верхних граней
		\item \textbf{\(a \cdot b\)} --- наибольший элемент множества нижних граней
	\end{itemize}
	\label{org806dc29}
\end{definition}
\begin{definition}
	\textbf{Решетка} = \(\langle A, \le \rangle\) --- структура, где для каждых \(a, b\) есть как \(a + b\), так и \(a \cdot b\), \\
	т.е. \(a \in A, b \in B \implies a + b \in A\) и \(a \cdot b \in A\)
	\label{org2f07bad}
\end{definition}
\subsection{Дистрибутивная решетка}
\label{sec:orgeac75a4}
\begin{definition}
	\textbf{Дистрибутивная решетка} если всегда  \(a \cdot (b + c) = a \cdot b + a \cdot c\)
	\label{org97cdb86}
\end{definition}
\begin{lemma}
	В дистрибутивной решетке \(a + b\cdot c = (a + b) \cdot(a + c)\)
	\label{org35529c3}
\end{lemma}
\subsection{Импликативная решетка}
\label{sec:orgb430b6a}
\begin{definition}
	\textbf{Псевдодополнение} \(a \to b = \text{наиб.}\{c \big| a \cdot c \le b\}\)
	\label{orge9b8efb}
\end{definition}
\begin{definition}
	\textbf{Импликативная решетка} --- решетка, где для любых \(a, b\) есть \(a \to b\)
	\label{orga61a1cc}
\end{definition}
\begin{definition}
	\textbf{0} --- наименьший элемент решетки, \textbf{1} --- наибольший элемент решетки
	\label{org93d0311}
\end{definition}
\begin{lemma}
	В импликативной решетке всегда есть \(1\).
	\label{org0de4829}
\end{lemma}
\begin{lemma}
	Импликативная решетка дистрибутивна
\end{lemma}
\subsection{Алгебра Гейтинга}
\label{sec:org6872e0a}
\begin{definition}
	\textbf{Псевдобулева алгебра (алгебра Гейтинга)} --- импликативная решетка с \(0\)
	\label{org0c2de51}
\end{definition}
\begin{theorem}
	Любая алгебра Гейтинга --- модель ИИВ
	\label{orge06a538}
\end{theorem}
\begin{examp}
	см. \ref{sec:org3f98477}
\end{examp}
\begin{definition}
	\textbf{Гомоморфизм} алгебр Гейтинга \\
	\begin{itemize}
		\item \(\varphi: \A \to \B\)
		\item \(\varphi(a \star b) = \varphi(a)\star\varphi(b)\)
		\item \(\varphi(1_\A) = 1_\B\)
		\item \(\varphi(0_\A) = 0_\B\)
	\end{itemize}
	\label{org958689d}
\end{definition}
\begin{theorem}
	\(a \le b\), то \(\varphi(a) \le \varphi(b)\)
	\label{orgea1f678}
\end{theorem}
\begin{definition}
	\-
	\begin{itemize}
		\item \(\alpha\) --- формула ИИВ
		\item \(f, g\): оценки ИИВ
		\item \(f\): ИИВ \(\to\) \(\A\)
		\item \(g\): ИИВ \(\to\) \(\B\)
	\end{itemize}
	\(\varphi\) согласована с \(f, g\), если \(\varphi(f(\alpha)) = g(\alpha)\)
	\label{org9fee5f4}
\end{definition}
\begin{theorem}
	если \(\varphi: \A \to \B\) согласована с \(f, g\) и оценка \(\llbracket \alpha \rrbracket_g \neq 1_\B\), то \(\llbracket \alpha \rrbracket_f \neq 1_\A\)
	\label{org813ee39}
\end{theorem}
\begin{theorem}
	Алгебра Гейтинга --- полная модель ИИВ
	\label{orgcb8f948}
\end{theorem}
\subsection{Булева алгебра}
\label{sec:orgc8e9721}
\begin{definition}
	\textbf{Булева алгебра} --- псевдобулева алгебра, такая что \(a + (a \to 0) = 1\)
	\label{orga6556e1}
\end{definition}
\begin{examp}
	см. \ref{sec:org5fbc0b1}
\end{examp}
\begin{examp}
	\-
	\begin{center}
		\begin{tikzpicture}
			\node (A) at (0, 0) {\(1\)};
			\node (B) at (-1, -1) {\(a\)};
			\node (C) at (1, -1) {\(b\)};
			\node (D) at (0, -2) {\(0\)};
			\draw[->] (A) -- (B);
			\draw[->] (A) -- (C);
			\draw[->] (B) -- (D);
			\draw[->] (C) -- (D);
		\end{tikzpicture}
	\end{center}
	\begin{itemize}
		\item \(a \cdot 0 = 0\)
		\item \(1\cdot b = b\)
		\item \(a \cdot b = 0\)
		\item \(a + b = 1\)
		\item \(a \to b = \text{наиб.}\{x \big| a\cdot x \le b\} = b\) \\
		      \(\{x \big| a \cdot x \le \} = \{0, b\}\)
		\item \(a \to 1 = 1\)
		\item \(a \to 0 = 0\)
	\end{itemize}
	Можем представить в виде пары \(\langle x, y \rangle\)
	\begin{itemize}
		\item \(a = \langle 1, 0 \rangle\)
		\item \(b = \langle 0 , 1\rangle\)
		\item \(1 = \langle 1, 1 \rangle\)
		\item \(0 = \langle 0, 0 \rangle\)
	\end{itemize}
	\label{org1090a03}
\end{examp}
\begin{theorem}
	Любая булева алгебра --- модель КИВ
	\label{orgf408481}
\end{theorem}
\subsection{Геделева алгебра}
\label{sec:org479435c}
\begin{definition}
	Гёделева алгебра --- алгебра Гейтинга, такая что из \(\alpha + \beta = 1\) следует что \(\alpha = 1\) или \(\beta = 1\) \\
	\label{org2cc5286}
\end{definition}
\subsection{Операция \(\Gamma(A)\)}
\label{sec:org5eb1202}
\begin{definition}
	Пусть \(\A\) --- алгебра Гейтинга, тогда:
	\begin{enumerate}
		\item \(\Gamma(\A)\) \\
		      \begin{center}
			      \begin{tikzpicture}
				      \draw (-1, 0) circle[radius=0.5cm] node {\(\A\)};
				      \draw (1, 0) circle[radius=0.5cm] node {\(\A\)};
				      \node (0, 0) {\(\Rightarrow\)};
				      \draw (-1, 0.5) circle[radius=1pt,fill=black] node[above] {\(1\)};
				      \draw (1, 0.5) circle[radius=1pt,fill=black] node[above right] {\(\omega\)};
				      \draw (1, 1.5) circle[radius=1pt,fill=black] node[above] {\(1\)};
				      \draw (1, 1.5) -- (1, 0.5);
			      \end{tikzpicture}
		      \end{center}

		      Добавим новый элемент \(1_{\Gamma(\A)}\) переименуем \(1_\A\) в  \(\omega\)
	\end{enumerate}
	\label{org9782613}
\end{definition}
\begin{theorem}
	\-
	\begin{itemize}
		\item \(\Gamma(\A)\) --- алгебра Гейтинга
		\item \(\Gamma(\A)\) --- Геделева
	\end{itemize}
	\label{org67330ed}
\end{theorem}
\subsection{Алгебра Линденбаума}
\label{sec:org84b5e2c}
\begin{definition}
	\(X\) --- все формулы логики
	\begin{itemize}
		\item \(\alpha \le \beta\) --- это \(\alpha \vdash \beta\)
		\item \(\alpha \approx \beta\), если \(\alpha \vdash \beta\) и \(\beta \vdash \alpha\)
		\item \([\alpha]_\approx = \{\gamma \big| \gamma \approx \alpha\}\) --- класс эквивалентности
		\item \(X/_\approx = \{[\alpha]_\approx \big| \alpha \in X\}\)
	\end{itemize}
	\(\pair{X/_\approx, \le}\) --- алгебра Гейтинга
	\label{org59ff855}
\end{definition}
\begin{property}
	\(\langle X/_\approx, \le \rangle\) --- алгебра Линденбаума, где \(X, \approx\) --- из интуиционистской логики
	\label{orge71db6e}
\end{property}
\subsection{Свойство дизъюнктивности ИИВ}
\label{sec:orgeea008a}
\begin{definition}
	\textbf{Дизъюнктивность} ИИВ: \(\vdash \alpha \vee \beta\) влечет \(\vdash \alpha\) или \(\vdash \beta\)
	\label{org07d0876}
\end{definition}
\begin{theorem}
	ИИВ дизъюнктивно
	\label{orgd9ed491}
\end{theorem}
\subsection{Свойство нетабличности ИИВ}
\label{sec:org2fd4f34}
\begin{definition}
	Назовем модель \textbf{табличной} для ИИВ:
	\begin{itemize}
		\item \(V\) --- множество истинностных значений \\
		      \(f_\to,f_\&, f_V: V^2 \to V\), \(f_\neg: V \to V\) \\
		      Выделенные значения \(T \in V\) \\
		      \(\llbracket p_i \rrbracket \in V\) \(f_\P : p_i \to V\)
		\item \(\eval{p_i} = f_\P(p_i)\) \\
		      \(\llbracket\alpha \star \beta\rrbracket = f_\star(\llbracket\alpha\rrbracket, \llbracket\beta\rrbracket)\) \\
		      \(\llbracket\neg \alpha\rrbracket = f_\neg(\llbracket\alpha\rrbracket)\)
	\end{itemize}
	\sout{Если \(\vdash \alpha\), то} \(\vDash \alpha\) означает, что \(\llbracket\alpha\rrbracket = T\), при любой \(f_\P\)
	\label{org2678362}
\end{definition}
\begin{theorem}
	У ИИВ не существует полной конечной табличной модели
	\label{org8999247}
\end{theorem}
\subsection{Модель Крипке, Вынужденность}
\label{sec:org1cad80a}
\begin{defintion}
	\-
	\begin{enumerate}
		\item \(W = \{W_i\}\) --- множество миров
		\item частичный порядок(\(\succeq\))
		\item отношение вынужденности: \(W_j \Vdash p_i\) \\
		      \((\Vdash)  \subseteq W \times \P\) \\
		      При этом, если \(W_j \Vdash p_i\) и \(W_j \preceq W_k\), то \(W_k \Vdash p\)
	\end{enumerate}
	\label{org7ec2d2e}
\end{defintion}
\begin{definition}
	\-
	\begin{enumerate}
		\item \(W_i \Vdash \alpha\) и \(W_i \Vdash \beta\), тогда (и только тогда) \(W_i \Vdash \alpha \& \beta\) \\
		\item \(W_i \Vdash \alpha\) или \(W_i \Vdash \beta\), то \(W_i \Vdash \alpha \vee \beta\)
		\item Пусть во всех \(W_i \preceq W_j\) всегда когда \(W_j \Vdash \alpha\) имеет место \(W_j \Vdash \beta\) \\
		      Тогда \(W_i \Vdash \alpha \to \beta\)
		\item \(W_i \Vdash \neg \alpha\) --- \(\alpha\) не вынуждено нигде, начиная с \(W_i\):
		      \(W_i \preceq W_j\), то \(W_j \not\Vdash \alpha\)
	\end{enumerate}
	\label{org1570c1d}
\end{definition}
\begin{theorem}
	Если \(W_i \Vdash \alpha\) и \(W_i \preceq W_j\), то \(W_j \Vdash \alpha\)
	\label{org21d327d}
\end{theorem}
\begin{definition}
	Если \(W_i \Vdash \alpha\) при всех \(W_i \in W\), то \(\vDash \alpha\)
	\label{org2a7aee4}
\end{definition}
\begin{theorem}
	ИИВ корректна в модели Крипке
	\label{org65995c8}
\end{theorem}
\section{Исчиление предикатов}
\label{sec:org1ae18f0}
\subsection{Предикатные и функциональные символы, константы и пропозициональные переменные}
\label{sec:org54441cf}
\begin{definition}
	Язык исчисления предикатов
	\begin{itemize}
		\item логические выражения ``предикаты``/``формулы``
		\item предметные выражния ``термы``
	\end{itemize}
	\(\Theta\) --- метаперменные для термов \\
	Термы:
	\begin{itemize}
		\item Атомы:
		      \begin{itemize}
			      \item \(a, b, c, d, \dots\) --- предметные переменные
			      \item \(x, y, z\) --- метапеременные для предметных перменных
		      \end{itemize}
		\item Функциональные Символы
		      \begin{itemize}
			      \item \(f, g, h\) --- Функциональные символы(метаперемнные)
			      \item \(f(\Theta_1, \dots \Theta_n)\) --- применение функциональных символов
		      \end{itemize}
		\item Логические выражения: \\
		      \color{gray}Если \(n = 0\), будем писать \(f, g\) --- без скобок\color{black}
		      \begin{itemize}
			      \item \(P\) --- метаперменные для предикатных символов
			      \item \(A, B, C\) --- предикатный символ
			      \item \(P(\Theta_1, \dots, \Theta_n)\) --- применение предикатных символов
			      \item \(\&, \vee, \neg, \to\) --- Cвязки
			      \item \(\forall x.\varphi\) и \(\exists x.\varphi\) --- кванторы \\
			            \color{gray}``<квантор> <переменная>.<выражение>``\color{black} \\
		      \end{itemize}
	\end{itemize}
	\label{org2d702b9}
\end{definition}
\subsubsection{Теория моделей}
\label{sec:orgb68d66d}
Оценка формулы в исчислении предикатов:
\begin{enumerate}
	\item Фиксируем \(D\) --- предметное множетво
	\item Кажодму \(f_i(x_1, \dots, x_n)\) сопоставим функцию \(D^n \to D\)
	\item Каждому \(P_j(x_1, \dots, x_m)\) сопоставим функцию(предикат) \(D^2 \to V\)
	\item Каждой \(x_i\) сопоставим элемент из \(D\)
\end{enumerate}
\begin{examp}
	\[\forall x.\forall y.\ E(x, y)\]
	Чтобы определить формулу сначала определим \(D = \N\)
	\[ E(x, y) = \begin{cases}\text{И} & ,x = y \\ \text{Л} &, x\neq y\end{cases} \]
	\begin{itemize}
		\item \(\llbracket x \rrbracket = f_{x_i}\)
		\item \(\llbracket \alpha \star \beta \rrbracket\) --- смотри ИИВ
		\item \(\llbracket P_i(\Theta_1, \dots , \Theta_n) \rrbracket = f_{P_i}(\llbracket \Theta_1 \rrbracket, \dots, \llbracket \Theta_n \rrbracket)\)
		\item \(\llbracket f_j(\Theta_1 , \dots, \Theta_n ) \rrbracket = f_{f_j}(\llbracket \Theta_1 \rrbracket, \dots, \llbracket \Theta_n \rrbracket)\)
		\item \[ \llbracket \forall x. \varphi \rrbracket = \begin{cases} \text{И} & , \text{если } \llbracket \varphi \rrbracket^{f_x = k} = \text{И}\text{ при всех } k \in D  \\ \text{Л} &,\text{иначе}\end{cases} \]
		\item \[ \llbracket \exists x.\varphi \rrbracket = \begin{cases} \text{И} &, \text{если } \llbracket \varphi \rrbracket^{f_x = k} = \text{И при некотором } k \in D \\ \text{Л} &,\text{иначе} \end{cases} \]
	\end{itemize}
	\[ \llbracket \forall x.\forall y.E(x, y) \rrbracket = \text{Л} \]
	т.к. \(\llbracket E(x, y) \rrbracket^{x:=1,\ y:=2} = \text{Л}\)
\end{examp}
\newcommand{\colorboxed}[2]{\,\color{#1}\fbox{\color{black}#2}\color{black}\,}

\begin{examp}
	\[ \forall \colorboxed{green}{\varepsilon > \colorboxed{blue}{0}}\ \exists N\ \forall \colorboxed{green}{\colorboxed{blue}{n} > \colorboxed{blue}{N}}\quad \colorboxed{green}{\colorboxed{blue}{|a_n - a|} < \colorboxed{blue}{\varepsilon}} \]
	Синим отмечены функциональные конструкции(термы), зеленым предикатные
	\[ \forall \varepsilon. (\varepsilon > 0) \to \exists N. \forall n. (n > N) \to (|a_n - a| < \varepsilon) \]
	Обозначим:
	\begin{itemize}
		\item \((>)(a, b) = G(a, b)\) --- предикат
		\item \(|\bullet|(a) = m_|(a)\)
		\item \((-)(a, b) = m_-(a, b)\)
		\item \(0() = m_0\)
		\item \(a_\bullet(n) = m_a(n)\)
	\end{itemize}
	\[ \forall e. \colorboxed{green}{G(\colorboxed{blue}{e}, \colorboxed{blue}{m_0})} \to \exists n_0.\forall n. \colorboxed{green}{G(n, n_0)}\to \colorboxed{green}{G\bigg(e, \colorboxed{blue}{m_|\Big(m_- \big(m_a(n), a\big)\Big)}\bigg)} \]
\end{examp}
\subsection{Свободные и связанные вхождения предметных переменных в формулу}
\label{sec:orge2af0e7}
\subsubsection{Вхождение}
\label{sec:org9792819}
\begin{examp}
	\[ (P(\underset{1}{x}) \vee Q(\underset{2}{x})) \to (R(\underset{3}{x}) \& (\underbrace{\forall \underset{4}{x}. P_1(\underset{5}{x})}_{\text{область }\forall\text{ по }x})) \]
	1, 2, 3 --- свободные, 5 --- связанное, по пермененной 4
\end{examp}
\begin{examp}
	\[ \underbrace{\forall x. \forall y. \forall x. \forall y. \forall x. P(x)}_{\text{область }\forall\text{ по }x} \]
	Здесь \(x\) в \(P(x)\) связано. \(x\) не входит свободно в эту формулу, потому что нет свободных вхождений
\end{examp}
\begin{definition}
	Переменная \(x\) входит свободно если существует свободное вхождение
\end{definition}
\begin{definition}
	Вхождение свободно, если не связано
\end{definition}
Можно относится к свободно входящим перменным как с перменным из библиотеки, т.е. мы не имеем права их переименовывать
\begin{examp}
	Некорректная формула
	\begin{description}
		\item[{\(\alpha_1\)}] \(x = 0 \to x = 0\)
		\item[{\(\alpha_2\)}] \color{red}\((\exists x. x = 0) \to (x = 0)\) --- не доказано\color{black}
		\item[{\(\alpha_2'\)}] \((\exists t. x = 0) \to (x = 0)\) --- (правило \(\exists\))
	\end{description}
\end{examp}
\begin{examp}
	\-
	\begin{description}
		\item[{\((n)\)}] \(x = 0 \to y = 0\) --- откуда то
		\item[{\((n + 1)\)}] \((\exists x. x = 0) \to (y = 0)\) --- (правило \(\exists\))
	\end{description}
\end{examp}
\subsubsection{Свободные подстановки}
\label{sec:org1cdf525}
\begin{definition}
	\(\Theta\) свободен для подстановки вместо \(x\) в \(\varphi\), если никакая свободная перменная в \(\Theta\) не станет связанной в \(\varphi[x := \Theta]\)
\end{definition}
\begin{definition}
	\(\varphi[x := \Theta]\) --- "Заменить все свободные вхождения x в \(\varphi\) на \(\Theta\)"
\end{definition}
\begin{examp}
	\[ (\forall x. \forall y. \forall x. P(x))[x := y] \equiv \forall x. \forall y. \forall x. P(x) \]
\end{examp}
\begin{examp}
	\[ (P(x) \vee \forall x. P(x))[x := y] \equiv P(y) \vee \forall x. P(x) \]
\end{examp}
\begin{examp}
	\[ (\forall y. x = y)\ [x := \underbrace{y}_{\equiv \Theta}] \equiv \forall y. \underset{1}{y} = y\]
	\(FV(\Theta) = \{y\}\) --- свободные перменные в \(\Theta\). Вхождение \(y\) с номером 1 стало связанным
\end{examp}
\begin{examp}
	\[ P(x) \& \forall y. x = y\ [x := y + z] \equiv P(y + z) \& \forall y. \underset{1}{y} + z = y \]
	Здесь при подстановке вхождение \(y\) с номером 1 cтало связанным. \(x\) --- библиотечная функция, переименовали \(x\) во что-то другое.
\end{examp}
\subsection{Свобода для подстановки, Правила вывода для кванторов, аксиомы исчисления предикатов для кванторов, оценки и модели в исчислении предикатов}
\label{sec:orgd0b59a7}
\subsubsection{Теория доказательств}
\label{sec:org4a1d4ab}
Все аксимомы И.В + M.P.
\begin{description}
	\item[{(cхема 11)}] \((\forall x. \varphi) \to \varphi[x:=\Theta]\)
	\item[{(схема 12)}] \(\varphi[x:=\Theta]\to \exists x. \varphi\)
\end{description}
Если \(\Theta\) свободен для подстановки вместо \(x\) в \(\varphi\).
\begin{definition}
	\textbf{Свободен для подстановки} --- никакое свободное вхождение \(x\) в \(\Theta\) не станет связанным
\end{definition}
\begin{examp}
	\-
	\begin{minted}[frame=lines,linenos=true,mathescape]{c}
  int y;
  int f(int x) {
          x = y;
  }
\end{minted}
	Заменим \texttt{y := x}. Код сломается, т.к. у нас нет свобод для подстановки
\end{examp}
\begin{description}
	\item[{(Правило \(\forall\))}] \[\frac{\varphi \to \psi}{\varphi \to \forall x. \psi}\]
	\item[{(Правило \(\exists\))}] \[ \frac{\psi \to \varphi}{(\exists x.\psi) \to \varphi} \]
\end{description}
В обоих правилах \(x\) не входит свободно в \(\varphi\)
\begin{examp}
	\[ \frac{x = 5 \to x^2 = 25}{x = 5 \to \forall x. x^2 = 25} \]
	Между \(x\) и \(x^2\) была связь, мы ее разрушили. Нарушено ограничение
\end{examp}
\begin{examp}
	\[ \exists y. x = y \]
	\[ \forall x. \exists y. x = y \to \exists y. y + 1 = y \]
	Делаем замену \texttt{x := y+1}. Нарушено требование свобод для подстановки. \(y\) входит в область действия квантора \(\exists\) и поэтому свободная переменная \(x\) стала связанная.
\end{examp}
\subsection{Теорема о дедукции для исчисления предикатов}
\label{sec:org595d2e3}
\begin{theorem}
	Пусть задана \(\Gamma,\ \alpha,\beta\)
	\begin{enumerate}
		\item Если \(\Gamma, \alpha \vdash \beta\), то \(\Gamma \vdash \alpha \to \beta\), при условии, если в доказательстве \(\Gamma, \alpha \vdash \beta\) не применялись правила для \(\forall, \exists\) по перменным, входяшим свободно в \(\alpha\)
		\item Если \(\Gamma \vdash \alpha \to \beta\), то \(\Gamma, \alpha \vdash \beta\)
	\end{enumerate}
	\label{org00a29a3}
\end{theorem}
\subsection{Теорема о корректности для исчисления предикатов}
\label{sec:org69bf234}
\begin{definition}[Условие для корректности]
	Правила для кванторов по свободным перменным из \(\Gamma\) запрещены. \\
	\uline{Тогда} \(\Gamma \vdash \alpha\) влечет \(\Gamma \vDash \alpha\)
	\label{org93b50c8}
\end{definition}
\subsection{Полное множество (бескванторных) формул}
\label{sec:orgaeb618c}
\begin{definition}
	\(\Gamma\) --- \textbf{непротиворечивое} множество формул, если \(\Gamma \not\vdash \alpha \& \neg \alpha\) ни при каком \(\alpha\)
	\label{org8b6f465}
\end{definition}
\begin{definition}
	Полное непротиворечивое замкнутых бескванторных формул --- такое, что для каждой замкнутой бескванторной формулы \(\alpha\): либо \(\alpha \in \Gamma\), либо \(\neg \alpha \in \Gamma\)
	\label{org6a0bc4e}
\end{definition}
\begin{theorem}
	Если \(\Gamma\) --- непротиворечивое множество з.б. фомул и \(\alpha\) --- з.б.  формула. \\
	То либо \(\Gamma \cup \{\alpha\}\), либо \(\Gamma \cup \{\neg \alpha\}\) --- непр. мн. з.б. формул
	\label{orgc036649}
\end{theorem}
\begin{theorem}
	Если \(\Gamma\) --- непр. мн. з.б. фомул, то можно построить \(\Delta\) --- полное непр. мн. з.б. формул. \(\Gamma \subseteq \Delta\) и в языке --- счетное количество формул
	\label{orgebc0c64}
\end{theorem}
\begin{definition}
	\(\varphi_1, \varphi_2, \varphi_3, \dots\) --- формулы з.б. \\
	\begin{itemize}
		\item \(\Gamma_0 = \Gamma\)
		\item \(\Gamma_1 = \Gamma_0 \cup \{\varphi_1\}\) либо \(\Gamma_0 \cup \{\neg \varphi_1\}\) --- смотря что непротиворечивое
		\item \(\Gamma_2 = \Gamma_1 \cup \{\varphi_2\}\) либо \(\Gamma_1 \cup \{\neg \varphi_2\}\)
	\end{itemize}
	\[ \Gamma^* = \bigcup_i \Gamma_i \]
	\label{org4398823}
\end{definition}
\begin{property}
	\(\Gamma^*\) --- полное
	\label{org2bcf8b7}
\end{property}
\begin{property}
	\(\Gamma^*\) --- непротиворечивое
	\label{org56a8999}
\end{property}
\begin{theorem}
	Любое полное непротиворечивое множество замкнутых бескванторных формул \(\Gamma\) имеет модель, т.е. существует оценка \(\llbracket \rrbracket\): если \(\gamma \in \Gamma\), то \(\llbracket \gamma \rrbracket = \text{И}\)
	\label{org725c6c0}
\end{theorem}
\begin{theorem}
	Если \(\Gamma_i\) --- непротиворечиво, то \(\Gamma_{i + 1}\) --- непротиворечиво
	\label{org693c96c}
\end{theorem}
\begin{theorem}
	\(\Gamma*\) --- непротиворечиво
	\label{orgd717523}
\end{theorem}
\begin{corollary}
	\(\Gamma^\triangle = \Gamma*\) без формул с \(\forall, \exists\)
	\label{org9267a44}
\end{corollary}
\begin{definition}
	\textbf{Предваренная нормальная форма} --- формула, где \(\forall \exists \forall \dots(\tau)\), \(\tau\) --- формула без кванторов
	\label{org50a2fe1}
\end{definition}
\begin{theorem}
	Если \(\varphi\) --- формула, то существует \(\psi\) --- в п.ф., то \(\varphi \to \psi\) и \(\psi \to \varphi\)
	\label{org04bf09a}
\end{theorem}
\subsection{Модель для формулы}
\label{sec:org5b2985d}
\begin{definition}
	\textbf{Моделью} для непротиворечивого множества замкнутых бескванторных формул \(\Gamma\) --- такая модель, что каждая формула из \(\Gamma\) оценивается в И
	\label{org8fc9248}
\end{definition}
\subsection{Теорема Гёделя о полноте исчисления предикатов}
\label{sec:org0391102}
\begin{theorem}[Геделя о полноте]
	Если \(\Gamma\) --- полное неротиворечивое множество замкнутых(не бескванторных) фомул, то оно имеет модель
	\label{org9a210a4}
\end{theorem}
\begin{theorem}[Геделя о полноте ИП]
	У любого н.м.з.ф. (непротиворечивого множества замкнутых формул) ИП существует модель
	\label{org85b8706}
\end{theorem}
\subsection{Следствие из теоремы Гёделя о полноте исчисления предикатов}
\label{sec:org276456e}
\begin{corollary}
	Пусть \(\vDash \alpha\), тогда \(\vdash \alpha\)
	\label{orgece4af2}
\end{corollary}
\subsection{Неразрешимость исчисления предикатов (формулировка, что такое неразрешимость).}
\label{sec:orgd9ea79a}
\begin{definition}
	\textbf{Язык} --- множество слов. Язык \(\mathcal{L}\) разрешим, если существует \(A\) --- алгоритм, что по слову \(w\): \\
	\(A(w)\) --- останавливается в `1`, если \(w \in \mathcal{L}\) и `0`, если \(w \not\in \mathcal{L}\)
	\label{orgd24a798}
\end{definition}
\begin{remark}
	Проблема останова: не существует алгоритма, который по программе для машина Тьюринга ответит, остановится она или нет. \\
	Пусть \(\mathcal{L}'\) --- язык всех останов программы для машины Тьюринга. \(\mathcal{L}'\) неразрешим
	\label{org8241621}
\end{remark}
\begin{theorem}
	ИП неразрешимо
	\label{org41cbf0f}
\end{theorem}
\section{Арифметика и теории первого порядка}
\label{sec:orgc309b90}
\subsection{Теория первого порядка}
\label{sec:org541226c}
\begin{definition}
	\textbf{Теория I порядка} --- Исчесление предикатов + нелогические функции + предикатные символы + нелогические (математические) аксиомы.
	\label{org3c4e85c}
\end{definition}
\subsection{Модели и структуры теорий первого порядка}
\label{sec:org84f95d2}
Назовём \textbf{структурой} теории первого порядка такую модель исчисления предикатов, что для всех нелогических
функциональных и предикатных символов теории в ней задана оценка.
Назовём \textbf{моделью} теории первого порядка такую структуру, что все нелогические аксиомы данной теории в ней
истинны.
\subsection{Аксиоматика Пеано}
\label{sec:org61bde88}
\begin{definition}
	Будем говорить, что \(N\) соответсвует \textbf{аксиоматике Пеано} если:
	\begin{itemize}
		\item задан \(('): N \to N\) --- инъективная функция (для разных элементов, разные значения)
		\item задан \(0 \in N\): нет \(a \in N\), что \(a' = 0\)
		\item если \(P(x)\) --- некоторое утверждение, зависящее от \(x \in N\), такое, что \(P(0)\) и всегда, когда \(P(x)\), также и \(P(x')\). Тогда \(P(x)\)
	\end{itemize}
	\label{orgacd9e1a}
\end{definition}
\subsection{Определение операций (сложение, умножение, возведение в степень)}
\label{sec:orgfee98d5}
\begin{definition}
	\[ a + b = \begin{cases}
			a        & b = 0  \\
			(a + c)' & b = c'
		\end{cases}\]
	\label{org3179b45}
\end{definition}
\begin{definition}
	\[ a \cdot b = \begin{cases}
			0               & b = 0  \\
			(a \cdot c) + a & b = c'
		\end{cases}\]
	\label{org8871ba9}
\end{definition}
\begin{definition}
	\[ a^b = \begin{cases}
			1            & b = 0  \\
			(a^c)\cdot a & b = c'
		\end{cases}\]
	\label{org1245388}
\end{definition}
\subsection{Формальная арифметика (язык, схема аксиом индукции и общая характеристика остальных аксиом).}
\label{sec:orgd643e78}
\subsubsection{Формальная арифметика}
\label{sec:orgd4bf0ba}
\begin{definition}
	Исчесление предикатов:
	\begin{itemize}
		\item Функциональные символы:
		      \begin{itemize}
			      \item \(0\) --- 0-местный
			      \item \((')\) --- 1-местный
			      \item \((\cdot)\) --- 2-местный
			      \item \((+)\) --- 2-местный
		      \end{itemize}
		\item \((=)\) --- 2-местный предикатный символ
	\end{itemize}
	Аксимомы:
	\begin{enumerate}
		\item \(a = b \to a' = b'\)
		\item \(a = b \to a = c \to b = c\)
		\item \(a' = b' \to a= b\)
		\item \(\neg a' = 0\)
		\item \(a + b' = (a + b)'\)
		\item \(a + 0 = a\)
		\item \(a\cdot 0 = 0\)
		\item \(a\cdot b' = a\cdot b + a\)
		\item Схема аксиом индукции:
		      \[ (\psi[x:=0])\&(\forall x. \psi \to (\psi[x:=x'])) \to \psi \]
		      \(x\) входит свободно в \(\psi\)
	\end{enumerate}
\end{definition}
\beginproperty
\begin{property}
	\[ ((a + 0 = a) \to (a + 0 = a) \to (a = a)) \]
\end{property}
\begin{proof}
	\[ \forall a. \forall b. \forall c. a = b \to a = c \to b = c \]
	\[ (\forall a. \forall b. \forall c. a = b \to a = c \to b = c) \to \forall b. \forall c. (a + 0 = b \to a + 0 = c \to b = c) \]
	\[ \forall b. \forall c. a + 0 = b \to a + 0 = c\to b = c \]
	\[ (\forall b. \forall c. a + 0 = b \to a + 0 = c \to b = c) \to \forall c.(a + 0 = a \to a + 0 = c \to a=c) \]
	\[ \forall c. a + 0 = a \to a + 0 = c \to a = c \]
	\[ (\forall c. a + 0 = a \to a + 0 = c \to a = c) \to a+0 = a \to a + 0 = a \to a= a \]
	\[ a + 0  = a \to a + 0 = a \to a = a \]
	\[ a + 0 = a \]
	\[ a + 0 = a \to a = a \]
	\[ a = a \]
	\[ \forall b. \forall c. a = b \to a = c \to b = c \]
	\[ (0 = 0 \to 0 = 0 \to 0 = 0) \]
	\[ (\forall b. \forall c. a = b \to a = c\ to b = c) \to (0 = 0 \to 0 = 0 \to 0 = 0) \to \phi \]
	\fixme
\end{proof}
\begin{definition}
	\(\exists! x.\varphi(x) \equiv (\exists x. \varphi(x))\&\forall p.\forall q. \varphi(p)\&\varphi(q) \to p = q\) \\
	Можно также записать \(\exists ! x.\neg \exists s. s' = x\) или \((\forall q.(\exists x. x' = q)\vee q= 0)\)
\end{definition}
\begin{definition}
	\(a \le b\) --- сокращение для \(\exists n. a + n = b\)
\end{definition}
\begin{definition}
	\[ \overline{n} = 0^{(n)}\]
	\[ 0^{(n)} = \begin{cases}
			0            & n = 0 \\
			0^{(n - 1)'} & n > 0
		\end{cases}\]
\end{definition}
\end{document}
