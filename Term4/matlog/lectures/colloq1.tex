% Created 2021-04-21 Wed 20:44
% Intended LaTeX compiler: pdflatex

\documentclass[english]{article}
\usepackage[T1, T2A]{fontenc}
\usepackage[lutf8]{luainputenc}
\usepackage[english, russian]{babel}
\usepackage{minted}
\usepackage{graphicx}
\usepackage{longtable}
\usepackage{hyperref}
\usepackage{xcolor}
\usepackage{natbib}
\usepackage{amssymb}
\usepackage{stmaryrd}
\usepackage{amsmath}
\usepackage{caption}
\usepackage{mathtools}
\usepackage{amsthm}
\usepackage{tikz}
\usepackage{grffile}
\usepackage{extarrows}
\usepackage{wrapfig}
\usepackage{algorithm}
\usepackage{algorithmic}
\usepackage{lipsum}
\usepackage{rotating}
\usepackage{placeins}
\usepackage[normalem]{ulem}
\usepackage{amsmath}
\usepackage{textcomp}
\usepackage{capt-of}

\usepackage{geometry}
\geometry{a4paper,left=2.5cm,top=2cm,right=2.5cm,bottom=2cm,marginparsep=7pt, marginparwidth=.6in}
 \usepackage{hyperref}
 \hypersetup{
     colorlinks=true,
     linkcolor=blue,
     filecolor=orange,
     citecolor=black,      
     urlcolor=cyan,
     }

\usetikzlibrary{decorations.markings}
\usetikzlibrary{cd}
\usetikzlibrary{patterns}
\usetikzlibrary{automata, arrows}

\newcommand\addtag{\refstepcounter{equation}\tag{\theequation}}
\newcommand{\eqrefoffset}[1]{\addtocounter{equation}{-#1}(\arabic{equation}\addtocounter{equation}{#1})}
\newcommand{\llb}{\llbracket}
\newcommand{\rrb}{\rrbracket}


\newcommand{\R}{\mathbb{R}}
\renewcommand{\C}{\mathbb{C}}
\newcommand{\N}{\mathbb{N}}
\newcommand{\A}{\mathfrak{A}}
\newcommand{\B}{\mathfrak{B}}
\newcommand{\rank}{\mathop{\rm rank}\nolimits}
\newcommand{\const}{\var{const}}
\newcommand{\grad}{\mathop{\rm grad}\nolimits}

\newcommand{\todo}{{\color{red}\fbox{\text{Доделать}}}}
\newcommand{\fixme}{{\color{red}\fbox{\text{Исправить}}}}

\newcounter{propertycnt}
\setcounter{propertycnt}{1}
\newcommand{\beginproperty}{\setcounter{propertycnt}{1}}

\theoremstyle{plain}
\newtheorem{propertyinner}{Свойство}
\newenvironment{property}{
  \renewcommand\thepropertyinner{\arabic{propertycnt}}
  \propertyinner
}{\endpropertyinner\stepcounter{propertycnt}}
\newtheorem{axiom}{Аксиома}
\newtheorem{lemma}{Лемма}
\newtheorem{manuallemmainner}{Лемма}
\newenvironment{manuallemma}[1]{%
  \renewcommand\themanuallemmainner{#1}%
  \manuallemmainner
}{\endmanuallemmainner}

\theoremstyle{remark}
\newtheorem*{remark}{Примечание}
\newtheorem*{solution}{Решение}
\newtheorem{corollary}{Следствие}[theorem]
\newtheorem*{examp}{Пример}
\newtheorem*{observation}{Наблюдение}

\theoremstyle{definition}
\newtheorem{task}{Задача}
\newtheorem{theorem}{Теорема}[section]
\newtheorem*{definition}{Определение}
\newtheorem*{symb}{Обозначение}
\newtheorem{manualtheoreminner}{Теорема}
\newenvironment{manualtheorem}[1]{%
  \renewcommand\themanualtheoreminner{#1}%
  \manualtheoreminner
}{\endmanualtheoreminner}
\captionsetup{justification=centering,margin=2cm}
\newenvironment{colored}[1]{\color{#1}}{}

\tikzset{->-/.style={decoration={
  markings,
  mark=at position .5 with {\arrow{>}}},postaction={decorate}}}
\makeatletter
\newcommand*{\relrelbarsep}{.386ex}
\newcommand*{\relrelbar}{%
  \mathrel{%
    \mathpalette\@relrelbar\relrelbarsep
  }%
}
\newcommand*{\@relrelbar}[2]{%
  \raise#2\hbox to 0pt{$\m@th#1\relbar$\hss}%
  \lower#2\hbox{$\m@th#1\relbar$}%
}
\providecommand*{\rightrightarrowsfill@}{%
  \arrowfill@\relrelbar\relrelbar\rightrightarrows
}
\providecommand*{\leftleftarrowsfill@}{%
  \arrowfill@\leftleftarrows\relrelbar\relrelbar
}
\providecommand*{\xrightrightarrows}[2][]{%
  \ext@arrow 0359\rightrightarrowsfill@{#1}{#2}%
}
\providecommand*{\xleftleftarrows}[2][]{%
  \ext@arrow 3095\leftleftarrowsfill@{#1}{#2}%
}
\makeatother

\newenvironment{rualgo}[1][]
  {\begin{algorithm}[#1]
     \selectlanguage{russian}%
     \floatname{algorithm}{Алгоритм}%
     \renewcommand{\algorithmicif}{{\color{red}\textbf{если}}}%
     \renewcommand{\algorithmicthen}{{\color{red}\textbf{тогда}}}%
     \renewcommand{\algorithmicelse}{{\color{red}\textbf{иначе}}}%
     \renewcommand{\algorithmicend}{{\color{red}\textbf{конец}}}%
     \renewcommand{\algorithmicfor}{{\color{red}\textbf{для}}}%
     \renewcommand{\algorithmicto}{{\color{red}\textbf{до}}}%
     \renewcommand{\algorithmicdo}{{\color{red}\textbf{делать}}}%
     \renewcommand{\algorithmicwhile}{{\color{red}\textbf{пока}}}%
     \renewcommand{\algorithmicrepeat}{{\color{red}\textbf{повторять}}}%
     \renewcommand{\algorithmicuntil}{{\color{red}\textbf{до тех пор пока}}}%
     \renewcommand{\algorithmicloop}{{\color{red}\textbf{повторять}}}%
     \renewcommand{\algorithmicnot}{{\color{blue}\textbf{не}}}%
     \renewcommand{\algorithmicand}{{\color{blue}\textbf{и}}}%
     \renewcommand{\algorithmicor}{{\color{blue}\textbf{или}}}%
     \renewcommand{\algorithmicrequire}{{\color{blue}\textbf{Ввод}}}%
     \renewcommand{\algorithmicensure}{{\color{blue}\textbf{Вывод}}}%
     \renewcommand{\algorithmicreturn}{{\color{red}\textbf{Вернуть}}}%
     \renewcommand{\algorithmicrtrue}{{\color{blue}\textbf{истинна}}}%
     \renewcommand{\algorithmicrfalse}{{\color{blue}\textbf{ложь}}}%
     % Set other language requirements
  }
  {\end{algorithm}}
\author{Ilya Yaroshevskiy}
\date{\today}
\title{Коллоквиум 1}
\hypersetup{
 pdfauthor={Ilya Yaroshevskiy},
 pdftitle={Коллоквиум 1},
 pdfkeywords={},
 pdfsubject={},
 pdfcreator={Emacs 28.0.50 (Org mode 9.4.4)}, 
 pdflang={English}}
\begin{document}

\maketitle
\tableofcontents


\section{Топология}
\label{sec:org5e36104}
\subsection{Топологическое пространство, открытое и замкнутое множество}
\label{sec:org172b294}
\begin{definition}
Рассмотрим множество \(X\) --- \textbf{носитель}. Рассмотрим \(\Omega \subseteq 2^X\) --- подмножество подмножеств \(X\) --- \textbf{топология}.
\begin{enumerate}
\item \(\bigcup X_i \in \Omega\), где \(X_i \in \Omega\)
\item \(X_1 \cap \dots \cap X_n \in \Omega\), если \(X_i \in \Omega\)
\item \(\emptyset, X \in \Omega\)
\end{enumerate}
\label{orgc32cd87}
\end{definition}
\subsection{Внутренность и замыкание множества}
\label{sec:org303a73f}
\begin{definition}
\[ (X)^\circ = \text{наиб.}\{w \big| w \subseteq X, w\text{ --- откр.}\}\]
\label{orga525d7c}
\end{definition}
\subsection{Топология стрелки}
\label{sec:orgf98160b}
\subsection{Дискретная топология}
\label{sec:orgfa67b97}
\begin{examp}
Дискретная топология: \(\Omega = 2^X\) --- любое множество открыто. Тогда \(\langle \Omega, \le \rangle\) --- булева алгебра
\label{org5a12b02}
\end{examp}
\subsection{Топология на частично упорядоченном множестве}
\label{sec:org94bb5aa}
\subsection{Индуцированная топология}
\label{sec:org6009c21}
\subsection{Связность}
\label{sec:org252f409}
\section{Исчисление высказываний}
\label{sec:org2d7b466}
\subsection{Метапеременные, пропозициональные переменные, Высказывания}
\label{sec:org76419e8}
\subsubsection{Язык}
\label{sec:orga79036d}
\begin{enumerate}
\item Пропозициональные переменные \\
\(A'_i\) --- большая буква начала латинского алфавита
\item Связки \\
\(\underbrace{\alpha}_\text{\color{green}метапеременная}, \beta\) --- высказывания \\
Тогда \((\alpha \to \beta),(\alpha \& \beta),(\alpha \vee \beta), (\neg \alpha)\) --- высказывания
\end{enumerate}
\subsubsection{Мета и предметные}
\label{sec:org99b1e11}
\begin{itemize}
\item \(\alpha, \beta, \gamma, \dots, \varphi, \psi, \dots\) --- метапеременные для выражений
\item \(X, Y, Z\) --- метапеременные для предметные переменные
\end{itemize}
Метавыражение: \(\alpha \to \beta\) \\
Предметное выражение: \(A \to (A \to A)\) (заменили \(\alpha\) на \(A\), \(\beta\) на \((A \to A)\) )
\begin{examp}
Черным --- предметные выражения, Синим --- метавыражения
\[ (\color{blue}X \color{black}\to\color{blue} Y\color{black})\color{blue}[X \coloneqq A, Y \coloneqq B] \color{black} \equiv A \to B \]
\[ (\color{blue}\alpha \color{black} \to (A \to \color{blue}X \color{black}))\color{blue}[\alpha \coloneqq A, X \coloneqq B] \equiv \color{black} A \to (A \to B) \]
\[ (\color{blue}\alpha \color{black} \to (A \to \color{blue}X \color{black}))\color{blue}[\alpha \coloneqq (A \to P), X \coloneqq B] \equiv \color{black} (A \to P) \to (A \to B) \]
\end{examp}
\subsection{Схемы аксиом, доказуемость}
\label{sec:orge61950f}
\subsubsection{Теория доказательств}
\label{sec:orge505e10}
\begin{definition}
\textbf{Схема высказывания} --- строка соответсвующая определению высказывания, с:
\begin{itemize}
\item метапеременными \(\alpha, \beta, \dots\)
\end{itemize}
\end{definition}
\begin{definition}
Аксиома --- высказывания:
\begin{enumerate}
\item \(\alpha \to (\beta \to \alpha)\)
\item \((\alpha \to \beta) \to (\alpha \to \beta \to \gamma) \to (\alpha \to \gamma)\)
\item \(\alpha \to \beta \to \alpha \& \beta\)
\item \(\alpha \& \beta \to \alpha\)
\item \(\alpha \& \beta \to \beta\)
\item \(\alpha \to \alpha \vee \beta\)
\item \(\beta \to \alpha \vee \beta\)
\item \((\alpha \to \gamma) \to (\beta \to \gamma) \to (\alpha \vee \beta \to \gamma)\)
\item \((\alpha \to \beta) \to (\alpha \to \neg \beta) \to \neg \alpha\)
\item \(\neg\neg \alpha \to \alpha\)
\end{enumerate}
\end{definition}
\subsection{Правило Modus Ponens, доказательство, вывод из гипотез}
\label{sec:org162008d}
\subsubsection{Правило Modus Ponens и доказательство}
\label{sec:org96d64c8}
\begin{definition}
\textbf{Доказательство} (вывод) --- последовательность высказываний \(\alpha_1, \dots, \alpha_n\), где \(\alpha_i\):
\begin{itemize}
\item аксиома
\item существует \(k, l < i\), что \(\alpha_k = \alpha_l \to \alpha\) \\
\[ \frac{A,\ A \to B}{B} \]
\end{itemize}
\end{definition}
\begin{examp}
\(\vdash A \to A\)
\begin{center}
\begin{tabular}{r|ll}
1 & \(A \to A\ \to A\) & (схема аксиом 1)\\
2 & \(A \to (A \to A) \to A\) & (схема аксиом 1)\\
3 & \((A \to (A \to A)) \to (A \to (A \to A) \to A) \to (A \to A)\) & (схема аксиом 2)\\
4 & \((A \to (A \to A) \to A) \to (A \to A)\) & (M.P. 1 и 3)\\
5 & \(A \to A\) & (M.P. 2 и 4)\\
\end{tabular}
\end{center}
\end{examp}
\begin{definition}
Доказательством высказывания \(\beta\) --- список высказываний \(\alpha_1, \dots, \alpha_n\)
\begin{itemize}
\item \(\alpha_1, \dots, \alpha_n\) --- доказательство
\item \(\alpha_n \equiv \beta\)
\end{itemize}
\end{definition}
\subsection{Множество истинностных значений, модель (оценка перменных), Оценка высказывания}
\label{sec:orgff00222}
\subsubsection{Теория моделей}
\label{sec:orge560fb5}
\begin{itemize}
\item \(\mathcal{P}\) --- множество предметных переменных
\item \(\llb\cdot\rrb: \mathcal{T} \to V\), где \(\mathcal{T}\) --- множество высказываний, \(V = \{\text{И}, \text{Л}\}\) --- множество истиностных значений
\end{itemize}



\begin{enumerate}
\item \(\llb x \rrb: \mathcal{P} \to V\) --- задается при оценке \\
\(\llb \rrb^{A \coloneqq v_1, B \coloneqq v_2}\):
\begin{itemize}
\item \(\mathcal{P} = v_1\)
\item \(\mathcal{P} = v_2\)
\end{itemize}
\item \(\llb \alpha \star \beta \rrb = \llb \alpha \rrb \color{blue}\underbrace{\star}_{\substack{\text{определенно} \\ \text{ественным образом}}}\color{black} \llb \beta \rrb\), где \(\star \in [\&, \vee, \neg, \to]\)
\begin{examp}
\[ \llb A \to A \rrb^{A \coloneqq \text{И}, B \coloneqq \text{Л}} = \llb A \rrb^{A \coloneqq \text{И}, B \coloneqq \text{Л}} \color{blue}\to\color{black}\llb A \rrb^{A \coloneqq \text{И}, B \coloneqq \text{Л}} = \color{blue} \text{И} \to \text{И} = \text{И} \]
Также можно записать так:
\[ \llb A \to A \rrb^{A \coloneqq \text{И}, B \coloneqq \text{Л}} = f_\to(\llb A \rrb^{A \coloneqq \text{И}, B \coloneqq \text{Л}}, \llb A \rrb^{A \coloneqq \text{И}, B \coloneqq \text{Л}}) = f_\to(\color{blue} \text{И}\color{black} , \color{blue}\text{И}\color{black}) \color{blue}= \text{И} \]
, где \(f_\to\) определена так:
\begin{center}
\begin{tabular}{ll|l}
\(a\) & \(b\) & \(f_\to\)\\
\hline
И & И & И\\
И & Л & Л\\
Л & И & И\\
Л & Л & И\\
\end{tabular}
\end{center}
\end{examp}
\end{enumerate}
\subsection{Общезначимость}
\label{sec:orgf99440b}
\begin{examp}
\(\vDash \alpha\) --- \(\alpha\) общезначимо
\label{org0078f4f}
\end{examp}
\subsection{Выполнимость}
\label{sec:org5db5c37}
\subsection{Невыполнимость}
\label{sec:org7d3be04}
\subsection{Следование}
\label{sec:orge3ee7f1}
\begin{definition}
Следование: \(\Gamma \vDash \alpha\), если
\begin{itemize}
\item \(\Gamma = \gamma_1, \dots, \gamma_n\)
\item Всегда когда все \(\llb \gamma_i \rrb = \text{И}\), то \(\llb \alpha \rrb = \text{И}\)
\end{itemize}
\label{org1df6c41}
\end{definition}
\subsection{Корректность}
\label{sec:orgfe58268}
\begin{definition}
\sout{Теория} Исчисление высказываний корректна, если при любом \(\alpha\) из \(\vdash \alpha\) следует \(\vDash \alpha\)
\label{orgfbed2ab}
\end{definition}
\subsection{Полнота}
\label{sec:org17247de}
\begin{definition}
Исчисление полно, если при любом \(\alpha\) из \(\vDash \alpha\) следует \(\vdash \alpha\)
\label{org0d85fd2}
\end{definition}
\subsection{Противоречивость}
\label{sec:orgee8ce80}
\subsection{Теорема о дедукции}
\label{sec:orgaa4b5d4}
\begin{theorem}[о дедукции]
\(\Gamma, \alpha \vdash \beta\) \uline{тогда и только тогда, когда} \(\Gamma \vdash \alpha \to \beta\)
\label{org16b9213}
\end{theorem}
\subsection{Теорема о корректности}
\label{sec:orgb2c5a2d}
\begin{theorem}[о корректности]
Пусть \(\vdash \alpha\) \\
\uline{Тогда} \(\vDash \alpha\)
\label{org1047dec}
\end{theorem}
\subsection{Теорема о полноте ИВ}
\label{sec:org7cda326}
\begin{theorem}[о полноте]
Пусть \(\vDash \alpha\), тогда \(\vdash \alpha\)
\label{org3b9fe39}
\end{theorem}
\section{Интуиционистское исчисление высказываний}
\label{sec:org0f0389a}
\subsection{Закон исключенного третьего}
\label{sec:orgaa08da2}
\subsection{Закон снятия двойного отрицания}
\label{sec:org2d8e7a4}
\subsection{Закон Пирса}
\label{sec:org54b543f}
\subsection{ВНК-интерпретация логических связок}
\label{sec:orgc1c2b28}
\subsubsection{Интуиционистская логика}
\label{sec:org4fd3e20}
\(A \vee B\) --- плохо
\begin{examp}
Докажем: существует \(a, b\), что \(a, b \in \R \setminus \mathbb{Q}\), но \(a^b \in \mathbb{Q}\) \\
Пусть \(a = b = \sqrt{2}\). Рассмотрим \(\sqrt{2}^{\sqrt{2}} \in \R \setminus \mathbb{Q}\)
\begin{itemize}
\item Если да, то ОК
\item Если нет, то возьмем \(a = \sqrt{2}^{\sqrt{2}}, b = \sqrt{2}\), \(a^b = (\sqrt{2}^{\sqrt{2}})^{\sqrt{2}} = \sqrt{2}^{2} = 2\)
\end{itemize}
\end{examp}
\begin{defintion}
ВНК-интерпретация. \(\alpha, \beta\)
\begin{itemize}
\item \(\alpha \& \beta\) --- есть \(\alpha, \beta\)
\item \(\alpha \vee \beta\) --- есть \(\alpha\) либо \(\beta\) и мы знаем какое
\item \(\alpha \to \beta\) --- есть способ перестроить \(\alpha\) в \(\beta\)
\item \(\perp\) --- конструкция без построения \(\neg \alpha \equiv \alpha \to \perp\)
\end{itemize}
\end{defintion}
\subsection{Теорема Гливенко}
\label{sec:orgbef31f7}
\subsection{Решетка}
\label{sec:orga1d880e}
\begin{definition}
Фиксируем \(A\) \\
Частичный порядок --- антисимметричное, транзитивное, рефлексивное отношение \\
Линейный --- сравнимы любые 2 элемента \\
\begin{itemize}
\item \(a \le b \vee b \le a\)
\item \textbf{Наименьший элемент} \(S\) --- такой \(k \in S\), что если \(x \in S\), то \(k \le x\)
\item \textbf{Минимальный элемент} \(S\) --- такой \(k \in S\), что нет \(x \in S\), что \(x \le k\)
\end{itemize}
\label{orgb4c067a}
\end{definition}
\begin{definition}
\-
\begin{itemize}
\item \textbf{Множество верхних граней} \(a\) и \(b\): \(\{x \big| a \le x \& b \le x\}\)
\item \textbf{Множество нижних граней} \(a\) и \(b\): \(\{x \big| x \le a \& x \le b\}\)
\end{itemize}
\label{org89daca7}
\end{definition}
\begin{definition}
\-
\begin{itemize}
\item \textbf{\(a + b\)} --- нименьший элемент множества верхних граней
\item \textbf{\(a \cdot b\)} --- наибольший элемент множества нижних граней
\end{itemize}
\label{org5312866}
\end{definition}
\begin{definition}
\textbf{Решетка} = \(\langle A, \le \rangle\) --- структура, где для каждых \(a, b\) есть как \(a + b\), так и \(a \cdot b\), \\
т.е. \(a \in A, b \in B \implies a + b \in A\) и \(a \cdot b \in A\)
\label{org118aef2}
\end{definition}
\subsection{Дистрибутивная решетка}
\label{sec:org88037ff}
\begin{definition}
\textbf{Дистрибутивная решетка} если всегда  \(a \cdot (b + c) = a\codt b + a \cdot c\)
\label{org004cb90}
\end{definition}
\subsection{Импликативная решетка}
\label{sec:org8545684}
\begin{definition}
\textbf{Импликативная решетка} --- решетка, где для любых \(a, b\) есть \(a \to b\)
\label{orgc22fd32}
\end{definition}
\subsection{Алгебра Гейтинга}
\label{sec:org9f8d5e9}
\begin{definition}
\textbf{Псевдобулева алгебра (алгебра Гейтинга)} --- импликативная решетка с \(0\)
\label{org176cd26}
\end{definition}
\subsection{Булева алгебра}
\label{sec:orgc69eb7e}
\begin{definition}
\textbf{Булева алгебра} --- псевдобулева алгебра, такая что \(a + (a \to 0) = 1\)
\label{org7e00b01}
\end{definition}
\subsection{Геделева алгебра}
\label{sec:orgdd9f3a6}
\begin{definition}
Гёделева алгебра --- алгебра Гейтинга, такая что из \(\alpha + \beta = 1\) следует что \(\alpha = 1\) или \(\beta = 1\) \\
\label{orgad11f3d}
\end{definition}
\subsection{Операция \(\Gamma(A)\)}
\label{sec:org91bf212}
\begin{definition}
Пусть \(\A\) --- алгебра Гейтинга, тогда:
\begin{enumerate}
\item \(\Gamma(\A)\) \\
\begin{center}
\begin{tikzpicture}
\draw (-1, 0) circle[radius=0.5cm] node {\(\A\)};
\draw (1, 0) circle[radius=0.5cm] node {\(\A\)};
\node (0, 0) {\(\Rightarrow\)};
\draw (-1, 0.5) circle[radius=1pt,fill=black] node[above] {\(1\)};
\draw (1, 0.5) circle[radius=1pt,fill=black] node[above right] {\(\omega\)};
\draw (1, 1.5) circle[radius=1pt,fill=black] node[above] {\(1\)};
\draw (1, 1.5) -- (1, 0.5);
\end{tikzpicture}
\end{center}

Добавим новый элемент \(1_{\Gamma(\A)}\) перенеименуем \(1_\A\) в  \(\omega\)
\end{enumerate}
\label{org7bb5345}
\end{definition}
\subsection{Алгебра Линденбаума}
\label{sec:orga096b36}
\subsection{Свойство дизъюнктивности ИИВ}
\label{sec:orgfa96d89}
\begin{definition}
\textbf{Дизъюнктивность} ИИВ: \(\vdash \alpha \vee \beta\) влечет \(\vdash \alpha\) или \(\vdash \beta\)
\label{orgd754ffd}
\end{definition}
\subsection{Свойство нетабличности ИИВ}
\label{sec:orgdfc67d2}
\subsection{Модель Крипке, Вынужденность}
\label{sec:orgbd9a4a2}
\begin{defintion}
\-
\begin{enumerate}
\item \(W = \{W_i\}\) --- множество миров
\item частичный порядок(\(\succeq\))
\item отношение вынужденности: \(W_j \Vdash p_i\) \\
\((\Vdash)  \subseteq W \times \P\) \\
При этом, если \(W_j \Vdash p_i\) и \(W_j \preceq W_k\), то \(W_j \Vdash p\)
\end{enumerate}
\label{orgea0b0cc}
\end{defintion}
\section{Исчиление предикатов}
\label{sec:orga5bbbc9}
\subsection{Предикатные и функциональные символы}
\label{sec:org0940075}
\subsubsection{Исчисление предикатов}
\label{sec:orgec0474d}
\begin{definition}
Язык исчисление предикатов
\begin{itemize}
\item логические выражения ``предикаты``/``формулы``
\item предметные выражния ``термы``
\end{itemize}
\(\Theta\) --- метаперменные для термов \\
Термы:
\begin{itemize}
\item Атомы:
\begin{itemize}
\item \(a, b, c, d, \dots\) --- предметные переменные
\item \(x, y, z\) --- метапеременные для предметных перменных
\end{itemize}
\item Функциональные Символы
\begin{itemize}
\item \(f, g, h\) --- Функциональные символы(метаперемнные)
\item \(f(\Theta_1, \dots \Theta_n)\) --- применение функциональных символов
\end{itemize}
\item Логические выражения: \\
\color{gray}Если \(n = 0\), будем писать \(f, g\) --- без скобок\color{black}
\begin{itemize}
\item \(P\) --- метаперменные для предикатных символов
\item \(A, B, C\) --- предикатный символ
\item \(P(\Theta_1, \dots, \Theta_n)\) --- применение предикатных символов
\item \(\&, \vee, \neg, \to\) --- Cвязки
\item \(\forall x.\varphi\) и \(\exists x.\varphi\) --- кванторы \\
\color{gray}``<квантор> <переменная>.<выражение>``\color{black} \\
\end{itemize}
\end{itemize}
\end{definition}
\begin{enumerate}
\item Сокращение записи
\label{sec:org89292cc}
И.В + жадность \(\forall, \exists\) \\
Метавыражение:
\[ \forall x. \color{green}(\color{black}P(x) \& \color{green}(\color{black}\forall y. P(y) \color{green}))\color{black} \]
Квантор съедает все что дают, т.е. имеет минимальный приоритет. \\
Правильный вариант(настоящее выражние):
\[ \forall a. B(A) \& \forall b. B(b) \]
\end{enumerate}
\subsection{Константы и пропозициональные переменные}
\label{sec:org5ddc062}
\subsection{Свободные и связанные вхождения предметных переменных в формулу}
\label{sec:org34a6107}
\subsubsection{Вхождение}
\label{sec:orgbfdc2e6}
\begin{examp}
\[ (P(\underset{1}{x}) \vee Q(\underset{2}{x})) \to (R(\underset{3}{x}) \& (\underbrace{\forall \underset{4}{x}. P_1(\underset{5}{x})}_{\text{область }\forall\text{ по }x})) \]
1, 2, 3 --- свободные, 5 --- связанное, по пермененной 4
\end{examp}
\begin{examp}
\[ \underbrace{\forall x. \forall y. \forall x. \forall y. \forall x. P(x)}_{\text{область }\forall\text{ по }x} \]
Здесь \(x\) в \(P(x)\) связано. \(x\) не входит свободно в эту формулу, потому что нет свободных вхождений
\end{examp}
\begin{definition}
Переменная \(x\) входит свободно если существует свободное вхождение
\end{definition}
\begin{definition}
Вхождение свободно, если не связано
\end{definition}
Можно относится к свободно входящим перменным как с перменным из библиотеки, т.е. мы не имеем права их переименовывать
\begin{examp}
Некорректная формула
\begin{description}
\item[{\(\alpha_1\)}] \(x = 0 \to x = 0\)
\item[{\(\alpha_2\)}] \color{red}\((\exists x. x = 0) \to (x = 0)\) --- не доказано\color{black}
\item[{\(\alpha_2'\)}] \((\exists t. x = 0) \to (x = 0)\) --- (правило \(\exists\))
\end{description}
\end{examp}
\begin{examp}
\-
\begin{description}
\item[{\((n)\)}] \(x = 0 \to y = 0\) --- откуда то
\item[{\((n + 1)\)}] \((\exists x. x = 0) \to (y = 0)\) --- (правило \(\exists\))
\end{description}
\end{examp}
\subsubsection{Свободные подстановки}
\label{sec:org7ca4d51}
\begin{definition}
\(\Theta\) свободен для подстановки вместо \(x\) в \(\varphi\), если никакая свободная перменная в \(\Theta\) не станет связанной в \(\varphi[x := \Theta]\)
\end{definition}
\begin{definition}
\(\varphi[x := \Theta]\) --- "Заменить все свободные вхождения x в \(\varphi\) на \(\Theta\)"
\end{definition}
\begin{examp}
\[ (\forall x. \forall y. \forall x. P(x))[x := y] \equiv \forall x. \forall y. \forall x. P(x) \]
\end{examp}
\begin{examp}
\[ P(x) \vee \forall x. P(x)\ [x := y] \equiv P(y) \vee \forall x. P(y) \]
\end{examp}
\begin{examp}
\[ (\forall y. x = y)\ [x := \underbrace{y}_{\equiv \Theta}] \equiv \forall y. \underset{1}{y} = y\]
\(FV(\Theta) = \{y\}\) --- свободные перменные в \(\Theta\). Вхождение \(y\) с номером 1 стало связанным
\end{examp}
\begin{examp}
\[ P(x) \& \forall y. x = y\ [x := y + z] \equiv P(y + z) \& \forall y. \underset{1}{y} + z = y \]
Здесь при подстановке вхождение \(y\) с номером 1 cтало связанным. \(x\) --- библиотечная функция, переименовали \(x\) во что-то другое.
\end{examp}
\subsection{Свобода для подстановки, Правила вывода для кванторов, аксиомы исчисления предикатов для кванторов, оценки и модели в исчислении предикатов}
\label{sec:org4273f4d}
\subsubsection{Теория доказательств}
\label{sec:orga48abc5}
Все аксимомы И.В + M.P.
\begin{description}
\item[{(cхема 11)}] \((\forall x. \varphi) \to \varphi[x:=\Theta]\)
\item[{(схема 12)}] \(\varphi[x:=\Theta]\to \exists x. \varphi\)
\end{description}
Если \(\Theta\) свободен для подстановки вместо \(x\) в \(\varphi\).
\begin{definition}
\textbf{Свободен для подстановки} --- никакое свободное вхождение \(x\) в \(\Theta\) не станет связанным
\end{definition}
\begin{examp}
\-
\begin{minted}[frame=lines,linenos=true,mathescape]{c}
int y;
int f(int x) {
	x = y;
}
\end{minted}
Заменим \texttt{y := x}. Код сломается, т.к. у нас нет свобод для подстановки
\end{examp}
\begin{description}
\item[{(Правило \(\forall\))}] \[\frac{\varphi \to \psi}{\varphi \to \forall x. \psi}\]
\item[{(Правило \(\exists\))}] \[ \frac{\psi \to \varphi}{\exists x.\psi \to \varphi} \]
\end{description}
В обоих правилах \(x\) не входит свободно в \(\varphi\)
\begin{examp}
\[ \frac{x = 5 \to x^2 = 25}{x = 5 \to \forall x. x^2 = 25} \]
Между \(x\) и \(x^2\) была связь, мы ее разрушили. Нарушено ограничение
\end{examp}
\begin{examp}
\[ \exists y. x = y \]
\[ \forall x. \exists y. x = y \to \exists y. y + 1 = y \]
Делаем замену \texttt{x := y+1}. Нарушено требование свобод для подстановки. \(y\) входит в область действия квантора \(\exists\) и поэтому свободная переменная \(x\) стала связанная.
\end{examp}
\subsection{Теорема о дедукции для исчисления предикатов}
\label{sec:org1108256}
\begin{theorem}
Пусть задана \(\Gamma,\ \alpha,\beta\)
\begin{enumerate}
\item Если \(\Gamma, \alpha \vdash \beta\), то \(\Gamma \vdash \alpha \to \beta\), при условии, если \(b\) в доказательстве \(\Gamma, \alpha \to \beta\) не применялись правила для \(\forall, \exists\) по перменным, входяшим свободно в \(\alpha\)
\item Если \(\Gamma \vdash \alpha \to \beta\), то \(\Gamma, \alpha \vdash \beta\)
\end{enumerate}
\label{org7e06597}
\end{theorem}
\subsection{Теорема о корректности для исчисления предикатов}
\label{sec:org0fe561e}
\subsection{Полное множество (бескванторных) формул}
\label{sec:org1cc34bf}
\begin{definition}
\(\Gamma\) --- \textbf{непротиворечивое} множество формул, если \(\Gamma \not\vdash \alpha \& \neg \alpha\) ни при каком \(\alpha\)
\label{org49f0f53}
\end{definition}
\begin{definition}
Полное непротиворечивое замкнутых бескванторных формул --- такое, что для каждой замкнутой бескванторной формулы \(\alpha\): либо \(\alpha \in \Gamma\), либо \(\neg \alpha \in \Gamma\)
\label{org5eb2a14}
\end{definition}
\subsection{Модель для формулы}
\label{sec:orgeaa661b}
\subsubsection{Теория моделей}
\label{sec:org6720670}
Оценка формулы в исчислении предикатов:
\begin{enumerate}
\item Фиксируем \(D\) --- предметное множетво
\item Кажодму \(f_i(x_1, \dots, x_n)\) сопоставим функцию \(D^n \to D\)
\item Каждому \(P_j(x_1, \dots, x_m)\) сопоставим функцию(предикат) \(D^2 \to V\)
\item Каждой \(x_i\) сопоставим элемент из \(D\)
\end{enumerate}
\begin{examp}
\[\forall x.\forall y.\ E(x, y)\]
Чтобы определить формулу сначала определим \(D = \N\) 
\[ E(x, y) = \begin{cases}\text{И} & ,x = y \\ \text{Л} &, x\neq y\end{cases} \]
\begin{itemize}
\item \(\llbracket x \rrbracket = f_{x_i}\)
\item \(\llbracket \alpha \star \beta \rrbracket\) --- смотри ИИВ
\item \(\llbracket P_i(\Theta_1, \dots , \Theta_n) \rrbracket = f_{P_i}(\llbracket \Theta_1 \rrbracket, \dots, \llbracket \Theta_n \rrbracket)\)
\item \(\llbracket f_j(\Theta_1 , \dots, \Theta_n ) \rrbracket = f_{f_j}(\llbracket \Theta_1 \rrbracket, \dots, \llbracket \Theta_n \rrbracket)\)
\item \[ \llbracket \forall x. \varphi \rrbracket = \begin{cases} \text{И} & , \text{если } \llbracket \varphi \rrbracket^{f_x = k} = \text{И}\text{ при всех } k \in D  \\ \text{Л} &,\text{иначе}\end{cases} \]
\item \[ \llbracket \exists x.\varphi \rrbracket = \begin{cases} \text{И} &, \text{если } \llbracket \varphi \rrbracket^{f_x = k} = \text{И при некотором } k \in D \\ \text{Л} &,\text{иначе} \end{cases} \]
\end{itemize}
\[ \llbracket \forall x.\forall y.E(x, y) \rrbracket = \text{Л} \]
т.к. \(\llbracket E(x, y) \rrbracket^{x:=1,\ y:=2} = \text{Л}\)
\end{examp}
\newcommand{\colorboxed}[2]{\,\color{#1}\fbox{\color{black}#2}\color{black}\,}

\begin{examp}
\[ \forall \colorboxed{green}{\varepsilon > \colorboxed{blue}{0}}\ \exists N\ \forall \colorboxed{green}{\colorboxed{blue}{n} > \colorboxed{blue}{N}}\quad \colorboxed{green}{\colorboxed{blue}{|a_n - a|} < \colorboxed{blue}{\varepsilon}} \]
Синим отмечены функциональные конструкции(термы), зеленым предикатные
\[ \forall \varepsilon. (\varepsilon > 0) \to \exists N. \forall n. (n > N) \to (|a_n - a| < \varepsilon) \]
Обозначим:
\begin{itemize}
\item \((>)(a, b) = G(a, b)\) --- предикат
\item \(|\bullet|(a) = m_|(a)\)
\item \((-)(a, b) = m_-(a, b)\)
\item \(0() = m_0\)
\item \(a_\bullet(n) = m_a(n)\)
\end{itemize}
\[ \forall e. \colorboxed{green}{G(\colorboxed{blue}{e}, \colorboxed{blue}{m_0})} \to \exists n_0.\forall n. \colorboxed{green}{G(n, n_0)}\to \colorboxed{green}{G\bigg(e, \colorboxed{blue}{m_|\Big(m_- \big(m_a(n), a\big)\Big)}\bigg)} \]
\end{examp}
\subsection{Теорема Гёделя о полноте исчисления предикатов}
\label{sec:org1d28c9b}
\begin{theorem}[Геделя о полноте]
Если \(\Gamma\) --- полное неротиворечивое множество замкнутых(не бескванторных) фомул, то оно имеет модель
\label{org9222f6f}
\end{theorem}
\subsection{Следствие из теоремы Гёделя о полноте исчисления предикатов}
\label{sec:org818bc40}
\begin{corollary}
Пусть \(\vDash \alpha\), тогда \(\vdash \alpha\)
\label{org3398134}
\end{corollary}
\subsection{Неразрешимость}
\label{sec:org7a384a1}
\subsection{Исчисления предикатов (формулировка, что такое неразрешимость).}
\label{sec:orga6fd0cc}
\begin{theorem}
ИП неразрешимо
\label{org8dd2b4e}
\end{theorem}
\section{Арифметика и теории первого порядка}
\label{sec:org44800f2}
\subsection{Теория первого порядка}
\label{sec:org502c023}
\begin{definition}
\textbf{Теория I порядка} --- Исчесление предикатов + нелогические функции + предикатные символы + нелогические (математические) аксиомы.
\label{orgdfc51ca}
\end{definition}
\subsection{Модели и структуры теорий первого порядка}
\label{sec:org37c961a}
\subsection{Аксиоматика Пеано}
\label{sec:orgcf1b626}
\begin{definition}
Будем говорить, что \(N\) соответсвует \textbf{аксиоматике Пеано} если:
\begin{itemize}
\item задан \(('): N \to N\) --- инъективная функция (для разных элементов, разные значения)
\item задан \(0 \in N\): нет \(a \in N\), что \(a' = 0\)
\item если \(P(x)\) --- некоторое утверждение, зависящее от \(x \in N\), такое, что \(P(0)\) и всегда, когда \(P(x)\), также и \(P(x')\). Тогда \(P(x)\)
\end{itemize}
\label{org2602401}
\end{definition}
\subsection{Определение операций (сложение, умножение, возведение в степень)}
\label{sec:orgad34fc6}
\begin{definition}
\[ a + b = \begin{cases}
a & b = 0 \\
(a + c)' & b = c'
\end{cases}\]
\label{orgeb67664}
\end{definition}
\begin{definition}
\[ a \cdot b = \begin{cases}
0 & b = 0 \\
(a \cdot c) + a & b = c'
\end{cases}\]
\label{org9dfb769}
\end{definition}
\begin{definition}
\[ a^b = \begin{cases}
1 & b = 0 \\
(a^c)\cdot a & b = c'
\end{cases}\]
\label{org4977d07}
\end{definition}
\subsection{Формальная арифметика (язык, схема аксиом индукции и общая характеристика остальных аксиом).}
\label{sec:orgabf1577}
\subsubsection{Формальная арифметика}
\label{sec:orge5d2f10}
\begin{definition}
Исчесление предикатов:
\begin{itemize}
\item Функциональные символы:
\begin{itemize}
\item \(0\) --- 0-местный
\item \((')\) --- 1-местный
\item \((\cdot)\) --- 2-местный
\item \((+)\) --- 2-местный
\end{itemize}
\item \((=)\) --- 2-местный предикатный символ
\end{itemize}
Аксимомы:
\begin{enumerate}
\item \(a = b \to a' = b'\)
\item \(a = b \to a = c \to b = c\)
\item \(a' = b' \to a= b\)
\item \(\neg a' = 0\)
\item \(a + b' = (a + b)'\)
\item \(a + 0 = a\)
\item \(a\cdot 0 = 0\)
\item \(a\cdot b' = a\cdot b + a\)
\item Схема аксиом индукции:
\[ (\psi[x:=0])\&(\forall x. \psi \to (\psi[x:=x'])) \to \psi \]
\(x\) входит свободно в \(\psi\)
\end{enumerate}
\end{definition}
\beginproperty
\begin{property}
\[ ((a + 0 = a) \to (a + 0 = a) \to (a = a)) \]
\end{property}
\begin{proof}
\[ \forall a. \forall b. \forall c. a = b \to a = c \to b = c \]
\[ (\forall a. \forall b. \forall c. a = b \to a = c \to b = c) \to \forall b. \forall c. (a + 0 = b \to a + 0 = c \to b = c) \]
\[ \forall b. \forall c. a + 0 = b \to a + 0 = c\to b = c \]
\[ (\forall b. \forall c. a + 0 = b \to a + 0 = c \to b = c) \to \forall c.(a + 0 = a \to a + 0 = c \to a=c) \]
\[ \forall c. a + 0 = a \to a + 0 = c \to a = c \]
\[ (\forall c. a + 0 = a \to a + 0 = c \to a = c) \to a+0 = a \to a + 0 = a \to a= a \]
\[ a + 0  = a \to a + 0 = a \to a = a \]
\[ a + 0 = a \]
\[ a + 0 = a \to a = a \]
\[ a = a \]
\[ \forall b. \forall c. a = b \to a = c \to b = c \]
\[ (0 = 0 \to 0 = 0 \to 0 = 0) \]
\[ (\forall b. \forall c. a = b \to a = c\ to b = c) \to (0 = 0 \to 0 = 0 \to 0 = 0) \to \phi \]
\fixme
\end{proof}
\begin{definition}
\(\exists! x.\varphi(x) \equiv (\exists x. \varphi(x))\&\forall p.\forall q. \varphi(p)\&\varphi(q) \to p = q\) \\
Можно также записать \(\exists ! x.\neg \exists s. s' = x\) или \((\forall q.(\exists x. x' = q)\vee q= 0)\)
\end{definition}
\begin{definition}
\(a \le b\) --- сокращение для \(\exists n. a + n = b\)
\end{definition}
\begin{definition}
\[ \overline{n} = 0^{(n)}\]
\[ 0^{(n)} = \begin{cases}
0 & n = 0 \\
0^{(n - 1)'} & n > 0
\end{cases}\]
\end{definition}
\begin{definition}
\(W \subseteq \N_0^n\). \(W\) --- выразимое в формальной арифметике. отношение, если существует формула \(\omega\) со свободными переменными \(x_1,\dots,x_n\). Пусть \(k_1,\dots,k_n \in \N\)
\begin{itemize}
\item \((k_1,\dots,k_n) \in W\), тогда \(\vdash \omega[x_1:=\overline{k_1}, \dots, x_n := \overline{k_n}]\)
\item \((k_1,\dots,k_n) \not\in W\), тогда \(\vdash \neg \omega[x_1:=\overline{k_1},\dots,x_n:=\overline{k_n}]\)
\end{itemize}
\[ \omega[x_1:=\Theta_1,\dots,x_n:=\Theta_n] \equiv \omega(\Theta_1, \dots, \Theta_n) \]
\end{definition}
\begin{definition}
\(f: \N^n \to \N\) --- представим в формальной арифметике, если найдется \(\varphi\) --- фомула с \(n + 1\) свободными переменными \(k_1, \dots, k_{n + 1} \in \N\)
\begin{itemize}
\item \(f(k_1,\dots,k_n) = k_{n + 1}\), то \(\vdash \varphi(\overline{k_1},\dots,\overline{k_{n + 1}})\) \\
\item \(\vdash \exists! x.\varphi(\overline{k_1},\dots,\overline{k_n},x)\)
\end{itemize}
\end{definition}
\end{document}
