% Created 2021-03-19 Fri 15:00
% Intended LaTeX compiler: pdflatex

\documentclass[english]{article}
\usepackage[T1, T2A]{fontenc}
\usepackage[lutf8]{luainputenc}
\usepackage[english, russian]{babel}
\usepackage{minted}
\usepackage{graphicx}
\usepackage{longtable}
\usepackage{hyperref}
\usepackage{xcolor}
\usepackage{natbib}
\usepackage{amssymb}
\usepackage{stmaryrd}
\usepackage{amsmath}
\usepackage{caption}
\usepackage{mathtools}
\usepackage{amsthm}
\usepackage{tikz}
\usepackage{grffile}
\usepackage{extarrows}
\usepackage{wrapfig}
\usepackage{rotating}
\usepackage{placeins}
\usepackage[normalem]{ulem}
\usepackage{amsmath}
\usepackage{textcomp}
\usepackage{capt-of}

\usepackage{geometry}
\geometry{a4paper,left=2.5cm,top=2cm,right=2.5cm,bottom=2cm,marginparsep=7pt, marginparwidth=.6in}

 \usepackage{hyperref}
 \hypersetup{
     colorlinks=true,
     linkcolor=blue,
     filecolor=orange,
     citecolor=black,      
     urlcolor=cyan,
     }

\usetikzlibrary{decorations.markings}
\usetikzlibrary{cd}
\usetikzlibrary{patterns}
\usetikzlibrary{automata, arrows}

\newcommand\addtag{\refstepcounter{equation}\tag{\theequation}}
\newcommand{\eqrefoffset}[1]{\addtocounter{equation}{-#1}(\arabic{equation}\addtocounter{equation}{#1})}


\newcommand{\R}{\mathbb{R}}
\renewcommand{\C}{\mathbb{C}}
\newcommand{\N}{\mathbb{N}}
\newcommand{\rank}{\text{rank}}
\newcommand{\const}{\text{const}}
\newcommand{\grad}{\text{grad}}

\theoremstyle{plain}
\newtheorem{axiom}{Аксиома}
\newtheorem{lemma}{Лемма}
\newtheorem{manuallemmainner}{Лемма}
\newenvironment{manuallemma}[1]{%
  \renewcommand\themanuallemmainner{#1}%
  \manuallemmainner
}{\endmanuallemmainner}

\theoremstyle{remark}
\newtheorem*{remark}{Примечание}
\newtheorem*{solution}{Решение}
\newtheorem{corollary}{Следствие}[theorem]
\newtheorem*{examp}{Пример}
\newtheorem*{observation}{Наблюдение}

\theoremstyle{definition}
\newtheorem{task}{Задача}
\newtheorem{theorem}{Теорема}[section]
\newtheorem*{definition}{Определение}
\newtheorem*{symb}{Обозначение}
\newtheorem{manualtheoreminner}{Теорема}
\newenvironment{manualtheorem}[1]{%
  \renewcommand\themanualtheoreminner{#1}%
  \manualtheoreminner
}{\endmanualtheoreminner}
\captionsetup{justification=centering,margin=2cm}
\newenvironment{colored}[1]{\color{#1}}{}

\tikzset{->-/.style={decoration={
  markings,
  mark=at position .5 with {\arrow{>}}},postaction={decorate}}}
\makeatletter
\newcommand*{\relrelbarsep}{.386ex}
\newcommand*{\relrelbar}{%
  \mathrel{%
    \mathpalette\@relrelbar\relrelbarsep
  }%
}
\newcommand*{\@relrelbar}[2]{%
  \raise#2\hbox to 0pt{$\m@th#1\relbar$\hss}%
  \lower#2\hbox{$\m@th#1\relbar$}%
}
\providecommand*{\rightrightarrowsfill@}{%
  \arrowfill@\relrelbar\relrelbar\rightrightarrows
}
\providecommand*{\leftleftarrowsfill@}{%
  \arrowfill@\leftleftarrows\relrelbar\relrelbar
}
\providecommand*{\xrightrightarrows}[2][]{%
  \ext@arrow 0359\rightrightarrowsfill@{#1}{#2}%
}
\providecommand*{\xleftleftarrows}[2][]{%
  \ext@arrow 3095\leftleftarrowsfill@{#1}{#2}%
}
\makeatother
\author{Ilya Yaroshevskiy}
\date{\today}
\title{Лекция 6}
\hypersetup{
 pdfauthor={Ilya Yaroshevskiy},
 pdftitle={Лекция 6},
 pdfkeywords={},
 pdfsubject={},
 pdfcreator={Emacs 28.0.50 (Org mode )}, 
 pdflang={English}}
\begin{document}

\maketitle
\tableofcontents


\section{Исчисление предикатов}
\label{sec:org7992cd6}
\subsection{Расставление скобок}
\label{sec:org89fa83a}
Кванторы имеют наименьший приоритет
\begin{examp}
\[ \forall x. A \& B \& \forll y. C \& D \vee \exists z. E \]
\[ (\forall x. (A \& B \& \forall y. (C \& D \vee \exists z. (E)))) \]
\end{examp}
Еще раз про правила только со скобками
\begin{enumerate}
\item \[ \frac{\varphi \to \psi}{(\exists. \varphi) \to \psi} \]
\item \[ \frac{\psi \to \varphi}{\psi \to (\forall x. \varphi)} \]
\end{enumerate}
\begin{examp}
\[ \frac{\varphi \to \psi}{\exists x.(\varphi \to \psi)} \]
--- можно доказать, но это не правило вывода для \(\exists\)
\end{examp}
\begin{definition}
\-
\(\alpha_1, \dots, \alpha_n\) --- доказательство
\begin{itemize}
\item если \(\alpha_i\) --- аксимома
\item либо существует \(j, k < i\), что \(\alpha_k = \alpha_j \to \alpha_i\)
\item либо существует \(\alpha_j:\ \alpha_j = \varphi \to \psi\) и \(\alpha_i = (\exists x. \varphi) \to \psi\) причем \(x\) не входит свободно в \(\psi\)
\item либо существует \(j: \alpha_j = \psi \to \varphi\) и \(\alpha_i = \psi \to \forall x. \varphi\) причем \(x\) не входит свободно в \(\psi\)
\end{itemize}
\end{definition}
\subsection{Вхождение}
\label{sec:orgf61cd9e}
\begin{examp}
\[ (P(\underset{1}{x}) \vee Q(\underset{2}{x})) \to (R(\underset{3}{x}) \& (\underbrace{\forall \underset{4}{x}. P_1(\underset{5}{x})}_{\text{область }\forall\text{ по }x})) \]
1, 2, 3 --- свободные, 5 --- связанное, по пермененной 4
\end{examp}
\begin{examp}
\[ \underbrace{\forall x. \forall y. \forall x. \forall y. \forall x. P(x)}_{\text{область }\forall\text{ по }x} \]
Здесь \(x\) в \(P(x)\) связано. \(x\) не входит свободно в эту формулу, потому что нет свободных вхождений
\end{examp}
\begin{definition}
Переменная \(x\) входит свободно если существует свободное вхождение
\end{definition}
\begin{definition}
Вхождение свободно, если не связано
\end{definition}
Можно относится к свободно входящим перменным как с перменным из библиотеки, т.е. мы не имеем права их переименовывать
\begin{examp}
Некорректная формула
\begin{description}
\item[{\(\alpha_1\)}] \(x = 0 \to x = 0\)
\item[{\(\alpha_2\)}] \color{red}\((\exists x. x = 0) \to (x = 0)\) --- не доказано\color{black}
\item[{\(\alpha_2'\)}] \((\exists t. x = 0) \to (x = 0)\) --- (правило \(\exists\))
\end{description}
\end{examp}
\begin{examp}
\-
\begin{description}
\item[{\((n)\)}] \(x = 0 \to y = 0\) --- откуда то
\item[{\((n + 1)\)}] \((\exists x. x = 0) \to (y = 0)\) --- (правило \(\exists\))
\end{description}
\end{examp}
\subsection{Свободные подстановки}
\label{sec:org8c6e18f}
\begin{definition}
\(\Theta\) свободен для подстановки вместо \(x\) в \(\varphi\), если никакая свободная перменная в \(\Theta\) не станет связанной в \(\varphi[x := \Theta]\)
\end{definition}
\begin{definition}
\(\varphi[x := \Theta]\) --- "Заменить все свободные вхождения x в \(\varphi\) на \(\Theta\)"
\end{definition}
\begin{examp}
\[ (\forall x. \forall y. \forall x. P(x))[x := y] \equiv \forall x. \forall y. \forall x. P(x) \]
\end{examp}
\begin{examp}
\[ P(x) \vee \forall x. P(x)\ [x := y] \equiv P(y) \vee \forall x. P(y) \]
\end{examp}
\begin{examp}
\[ (\forall y. x = y)\ [x := \underbrace{y}_{\equiv \Theta}] \equiv \forall y. \underset{1}{y} = y\]
\(FV(\Theta) = \{y\}\) --- свободные перменные в \(\Theta\). Вхождение \(y\) с номером 1 стало связанным
\end{examp}
\begin{examp}
\[ P(x) \& \forall y. x = y\ [x := y + z] \equiv P(y + z) \& \forall y. \underset{1}{y} + z = y \]
Здесь при подстановке вхождение \(y\) с номером 1 cтало связанным. \(x\) --- библиотечная функция, переименовали \(x\) во что-то другое.
\end{examp}
\subsection{Пример доказательства}
\label{sec:orgb9aa50d}
\begin{lemma}
Пусть \(\vdash \alpha\). Тогда \(\vdash \forall x. \alpha\)
\end{lemma}
\begin{proof}
\-
\begin{enumerate}
\item Т.к. \(\vdash \alpha\), то существует \(\gamma_1, \dots, \gamma_2: \gamma_n = \alpha\)
\[ \begin{matrix}
   (1) & \gamma_1 &  \\
   \vdots & \vdots &  \\
   (n) & \gamma_n (\equiv \alpha) &  \\
   (n + 1) & A\& A \to A & \text{(акс)} \\
   (n + 2) & \alpha \to ((A \& A \to A) \to \alpha) & \text{(акс)} \\
   (n + 3) & (A \& A \to A) \to \alpha & (\text{M.P } n, n + 2) \\
   (n + 4) & (A \& A \to A) \to \forall x.\alpha & (\text{введение }\forall\ n + 3) \\
   (n + 5) & \forall x. \alpha & (\text{M.P. } n + 1, n + 4)
   \end{matrix} \]
\end{enumerate}
\end{proof}
\subsection{Теорема о дедукции}
\label{sec:org9752186}
\begin{theorem}
Пусть задана \(\Gamma,\ \alpha,\beta\)
\begin{enumerate}
\item Если \(\Gamma, \alpha \vdash \beta\), то \(\Gamma \vdash \alpha \to \beta\), при условии, если \(b\) в доказательстве \(\Gamma, \alpha \to \beta\) не применялись правила для \(\forall, \exists\) по перменным, входяшим свободно в \(\alpha\)
\item Если \(\Gamma \vdash \alpha \to \beta\), то \(\Gamma, \alpha \vdash \beta\)
\end{enumerate}
\end{theorem}
\subsubsection{{\bfseries\sffamily TODO} Доказательсво}
\label{sec:org725cd63}
\end{document}
