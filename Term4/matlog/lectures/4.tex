% Created 2021-03-11 Thu 22:26
% Intended LaTeX compiler: pdflatex
\documentclass[english]{article}
\usepackage[T1, T2A]{fontenc}
\usepackage[lutf8]{luainputenc}
\usepackage[english, russian]{babel}
\usepackage{minted}
\usepackage{graphicx}
\usepackage{longtable}
\usepackage{hyperref}
\usepackage{xcolor}
\usepackage{natbib}
\usepackage{amssymb}
\usepackage{stmaryrd}
\usepackage{amsmath}
\usepackage{caption}
\usepackage{mathtools}
\usepackage{amsthm}
\usepackage{tikz}
\usepackage{grffile}
\usepackage{extarrows}
\usepackage{wrapfig}
\usepackage{rotating}
\usepackage{placeins}
\usepackage[normalem]{ulem}
\usepackage{amsmath}
\usepackage{textcomp}
\usepackage{capt-of}
\usepackage{stmaryrd}

\usepackage{geometry}
\geometry{a4paper,left=2.5cm,top=2cm,right=2.5cm,bottom=2cm,marginparsep=7pt, marginparwidth=.6in}

 \usepackage{hyperref}
 \hypersetup{
     colorlinks=true,
     linkcolor=blue,
     filecolor=orange,
     citecolor=black,      
     urlcolor=cyan,
     }

\usetikzlibrary{decorations.markings}
\usetikzlibrary{cd}
\usetikzlibrary{patterns}

\newcommand\addtag{\refstepcounter{equation}\tag{\theequation}}
\newcommand{\eqrefoffset}[1]{\addtocounter{equation}{-#1}(\arabic{equation}\addtocounter{equation}{#1})}


\newcommand{\R}{\mathbb{R}}
\renewcommand{\C}{\mathbb{C}}
\newcommand{\N}{\mathbb{N}}
\newcommand{\rank}{\text{rank}}
\newcommand{\const}{\text{const}}
\newcommand{\grad}{\text{grad}}

\theoremstyle{plain}
\newtheorem{axiom}{Аксиома}
\newtheorem{lemma}{Лемма}
\newtheorem{manuallemmainner}{Лемма}
\newenvironment{manuallemma}[1]{%
  \renewcommand\themanuallemmainner{#1}%
  \manuallemmainner
}{\endmanuallemmainner}

\theoremstyle{remark}
\newtheorem*{remark}{Примечание}
\newtheorem*{solution}{Решение}
\newtheorem{corollary}{Следствие}[theorem]
\newtheorem*{examp}{Пример}
\newtheorem*{observation}{Наблюдение}

\theoremstyle{definition}
\newtheorem{task}{Задача}
\newtheorem{theorem}{Теорема}[section]
\newtheorem*{definition}{Определение}
\newtheorem*{symb}{Обозначение}
\newtheorem{manualtheoreminner}{Теорема}
\newenvironment{manualtheorem}[1]{%
  \renewcommand\themanualtheoreminner{#1}%
  \manualtheoreminner
}{\endmanualtheoreminner}
\captionsetup{justification=centering,margin=2cm}
\newenvironment{colored}[1]{\color{#1}}{}

\tikzset{->-/.style={decoration={
  markings,
  mark=at position .5 with {\arrow{>}}},postaction={decorate}}}
\makeatletter
\newcommand*{\relrelbarsep}{.386ex}
\newcommand*{\relrelbar}{%
  \mathrel{%
    \mathpalette\@relrelbar\relrelbarsep
  }%
}
\newcommand*{\@relrelbar}[2]{%
  \raise#2\hbox to 0pt{$\m@th#1\relbar$\hss}%
  \lower#2\hbox{$\m@th#1\relbar$}%
}
\providecommand*{\rightrightarrowsfill@}{%
  \arrowfill@\relrelbar\relrelbar\rightrightarrows
}
\providecommand*{\leftleftarrowsfill@}{%
  \arrowfill@\leftleftarrows\relrelbar\relrelbar
}
\providecommand*{\xrightrightarrows}[2][]{%
  \ext@arrow 0359\rightrightarrowsfill@{#1}{#2}%
}
\providecommand*{\xleftleftarrows}[2][]{%
  \ext@arrow 3095\leftleftarrowsfill@{#1}{#2}%
}
\makeatother
\author{Ilya Yaroshevskiy}
\date{\today}
\title{Лекция 4}
\hypersetup{
 pdfauthor={Ilya Yaroshevskiy},
 pdftitle={Лекция 4},
 pdfkeywords={},
 pdfsubject={},
 pdfcreator={Emacs 28.0.50 (Org mode )}, 
 pdflang={English}}
\begin{document}

\maketitle
\tableofcontents

\renewcommand{\P}{\mathcal{P}}
\newcommand{\A}{\mathcal{A}}
\newcommand{\L}{\mathcal{L}}
\newcommand{\B}{\mathcal{B}}


\begin{definition}
\textbf{Предпорядок} --- транзитивное, рефлексивнре
\end{definition}
\begin{definition}
\textbf{Отношение порядка} (частичный) --- антисимметричное, транзитивное, рефлексивное
\end{definition}
\begin{definition}
\textbf{Линейный порядок} --- порядок в котором \(a \preceq b\) или \(b \preceq a\)
\end{definition}
\begin{definition}
\textbf{Полный порядок} --- линейный, каждое подмножество имеет наименьший элемент. 
\end{definition}
\begin{examp}
\(\N\) --- вполне упорядоченное множество
\end{examp}
\begin{examp}
\(\R\) --- не вполне упорядоченной множество
\begin{itemize}
\item \((0, 1)\) не имееи наименььшего
\item \(\R\) не имеет наименьшего
\end{itemize}
\end{examp}
\section{Табличные модели}
\label{sec:org98800a3}
\begin{definition}
Назовем модель \textbf{табличной} для ИИВ:
\begin{itemize}
\item \(V\) --- множество истинностных значений \\
\(f_\to,f_\&, f_V: V^2 \to V\), \(f_\neg: V \to V\) \\
Выделенные значения \(T \in V\) \\
\(\llbracketp+i\rrbracket \in V\) \(f_p : p_i \to V\)
\item \(p_i = f_\P(p_i)\) \\
\(\llbracket\alpha \star \beta\rrbracket = f_\star(\llbracket\alpha\rrbracket, \llbracket\beta\rrbracket)\) \\
\(\llbracket\neg \alpha\rrbracket = f_\neg(\llbracket\alpha\rrbracket)\)
\end{itemize}
\sout{Если \(\vdash \alpha\), то} \(\vDash \alpha\) означает, что \(\llbracket\alpha\rrbracket = T\), при любой \(f_\P\)
\end{definition}
\begin{definition}
Конечная модель: модель где \(V\) --- конечно
\end{definition}
\begin{theorem}
У ИИВ не существует полной табличной модели
\end{theorem}
\section{Модели Крипке}
\label{sec:orgf2d3b60}
\begin{center}
\begin{tikzpicture}
\node at (0,0) (A) {\( P = NP? \)};
\node at (2, 2) (B) {все банки лопнут, RSA сломают!!!};
\node at (2, -2) (C) {RSA устоит};
\draw[->] (A) -- node[above] {\(+\)} (B);
\draw[->] (A) -- node[below] {\(-\)} (C);
\end{tikzpicture}
\end{center}
\begin{defintion}
\-
\begin{enumerate}
\item \(W = \{W_i\}\) --- множество миров
\item частичный порядок(\(\succeq\))
\item отношение вынужденности: \(W_j \Vdash p_i\) \\
\((\Vdash)  \subseteq W \times \P\) \\
При этом, если \(W_j \Vdash p_i\) и \(W_j \preceq W_k\), то \(W_j \Vdash p\)
\end{enumerate}
\end{defintion}
\begin{definition}
\-
\begin{enumerate}
\item \(W_i \Vdash \alpha\) и \(W_i \Vdash \beta\), тогда (и только тогда) \(W_i \Vdash \alpha \& \beta\) \\
\item \(W_i \Vdash \alpha\) или \(W_i \Vdash \beta\), то \(W_i \Vdash \alpha \vee \beta\)
\item Пусть во всех \(W_i \preceq W_j\) всегда когда \(W_j \Vdash \alpha\) имеет место \(W_j \Vdash \beta\) \\
Тогда \(W_i \Vdash \alpha \to \beta\)
\item \(W_i \Vdash \alpha\) --- \(\alpha\) не вынуждено нигде, начиная с \(W_i\):
\(W_i \preceq W_j\), то \(W_j \not\Vdash \alpha\)
\end{enumerate}
\end{definition}
\begin{theorem}
Если \(W_i \Vdash \alpha\) и \(W_i \preceq W_j\), то \(W_j \Vdash \alpha\)
\end{theorem}
\begin{definition}
Если \(W_i \Vdash \alpha\) при всех \(W_i \in W\), то \(\vDash \alpha\)
\end{definition}
\begin{theorem}
ИИВ корректна в модели Крипке
\end{theorem}
\begin{proof}
\begin{enumerate}
\item \(\langle W, \Omega \rangle\) --- топология, где \(\Omega = \{w \subseteq W | \text{если }W_i \in w,\ W_i \preceq W_j,\text{ то } W_j \in w\}\) \\
\item \(\{W_k | W_k \Vdash p_j\}\) --- открытое множество \\
Примем \(\llbracket p_j \rrbracket = \{W_k | W_k \Vdash p_j\}\) \\
Аналогично \(\llbracket \alpha \rrbracket = \{W_k | W_k \Vdash \alpha\}\)
\end{enumerate}
\end{proof}
\section{Доказательство нетабличности}
\label{sec:org3547f77}
Пусть существует конечная табличная модель \(|V| = n\)
\[ \varphi_n =  \bigvee_{\substack{1 \le i, j \le n + 1 \\ i \neq j}} (p_i \to p_j \&p_j \to p_i)\]
\begin{enumerate}
\item \(\not\vdash\varphi\)
\begin{center}
\begin{tikzpicture}
\node[anchor=west] at (0, 0) (A) {\(W_0\)};
\node[anchor=west] at (1, 2) (B) {\(W_1\)};
\node[anchor=west] at (1, 1) (C) {\(W_2\)};
\node[anchor=west] at (1, 0) (D) {\(\vdots\)};
\node[anchor=west] at (1, -1) (E) {\(W_{n + 1}\)};
\draw[->] (A) -- (B);
\draw[->] (A) -- (C);
\draw[->] (A) -- (E);
\node[anchor=west] at (2, 2) {\(p_1\)};
\node[anchor=west] at (2, 1) {\(p_2\)};
\node[anchor=west] at (2, -1) {\(p_{n + 1}\)};
\end{tikzpicture}
\end{center}
\[ W_1 \not\Vdash (p_i \to p_k)\&(p_k\to p_1),\ k\neq 1 \]
Значит \[ \not\vDash (p_i\to p_j)\&(p_j\to p_i) \]
\[ \not\vDash \bigvee (p_i\to p_j)\&(p_j\to p_i) \]
\[ \not\vdash\varphi_n \]
\item \(\vDash_V \varphi_n\): по признаку Дирихле найдутся \(i\neq j:\llbracket p_i \rrbracket = \llbracket p_j \rrbracket\) \\
\(\llbracket p_i \to p_j \rrbracket = \text{И}\) и \(\llbracket \varphi_n \rrbracket = \text{И}\) \\
Значит \(\vdash \varphi_n\) --- противоречие
\end{enumerate}
\begin{definition}
\textbf{Дизъюнктивность} ИИВ: \(\vdash \alpha \vee \beta\) влечет \(\vdash \alpha\) или \(\vdash \beta\)
\end{definition}
\begin{definition}
Гёделева алгебра --- алгебра Гейтинга, такая что из \(\alpha + \beta = 1\) следует что \(\alpha = 1\) или \(\beta = 1\) \\
\end{definition}
\begin{definition}
Пусть \(\A\) --- алгебра Гейтинга, тогда:
\begin{enumerate}
\item \(\Gamma(\A)\) \\
\begin{center}
\begin{tikzpicture}
\draw (-1, 0) circle[radius=0.5cm] node {\(\A\)};
\draw (1, 0) circle[radius=0.5cm] node {\(\A\)};
\node (0, 0) {\(\Rightarrow\)};
\draw (-1, 0.5) circle[radius=1pt,fill=black] node[above] {\(1\)};
\draw (1, 0.5) circle[radius=1pt,fill=black] node[above right] {\(\omega\)};
\draw (1, 1.5) circle[radius=1pt,fill=black] node[above] {\(1\)};
\draw (1, 1.5) -- (1, 0.5);
\end{tikzpicture}
\end{center}

Добавим новый элемент \(1_{\Gamma(\A)}\) перенеименуем \(1_\A\) в  \(\omega\)
\end{enumerate}
\end{definition}
\begin{theorem}
\-
\begin{itemize}
\item \(\Gamma(\A)\) --- алгебра Гейтинга
\item \(\Gamma(\A)\) --- Геделева
\end{itemize}
\end{theorem}
\begin{definition}
\textbf{Гомоморфизм} алгебр Гейтинга \\
\begin{itemize}
\item \(\varphi: \A \to \B\)
\item \(\varphi(a \star b) = \varphi(a)\star\varphi(b)\)
\item \(\varphi(1_\A) = 1_\B\)
\item \(\varphi(0_\A) = 0_\B\)
\end{itemize}
\end{definition}
\begin{theorem}
\(a \le b\), то \(\varphi(a) \le \varphi(b)\)
\end{theorem}
\begin{definition}
\-
\begin{itemize}
\item \(\alpha\) --- формула ИИВ
\item \(f, g\): оценки ИИВ
\item \(f\): ИИВ \(\to\) \(\A\)
\item \(g\): ИИВ \(\to\) \(\B\)
\end{itemize}
\(\varphi\) согласованы \(f, g\), если \(\varphi(f(\alpha)) = g(\alpha)\)
\end{definition}
\begin{theorem}
если \(\varphi: \A \to \B\) согласована с \(f, g\) и оценка \(\llbracket \alpha \rrbracket_g \neq 1_\B\), то \(\llbracket \alpha \rrbracket_f \neq 1_\A\)
\end{theorem}
\begin{theorem}
ИИВ дизъюнктивно
\end{theorem}
\begin{proof}
Рассмторим алгебру Линденбаума: \(\mathcal{L}\) \\
Рассмотрим \(\Gamma(\mathcal{L})\) \\
\begin{itemize}
\item \(\varphi: \Gamma(\mathcal{L}) \to \mathcal{L}\)
\end{itemize}
\[ \varphi(x) = \begin{cases}1_\mathcal{L} & ,\substack{x =\omega \\ x = 1_{\Gamma(\mathcal{L})}} \\ x & , \text{иначе}\end{cases} \] 
\(\varphi\) --- гомоморфизм \\
Пусть \(\vdash \alpha \vee \beta\), тогда \(\llbracket \alpha \vee \beta \rrbracket_{\Gamma(\mathcal{L})} = 1_{\Gamma(\mathcal{L})}\) \\
\(\llbracket \alpha + \beta \rrbracket = 1\), и т.к. \(\Gamma(\mathcal{L})\) --- Геделева то \(\llbracket \alpha \rrbracket = 1\) или \(\llbracket \beta \rrbracket = 1\) \\
Пусть \(\not \vdash \alpha\) и \(\not \vdash \beta\), тогда \(\varphi(\llbracket \alpha \rrbracket) \neq 1_\mathcal{L}\) и \(\varphi(\llbracket \beta \rrbracket) \neq 1_\mathcal{L}\), т.е. \(\llbracket \alpha \rrbracket_\mathcal{L} \neq 1_\mathcal{L}\) и \(\llbracket \beta \rrbracket_\mathcal{L} \neq 1_\mathcal{L}\), тогда \(\llbracket \alpha \rrbracket_{\Gamma(\mathcal{L})} \neq 1_{\Gamma(\mathcal{L})}\) и \(\llbracket \beta \rrbracket_{\Gamma(\mathcal{L})} \neq 1_{\Gamma(\mathcal{L})}\) \(\Rightarrow\) Противоречие
\end{proof}
\end{document}
