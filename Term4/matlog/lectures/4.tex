% Created 2022-06-11 Sat 01:27
% Intended LaTeX compiler: pdflatex

\documentclass[english]{article}
\usepackage[T1, T2A]{fontenc}
\usepackage[lutf8]{luainputenc}
\usepackage[english, russian]{babel}
\usepackage{minted}
\usepackage{graphicx}
\usepackage{longtable}
\usepackage{hyperref}
\usepackage{xcolor}
\usepackage{natbib}
\usepackage{amssymb}
\usepackage{stmaryrd}
\usepackage{amsmath}
\usepackage{caption}
\usepackage{mathtools}
\usepackage{amsthm}
\usepackage{tikz}
\usepackage{fancyhdr}
\usepackage{lastpage}
\usepackage{titling}
\usepackage{grffile}
\usepackage{extarrows}
\usepackage{wrapfig}
\usepackage{algorithm}
\usepackage{algorithmic}
\usepackage{lipsum}
\usepackage{rotating}
\usepackage{placeins}
\usepackage[normalem]{ulem}
\usepackage{amsmath}
\usepackage{textcomp}
\usepackage{svg}
\usepackage{capt-of}
\usepackage{stmaryrd}

\usepackage{geometry}
\geometry{a4paper,left=2.5cm,top=2cm,right=2.5cm,bottom=2cm,marginparsep=7pt, marginparwidth=.6in}
 \usepackage{hyperref}
 \hypersetup{
     colorlinks=true,
     linkcolor=blue,
     filecolor=orange,
     citecolor=black,      
     urlcolor=cyan,
     }

\usetikzlibrary{decorations.markings}
\usetikzlibrary{cd}
\usetikzlibrary{patterns}
\usetikzlibrary{automata, arrows}

\newcommand\addtag{\refstepcounter{equation}\tag{\theequation}}
\newcommand{\eqrefoffset}[1]{\addtocounter{equation}{-#1}(\arabic{equation}\addtocounter{equation}{#1})}
\newcommand{\llb}{\llbracket}
\newcommand{\rrb}{\rrbracket}


\newcommand{\R}{\mathbb{R}}
\renewcommand{\C}{\mathbb{C}}
\newcommand{\N}{\mathbb{N}}
\newcommand{\A}{\mathfrak{A}}
\newcommand{\B}{\mathfrak{B}}
\newcommand{\rank}{\mathop{\rm rank}\nolimits}
\newcommand{\const}{\var{const}}
\newcommand{\grad}{\mathop{\rm grad}\nolimits}
\newcommand{\custombracket}[3]{\left#1 #3 \right#2}
\newcommand{\custombracketsame}[2]{\custombracket{#1}{#1}{#2}}
\newcommand{\pair}[1]{\custombracket{\langle}{\rangle}{#1}}
\newcommand{\eval}[1]{\custombracket{\llbracket}{\rrbracket}{#1}}

\newcommand{\todo}{{\color{red}\fbox{\text{Доделать}}}}
\newcommand{\fixme}{{\color{red}\fbox{\text{Исправить}}}}

\newcounter{propertycnt}
\setcounter{propertycnt}{1}
\newcommand{\beginproperty}{\setcounter{propertycnt}{1}}

\theoremstyle{plain}
\newtheorem{propertyinner}{\protect\propertyname}
\newenvironment{property}{
  \renewcommand\thepropertyinner{\arabic{propertycnt}}
  \propertyinner
}{\endpropertyinner\stepcounter{propertycnt}}
\newtheorem{axiom}{\protect\axiomname}
\newtheorem{lemma}{\protect\lemmaname}
\newtheorem*{statement}{\protect\statementname}
\newtheorem{manuallemmainner}{\protect\lemmaname}
\newenvironment{manuallemma}[1]{%
  \renewcommand\themanuallemmainner{#1}%
  \manuallemmainner
}{\endmanuallemmainner}

\theoremstyle{remark}
\newtheorem*{remark}{\protect\remarkname}
\newtheorem*{solution}{\protect\solutionname}
\newtheorem{corollary}{\protect\corollaryname}[theorem]
\newtheorem*{examp}{\protect\exampname}
\newtheorem*{observation}{\protect\observationname}

\theoremstyle{definition}
\newtheorem{task}{\protect\taskname}
\newtheorem{theorem}{\protect\theoremname}[section]
\newtheorem*{definition}{\protect\definitionname}
\newtheorem*{symb}{\protect\symbname}
\newtheorem{manualtheoreminner}{\protect\theoremname}
\newenvironment{manualtheorem}[1]{%
  \renewcommand\themanualtheoreminner{#1}%
  \manualtheoreminner
}{\endmanualtheoreminner}

\newtheoremstyle{colon}%
{}
{}
{}%bodyfont
{}%indent
{\bfseries}%headfont
{:}%head punctuation
{ }%space after head
{}
\theoremstyle{colon}
\newtheorem*{answer}{\protect\answername}

\newcommand{\propertyname}{}
\newcommand{\axiomname}{}
\newcommand{\lemmaname}{}
\newcommand{\statementname}{}
\newcommand{\remarkname}{}
\newcommand{\solutionname}{}
\newcommand{\corollaryname}{}
\newcommand{\exampname}{}
\newcommand{\observationname}{}
\newcommand{\taskname}{}
\newcommand{\theoremname}{}
\newcommand{\definitionname}{}
\newcommand{\symbname}{}
\newcommand{\answername}{}
\addto\captionsrussian{%
  \renewcommand{\propertyname}{Свойство}%
  \renewcommand{\axiomname}{Аксиома}%
  \renewcommand{\lemmaname}{Лемма}%
  \renewcommand{\statementname}{Утверждение}%
  \renewcommand{\remarkname}{Замечание}%
  \renewcommand{\solutionname}{Решение}%
  \renewcommand{\corollaryname}{Следствие}%
  \renewcommand{\exampname}{Пример}%
  \renewcommand{\observationname}{Наблюдение}%
  \renewcommand{\taskname}{Задача}%
  \renewcommand{\theoremname}{Теорема}%
  \renewcommand{\definitionname}{Определение}%
  \renewcommand{\symbname}{Обозначение}%
  \renewcommand{\answername}{Ответ}%
}
\addto\captionsenglish{%
  \renewcommand{\propertyname}{Property}%
  \renewcommand{\axiomname}{Axiom}%
  \renewcommand{\lemmaname}{Lemma}%
  \renewcommand{\statementname}{Statement}%
  \renewcommand{\remarkname}{Remark}%
  \renewcommand{\solutionname}{Solution}%
  \renewcommand{\corollaryname}{Corollary}%
  \renewcommand{\exampname}{Example}%
  \renewcommand{\observationname}{Observation}%
  \renewcommand{\taskname}{Task}%
  \renewcommand{\theoremname}{Theorem}%
  \renewcommand{\definitionname}{Definition}%
  \renewcommand{\answername}{Answer}%
}


\captionsetup{justification=centering,margin=2cm}
\newenvironment{colored}[1]{\color{#1}}{}

\tikzset{->-/.style={decoration={
  markings,
  mark=at position .5 with {\arrow{>}}},postaction={decorate}}}
\makeatletter
\newcommand*{\relrelbarsep}{.386ex}
\newcommand*{\relrelbar}{%
  \mathrel{%
    \mathpalette\@relrelbar\relrelbarsep
  }%
}
\newcommand*{\@relrelbar}[2]{%
  \raise#2\hbox to 0pt{$\m@th#1\relbar$\hss}%
  \lower#2\hbox{$\m@th#1\relbar$}%
}
\providecommand*{\rightrightarrowsfill@}{%
  \arrowfill@\relrelbar\relrelbar\rightrightarrows
}
\providecommand*{\leftleftarrowsfill@}{%
  \arrowfill@\leftleftarrows\relrelbar\relrelbar
}
\providecommand*{\xrightrightarrows}[2][]{%
  \ext@arrow 0359\rightrightarrowsfill@{#1}{#2}%
}
\providecommand*{\xleftleftarrows}[2][]{%
  \ext@arrow 3095\leftleftarrowsfill@{#1}{#2}%
}
\makeatother

\newenvironment{rualgo}[1][]
  {\begin{algorithm}[#1]
     \selectlanguage{russian}%
     \floatname{algorithm}{Алгоритм}%
     \renewcommand{\algorithmicif}{{\color{red}\textbf{если}}}%
     \renewcommand{\algorithmicthen}{{\color{red}\textbf{тогда}}}%
     \renewcommand{\algorithmicelse}{{\color{red}\textbf{иначе}}}%
     \renewcommand{\algorithmicend}{{\color{red}\textbf{конец}}}%
     \renewcommand{\algorithmicfor}{{\color{red}\textbf{для}}}%
     \renewcommand{\algorithmicto}{{\color{red}\textbf{до}}}%
     \renewcommand{\algorithmicdo}{{\color{red}\textbf{делать}}}%
     \renewcommand{\algorithmicwhile}{{\color{red}\textbf{пока}}}%
     \renewcommand{\algorithmicrepeat}{{\color{red}\textbf{повторять}}}%
     \renewcommand{\algorithmicuntil}{{\color{red}\textbf{до тех пор пока}}}%
     \renewcommand{\algorithmicloop}{{\color{red}\textbf{повторять}}}%
     \renewcommand{\algorithmicnot}{{\color{blue}\textbf{не}}}%
     \renewcommand{\algorithmicand}{{\color{blue}\textbf{и}}}%
     \renewcommand{\algorithmicor}{{\color{blue}\textbf{или}}}%
     \renewcommand{\algorithmicrequire}{{\color{blue}\textbf{Ввод}}}%
     \renewcommand{\algorithmicensure}{{\color{blue}\textbf{Вывод}}}%
     \renewcommand{\algorithmicreturn}{{\color{red}\textbf{Вернуть}}}%
     \renewcommand{\algorithmicrtrue}{{\color{blue}\textbf{истинна}}}%
     \renewcommand{\algorithmicrfalse}{{\color{blue}\textbf{ложь}}}%
     % Set other language requirements
  }
  {\end{algorithm}}

\newenvironment{enalgo}[1][]
  {\begin{algorithm}[#1]
     \selectlanguage{english,russian}%
     \floatname{algorithm}{Program}%
     \renewcommand{\algorithmicif}{{\color{red}\textbf{if}}}%
     \renewcommand{\algorithmicthen}{{\color{red}\textbf{then}}}%
     \renewcommand{\algorithmicelse}{{\color{red}\textbf{else}}}%
     \renewcommand{\algorithmicend}{{\color{red}\textbf{end}}}%
     \renewcommand{\algorithmicfor}{{\color{red}\textbf{for}}}%
     \renewcommand{\algorithmicto}{{\color{red}\textbf{to}}}%
     \renewcommand{\algorithmicdo}{{\color{red}\textbf{do}}}%
     \renewcommand{\algorithmicwhile}{{\color{red}\textbf{while}}}%
     \renewcommand{\algorithmicrepeat}{{\color{red}\textbf{repeat}}}%
     \renewcommand{\algorithmicuntil}{{\color{red}\textbf{until}}}%
     \renewcommand{\algorithmicloop}{{\color{red}\textbf{loop}}}%
     \renewcommand{\algorithmicnot}{{\color{blue}\textbf{not}}}%
     \renewcommand{\algorithmicand}{{\color{blue}\textbf{and}}}%
     \renewcommand{\algorithmicor}{{\color{blue}\textbf{or}}}%
     \renewcommand{\algorithmicrequire}{{\color{blue}\textbf{Input}}}%
     \renewcommand{\algorithmicensure}{{\color{blue}\textbf{Output}}}%
     \renewcommand{\algorithmicreturn}{{\color{red}\textbf{return}}}%
     \renewcommand{\algorithmicrtrue}{{\color{blue}\textbf{true}}}%
     \renewcommand{\algorithmicrfalse}{{\color{blue}\textbf{false}}}%
     % Set other language requirements
  }
  {\end{algorithm}}

\pagestyle{fancy}
\fancyhf{}
\fancyhead[R]{\thetitle}
\fancyfoot[L]{ITMO y2019}
\fancyfoot[C]{Page \thepage \hspace{1pt} of \pageref*{LastPage}}
\renewcommand{\footrulewidth}{0.4pt}

\author{Ilya Yaroshevskiy}
\date{\today}
\title{Лекция 4}
\hypersetup{
	pdfauthor={Ilya Yaroshevskiy},
	pdftitle={Лекция 4},
	pdfkeywords={},
	pdfsubject={},
	pdfcreator={Emacs 28.1 (Org mode 9.5.3)},
	pdflang={English}}
\begin{document}

\maketitle
\tableofcontents

\renewcommand{\P}{\mathcal{P}}
\newcommand{\A}{\mathcal{A}}
\newcommand{\L}{\mathcal{L}}
\newcommand{\B}{\mathcal{B}}

\section{Упорядоченность}
\label{sec:org9cf4f37}
\begin{definition}
	\textbf{Предпорядок} --- транзитивное, рефлексивнре
\end{definition}
\begin{definition}
	\textbf{Отношение порядка} (частичный) --- антисимметричное, транзитивное, рефлексивное
\end{definition}
\begin{definition}
	\textbf{Линейный порядок} --- порядок в котором \(a \preceq b\) или \(b \preceq a\)
\end{definition}
\begin{definition}
	\textbf{Полный порядок} --- линейный, каждое подмножество имеет наименьший элемент.
\end{definition}
\begin{examp}
	\(\N\) --- вполне упорядоченное множество
\end{examp}
\begin{examp}
	\(\R\) --- не вполне упорядоченной множество
	\begin{itemize}
		\item \((0, 1)\) не имееи наименььшего
		\item \(\R\) не имеет наименьшего
	\end{itemize}
\end{examp}
\section{Табличные модели}
\label{sec:org6895ad5}
\begin{definition}
	Назовем модель \textbf{табличной} для ИИВ:
	\begin{itemize}
		\item \(V\) --- множество истинностных значений \\
		      \(f_\to,f_\&, f_V: V^2 \to V\), \(f_\neg: V \to V\) \\
		      Выделенные значения \(T \in V\) \\
		      \(\llbracket p_i \rrbracket \in V\) \(f_\P : p_i \to V\)
		\item \(\eval{p_i} = f_\P(p_i)\) \\
		      \(\llbracket\alpha \star \beta\rrbracket = f_\star(\llbracket\alpha\rrbracket, \llbracket\beta\rrbracket)\) \\
		      \(\llbracket\neg \alpha\rrbracket = f_\neg(\llbracket\alpha\rrbracket)\)
	\end{itemize}
	\sout{Если \(\vdash \alpha\), то} \(\vDash \alpha\) означает, что \(\llbracket\alpha\rrbracket = T\), при любой \(f_\P\)
	\label{orgbefdb37}
\end{definition}
\begin{definition}
	Конечная модель: модель где \(V\) --- конечно
	\label{org9492f1f}
\end{definition}
\begin{theorem}
	У ИИВ не существует полной конечной табличной модели
	\label{org47ec0b1}
\end{theorem}
\section{Модели Крипке}
\label{sec:orga72f767}
\begin{center}
	\begin{tikzpicture}
		\node at (0,0) (A) {\( P = NP? \)};
		\node at (2, 2) (B) {все банки лопнут, RSA сломают!!!};
		\node at (2, -2) (C) {RSA устоит};
		\draw[->] (A) -- node[above] {\(+\)} (B);
		\draw[->] (A) -- node[below] {\(-\)} (C);
	\end{tikzpicture}
\end{center}
\begin{defintion}
	\-
	\begin{enumerate}
		\item \(W = \{W_i\}\) --- множество миров
		\item частичный порядок(\(\succeq\))
		\item отношение вынужденности: \(W_j \Vdash p_i\) \\
		      \((\Vdash)  \subseteq W \times \P\) \\
		      При этом, если \(W_j \Vdash p_i\) и \(W_j \preceq W_k\), то \(W_k \Vdash p\)
	\end{enumerate}
	\label{org8f83648}
\end{defintion}
\begin{definition}
	\-
	\begin{enumerate}
		\item \(W_i \Vdash \alpha\) и \(W_i \Vdash \beta\), тогда (и только тогда) \(W_i \Vdash \alpha \& \beta\) \\
		\item \(W_i \Vdash \alpha\) или \(W_i \Vdash \beta\), то \(W_i \Vdash \alpha \vee \beta\)
		\item Пусть во всех \(W_i \preceq W_j\) всегда когда \(W_j \Vdash \alpha\) имеет место \(W_j \Vdash \beta\) \\
		      Тогда \(W_i \Vdash \alpha \to \beta\)
		\item \(W_i \Vdash \neg \alpha\) --- \(\alpha\) не вынуждено нигде, начиная с \(W_i\):
		      \(W_i \preceq W_j\), то \(W_j \not\Vdash \alpha\)
	\end{enumerate}
	\label{orge32b2c9}
\end{definition}
\begin{theorem}
	Если \(W_i \Vdash \alpha\) и \(W_i \preceq W_j\), то \(W_j \Vdash \alpha\)
	\label{orga642aa1}
\end{theorem}
\begin{definition}
	Если \(W_i \Vdash \alpha\) при всех \(W_i \in W\), то \(\vDash \alpha\)
	\label{org57f714b}
\end{definition}
\begin{theorem}
	ИИВ корректна в модели Крипке
	\label{org29cc3fc}
\end{theorem}
\begin{proof}
	\begin{enumerate}
		\item \(\langle W, \Omega \rangle\) --- топология, где \(\Omega = \{w \subseteq W | \text{если }W_i \in w,\ W_i \preceq W_j,\text{ то } W_j \in w\}\) \\
		\item \(\{W_k | W_k \Vdash p_j\}\) --- открытое множество \\
		      Примем \(\llbracket p_j \rrbracket = \{W_k | W_k \Vdash p_j\}\) \\
		      Аналогично \(\llbracket \alpha \rrbracket = \{W_k | W_k \Vdash \alpha\}\)
	\end{enumerate}
\end{proof}
\section{Доказательство нетабличности}
\label{sec:orgec6ba08}
Пусть существует конечная табличная модель \(|V| = n\)
\[ \varphi_n =  \bigvee_{\substack{1 \le i, j \le n + 1 \\ i \neq j}} (p_i \to p_j \&p_j \to p_i)\]
\begin{enumerate}
	\item \(\not\vdash\varphi\)
	      \begin{center}
		      \begin{tikzpicture}
			      \node[anchor=west] at (0, 0) (A) {\(W_0\)};
			      \node[anchor=west] at (1, 2) (B) {\(W_1\)};
			      \node[anchor=west] at (1, 1) (C) {\(W_2\)};
			      \node[anchor=west] at (1, 0) (D) {\(\vdots\)};
			      \node[anchor=west] at (1, -1) (E) {\(W_{n + 1}\)};
			      \draw[->] (A) -- (B);
			      \draw[->] (A) -- (C);
			      \draw[->] (A) -- (E);
			      \node[anchor=west] at (2, 2) {\(p_1\)};
			      \node[anchor=west] at (2, 1) {\(p_2\)};
			      \node[anchor=west] at (2, -1) {\(p_{n + 1}\)};
		      \end{tikzpicture}
	      \end{center}
	      \[ W_1 \not\Vdash (p_i \to p_k)\&(p_k\to p_1),\ k\neq 1 \]
	      Значит \[ \not\vDash (p_i\to p_j)\&(p_j\to p_i) \]
	      \[ \not\vDash \bigvee (p_i\to p_j)\&(p_j\to p_i) \]
	      \[ \not\vdash\varphi_n \]
	\item \(\vDash_V \varphi_n\): по признаку Дирихле найдутся \(i\neq j:\llbracket p_i \rrbracket = \llbracket p_j \rrbracket\) \\
	      \(\llbracket p_i \to p_j \rrbracket = \text{И}\) и \(\llbracket \varphi_n \rrbracket = \text{И}\) \\
	      Значит \(\vdash \varphi_n\) --- противоречие
\end{enumerate}
\begin{definition}
	\textbf{Дизъюнктивность} ИИВ: \(\vdash \alpha \vee \beta\) влечет \(\vdash \alpha\) или \(\vdash \beta\)
	\label{org1a36943}
\end{definition}
\begin{definition}
	Гёделева алгебра --- алгебра Гейтинга, такая что из \(\alpha + \beta = 1\) следует что \(\alpha = 1\) или \(\beta = 1\) \\
	\label{org1a81ffa}
\end{definition}
\begin{definition}
	Пусть \(\A\) --- алгебра Гейтинга, тогда:
	\begin{enumerate}
		\item \(\Gamma(\A)\) \\
		      \begin{center}
			      \begin{tikzpicture}
				      \draw (-1, 0) circle[radius=0.5cm] node {\(\A\)};
				      \draw (1, 0) circle[radius=0.5cm] node {\(\A\)};
				      \node (0, 0) {\(\Rightarrow\)};
				      \draw (-1, 0.5) circle[radius=1pt,fill=black] node[above] {\(1\)};
				      \draw (1, 0.5) circle[radius=1pt,fill=black] node[above right] {\(\omega\)};
				      \draw (1, 1.5) circle[radius=1pt,fill=black] node[above] {\(1\)};
				      \draw (1, 1.5) -- (1, 0.5);
			      \end{tikzpicture}
		      \end{center}

		      Добавим новый элемент \(1_{\Gamma(\A)}\) переименуем \(1_\A\) в  \(\omega\)
	\end{enumerate}
	\label{org4da24e6}
\end{definition}
\begin{theorem}
	\-
	\begin{itemize}
		\item \(\Gamma(\A)\) --- алгебра Гейтинга
		\item \(\Gamma(\A)\) --- Геделева
	\end{itemize}
	\label{orgdc8c100}
\end{theorem}
\begin{definition}
	\textbf{Гомоморфизм} алгебр Гейтинга \\
	\begin{itemize}
		\item \(\varphi: \A \to \B\)
		\item \(\varphi(a \star b) = \varphi(a)\star\varphi(b)\)
		\item \(\varphi(1_\A) = 1_\B\)
		\item \(\varphi(0_\A) = 0_\B\)
	\end{itemize}
	\label{org7fb5121}
\end{definition}
\begin{theorem}
	\(a \le b\), то \(\varphi(a) \le \varphi(b)\)
	\label{org6aae8cf}
\end{theorem}
\begin{definition}
	\-
	\begin{itemize}
		\item \(\alpha\) --- формула ИИВ
		\item \(f, g\): оценки ИИВ
		\item \(f\): ИИВ \(\to\) \(\A\)
		\item \(g\): ИИВ \(\to\) \(\B\)
	\end{itemize}
	\(\varphi\) согласована с \(f, g\), если \(\varphi(f(\alpha)) = g(\alpha)\)
	\label{org557f5b0}
\end{definition}
\begin{theorem}
	если \(\varphi: \A \to \B\) согласована с \(f, g\) и оценка \(\llbracket \alpha \rrbracket_g \neq 1_\B\), то \(\llbracket \alpha \rrbracket_f \neq 1_\A\)
	\label{org3be227f}
\end{theorem}
\begin{theorem}
	ИИВ дизъюнктивно
	\label{orgf52e300}
\end{theorem}
\begin{proof}
	Рассмторим алгебру Линденбаума: \(\mathcal{L}\) \\
	Рассмотрим \(\Gamma(\mathcal{L})\) \\
	\begin{itemize}
		\item \(\varphi: \Gamma(\mathcal{L}) \to \mathcal{L}\)
	\end{itemize}
	\[ \varphi(x) = \begin{cases}1_\mathcal{L} & ,\substack{x =\omega \\ x = 1_{\Gamma(\mathcal{L})}} \\ x & , \text{иначе}\end{cases} \]
	\(\varphi\) --- гомоморфизм \\
	Пусть \(\vdash \alpha \vee \beta\), тогда \(\llbracket \alpha \vee \beta \rrbracket_{\Gamma(\mathcal{L})} = 1_{\Gamma(\mathcal{L})}\) \\
	\(\llbracket \alpha + \beta \rrbracket = 1\), и т.к. \(\Gamma(\mathcal{L})\) --- Геделева то \(\llbracket \alpha \rrbracket = 1\) или \(\llbracket \beta \rrbracket = 1\) \\
	Пусть \(\not \vdash \alpha\) и \(\not \vdash \beta\), тогда \(\varphi(\llbracket \alpha \rrbracket) \neq 1_\mathcal{L}\) и \(\varphi(\llbracket \beta \rrbracket) \neq 1_\mathcal{L}\), т.е. \(\llbracket \alpha \rrbracket_\mathcal{L} \neq 1_\mathcal{L}\) и \(\llbracket \beta \rrbracket_\mathcal{L} \neq 1_\mathcal{L}\), тогда \(\llbracket \alpha \rrbracket_{\Gamma(\mathcal{L})} \neq 1_{\Gamma(\mathcal{L})}\) и \(\llbracket \beta \rrbracket_{\Gamma(\mathcal{L})} \neq 1_{\Gamma(\mathcal{L})}\) \(\Rightarrow\) Противоречие
\end{proof}
\end{document}
