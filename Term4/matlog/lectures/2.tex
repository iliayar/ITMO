% Created 2021-04-19 Mon 21:12
% Intended LaTeX compiler: pdflatex

\documentclass[english]{article}
\usepackage[T1, T2A]{fontenc}
\usepackage[lutf8]{luainputenc}
\usepackage[english, russian]{babel}
\usepackage{minted}
\usepackage{graphicx}
\usepackage{longtable}
\usepackage{hyperref}
\usepackage{xcolor}
\usepackage{natbib}
\usepackage{amssymb}
\usepackage{stmaryrd}
\usepackage{amsmath}
\usepackage{caption}
\usepackage{mathtools}
\usepackage{amsthm}
\usepackage{tikz}
\usepackage{grffile}
\usepackage{extarrows}
\usepackage{wrapfig}
\usepackage{algorithm}
\usepackage{algorithmic}
\usepackage{lipsum}
\usepackage{rotating}
\usepackage{placeins}
\usepackage[normalem]{ulem}
\usepackage{amsmath}
\usepackage{textcomp}
\usepackage{capt-of}

\usepackage{geometry}
\geometry{a4paper,left=2.5cm,top=2cm,right=2.5cm,bottom=2cm,marginparsep=7pt, marginparwidth=.6in}
 \usepackage{hyperref}
 \hypersetup{
     colorlinks=true,
     linkcolor=blue,
     filecolor=orange,
     citecolor=black,      
     urlcolor=cyan,
     }

\usetikzlibrary{decorations.markings}
\usetikzlibrary{cd}
\usetikzlibrary{patterns}
\usetikzlibrary{automata, arrows}

\newcommand\addtag{\refstepcounter{equation}\tag{\theequation}}
\newcommand{\eqrefoffset}[1]{\addtocounter{equation}{-#1}(\arabic{equation}\addtocounter{equation}{#1})}
\newcommand{\llb}{\llbracket}
\newcommand{\rrb}{\rrbracket}


\newcommand{\R}{\mathbb{R}}
\renewcommand{\C}{\mathbb{C}}
\newcommand{\N}{\mathbb{N}}
\newcommand{\A}{\mathfrak{A}}
\newcommand{\B}{\mathfrak{B}}
\newcommand{\rank}{\mathop{\rm rank}\nolimits}
\newcommand{\const}{\var{const}}
\newcommand{\grad}{\mathop{\rm grad}\nolimits}

\newcommand{\todo}{{\color{red}\fbox{\text{Доделать}}}}
\newcommand{\fixme}{{\color{red}\fbox{\text{Исправить}}}}

\newcounter{propertycnt}
\setcounter{propertycnt}{1}
\newcommand{\beginproperty}{\setcounter{propertycnt}{1}}

\theoremstyle{plain}
\newtheorem{propertyinner}{Свойство}
\newenvironment{property}{
  \renewcommand\thepropertyinner{\arabic{propertycnt}}
  \propertyinner
}{\endpropertyinner\stepcounter{propertycnt}}
\newtheorem{axiom}{Аксиома}
\newtheorem{lemma}{Лемма}
\newtheorem{manuallemmainner}{Лемма}
\newenvironment{manuallemma}[1]{%
  \renewcommand\themanuallemmainner{#1}%
  \manuallemmainner
}{\endmanuallemmainner}

\theoremstyle{remark}
\newtheorem*{remark}{Примечание}
\newtheorem*{solution}{Решение}
\newtheorem{corollary}{Следствие}[theorem]
\newtheorem*{examp}{Пример}
\newtheorem*{observation}{Наблюдение}

\theoremstyle{definition}
\newtheorem{task}{Задача}
\newtheorem{theorem}{Теорема}[section]
\newtheorem*{definition}{Определение}
\newtheorem*{symb}{Обозначение}
\newtheorem{manualtheoreminner}{Теорема}
\newenvironment{manualtheorem}[1]{%
  \renewcommand\themanualtheoreminner{#1}%
  \manualtheoreminner
}{\endmanualtheoreminner}
\captionsetup{justification=centering,margin=2cm}
\newenvironment{colored}[1]{\color{#1}}{}

\tikzset{->-/.style={decoration={
  markings,
  mark=at position .5 with {\arrow{>}}},postaction={decorate}}}
\makeatletter
\newcommand*{\relrelbarsep}{.386ex}
\newcommand*{\relrelbar}{%
  \mathrel{%
    \mathpalette\@relrelbar\relrelbarsep
  }%
}
\newcommand*{\@relrelbar}[2]{%
  \raise#2\hbox to 0pt{$\m@th#1\relbar$\hss}%
  \lower#2\hbox{$\m@th#1\relbar$}%
}
\providecommand*{\rightrightarrowsfill@}{%
  \arrowfill@\relrelbar\relrelbar\rightrightarrows
}
\providecommand*{\leftleftarrowsfill@}{%
  \arrowfill@\leftleftarrows\relrelbar\relrelbar
}
\providecommand*{\xrightrightarrows}[2][]{%
  \ext@arrow 0359\rightrightarrowsfill@{#1}{#2}%
}
\providecommand*{\xleftleftarrows}[2][]{%
  \ext@arrow 3095\leftleftarrowsfill@{#1}{#2}%
}
\makeatother

\newenvironment{rualgo}[1][]
  {\begin{algorithm}[#1]
     \selectlanguage{russian}%
     \floatname{algorithm}{Алгоритм}%
     \renewcommand{\algorithmicif}{{\color{red}\textbf{если}}}%
     \renewcommand{\algorithmicthen}{{\color{red}\textbf{тогда}}}%
     \renewcommand{\algorithmicelse}{{\color{red}\textbf{иначе}}}%
     \renewcommand{\algorithmicend}{{\color{red}\textbf{конец}}}%
     \renewcommand{\algorithmicfor}{{\color{red}\textbf{для}}}%
     \renewcommand{\algorithmicto}{{\color{red}\textbf{до}}}%
     \renewcommand{\algorithmicdo}{{\color{red}\textbf{делать}}}%
     \renewcommand{\algorithmicwhile}{{\color{red}\textbf{пока}}}%
     \renewcommand{\algorithmicrepeat}{{\color{red}\textbf{повторять}}}%
     \renewcommand{\algorithmicuntil}{{\color{red}\textbf{до тех пор пока}}}%
     \renewcommand{\algorithmicloop}{{\color{red}\textbf{повторять}}}%
     \renewcommand{\algorithmicnot}{{\color{blue}\textbf{не}}}%
     \renewcommand{\algorithmicand}{{\color{blue}\textbf{и}}}%
     \renewcommand{\algorithmicor}{{\color{blue}\textbf{или}}}%
     \renewcommand{\algorithmicrequire}{{\color{blue}\textbf{Ввод}}}%
     \renewcommand{\algorithmicensure}{{\color{blue}\textbf{Вывод}}}%
     \renewcommand{\algorithmicreturn}{{\color{red}\textbf{Вернуть}}}%
     \renewcommand{\algorithmicrtrue}{{\color{blue}\textbf{истинна}}}%
     \renewcommand{\algorithmicrfalse}{{\color{blue}\textbf{ложь}}}%
     % Set other language requirements
  }
  {\end{algorithm}}
\author{Ilya Yaroshevskiy}
\date{\today}
\title{Лекция 2}
\hypersetup{
 pdfauthor={Ilya Yaroshevskiy},
 pdftitle={Лекция 2},
 pdfkeywords={},
 pdfsubject={},
 pdfcreator={Emacs 28.0.50 (Org mode 9.4.4)}, 
 pdflang={English}}
\begin{document}

\maketitle
\tableofcontents

\begin{symb}
\(\Gamma, \Delta, \Sigma\) --- списки высказываний
\end{symb}
\begin{definition}
Следование: \(\Gamma \vDash \alpha\), если
\begin{itemize}
\item \(\Gamma = \gamma_1, \dots, \gamma_n\)
\item Всегда когда все \(\llb \gamma_i \rrb = \text{И}\), то \(\llb \alpha \rrb = \text{И}\)
\end{itemize}
\end{definition}
\begin{examp}
\(\vDash \alpha\) --- \(\alpha\) общезначимо
\end{examp}
\begin{definition}
\sout{Теория} Исчисление высказываний корректна, если при любом \(\alpha\) из \(\vdash \alpha\) следует \(\vDash \alpha\)
\end{definition}
\begin{definition}
Исчисление полно, если при любом \(\alpha\) из \(\vDash \alpha\) следует \(\vdash \alpha\)
\end{definition}
\begin{theorem}[о дедукции]
\(\Gamma, \alpha \vdash \beta\) \uline{тогда и только тогда, когда} \(\Gamma \vdash \alpha \to \beta\)
\end{theorem}
\begin{proof}
\-
\begin{description}
\item[{\((\Leftarrow)\)}] Пусть \(\Gamma \vdash \alpha \to \beta\). \\
Т.е. существует доказательство \(\delta_1, \dots, \delta_n\), где \(\delta_n = \alpha \to \beta\) \\
Построим новое доказательство: \(\delta_1, \dots, \delta_n, \alpha(\text{гипотеза}), \beta(\text{M.P.})\) \\
Эта новая последовательность --- доказательство \(\Gamma, \alpha \vdash \beta\)
\item[{\((\Rightarrow)\)}] Рассмотрим \(\delta_1, \dots, \delta_n\) --- доказательство \(\Gamma, \alpha \vdash \beta\)
\begin{center}
\begin{tabular}{ll}
\(\sigma_1\) & \(\alpha \to \delta_1\)\\
\(\vdots\) & \(\vdots\)\\
\(\sigma_n\) & \(\alpha \to \delta_n\)\\
\end{tabular}
\end{center}
Утвреждение: последовательность \(\sigma_1, \dots, \sigma_n\) можно дополнить до доказательства, т.е. каждый \(\sigma_i\) --- аксиома, гипотеза или получается по M.P. Докажем по индукции: \\
\uline{База}: \(n = 0\) \\
\uline{Переход}: пусть \(\sigma_0, \dots, \sigma_n\) --- доказательсво. тогда \(\sigma_{n + 1} = \alpha \to \delta_{n + 1}\) по трем вариантам:
\begin{enumerate}
\item \(\delta_{n + 1}\) --- аксиома или гипотеза \(\not\equiv \alpha\)
\item \(\delta_{n + 1} \equiv \alpha\)
\item \(\delta_k \equiv \delta_l \to \delta_{n + 1},\ k,l\le n\)
\end{enumerate}
Докажем каждый из трех вариантов
\begin{enumerate}
\item \-
\begin{center}
\begin{tabular}{l|ll}
(n + 0.2) & \(\delta_{n + 1}\) & (аксиома или гипотеза)\\
(n + 0.4) & \(\deta_{n + 1} \to \alpha \to \delta_{n + 1}\) & (сх. акс. 1)\\
(n + 1) & \(\alpha \to \delta_{n + 1}\) & (M.P. \(n + 0.2, n + 0.4\))\\
\end{tabular}
\end{center}
\item \((n + 0.2, n + 0.4, n+0.6, n+0.8, n+1)\) --- доказательтво \(\alpha \to \alpha\)
\item \-
\begin{center}
\begin{tabular}{lll}
\((k)\) & \(\alpha \to (\sigma_l \to \sigma_{n + 1})\) & \\
\((l)\) & \(\alpha \to \sigma_l\) & \\
\((n + 0.2)\) & \((\alpha \to \delta_l) \to (\alpha \to (\delta_l \to \delta_{n + 1})) \to (\alpha \to \delta_{n + 1})\) & (сх. 2)\\
\((n + 0.4)\) & \((\alpha \to \delta_l \to \delta_{n + 1}) \to (\alpha \to \delta_{n + 1})\) & (M.P. \(n + 0.2, l\))\\
\((n + 1)\) & \(\alpha \to \delta_{n + 1}\) & (M.P. \(n + 0.4, k\))\\
\end{tabular}
\end{center}
\end{enumerate}
\end{description}
\end{proof}
\begin{theorem}[о корректности]
Пусть \(\vdash \alpha\) \\
\uline{Тогда} \(\vDash \alpha\)
\end{theorem}
\begin{proof}
Индукция по длине доказательства: каждая \(\llb \delta_i \rrb = \text{И}\), если \(\delta_1, \dots, \delta_k\) --- доказательство \(\alpha\) \\
Пусть \(\llb \delta_1 \rrb = \text{И}, \dots, \llb \delta_n \rrb = \text{И}\). Тогда осн. \(\delta_{n + 1}\):
\begin{enumerate}
\item \(\delta_{n + 1}\) --- аксиома
\begin{enumerate}
\item \(\delta_{n + 1} \equiv \alpha \to \beta \to \alpha\) (Сущесвуют \(\alpha, \beta\), что) \\
Пусть \(\delta_{n + 1} = A \to B \to A\). Тогда \(\alpha \equiv A, \beta \equiv B\) \\
\(\llb \alpha \to \beta \to \alpha \rrb ^{\llb \alpha \rrb \coloneqq a, \llb \beta \rrb \coloneqq b} = \text{И}\)
\begin{center}
\begin{tabular}{ll|l|l}
\(a\) & \(b\) & \(\beta \to \alpha\) & \(\alpha \to \beta \to \alpha\)\\
\hline
Л & Л & И & И\\
Л & И & Л & И\\
И & Л & И & И\\
И & И & И & И\\
\end{tabular}
\end{center}
\end{enumerate}
\item \(\delta_{n + 1}\) --- M.P. \(\delta_k = \delta_l \to \delta_{n + 1}\) \\
Фиксируем оценку \(\llb \delta_k \rrb = \llb \delta_l \rrb = \text{И}\), тогда \(\llb \delta_l \to \delta_{n + 1} \rrb = \text{И}\)
\begin{center}
\begin{tabular}{lll}
\(\llb \delta_l \rrb\) & \(\llb \delta_{n + 1} \rrb\) & \(\llb \delta_k \rrb = \llb \delta_l \to \delta_{n + 1} \rrb\)\\
\hline
\sout{Л} & \sout{Л} & \sout{И}\\
\sout{Л} & \sout{И} & \sout{И}\\
\sout{И} & \sout{Л} & \sout{Л}\\
И & И & И\\
\end{tabular}
\end{center}
Т.е. \(\llb \delta_{n + 1} \rrb = \text{И}\)
\end{enumerate}
\end{proof}
\begin{theorem}[о полноте]
Пусть \(\vDash \alpha\), тогда \(\vdash \alpha\)
\end{theorem}
\begin{symb}
\[ [\beta]^\alpha \equiv \begin{cases}
\alpha & \llb \beta \rrb = \text{И} \\
\neg \alpha & \llb \beta \rrb = \text{Л} 
\end{cases}\]
\end{symb}
\begin{proof}
Фиксируем набор перменных из \(\alpha\): \(P_1, \dots, P_n\) \\
Рассмотрим \(\llb \alpha \rrb^{P_1 \coloneqq x_1, \dots P_n \coloneqq x_n} = \text{И}\).
Докажем, что \(\underbrace{[x_1]^{P_1},\dots,[x_n]^{P_n}}_\Delta \vdash [\alpha]^\alpha\). \\
\uline{Индукция} по длине формулы (по структуре) \\
\uline{База}: \(\alpha \equiv P_i\) \([P_i]^{P_i} \vdash [P_i]^{P_i}\) \\
\uline{Переход}: пусть \(\eta, \zeta\): \(\Delta \vdash [\eta]^\eta, \Delta \vdash [\zeta]^\zeta\). Покажем, что \(\Delta \vdash [\eta \star \zeta]^{\eta \star \zeta}\), где \(\star\) --- все свзяки \\
Используя \hyperref[org134d6f1]{лемму}: \(\vDash \alpha\), т.е. \([x_1]^{P_1},\dots,[x_n]^{P_n} \vdash [\alpha]^\alpha\). Но \(\llb \alpha \rrb = \text{И}\) при любой оценке, \\
т.е. \([x_1]^{P_1},\dots,[x_n]^{P_n} \vdash \alpha\) при всех \(x_i\) \\
\[ \left.\begin{matrix}
[x_1]^{P_1},\dots,[x_{n - 1}]^{P_{n - 1}}, P_n \vdash \alpha \\
[x_1]^{P_1},\dots,[x_{n - 1}]^{P_{n - 1}}, \neg P_n \vdash \alpha
\end{matrix}\right| \xRightarrow{\text{лемма}} [x_1]^{P_1},\dots,[x_{n - 1}]^{P_{n - 1}} \vdash \alpha\]
\end{proof}
\begin{lemma}
\-
\begin{itemize}
\item \(\Gamma, \eta \vdash \zeta\)
\item \(\Gamma, \neg \eta \dash \zeta\)
\end{itemize}
\uline{Тогда} \(\Gamma \vdash \zeta\)
\label{org134d6f1}
\end{lemma}
\begin{lemma}
\([x_1]^{P_1},\dots,[x_n]^{P_n} \vdash \alpha\), то \([x_1]^{P_1},\dots,[x_{n - 1}]^{P_{n- 1}} \vdash \alpha\)
\label{orgdc84f5c}
\end{lemma}
\section{Интуиционистская логика}
\label{sec:org2cf52b4}
\(A \vee B\) --- плохо
\begin{examp}
Докажем: существует \(a, b\), что \(a, b \in \R \setminus \mathbb{Q}\), но \(a^b \in \mathbb{Q}\) \\
Пусть \(a = b = \sqrt{2}\). Рассмотрим \(\sqrt{2}^{\sqrt{2}} \in \R \setminus \mathbb{Q}\)
\begin{itemize}
\item Если да, то ОК
\item Если нет, то возьмем \(a = \sqrt{2}^{\sqrt{2}}, b = \sqrt{2}\), \(a^b = (\sqrt{2}^{\sqrt{2}})^{\sqrt{2}} = \sqrt{2}^{2} = 2\)
\end{itemize}
\end{examp}
\begin{defintion}
ВНК-интерпретация. \(\alpha, \beta\)
\begin{itemize}
\item \(\alpha \& \beta\) --- есть \(\alpha, \beta\)
\item \(\alpha \vee \beta\) --- есть \(\alpha\) либо \(\beta\) и мы знаем какое
\item \(\alpha \to \beta\) --- есть способ перестроить \(\alpha\) в \(\beta\)
\item \(\perp\) --- конструкция без построения \(\neg \alpha \equiv \alpha \to \perp\)
\end{itemize}
\end{defintion}
\end{document}
