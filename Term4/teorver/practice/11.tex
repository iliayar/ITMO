% Created 2021-04-20 Tue 13:04
% Intended LaTeX compiler: pdflatex

\documentclass[english]{article}
\usepackage[T1, T2A]{fontenc}
\usepackage[lutf8]{luainputenc}
\usepackage[english, russian]{babel}
\usepackage{minted}
\usepackage{graphicx}
\usepackage{longtable}
\usepackage{hyperref}
\usepackage{xcolor}
\usepackage{natbib}
\usepackage{amssymb}
\usepackage{stmaryrd}
\usepackage{amsmath}
\usepackage{caption}
\usepackage{mathtools}
\usepackage{amsthm}
\usepackage{tikz}
\usepackage{grffile}
\usepackage{extarrows}
\usepackage{wrapfig}
\usepackage{algorithm}
\usepackage{algorithmic}
\usepackage{lipsum}
\usepackage{rotating}
\usepackage{placeins}
\usepackage[normalem]{ulem}
\usepackage{amsmath}
\usepackage{textcomp}
\usepackage{capt-of}

\usepackage{geometry}
\geometry{a4paper,left=2.5cm,top=2cm,right=2.5cm,bottom=2cm,marginparsep=7pt, marginparwidth=.6in}
 \usepackage{hyperref}
 \hypersetup{
     colorlinks=true,
     linkcolor=blue,
     filecolor=orange,
     citecolor=black,      
     urlcolor=cyan,
     }

\usetikzlibrary{decorations.markings}
\usetikzlibrary{cd}
\usetikzlibrary{patterns}
\usetikzlibrary{automata, arrows}

\newcommand\addtag{\refstepcounter{equation}\tag{\theequation}}
\newcommand{\eqrefoffset}[1]{\addtocounter{equation}{-#1}(\arabic{equation}\addtocounter{equation}{#1})}
\newcommand{\llb}{\llbracket}
\newcommand{\rrb}{\rrbracket}


\newcommand{\R}{\mathbb{R}}
\renewcommand{\C}{\mathbb{C}}
\newcommand{\N}{\mathbb{N}}
\newcommand{\A}{\mathfrak{A}}
\newcommand{\B}{\mathfrak{B}}
\newcommand{\rank}{\mathop{\rm rank}\nolimits}
\newcommand{\const}{\var{const}}
\newcommand{\grad}{\mathop{\rm grad}\nolimits}

\newcommand{\todo}{{\color{red}\fbox{\text{Доделать}}}}
\newcommand{\fixme}{{\color{red}\fbox{\text{Исправить}}}}

\newcounter{propertycnt}
\setcounter{propertycnt}{1}
\newcommand{\beginproperty}{\setcounter{propertycnt}{1}}

\theoremstyle{plain}
\newtheorem{propertyinner}{Свойство}
\newenvironment{property}{
  \renewcommand\thepropertyinner{\arabic{propertycnt}}
  \propertyinner
}{\endpropertyinner\stepcounter{propertycnt}}
\newtheorem{axiom}{Аксиома}
\newtheorem{lemma}{Лемма}
\newtheorem{manuallemmainner}{Лемма}
\newenvironment{manuallemma}[1]{%
  \renewcommand\themanuallemmainner{#1}%
  \manuallemmainner
}{\endmanuallemmainner}

\theoremstyle{remark}
\newtheorem*{remark}{Примечание}
\newtheorem*{solution}{Решение}
\newtheorem{corollary}{Следствие}[theorem]
\newtheorem*{examp}{Пример}
\newtheorem*{observation}{Наблюдение}

\theoremstyle{definition}
\newtheorem{task}{Задача}
\newtheorem{theorem}{Теорема}[section]
\newtheorem*{definition}{Определение}
\newtheorem*{symb}{Обозначение}
\newtheorem{manualtheoreminner}{Теорема}
\newenvironment{manualtheorem}[1]{%
  \renewcommand\themanualtheoreminner{#1}%
  \manualtheoreminner
}{\endmanualtheoreminner}
\captionsetup{justification=centering,margin=2cm}
\newenvironment{colored}[1]{\color{#1}}{}

\tikzset{->-/.style={decoration={
  markings,
  mark=at position .5 with {\arrow{>}}},postaction={decorate}}}
\makeatletter
\newcommand*{\relrelbarsep}{.386ex}
\newcommand*{\relrelbar}{%
  \mathrel{%
    \mathpalette\@relrelbar\relrelbarsep
  }%
}
\newcommand*{\@relrelbar}[2]{%
  \raise#2\hbox to 0pt{$\m@th#1\relbar$\hss}%
  \lower#2\hbox{$\m@th#1\relbar$}%
}
\providecommand*{\rightrightarrowsfill@}{%
  \arrowfill@\relrelbar\relrelbar\rightrightarrows
}
\providecommand*{\leftleftarrowsfill@}{%
  \arrowfill@\leftleftarrows\relrelbar\relrelbar
}
\providecommand*{\xrightrightarrows}[2][]{%
  \ext@arrow 0359\rightrightarrowsfill@{#1}{#2}%
}
\providecommand*{\xleftleftarrows}[2][]{%
  \ext@arrow 3095\leftleftarrowsfill@{#1}{#2}%
}
\makeatother

\newenvironment{rualgo}[1][]
  {\begin{algorithm}[#1]
     \selectlanguage{russian}%
     \floatname{algorithm}{Алгоритм}%
     \renewcommand{\algorithmicif}{{\color{red}\textbf{если}}}%
     \renewcommand{\algorithmicthen}{{\color{red}\textbf{тогда}}}%
     \renewcommand{\algorithmicelse}{{\color{red}\textbf{иначе}}}%
     \renewcommand{\algorithmicend}{{\color{red}\textbf{конец}}}%
     \renewcommand{\algorithmicfor}{{\color{red}\textbf{для}}}%
     \renewcommand{\algorithmicto}{{\color{red}\textbf{до}}}%
     \renewcommand{\algorithmicdo}{{\color{red}\textbf{делать}}}%
     \renewcommand{\algorithmicwhile}{{\color{red}\textbf{пока}}}%
     \renewcommand{\algorithmicrepeat}{{\color{red}\textbf{повторять}}}%
     \renewcommand{\algorithmicuntil}{{\color{red}\textbf{до тех пор пока}}}%
     \renewcommand{\algorithmicloop}{{\color{red}\textbf{повторять}}}%
     \renewcommand{\algorithmicnot}{{\color{blue}\textbf{не}}}%
     \renewcommand{\algorithmicand}{{\color{blue}\textbf{и}}}%
     \renewcommand{\algorithmicor}{{\color{blue}\textbf{или}}}%
     \renewcommand{\algorithmicrequire}{{\color{blue}\textbf{Ввод}}}%
     \renewcommand{\algorithmicensure}{{\color{blue}\textbf{Вывод}}}%
     \renewcommand{\algorithmicreturn}{{\color{red}\textbf{Вернуть}}}%
     \renewcommand{\algorithmicrtrue}{{\color{blue}\textbf{истинна}}}%
     \renewcommand{\algorithmicrfalse}{{\color{blue}\textbf{ложь}}}%
     % Set other language requirements
  }
  {\end{algorithm}}
\author{Ilya Yaroshevskiy}
\date{\today}
\title{Практика 11}
\hypersetup{
 pdfauthor={Ilya Yaroshevskiy},
 pdftitle={Практика 11},
 pdfkeywords={},
 pdfsubject={},
 pdfcreator={Emacs 28.0.50 (Org mode 9.4.4)}, 
 pdflang={English}}
\begin{document}

\maketitle
\tableofcontents

\begin{task}
Имеются 2 случайные величина
\begin{center}
\begin{tabular}{l|lllll}
\(\xi\) & \(-2\) & \(-1\) & \(0\) & \(1\) & \(2\)\\
\hline
\(p\) & \(0.1\) & \(0.3\) & \(0.3\) & \(0.2\) & \(0.1\)\\
\end{tabular}
\end{center}


\begin{center}
\begin{tabular}{l|lll}
\(\eta\) & \(-3\) & \(1\) & \(4\)\\
\hline
\(p\) & \(0.4\) & \(0.4\) & \(0.2\)\\
\end{tabular}
\end{center}
Найти: \(\gamma = \xi^2 - \eta\)
\end{task}
\begin{solution}
\begin{center}
\begin{tabular}{l|rrr}
\(\xi^2\) & 0 & 1 & 4\\
\hline
\(p\) & 0.3 & 0.5 & 0.2\\
\end{tabular}
\end{center}


\begin{center}
\begin{tabular}{r|lll}
\(\eta\textbackslash\xi^2\) & 0 & 1 & 4\\
\hline
-3 & \(3\) & \(4\) & \(7\)\\
1 & \(-1\) & \(0\) & \(3\)\\
4 & \(-4\) & \(-3\) & \(0\)\\
\end{tabular}
\end{center}


\begin{center}
\begin{tabular}{l|rrrrrrr}
\(\gamma\) & -4 & -3 & -1 & 0 & 3 & 4 & 7\\
\hline
\(p\) & \(0.006\) & \(0.1\) & \(0.12\) & \(0.2 + 0.04\) & \(0.08 + 0.12\) & \(0.2\) & \(0.08\)\\
\end{tabular}
\end{center}
\end{solution}
\begin{task}
Игрок играет в орлянку по схеме с удвоением. Если он проиграл, то следущий раз удваивает ставку. Игрок играет до тех пор пока не выиграет. Случайная величина --- его выигрышь. Вычислить мат. ожидание и дисперсию.
\end{task}
\begin{solution}
\-
\begin{center}
\begin{tabular}{l|llllll}
\(\xi\) & \(1\) & \(1 = 2 - 1\) & \(1 = 4 - 2 - 1\) & \(\dots\) & \(1 = 2^{n - 1} - \sum_{i = 0}^{n - 2}2^i\) & \(\dots\)\\
\hline
\(p\) & \(\frac{1}{2}\) & \(\frac{1}{4}\) & \(\frac{1}{8}\) & \(\dots\) & \(\frac{1}{2^n}\) & \(\dots\)\\
\end{tabular}
\end{center}
\[ E\xi = 1 \cdot \sum_{n = 1}^\infty \frac{1}{2^n} = \frac{\frac{1}{2}}{1 - \frac{1}{2}} = 1 \]
\[ D\xi = 1^2 \cdot \sum_{n = 1}^\infty \frac{1}{2^n} - 1^2 = 0 \]
\end{solution}
\begin{task}
Та же задача, только \(n\) --- ограничено
\end{task}
\begin{solution}
\begin{center}
\begin{tabular}{l|llllrrll}
\(\xi\) & \(1\) & \(1\) & \(1\) & \(\dots\) & 1 & 1 & \(-\sum_{i = 0}^{n - 1}2 ^i\) & \\
\hline
\(p\) & \(\frac{1}{2}\) & \(\frac{1}{4}\) & \(\frac{1}{8}\) & \(\dots\) & \(\frac{1}{2^{n - 1}}\) & \(\frac{1}{2^{n + 1}}\) & \(\frac{1}{2^{n + 1}}\) & \\
\end{tabular}
\end{center}
\[ E\xi = 1\cdot \sum_{k = 1}^{n - 1} \frac{1}{2^k} + 1\cdot \frac{1}{2^{k + 1}} - \frac{1}{2^{n + 1}}\cdot \sum_{k = 0}^{n - 1}2^k =  \]
\end{solution}
\end{document}
