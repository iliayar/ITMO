% Created 2021-04-13 Tue 13:06
% Intended LaTeX compiler: pdflatex

\documentclass[english]{article}
\usepackage[T1, T2A]{fontenc}
\usepackage[lutf8]{luainputenc}
\usepackage[english, russian]{babel}
\usepackage{minted}
\usepackage{graphicx}
\usepackage{longtable}
\usepackage{hyperref}
\usepackage{xcolor}
\usepackage{natbib}
\usepackage{amssymb}
\usepackage{stmaryrd}
\usepackage{amsmath}
\usepackage{caption}
\usepackage{mathtools}
\usepackage{amsthm}
\usepackage{tikz}
\usepackage{grffile}
\usepackage{extarrows}
\usepackage{wrapfig}
\usepackage{algorithm}
\usepackage{algorithmic}
\usepackage{lipsum}
\usepackage{rotating}
\usepackage{placeins}
\usepackage[normalem]{ulem}
\usepackage{amsmath}
\usepackage{textcomp}
\usepackage{capt-of}

\usepackage{geometry}
\geometry{a4paper,left=2.5cm,top=2cm,right=2.5cm,bottom=2cm,marginparsep=7pt, marginparwidth=.6in}
 \usepackage{hyperref}
 \hypersetup{
     colorlinks=true,
     linkcolor=blue,
     filecolor=orange,
     citecolor=black,      
     urlcolor=cyan,
     }

\usetikzlibrary{decorations.markings}
\usetikzlibrary{cd}
\usetikzlibrary{patterns}
\usetikzlibrary{automata, arrows}

\newcommand\addtag{\refstepcounter{equation}\tag{\theequation}}
\newcommand{\eqrefoffset}[1]{\addtocounter{equation}{-#1}(\arabic{equation}\addtocounter{equation}{#1})}
\newcommand{\llb}{\llbracket}
\newcommand{\rrb}{\rrbracket}


\newcommand{\R}{\mathbb{R}}
\renewcommand{\C}{\mathbb{C}}
\newcommand{\N}{\mathbb{N}}
\newcommand{\A}{\mathfrak{A}}
\newcommand{\B}{\mathfrak{B}}
\newcommand{\rank}{\mathop{\rm rank}\nolimits}
\newcommand{\const}{\var{const}}
\newcommand{\grad}{\mathop{\rm grad}\nolimits}

\newcommand{\todo}{{\color{red}\fbox{\text{Доделать}}}}
\newcommand{\fixme}{{\color{red}\fbox{\text{Исправить}}}}

\newcounter{propertycnt}
\setcounter{propertycnt}{1}
\newcommand{\beginproperty}{\setcounter{propertycnt}{1}}

\theoremstyle{plain}
\newtheorem{propertyinner}{Свойство}
\newenvironment{property}{
  \renewcommand\thepropertyinner{\arabic{propertycnt}}
  \propertyinner
}{\endpropertyinner\stepcounter{propertycnt}}
\newtheorem{axiom}{Аксиома}
\newtheorem{lemma}{Лемма}
\newtheorem{manuallemmainner}{Лемма}
\newenvironment{manuallemma}[1]{%
  \renewcommand\themanuallemmainner{#1}%
  \manuallemmainner
}{\endmanuallemmainner}

\theoremstyle{remark}
\newtheorem*{remark}{Примечание}
\newtheorem*{solution}{Решение}
\newtheorem{corollary}{Следствие}[theorem]
\newtheorem*{examp}{Пример}
\newtheorem*{observation}{Наблюдение}

\theoremstyle{definition}
\newtheorem{task}{Задача}
\newtheorem{theorem}{Теорема}[section]
\newtheorem*{definition}{Определение}
\newtheorem*{symb}{Обозначение}
\newtheorem{manualtheoreminner}{Теорема}
\newenvironment{manualtheorem}[1]{%
  \renewcommand\themanualtheoreminner{#1}%
  \manualtheoreminner
}{\endmanualtheoreminner}
\captionsetup{justification=centering,margin=2cm}
\newenvironment{colored}[1]{\color{#1}}{}

\tikzset{->-/.style={decoration={
  markings,
  mark=at position .5 with {\arrow{>}}},postaction={decorate}}}
\makeatletter
\newcommand*{\relrelbarsep}{.386ex}
\newcommand*{\relrelbar}{%
  \mathrel{%
    \mathpalette\@relrelbar\relrelbarsep
  }%
}
\newcommand*{\@relrelbar}[2]{%
  \raise#2\hbox to 0pt{$\m@th#1\relbar$\hss}%
  \lower#2\hbox{$\m@th#1\relbar$}%
}
\providecommand*{\rightrightarrowsfill@}{%
  \arrowfill@\relrelbar\relrelbar\rightrightarrows
}
\providecommand*{\leftleftarrowsfill@}{%
  \arrowfill@\leftleftarrows\relrelbar\relrelbar
}
\providecommand*{\xrightrightarrows}[2][]{%
  \ext@arrow 0359\rightrightarrowsfill@{#1}{#2}%
}
\providecommand*{\xleftleftarrows}[2][]{%
  \ext@arrow 3095\leftleftarrowsfill@{#1}{#2}%
}
\makeatother

\newenvironment{rualgo}[1][]
  {\begin{algorithm}[#1]
     \selectlanguage{russian}%
     \floatname{algorithm}{Алгоритм}%
     \renewcommand{\algorithmicif}{{\color{red}\textbf{если}}}%
     \renewcommand{\algorithmicthen}{{\color{red}\textbf{тогда}}}%
     \renewcommand{\algorithmicelse}{{\color{red}\textbf{иначе}}}%
     \renewcommand{\algorithmicend}{{\color{red}\textbf{конец}}}%
     \renewcommand{\algorithmicfor}{{\color{red}\textbf{для}}}%
     \renewcommand{\algorithmicto}{{\color{red}\textbf{до}}}%
     \renewcommand{\algorithmicdo}{{\color{red}\textbf{делать}}}%
     \renewcommand{\algorithmicwhile}{{\color{red}\textbf{пока}}}%
     \renewcommand{\algorithmicrepeat}{{\color{red}\textbf{повторять}}}%
     \renewcommand{\algorithmicuntil}{{\color{red}\textbf{до тех пор пока}}}%
     \renewcommand{\algorithmicloop}{{\color{red}\textbf{повторять}}}%
     \renewcommand{\algorithmicnot}{{\color{blue}\textbf{не}}}%
     \renewcommand{\algorithmicand}{{\color{blue}\textbf{и}}}%
     \renewcommand{\algorithmicor}{{\color{blue}\textbf{или}}}%
     \renewcommand{\algorithmicrequire}{{\color{blue}\textbf{Предусловие}}}%
     \renewcommand{\algorithmicrensure}{{\color{blue}\textbf{Постусловие}}}%
     \renewcommand{\algorithmicrtrue}{{\color{blue}\textbf{истинна}}}%
     \renewcommand{\algorithmicrfalse}{{\color{blue}\textbf{ложь}}}%
     % Set other language requirements
  }
  {\end{algorithm}}
\author{Ilya Yaroshevskiy}
\date{\today}
\title{Практика 10}
\hypersetup{
 pdfauthor={Ilya Yaroshevskiy},
 pdftitle={Практика 10},
 pdfkeywords={},
 pdfsubject={},
 pdfcreator={Emacs 28.0.50 (Org mode 9.4.4)}, 
 pdflang={English}}
\begin{document}

\maketitle
\tableofcontents


\section{Правило трех \(\sigma\)}
\label{sec:orgfd57f33}
\subsection{Равномерное распределение}
\label{sec:org718c2c7}
\(\xi \in U_{a, b}\)
\[ E\xi = \frac{a + b}{2} \]
\[ \sigma = \frac{b - a}{2\sqrt{3}} \]
\[ p(E\xi - 3\sigma < \xi < E\xi + 3\sigma \) = 1 \]
\subsection{Покзательное распределение}
\label{sec:org29ce43c}
\(\xi \in E_\alpha\)
\[ p(E\xi - 3\sigma < \xi < E\xi + 3\sigma) = p \left(\frac{1}{\alpha} - \frac{3}{\alpha} < \xi < \frac{1}{\alpha} + \frac{3}{\alpha}\right) = \]
\[ = p \left(-\frac{2}{\alpha} < \xi < \frac{4}{\alpha}\right) = p(0 < \xi < \frac{4}{\alpha}) = 1 - e^{-4} \approx 0.9817 \]

\section{Задачи}
\label{sec:orgf15e6f5}
\begin{task}
\-
\begin{itemize}
\item \(E\xi = 2\)
\item \(E\eta = -3\)
\item \(D\xi = 1\)
\item \(D\eta = 4\)
\item \(\gamma = 3\xi - 5\eta\)
\end{itemize}
Найти \(E\gamma\), \(D\gamma\), если \(\xi\) и \(\eta\) независимы
\end{task}
\begin{solution}
\[ E\gamma = 3E\xi - 5E\eta = 3\cdot 2 - 5\cdot (-3) = 21 \]
\[ D\gamma = 3D\xi + 5D\eta = 3^2\cdot 1 + 5^2\cdot 4 = 109  \]
\end{solution}

\begin{task}
\[ p_\xi = \begin{cases}
0 & x < 0 \\
\frac{1}{2} \sin x & 0 \le x \le \pi \\
0 & x > \pi
\end{cases}\]
\begin{itemize}
\item \(\eta = -2\xi + \pi\)
\end{itemize}
Найти \(p_\eta\)
\end{task}
\begin{solution}
\(\xi \in [0; \pi]\), тогда \(\eta \in [-\pi, \pi]\)
\[ p_\eta = \frac{1}{|a|}\cdot f_\xi \left(\frac{x - b}{a}\right) \]
\[ f_\eta = \begin{cases}
0 & x < -\pi \\
\frac{1}{2}\cdot\frac{1}{2}\sin \left(\frac{x - \pi}{-2}\right) = \frac{1}{4}\cos\frac{x}{2} & -\pi \le x \le pi \\
0 & x > \pi
\end{cases} \]
\end{solution}
\begin{task}
\-
\begin{itemize}
\item \(\xi \in N_{0, 1}\)
\item \(\eta = e^{-\frac{\xi}{\sqrt{2}}}\)
\end{itemize}
Найти \(f_\eta\)
\end{task}
\begin{solution}
\(\eta \in (0, +\infty)\)
\[ f_\eta = \frac{1}{|h'(x)|}\cdot f_\xi(h(x)) \]
\[ g(x) = e^{-\frac{x}{\sqrt{2}}} \implies x = -\sqrt{2}\ln g(x) \implies h(x) = -\sqrt{2}\ln x\]
\[ f_\eta = \frac{1}{\left|\frac{-\sqrt{2}}{x}\right|}\cdot \frac{1}{\sqrt{2\pi}}\cdot e^{\textstyle-\frac{(-\sqrt{2}\ln x)^2}{2}} = \frac{|x|}{2\sqrt{\pi}}\cdot x^{\ln x} \]
\end{solution}
\begin{theorem}[Смирнова]
Пусть функция не является монотонной, тогда `обратная` функция распадается на несколько ветвей.
\[ f_\eta(x) = \sum_{i = 1}^k \frac{1}{|h_i'(x)|} \cdot f_\xi(h_i(x)) \]
\end{theorem}
\begin{examp}
\[ f_\xi = \begin{cases}
0 & x < 1 \\
\frac{4}{3x^2} & 1 \le x \le 4 \\
0 & x > 4
\end{cases}\]
\begin{itemize}
\item \(\eta = |\xi - 2|\)
\end{itemize}
\end{examp}
\begin{solution}
\(\xi \in [1, 4]\), \(\eta \in [0, 2]\)
\begin{itemize}
\item \(h_1(\eta) = \eta + 2\)
\item \(h_2(\eta) = -\eta + 2\)
\end{itemize}
\[ f_\eta = \begin{cases}
0 & x < 0 \\
\frac{4}{3}\cdot\frac{1}{(x + 2)^2} + \frac{4}{3}\cdot\frac{1}{(2 - x)^2} & 0 \le x \le 1 \\
\frac{4}{3}\cdot\frac{1}{(x + 2)^2} & 1 < x \le 2 \\
0 & x > 2
\end{cases}\]
\end{solution}
\end{document}
