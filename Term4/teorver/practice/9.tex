% Created 2021-04-06 Tue 13:00
% Intended LaTeX compiler: pdflatex

\documentclass[english]{article}
\usepackage[T1, T2A]{fontenc}
\usepackage[lutf8]{luainputenc}
\usepackage[english, russian]{babel}
\usepackage{minted}
\usepackage{graphicx}
\usepackage{longtable}
\usepackage{hyperref}
\usepackage{xcolor}
\usepackage{natbib}
\usepackage{amssymb}
\usepackage{stmaryrd}
\usepackage{amsmath}
\usepackage{caption}
\usepackage{mathtools}
\usepackage{amsthm}
\usepackage{tikz}
\usepackage{grffile}
\usepackage{extarrows}
\usepackage{wrapfig}
\usepackage{rotating}
\usepackage{placeins}
\usepackage[normalem]{ulem}
\usepackage{amsmath}
\usepackage{textcomp}
\usepackage{capt-of}

\usepackage{geometry}
\geometry{a4paper,left=2.5cm,top=2cm,right=2.5cm,bottom=2cm,marginparsep=7pt, marginparwidth=.6in}
 \usepackage{hyperref}
 \hypersetup{
     colorlinks=true,
     linkcolor=blue,
     filecolor=orange,
     citecolor=black,      
     urlcolor=cyan,
     }

\usetikzlibrary{decorations.markings}
\usetikzlibrary{cd}
\usetikzlibrary{patterns}
\usetikzlibrary{automata, arrows}

\newcommand\addtag{\refstepcounter{equation}\tag{\theequation}}
\newcommand{\eqrefoffset}[1]{\addtocounter{equation}{-#1}(\arabic{equation}\addtocounter{equation}{#1})}


\newcommand{\R}{\mathbb{R}}
\renewcommand{\C}{\mathbb{C}}
\newcommand{\N}{\mathbb{N}}
\newcommand{\A}{\mathfrak{A}}
\newcommand{\rank}{\mathop{\rm rank}\nolimits}
\newcommand{\const}{\var{const}}
\newcommand{\grad}{\mathop{\rm grad}\nolimits}

\newcommand{\todo}{{\color{red}\fbox{\text{Доделать}}}}
\newcommand{\fixme}{{\color{red}\fbox{\text{Исправить}}}}

\newcounter{propertycnt}
\setcounter{propertycnt}{1}
\newcommand{\beginproperty}{\setcounter{propertycnt}{1}}

\theoremstyle{plain}
\newtheorem{propertyinner}{Свойство}
\newenvironment{property}{
  \renewcommand\thepropertyinner{\arabic{propertycnt}}
  \propertyinner
}{\endpropertyinner\stepcounter{propertycnt}}
\newtheorem{axiom}{Аксиома}
\newtheorem{lemma}{Лемма}
\newtheorem{manuallemmainner}{Лемма}
\newenvironment{manuallemma}[1]{%
  \renewcommand\themanuallemmainner{#1}%
  \manuallemmainner
}{\endmanuallemmainner}

\theoremstyle{remark}
\newtheorem*{remark}{Примечание}
\newtheorem*{solution}{Решение}
\newtheorem{corollary}{Следствие}[theorem]
\newtheorem*{examp}{Пример}
\newtheorem*{observation}{Наблюдение}

\theoremstyle{definition}
\newtheorem{task}{Задача}
\newtheorem{theorem}{Теорема}[section]
\newtheorem*{definition}{Определение}
\newtheorem*{symb}{Обозначение}
\newtheorem{manualtheoreminner}{Теорема}
\newenvironment{manualtheorem}[1]{%
  \renewcommand\themanualtheoreminner{#1}%
  \manualtheoreminner
}{\endmanualtheoreminner}
\captionsetup{justification=centering,margin=2cm}
\newenvironment{colored}[1]{\color{#1}}{}

\tikzset{->-/.style={decoration={
  markings,
  mark=at position .5 with {\arrow{>}}},postaction={decorate}}}
\makeatletter
\newcommand*{\relrelbarsep}{.386ex}
\newcommand*{\relrelbar}{%
  \mathrel{%
    \mathpalette\@relrelbar\relrelbarsep
  }%
}
\newcommand*{\@relrelbar}[2]{%
  \raise#2\hbox to 0pt{$\m@th#1\relbar$\hss}%
  \lower#2\hbox{$\m@th#1\relbar$}%
}
\providecommand*{\rightrightarrowsfill@}{%
  \arrowfill@\relrelbar\relrelbar\rightrightarrows
}
\providecommand*{\leftleftarrowsfill@}{%
  \arrowfill@\leftleftarrows\relrelbar\relrelbar
}
\providecommand*{\xrightrightarrows}[2][]{%
  \ext@arrow 0359\rightrightarrowsfill@{#1}{#2}%
}
\providecommand*{\xleftleftarrows}[2][]{%
  \ext@arrow 3095\leftleftarrowsfill@{#1}{#2}%
}
\makeatother
\author{Ilya Yaroshevskiy}
\date{\today}
\title{Практика 9}
\hypersetup{
 pdfauthor={Ilya Yaroshevskiy},
 pdftitle={Практика 9},
 pdfkeywords={},
 pdfsubject={},
 pdfcreator={Emacs 28.0.50 (Org mode 9.4.4)}, 
 pdflang={English}}
\begin{document}

\maketitle
\tableofcontents

\begin{task}
\(\xi \in N(2, 3)\). Найти \(p(-1 < \xi < 7)\), \(p(|\xi - a| > 6.5)\)
\end{task}
\begin{solution}
\[ p(-1 < \xi < 7) = \Phi\left(\frac{7 - 2}{3}\right) - \Phi\left(\frac{-1 - 2}{3}\right) = \Phi(1.67) + \Phi(1) \]
\[ p(|\xi - a| > 6.5) = 1 - p(|\xi - a| < 6.5) = 1 - 2\Phi \left(\frac{6.5}{3}\right) = 1 - 2\Phi(2.17) \]
\end{solution}
\begin{task}
Прибор точно калиброван. Среднее квадратическое отклонение ошибки.
\end{task}
\begin{solution}
\(a = 0,\ \sigma = 0.5\)
\[ p(-1.5 < \xi < 0.5) = \Phi(1)  + \Phi(3) \]
\end{solution}
\begin{task}
Вероятность того, что ноормальная случаная величина отклонится от среднего занчения не более чем на 5 равна \(0ю95\)
\end{task}
\begin{solution}
\[ p(|\xi - a| < 5) = 0.95 = 2\Phi \left(\frac{5}{\sigma}\right) \]
\[ \Phi \left(\frac{5}{\sigma}\right) = 0.475 \]
\[ \frac{5}{\sigma} = 1.96 \]
\[ \sigma \approx 2.55 \]
\end{solution}
\begin{task}
Известно, что нормальная случайная величина имеет среднее значение \(a = 100\) и \(p(88 < \xi < 112) = 0.9973\). Найти \(p(95 < \xi < 107)\)
\end{task}
\begin{solution}
\[ p(88 < \xi < 112) = \Phi \left(\frac{112 - 100}{\sigma}\right) - \Phi \left(\frac{88 - 100}{\sigma}\right) = 2\Phi \left(\frac{12}{\sigma}\right) = 0.9973 \]
\[ \Phi \left(\frac{12}{\sigma}\right) = 0.49865 \]
\[ \frac{12}{\sigma} = 3 \]
\[ \sigma = 4 \]
\[ p(95 < \xi < 107) = \Phi(1.75) + \Phi(1.25) = 0.4599 + 0.3944 = 0.8543 \]
\end{solution}
\begin{task}
Трамвай ходит с интервалом ровно 15 мин. Случайная величина --- время его ожидания на остановке. Найти мат ожидание и дисперсию.
\end{task}
\begin{solution}
\(a = 0, b = 15\)
\end{solution}
\begin{task}
Среднее время работы прибора до поломки --- 10 лет. Найти вероятност того что он проработате не менее 20 лет.
\end{task}
\begin{solution}
\(\xi \in E_\alpha\)
\[ E\xi = \frac{1}{\alpha} = 10 \]
\[ \alpha = 0.1 \]
\[ p(\xi > 20) = 1 - p(0 < \xi < 20) = 1 - e^{-0\cdot0.1}  + e^{-20\cdot 0.1}\]
\end{solution}

\section{ДЗ}
\label{sec:org0d557fd}
\begin{task}
\(\xi \in N(3, 1^2)\). Найти \(p(1.8 < \xi < 3.8)\)
\end{task}
\end{document}
