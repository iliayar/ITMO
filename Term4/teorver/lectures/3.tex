% Created 2021-02-27 Sat 13:39
% Intended LaTeX compiler: pdflatex
\documentclass[english]{article}
\usepackage[T1, T2A]{fontenc}
\usepackage[lutf8]{luainputenc}
\usepackage[english, russian]{babel}
\usepackage{minted}
\usepackage{graphicx}
\usepackage{longtable}
\usepackage{hyperref}
\usepackage{xcolor}
\usepackage{natbib}
\usepackage{amssymb}
\usepackage{amsmath}
\usepackage{caption}
\usepackage{mathtools}
\usepackage{amsthm}
\usepackage{tikz}
\usepackage{grffile}
\usepackage{extarrows}
\usepackage{wrapfig}
\usepackage{rotating}
\usepackage{placeins}
\usepackage[normalem]{ulem}
\usepackage{amsmath}
\usepackage{textcomp}
\usepackage{capt-of}

\usepackage{geometry}
\geometry{a4paper,left=2.5cm,top=2cm,right=2.5cm,bottom=2cm,marginparsep=7pt, marginparwidth=.6in}

 \usepackage{hyperref}
 \hypersetup{
     colorlinks=true,
     linkcolor=blue,
     filecolor=orange,
     citecolor=black,      
     urlcolor=cyan,
     }

\usetikzlibrary{decorations.markings}
\usetikzlibrary{cd}
\usetikzlibrary{patterns}

\newcommand\addtag{\refstepcounter{equation}\tag{\theequation}}
\newcommand{\eqrefoffset}[1]{\addtocounter{equation}{-#1}(\arabic{equation}\addtocounter{equation}{#1})}


\newcommand{\R}{\mathbb{R}}
\renewcommand{\C}{\mathbb{C}}
\newcommand{\N}{\mathbb{N}}
\newcommand{\rank}{\text{rank}}
\newcommand{\const}{\text{const}}
\newcommand{\grad}{\text{grad}}

\theoremstyle{plain}
\newtheorem{axiom}{Аксиома}
\newtheorem{lemma}{Лемма}
\newtheorem{manuallemmainner}{Лемма}
\newenvironment{manuallemma}[1]{%
  \renewcommand\themanuallemmainner{#1}%
  \manuallemmainner
}{\endmanuallemmainner}

\theoremstyle{remark}
\newtheorem*{remark}{Примечание}
\newtheorem*{solution}{Решение}
\newtheorem{corollary}{Следствие}[theorem]
\newtheorem*{examp}{Пример}
\newtheorem*{observation}{Наблюдение}

\theoremstyle{definition}
\newtheorem{task}{Задача}
\newtheorem{theorem}{Теорема}[section]
\newtheorem*{definition}{Определение}
\newtheorem*{symb}{Обозначение}
\newtheorem{manualtheoreminner}{Теорема}
\newenvironment{manualtheorem}[1]{%
  \renewcommand\themanualtheoreminner{#1}%
  \manualtheoreminner
}{\endmanualtheoreminner}
\captionsetup{justification=centering,margin=2cm}
\newenvironment{colored}[1]{\color{#1}}{}

\tikzset{->-/.style={decoration={
  markings,
  mark=at position .5 with {\arrow{>}}},postaction={decorate}}}
\makeatletter
\newcommand*{\relrelbarsep}{.386ex}
\newcommand*{\relrelbar}{%
  \mathrel{%
    \mathpalette\@relrelbar\relrelbarsep
  }%
}
\newcommand*{\@relrelbar}[2]{%
  \raise#2\hbox to 0pt{$\m@th#1\relbar$\hss}%
  \lower#2\hbox{$\m@th#1\relbar$}%
}
\providecommand*{\rightrightarrowsfill@}{%
  \arrowfill@\relrelbar\relrelbar\rightrightarrows
}
\providecommand*{\leftleftarrowsfill@}{%
  \arrowfill@\leftleftarrows\relrelbar\relrelbar
}
\providecommand*{\xrightrightarrows}[2][]{%
  \ext@arrow 0359\rightrightarrowsfill@{#1}{#2}%
}
\providecommand*{\xleftleftarrows}[2][]{%
  \ext@arrow 3095\leftleftarrowsfill@{#1}{#2}%
}
\makeatother
\author{Ilya Yaroshevskiy}
\date{\today}
\title{Лекция 3}
\hypersetup{
 pdfauthor={Ilya Yaroshevskiy},
 pdftitle={Лекция 3},
 pdfkeywords={},
 pdfsubject={},
 pdfcreator={Emacs 28.0.50 (Org mode )}, 
 pdflang={English}}
\begin{document}

\maketitle
\tableofcontents

\begin{theorem}[Баеса]
\(] H1, H2, \dots, H_n ,\dots\) --- полн. у. соб. \\
\uline{Тогда} \[ P(H_k|A) = \frac{P(H_k)P(A|H_k)}{\sum_{i = 1}^{\infty}P(H_i)P(A|H_i)} \]
\end{theorem}
\begin{examp}
В первой коробке 4 белых и два черных шара, во второй й белый и два
черных. Из первой коробки во вторую переложили два шара, затем из
второй коробке доставли шар. Найти вероятность того что он оказался белый
\end{examp}
\begin{solution}
\begin{itemize}
\item \(] H_1\) --- переложили 2 белых
\item \(] H_2\) --- переложили 2 черных
\item \(] H_3\) --- переложили 1 черный и 1 белый
\item \(] A\) --- из второй коробки достали белый
\end{itemize}
\[ P(H_1) = \frac{4}{6}\cdot\frac{3}{5} =  \frac{6}{15} \]
\[ P(H_2) = \frac{2}{6}\cdot\frac{1}{5} = \frac{1}{15} \]
\[ P(H_3) = \frac{4}{6}\cdot\frac{2}{5} + \frac{2}{6}\cdot\frac{4}{5} = \frac{8}{15} \]
\[ \sum P(H_i) = 1\text{ --- верно} \]
\[ P(A|H_1) = \frac{3}{5} \]
\[ P(A|H_2) = \frac{1}{5} \]
\[ P(A|H_3) = \frac{2}{5} \]
По формуле полной вероятности:
\[ P(A) = P(H_1)\cdotP(A|H_1) + P(H_2)\cdotP(A|H_2) + P(H_3)\cdotP(A|H_3) = \frac{6}{15}\cdot\frac{3}{5} + \frac{1}{15}\cdot\frac{1}{5} + \frac{8}{15}\cdot\frac{2}{5} = \frac{7}{15}\]
\end{solution}
\begin{examp}
По статистике 1\% населения болен раком. Тест дает правильный результат
в 99\% случаев. Тест оказался положительным. Найти веротяность того что
человек болен.
\end{examp}
\begin{solution}
$]$ \left.\begin{array}{l}
$H_1$ --- болен \\
$H_2$ --- здоров
\end{array}\right\}
, \(A\) --- тест положительный
\begin{itemize}
\item \(P(H_1) = 0.01\)
\item \(P(H_2) = 0.99\)
\item \(P(A|H_1) = 0.99\)
\item \(P(A|H_2) = 0.01\)
\[ P(H_1|A) = \frac{P(H_1)P(A|H_1)}{P(H_1)P(A|H_1) + P(H_2)P(A|H_2)} = \frac{1}{2} \]
\end{itemize}
Сделаем второй тест:
\begin{itemize}
\item \(P(H_1) = 0.01\)
\item \(P(H_2) = 0.99\)
\item \(P(AA|H_1) = 0.99^2\)
\item \(P(AA|H_2) = 0.01^2\)
\end{itemize}
\[ P(H_1|AA) = \frac{0.99}{0.99 + 0.01} = 0.99 \]
\end{solution}
\end{document}
