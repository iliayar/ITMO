% Created 2021-03-13 Sat 15:16
% Intended LaTeX compiler: pdflatex
\documentclass[english]{article}
\usepackage[T1, T2A]{fontenc}
\usepackage[lutf8]{luainputenc}
\usepackage[english, russian]{babel}
\usepackage{minted}
\usepackage{graphicx}
\usepackage{longtable}
\usepackage{hyperref}
\usepackage{xcolor}
\usepackage{natbib}
\usepackage{amssymb}
\usepackage{stmaryrd}
\usepackage{amsmath}
\usepackage{caption}
\usepackage{mathtools}
\usepackage{amsthm}
\usepackage{tikz}
\usepackage{grffile}
\usepackage{extarrows}
\usepackage{wrapfig}
\usepackage{rotating}
\usepackage{placeins}
\usepackage[normalem]{ulem}
\usepackage{amsmath}
\usepackage{textcomp}
\usepackage{capt-of}

\usepackage{geometry}
\geometry{a4paper,left=2.5cm,top=2cm,right=2.5cm,bottom=2cm,marginparsep=7pt, marginparwidth=.6in}

 \usepackage{hyperref}
 \hypersetup{
     colorlinks=true,
     linkcolor=blue,
     filecolor=orange,
     citecolor=black,      
     urlcolor=cyan,
     }

\usetikzlibrary{decorations.markings}
\usetikzlibrary{cd}
\usetikzlibrary{patterns}
\usetikzlibrary{automata, arrows}

\newcommand\addtag{\refstepcounter{equation}\tag{\theequation}}
\newcommand{\eqrefoffset}[1]{\addtocounter{equation}{-#1}(\arabic{equation}\addtocounter{equation}{#1})}


\newcommand{\R}{\mathbb{R}}
\renewcommand{\C}{\mathbb{C}}
\newcommand{\N}{\mathbb{N}}
\newcommand{\rank}{\text{rank}}
\newcommand{\const}{\text{const}}
\newcommand{\grad}{\text{grad}}

\theoremstyle{plain}
\newtheorem{axiom}{Аксиома}
\newtheorem{lemma}{Лемма}
\newtheorem{manuallemmainner}{Лемма}
\newenvironment{manuallemma}[1]{%
  \renewcommand\themanuallemmainner{#1}%
  \manuallemmainner
}{\endmanuallemmainner}

\theoremstyle{remark}
\newtheorem*{remark}{Примечание}
\newtheorem*{solution}{Решение}
\newtheorem{corollary}{Следствие}[theorem]
\newtheorem*{examp}{Пример}
\newtheorem*{observation}{Наблюдение}

\theoremstyle{definition}
\newtheorem{task}{Задача}
\newtheorem{theorem}{Теорема}[section]
\newtheorem*{definition}{Определение}
\newtheorem*{symb}{Обозначение}
\newtheorem{manualtheoreminner}{Теорема}
\newenvironment{manualtheorem}[1]{%
  \renewcommand\themanualtheoreminner{#1}%
  \manualtheoreminner
}{\endmanualtheoreminner}
\captionsetup{justification=centering,margin=2cm}
\newenvironment{colored}[1]{\color{#1}}{}

\tikzset{->-/.style={decoration={
  markings,
  mark=at position .5 with {\arrow{>}}},postaction={decorate}}}
\makeatletter
\newcommand*{\relrelbarsep}{.386ex}
\newcommand*{\relrelbar}{%
  \mathrel{%
    \mathpalette\@relrelbar\relrelbarsep
  }%
}
\newcommand*{\@relrelbar}[2]{%
  \raise#2\hbox to 0pt{$\m@th#1\relbar$\hss}%
  \lower#2\hbox{$\m@th#1\relbar$}%
}
\providecommand*{\rightrightarrowsfill@}{%
  \arrowfill@\relrelbar\relrelbar\rightrightarrows
}
\providecommand*{\leftleftarrowsfill@}{%
  \arrowfill@\leftleftarrows\relrelbar\relrelbar
}
\providecommand*{\xrightrightarrows}[2][]{%
  \ext@arrow 0359\rightrightarrowsfill@{#1}{#2}%
}
\providecommand*{\xleftleftarrows}[2][]{%
  \ext@arrow 3095\leftleftarrowsfill@{#1}{#2}%
}
\makeatother
\author{Ilya Yaroshevskiy}
\date{\today}
\title{Лекция 5}
\hypersetup{
 pdfauthor={Ilya Yaroshevskiy},
 pdftitle={Лекция 5},
 pdfkeywords={},
 pdfsubject={},
 pdfcreator={Emacs 28.0.50 (Org mode )}, 
 pdflang={English}}
\begin{document}

\maketitle
\tableofcontents


\section{Схемы испытаний и соответствующие распределения}
\label{sec:orga2079de}
\begin{itemize}
\item \(n\) --- число испытаний
\item \(p\) --- вероятность при одном испытании
\item \(q = 1 - p\) --- веротяность неудачи при одном испытании
\end{itemize}


\begin{definition}
\[ k \to C^k_n p^k q^{n - k} \] --- биноминальное распределение с параметрами \(n\) и \(p\)
\end{definition}
\begin{symb}
\(B_{n,p} = B(n, p)\)
\end{symb}
\subsection{Схема до превого успешного испытания}
\label{sec:org08df334}
\begin{definition}
\textbf{Схема до первого успешного испытания}. Пусть проводится бесконечная
 серия испытаний, которая заканчивается после первого успеха под номером \(\tau\)
\end{definition}
\begin{theorem}
\(p(\tau = k) = q^{k - 1}p\)
\end{theorem}
\begin{proof}
\[ p(\tau = k) = p(\underbrace{\text{НН}\dots\text{Н}}_{k - 1}\underset{\substack{\uparrow \\ k}}{\text{У}}) = q^{k - 1}p\]
\end{proof}
\begin{definition}
\(k \to q^{k-1}p,\ 1 \le k \le \infty\) --- называется \textbf{геометрическим распределением} с параметром \(t\)
\end{definition}
\begin{symb}
G(p)
\end{symb}
\begin{remark}
Это распределение обладает так назыаемым свойством отсутствия после действия или свойством нестарения
\end{remark}
\begin{theorem}
\(] p(\tau = k) = q^{k - 1}p\) \\
\uline{Тогда} \(\forall n, k \in \N\ p(\tau > n + k| \tau > n) = p(\tau > k)\)
\end{theorem}
\begin{proof}
По формуле условной вероятности: \[ p(\tau > n + k |\tau > k) = \frac{p(\tau > n + k \text{ и } \tau > j)}{p(\tau > n)} = \frac{p(\tau > n + k)}{p(\tau > n)} \addtag\label{5_1_geom} \]
\(p(\tau > m) = p(\text{первые } m\text{ неудач}) = q^m\)
\[ \ref{5_1_geom} = \frac{q^{n + k}}{q^n} = q^k \]
\end{proof}
\begin{remark}
То, проработет ли девайс \(k\) часов после этого, не зависит от того сколько проработал до этого
\end{remark}
\begin{remark}
Также \(p(\tau = n + k|\tau > n) = p(\tau = k)\)
\end{remark}
\subsection{Испытание с несколькими исходами}
\label{sec:orgad3896a}
Пусть при \(n\) испытаниях могут произойти \(m\) несовместных исходов
\begin{itemize}
\item \(p_i\) --- вероятность \(i\)-го исхода при одном отдельном испытании
\end{itemize}
\begin{theorem}
Вероятность того, что при \(n\) испытаниях первый исход появится \(n_1\) раз, второй \(n_2\) раз, \dots{}, \(m\)-й \(n_m\) раз. \(n_1 + n_2 + \dots + n_m = m\)
\uline{Тогда} \[ p(n_1, n_2, \dots, n_m) = \frac{n!}{n_1!n_2!\dots n_m!}p_1^{n_1}p_2^{n_2}\dots p_m^{n_m} \]
\end{theorem}
\begin{proof}
\(A_1 = \underbrace{11\dots1}_{n_1}\underbrace{22\dots2}_{n_2}\dots \underbrace{m\dots m}_{n_m}\)
\[ p(A_1) = p_1^{n_1}\dots p_n^{n_m} \]
Остальные благоприятные исходы отличаются лишь расположением \(i\)-х исходов по \(n\) местам, а веротяности будут те-же. Всего таких исходов будет:
\[ C^{n_1}_nC^{n_2}_{n - n_1}C^{n_3}_{n - n_1 - n_2}\dots C^{n_m}_{n_m} = \frac{n!}{n_1!n_2!\dots n_m!} \] --- формула для перестановок с повторениями
\end{proof}
\begin{task}
Два одинаковых по силе шахматиста играют матч из 6 партий. Вероятность ничьи при одной пратии --- \(0.5\). Найти веротяность того, что второй игрок две партии выиграл, а три партии свел в ничью
\end{task}
\begin{solution}
Исходы:
\begin{enumerate}
\item первый выиграл
\item второй выиграл
\item ничья
\[ p_3 = \frac{1}{2};\ p_1 = p_2 = \frac{1}{2}\left(1 - \frac{1}{2}\right) = \frac{1}{4};\ n= 6 \]
\[ P(1, 2, 3) = \frac{6!}{1!2!3!}\cdot\left(\frac{1}{4}\right)^1\cdot\left(\frac{1}{4}\right)^2\cdot\left(\frac{1}{2}\right)^3  = \frac{15}{2^7}\]
\end{enumerate}
\end{solution}
\subsection{Урновая схема}
\label{sec:orgdd3e88b}
В урне \(N\) шаров. Из них \(K\) белых, а черных \(N - K\). Из нее выбираем \(n\) шаров без учета порядка. \(k\) --- число вынутых белых
\begin{theorem}[Схема в возвратом]
Вероятность вынуть белый шар не менятеся. \\
\uline{Тогда} \[ p = \frac{K}{N}\quad p_n(k) = C^k_np^k(1 - p)^{n - k} \]
--- биноминальное распределение
\end{theorem}
\begin{theorem}[Схема без возврата]
\uline{Тогда} \[ P_{N,K}(n, k) = \frac{C^k_K\cdot C^{n-k}_{N - K}}{C^n_N} \]
\end{theorem}
\begin{definition}
\[ k \to \frac{C^k_K\cdot C^{n - k}_{N - K}}{C^n_N},\ k \le K \]
назвается \textbf{гипергеометрическим} распределением веротяности
\end{definition}
\begin{lemma}
\[ C^k_K \sim \frac{K^k}{k!} \]
, при \(K \to \infty, K = \const\)
\end{lemma}
\begin{proof}
\[ C^k_K = \frac{K!}{k!(K - k)!} = \frac{K(K - 1)\dots(K - k + 1)}{K^k}\cdot \frac{K^k}{k!} = \underbrace{1 \cdot \left(1 - \frac{1}{K}\right)\cdot\left(1 - \frac{2}{K}\right) \dots \left(1 - \frac{k - 1}{K}\right)}_{\substack{\downarrow \\ 1}} \cdot\frac{K^k}{k!} \sim \frac{K^k}{k!}\]
\end{proof}
\begin{theorem}
\-
\begin{itemize}
\item \(N \to \infty\)
\item \(K \to \infty\)
\item \(\frac{K}{N} \to p \in (0, 1)\)
\item \(n\) и \(0 \le k \le K\) --- фиксированны
\end{itemize}
\uline{Тогда} \[ P_{N,K}(n,k) = \frac{C^k_KC^{n - k}_{N - K}}{C^n_N} \to C^k_np^k(1 - p)^{n - k} \]
\end{theorem}
\begin{proof}
\[ P_{N, K}(n, k) = \frac{C^k_KC^{n - k}_{N - K}}{C^n_N} \xrightarrow[N \to \infty]{} \frac{K^k}{k!}\cdot \frac{(N -K)^{n - k}}{(n - k)!}\cdot \frac{n!}{N^n} = \frac{n!}{k!\cdot(n- k)!}\cdot \frac{K^k}{N^k}\cdot\frac{(N - K)^{n - k}}{N^{n - k}} = \]
\[ = C^k_n\left(\frac{K}{N}\right)^k\left(1 - \frac{K}{N}\right)^{n -k} \xrightarrow[N \to \infty]{} C^k_n\cdot p^k \cdot ( 1- p)^{n - k}\]
\end{proof}
\subsection{Схемы Пуассона. Теорема Пуассона для схемы Бернулли}
\label{sec:orga4ac9e6}
Схема: вероятность успеха при одном отдельном испытании зависит от числа испытаний \(n\) таким образом, чтобы \(n \cdot p_n = \lambda\)(точнее \(np_n \xrightarrow[n \to \infty]{} \lambda\)) \\
Появление очень редких событий в длинном потоке испытаний
\begin{theorem}[Формула Пуассона]
Пусть \(n \to \infty,\ p_n \to 0\), так что \(np_n \to \lambda > 0\) \\
\uline{Тогда} вероятность \(k\) успехов при \(n\) испытаниях \(p(\nu_n = k) = C^k_np_n^k(1 - p_n)^{n -k} \xrightarrow[n \to \infty]{} \frac{\lambda^k}{k!}e^{-\lambda}\)
\end{theorem}
\begin{proof}
Положим \(\lambda_n = np_n\)
\[ p(\nu_n = k) = C^k_np_n^k(1 - p_n)^{n - k} \xrightarrow[n \to \infty]{} \frac{n^k}{k!}\cdot \frac{\lambda_n^k}{n^k}\cdot\left(1 - \frac{\lambda_n}{n}\right)^{n - k} = \frac{\lambda_n^k}{k!}\cdot\left(1 - \frac{\lambda_n}{n}\right)^n\cdot\left(1 - \frac{\lambda_n}{n}\right)^{-k} \xrightarrow[n \to \infty]{} \]
\[ \xrightarrow[n \to \infty]{} \frac{\lambda_n^k}{k!}\cdot\left(1 - \frac{\lambda_n}{n}\right)^n \xrightarrow[n \to \infty]{} \frac{\lambda_n^k}{k!}\cdot\left(\left(1 - \frac{\lambda_n}{n}\right)^{-\frac{n}{\lambda_n}}\right)^{-\lambda_n} \xrightarrow[n \to \infty]{} \frac{\lambda_n^k}{k!}e^{-\lambda_n} \xrightarrow[n \to \infty]{} \frac{\lambda^k}{k!}e^{-\lambda} \]
\end{proof}
\subsubsection{Оценка погрешности в формуле Пуссона}
\label{sec:org341aa8b}
\begin{theorem}
Пусть \(\nu_n\) -- число успешных про \(k\) в схеме Бернулли с вероятностью \(p\)
\[ \lambda = np\quad A \subset \{0, 1, 2, \dotsm n\}\text{ --- произвольное подмножество}\]
\uline{Тогда} погрешность
\[ \left|p(\nu_n \in A) - \sum_{k \in A} \frac{\lambda_k}{k!} e^{-\lambda}\right| \le \min(p, \lambda p) = \min(p, np^2) = \min\left(p, \frac{\lambda^2}{n}\right) \]
\end{theorem}
\begin{remark}
Формулу Пуасснона иногда называют формулой редких событий и применяем при малых \(p\), \(n \ge 100\)
\end{remark}
\begin{task}
Прибор состоит из 1000 элементов. Вероятность отказа каждого элемента \(\frac{1}{1000}\). Какова вероятность отказа больше двух элементов
\end{task}
\begin{solution}
\[ p_n(k) \approx \frac{\lambda^k}{k!}e^{-\lambda} \]
, где \(\lambda = np\)
\begin{itemize}
\item \(n = 1000\)
\item \(p = 0.001\)
\item \(\lambda = np = 1\)
\item \(k > 2\)
\end{itemize}
\[ p(\nu_n > 2) = 1 - p(\nu_n \le 2) = 1 - (p(0) + p(1) + p(2)) \approx 1 - \left(\frac{\lambda^0}{0!} e^{-\lambda} + \frac{\lambda^1}{1!}e^{-\lambda} + \frac{\lambda^2}{2!}e^{-\lambda}\right) = \]
\[ = 1 - 2.5e^{-1} \approx 0.0803\]
Погрешность \(\varepsilon \le \min(p, \lambda p) = 0.001\)
\end{solution}
\end{document}
