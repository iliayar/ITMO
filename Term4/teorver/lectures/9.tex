% Created 2021-04-13 Tue 12:50
% Intended LaTeX compiler: pdflatex

\documentclass[english]{article}
\usepackage[T1, T2A]{fontenc}
\usepackage[lutf8]{luainputenc}
\usepackage[english, russian]{babel}
\usepackage{minted}
\usepackage{graphicx}
\usepackage{longtable}
\usepackage{hyperref}
\usepackage{xcolor}
\usepackage{natbib}
\usepackage{amssymb}
\usepackage{stmaryrd}
\usepackage{amsmath}
\usepackage{caption}
\usepackage{mathtools}
\usepackage{amsthm}
\usepackage{tikz}
\usepackage{grffile}
\usepackage{extarrows}
\usepackage{wrapfig}
\usepackage{algorithm}
\usepackage{algorithmic}
\usepackage{lipsum}
\usepackage{rotating}
\usepackage{placeins}
\usepackage[normalem]{ulem}
\usepackage{amsmath}
\usepackage{textcomp}
\usepackage{capt-of}

\usepackage{geometry}
\geometry{a4paper,left=2.5cm,top=2cm,right=2.5cm,bottom=2cm,marginparsep=7pt, marginparwidth=.6in}
 \usepackage{hyperref}
 \hypersetup{
     colorlinks=true,
     linkcolor=blue,
     filecolor=orange,
     citecolor=black,      
     urlcolor=cyan,
     }

\usetikzlibrary{decorations.markings}
\usetikzlibrary{cd}
\usetikzlibrary{patterns}
\usetikzlibrary{automata, arrows}

\newcommand\addtag{\refstepcounter{equation}\tag{\theequation}}
\newcommand{\eqrefoffset}[1]{\addtocounter{equation}{-#1}(\arabic{equation}\addtocounter{equation}{#1})}
\newcommand{\llb}{\llbracket}
\newcommand{\rrb}{\rrbracket}


\newcommand{\R}{\mathbb{R}}
\renewcommand{\C}{\mathbb{C}}
\newcommand{\N}{\mathbb{N}}
\newcommand{\A}{\mathfrak{A}}
\newcommand{\B}{\mathfrak{B}}
\newcommand{\rank}{\mathop{\rm rank}\nolimits}
\newcommand{\const}{\var{const}}
\newcommand{\grad}{\mathop{\rm grad}\nolimits}

\newcommand{\todo}{{\color{red}\fbox{\text{Доделать}}}}
\newcommand{\fixme}{{\color{red}\fbox{\text{Исправить}}}}

\newcounter{propertycnt}
\setcounter{propertycnt}{1}
\newcommand{\beginproperty}{\setcounter{propertycnt}{1}}

\theoremstyle{plain}
\newtheorem{propertyinner}{Свойство}
\newenvironment{property}{
  \renewcommand\thepropertyinner{\arabic{propertycnt}}
  \propertyinner
}{\endpropertyinner\stepcounter{propertycnt}}
\newtheorem{axiom}{Аксиома}
\newtheorem{lemma}{Лемма}
\newtheorem{manuallemmainner}{Лемма}
\newenvironment{manuallemma}[1]{%
  \renewcommand\themanuallemmainner{#1}%
  \manuallemmainner
}{\endmanuallemmainner}

\theoremstyle{remark}
\newtheorem*{remark}{Примечание}
\newtheorem*{solution}{Решение}
\newtheorem{corollary}{Следствие}[theorem]
\newtheorem*{examp}{Пример}
\newtheorem*{observation}{Наблюдение}

\theoremstyle{definition}
\newtheorem{task}{Задача}
\newtheorem{theorem}{Теорема}[section]
\newtheorem*{definition}{Определение}
\newtheorem*{symb}{Обозначение}
\newtheorem{manualtheoreminner}{Теорема}
\newenvironment{manualtheorem}[1]{%
  \renewcommand\themanualtheoreminner{#1}%
  \manualtheoreminner
}{\endmanualtheoreminner}
\captionsetup{justification=centering,margin=2cm}
\newenvironment{colored}[1]{\color{#1}}{}

\tikzset{->-/.style={decoration={
  markings,
  mark=at position .5 with {\arrow{>}}},postaction={decorate}}}
\makeatletter
\newcommand*{\relrelbarsep}{.386ex}
\newcommand*{\relrelbar}{%
  \mathrel{%
    \mathpalette\@relrelbar\relrelbarsep
  }%
}
\newcommand*{\@relrelbar}[2]{%
  \raise#2\hbox to 0pt{$\m@th#1\relbar$\hss}%
  \lower#2\hbox{$\m@th#1\relbar$}%
}
\providecommand*{\rightrightarrowsfill@}{%
  \arrowfill@\relrelbar\relrelbar\rightrightarrows
}
\providecommand*{\leftleftarrowsfill@}{%
  \arrowfill@\leftleftarrows\relrelbar\relrelbar
}
\providecommand*{\xrightrightarrows}[2][]{%
  \ext@arrow 0359\rightrightarrowsfill@{#1}{#2}%
}
\providecommand*{\xleftleftarrows}[2][]{%
  \ext@arrow 3095\leftleftarrowsfill@{#1}{#2}%
}
\makeatother

\newenvironment{rualgo}[1][]
  {\begin{algorithm}[#1]
     \selectlanguage{russian}%
     \floatname{algorithm}{Алгоритм}%
     \renewcommand{\algorithmicif}{{\color{red}\textbf{если}}}%
     \renewcommand{\algorithmicthen}{{\color{red}\textbf{тогда}}}%
     \renewcommand{\algorithmicelse}{{\color{red}\textbf{иначе}}}%
     \renewcommand{\algorithmicend}{{\color{red}\textbf{конец}}}%
     \renewcommand{\algorithmicfor}{{\color{red}\textbf{для}}}%
     \renewcommand{\algorithmicto}{{\color{red}\textbf{до}}}%
     \renewcommand{\algorithmicdo}{{\color{red}\textbf{делать}}}%
     \renewcommand{\algorithmicwhile}{{\color{red}\textbf{пока}}}%
     \renewcommand{\algorithmicrepeat}{{\color{red}\textbf{повторять}}}%
     \renewcommand{\algorithmicuntil}{{\color{red}\textbf{до тех пор пока}}}%
     \renewcommand{\algorithmicloop}{{\color{red}\textbf{повторять}}}%
     \renewcommand{\algorithmicnot}{{\color{blue}\textbf{не}}}%
     \renewcommand{\algorithmicand}{{\color{blue}\textbf{и}}}%
     \renewcommand{\algorithmicor}{{\color{blue}\textbf{или}}}%
     \renewcommand{\algorithmicrequire}{{\color{blue}\textbf{Предусловие}}}%
     \renewcommand{\algorithmicrensure}{{\color{blue}\textbf{Постусловие}}}%
     \renewcommand{\algorithmicrtrue}{{\color{blue}\textbf{истинна}}}%
     \renewcommand{\algorithmicrfalse}{{\color{blue}\textbf{ложь}}}%
     % Set other language requirements
  }
  {\end{algorithm}}
\author{Ilya Yaroshevskiy}
\date{\today}
\title{Лекция 9}
\hypersetup{
 pdfauthor={Ilya Yaroshevskiy},
 pdftitle={Лекция 9},
 pdfkeywords={},
 pdfsubject={},
 pdfcreator={Emacs 28.0.50 (Org mode 9.4.4)}, 
 pdflang={English}}
\begin{document}

\maketitle
\tableofcontents


\section{Сингулярное распределение}
\label{sec:org2b34c0c}
\begin{definition}
Случайная величина \(\xi\) имеет \textbf{сингулярное распределение}, если существует Борелевское множество \(B \in \mathfrak{B}\), с нулевой мерой Лебега \(\lambda B = 0\), такое что \(p(\xi \in B) = 1\), но \(\forall x \in B\ p(\xi = x) = 0\) 
\end{definition}
\begin{remark}
\[ \forall x \in B\ p(\xi = x) = 0 \implies p(\xi \in B) = 0 \]
Иными словами, при сингулярном распределении, случайная величина распределена на несчетном множестве меры 0
\end{remark}
\begin{remark}
Функция распределения --- непрерывная функция, по свойству 7 функии распределения.
\end{remark}
\begin{examp}
Случайная величина \(\xi\), задана функция распределения, которая --- лестница Кантора
\todo
\end{examp}
\begin{theorem}[Лебега]
Пусть \(F_\xi(x)\) --- функция распределения произвольной случайной величины \(\xi\) \\
\uline{Тогда}
\[ F_\xi(x) = p_1F_1(x) + p_2F_2(x) + p_3F_3(x) \quad p_1 + p_2 + p_3 = 1\]
, где \(F_1(x)\) --- фунция дискретного распределения, \(F_2(x)\) --- функция абсолютно непрерывного распределения, \(F_3(x)\) --- функция сингулярного распределения. Т.е. все распределения делятся на дискретные, абсолютно непрерывные, сингуряные и их смеси
\end{theorem}
\section{Общий взгляд на математическое ожидание}
\label{sec:org2a5f7ca}
Пусть случайная величина \(\xi\) задана на вероятностном пространстве \((\Omega, \mathcal{F}, p)\). Математическим ожиданием случайной величины \(\xi\) называется интеграл Лебега:
\[ E\xi = \int_\Omega \xi(\omega) dp(\omega) \addtag\label{int_1_9} \], при условии, что данный интеграл существует. Использую интеграл Стилтьеса, эту формулу можно записать в виде:
\[ E\xi = \int_{-\infty}^{\infty} x dF_\xi(x) \addtag\label{int_2_9} \]
Из определения интеграла Стилтьеса можно получить геометрическую интерпретацию математического ожидания

Рассмотрим две ситуации:
\begin{enumerate}
\item Вероятностное пространство \((\Omega, \mathcal{F}, p)\) --- дискретное вероятностное пространство, т.е. \(\Omega\) состоит из н.б.ч.с числа точек. Тогда из \ref{int_1_9} получаем:
\[ E\xi = \sum_{i = 1}^\infty \xi(\omega_i) p(\omega_i) \]
\begin{examp}
\todo
\end{examp}
\item \((\Omega, \mathcal{F}, p)\) --- непрерывное вероятностное пространство. например \(\Omega \subset \R^m\), \(\omega = (x_1, x_2, \dots, x_m)\), тогда из \ref{int_2_9} получаем:
\[ E\xi = \iint\dots\int_\Omega \xi(x_1, x_2, \dots, x_m)\cdot p(x_1, x_2, \dots, x_m) dx_1 dx_2\dots dx_m \]
\begin{examp}
В круг радиуса 3 наугад бросается точка, случайная величина \(\xi\), расстояния от центра круга до данной точки. Найти мат. ожидание \(\xi\).
\[ \Omega = \{(x, y) \Big| x^2 + y^2 \le 9\} \]
\[ \xi = \sqrt{x^2 + y^2} \]
\[ p(x, y) = p = \const \]
Из условия нормировки:
\[ \int_\Omega \alpha p(\omega) = 1 \text{ или } \iint_\Omega p\,dx\,dy = 1 \implies \frac{1}{9 \pi} \]
\[ E\xi = \iint_\Omega \xi(x, y)\cdot p dx dy = \iint_\Omega \sqrt{x^2 + y^2} \frac{1}{9 \pi} dx dy = \]
\[ = \frac{1}{9\pi} \int_0^\pi d\varphi \int_0^3 \rho \cdot \rho d\rho = \frac{1}{9\pi}\cdot\pi\cdot\frac{\rho^3}{3} \bigg|_0^3 = 2 \]
\fixme
\end{examp}
\end{enumerate}
\section{Преобразование случайных величин}
\label{sec:org7de8dae}
\(\xi\) --- случайная величина на \((\Omega, \mathcal{F}, p)\), \(g: \R \to \R\). Тогда функция \(g(\xi)\)
\begin{definition}
Функция \(g(x): \R \to \R\) --- Борелевская функция, если \(\forall B \in mathfrak{B}\), \(g^{-1}(B) \in \mathfrak{B}\)
\end{definition}
\begin{theorem}
Если \(g(x)\) --- Борелевская функция и \(\xi\) --- случайная величина на \((\Omega, \mathcal{F}, p)\), то \(g(\xi)\) --- случайная величина на \((\Omega, \mathcal{F}, p)\)
\end{theorem}
\begin{proof}
\todo
\end{proof}
\begin{remark}
Если \(\xi\) --- дискретная случайная величина, то ее закон распределения находится просто из определения, поэтому в дальнейшем будем считать, что \(\xi\) имеет абсолютно непрерывное распределние
\end{remark}
\subsection{Стандартизация случайной величины}
\label{sec:orgc48b1eb}
\begin{definition}
Пусть имеется случайная величина \(\xi\) с соответствующей ей стандортной величиной:
\[ \eta = \frac{\xi - E\xi}{\sigma} \]
\end{definition}
\beginproperty
\begin{property}
\(E\eta = 0\), \(D\eta = 1\)
\end{property}
\begin{proof}
\todo
\end{proof}
\begin{remark}
При стандартизации тип распределения не всегда сохраняется
\end{remark}
\subsection{Линейное преобразование}
\label{sec:orgba7bdf0}
\begin{theorem}
Пусть случайная величина \(\xi\) имеет плотность \(f_\xi(x)\) \\
\uline{Тогда} случайная величина \(\eta = a\xi + b,\ a \neq 0\) имеет плотность:
\[ f_\eta(x) = \frac{1}{|a|}\cdot f_\xi \left(\frac{x - b}{a}\right) \]
\end{theorem}
\begin{proof}
\-
\begin{enumerate}
\item \(a > 0\), тогда:
\[ F_\eta(x) = p(a\xi + b < x) = p(\xi < \frac{x - b}{a}) = \int_{-\infty}^{\frac{x - b}{a}} f_\xi(t) dt \]
\[ = \left[\begin{matrix} t = \frac{y - b}{a} & dt = \frac{1}{a}dy & y = at + b \\ y(-\infty) = -\infty & y \left(\frac{x - b}{a}\right) = x &  \end{matrix}\right. = \int \]
\end{enumerate}
\todo
\end{proof}
\beginproperty
\begin{property}
Если \(\xi \in N(0, 1)\), то \(\eta = \sigma \xi + a \in N(a, \sigma^{-1})\)
\end{property}
\begin{proof}
\todo
\end{proof}
\begin{property}
Если \(\eta \in N(a, \sigma^2)\), то \(\xi = \frac{\eta - a}{\sigma} \in N(0, 1)\)
\end{property}
\begin{property}
Если \(\eta \in N(a, \sigma^2)\), то \(\xi = \gamma\eta + b \in N(a\gamma + b, \gamma^2\sigma^2)\)
\end{property}
\begin{property}
Если \(\xi \in U(0, 1)\), то \(\eta = a \xi + b \in U(b, a + b)\) при \(a > 0\)
\end{property}
\begin{property}
Если \(\xi \in E_\alpha\), то \(\eta = \alpha \xi \in E_1\)
\end{property}

\begin{theorem}
Пусть \(f_\xi(x)\) --- плотность случайной величины \(\xi\) и функция \(\g(x)\) --- монотонная. Тогда существует обратная \(h(t) = g^{-1}(x)\) и случайная величина \(\eta = g(\xi)\) имеет плотность:
\[ f_\eta(x) = \frac{1}{|h'(x)|} f _\xi(h(x)) \]
\end{theorem}
\subsection{Квантильное преобразование}
\label{sec:org1b9839a}
\begin{theorem}
Пусть функция распределения \(F(x)\) случайной величины \(\xi\) --- непрерывная, тогда случайная величина \(\eta = F(\xi) \in U(0, 1)\) --- имеет стандартное равномерное распределение
\end{theorem}
\begin{proof}
Ясно, что \(0 \le \eta \le 1\)
\begin{enumerate}
\item Предположим сначала, что \(F(x)\) --- строго возрастающая функция. Тогда она имеет обратную функцию \(F^{-1}(x)\) и
\[ F_\eta(x) = p(F(\xi) < x) = p(\xi < F^{-1}(x)) = \begin{cases}
   0 & x < 0 \\
   F(F^{-1}(x)) = x & 0 \le x \le 1 \\
   1 & x > 1
   \end{cases} \implies \]
\(\implies \eta \in U(0, 1)\)
\item Пусть функция не является строго возрастающей, т.е. у нее есть интервалы постоянства, в этом случае через \(F^{-1}(x)\) обозначим, самую левую точку такого интервала:
\[ F^{-1}(x) = \min_t\{t \Big| F(t) = x)\} \]
--- корректно, т.к. \(F(x)\) непрерывна слева. Тогда снова будет верна цепочка:
\[ F_\eta(x) = p(F(\xi) < x) = p(\xi < F^{-1}(x)) = F(F^{-1}(x)) = x\quad 0 \le x \le 1\]
\end{enumerate}
\end{proof}
Сформулируем теперь обратную теорему: \\
Пусть \(F(x)\) --- функция распределения случаайной величины \(\xi\), при чем не обязательно непрерывная. Обозначим через \(F^{-1}(x) = \inf\{t \Big| F(t) \ge x\}\)
\begin{theorem}
Пусть \(\eta \in U(0, 1)\), \(F(x)\) --- произвольная функция распределения. \\
\uline{Тогда} случайная величина \(\xi = F^{-1}(\eta)\) имеет функцию распределения \(F(x)\)
\end{theorem}
\begin{remark}
\(F^{-1}(\eta)\) называется квантильным преобразованием над случайной величиной \(\eta\)
\end{remark}
\begin{corollary}
Датчики случайных чисел обычно имеют стандартное равномерное распределение. Из теоремы следует, что при помощи датчика случайных числе и квантильного преобразования, мы можем смоделировать любое желаемое распределение, в том числе дискретное.
\end{corollary}
\begin{examp}
\(E_\alpha\):
\[ F(x) = \begin{cases}
0 & x < 0 \\
1 - e^{-\alpha x} & x \ge 0
\end{cases}\]
\[ \eta = 1 - e^{-\alpha x} \implies x = -\frac{1}{\alpha}\ln(1 - \eta) \]
Если \(\eta \in U(0, 1)\), то \(\xi = \frac{1}{\alpha} \ln(1 - \eta) \in E_\alpha\)
\end{examp}
\begin{examp}
\(N(0, 1)\):
\[ \Phi_0(x) = \frac{1}{\sqrt{2 \pi}} \int_{-\infty}^x  e^{-\frac{z^2}{2}} dz\]
\[ \Phi^{-1}_0 \in N(0, 1) \]
\end{examp}
\end{document}
