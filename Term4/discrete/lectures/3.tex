% Created 2021-03-05 Fri 23:12
% Intended LaTeX compiler: pdflatex
\documentclass[english]{article}
\usepackage[T1, T2A]{fontenc}
\usepackage[lutf8]{luainputenc}
\usepackage[english, russian]{babel}
\usepackage{minted}
\usepackage{graphicx}
\usepackage{longtable}
\usepackage{hyperref}
\usepackage{xcolor}
\usepackage{natbib}
\usepackage{amssymb}
\usepackage{amsmath}
\usepackage{caption}
\usepackage{mathtools}
\usepackage{amsthm}
\usepackage{tikz}
\usepackage{grffile}
\usepackage{extarrows}
\usepackage{wrapfig}
\usepackage{rotating}
\usepackage{placeins}
\usepackage[normalem]{ulem}
\usepackage{amsmath}
\usepackage{textcomp}
\usepackage{capt-of}

\usepackage{geometry}
\geometry{a4paper,left=2.5cm,top=2cm,right=2.5cm,bottom=2cm,marginparsep=7pt, marginparwidth=.6in}

 \usepackage{hyperref}
 \hypersetup{
     colorlinks=true,
     linkcolor=blue,
     filecolor=orange,
     citecolor=black,      
     urlcolor=cyan,
     }

\usetikzlibrary{decorations.markings}
\usetikzlibrary{cd}
\usetikzlibrary{patterns}

\newcommand\addtag{\refstepcounter{equation}\tag{\theequation}}
\newcommand{\eqrefoffset}[1]{\addtocounter{equation}{-#1}(\arabic{equation}\addtocounter{equation}{#1})}


\newcommand{\R}{\mathbb{R}}
\renewcommand{\C}{\mathbb{C}}
\newcommand{\N}{\mathbb{N}}
\newcommand{\rank}{\text{rank}}
\newcommand{\const}{\text{const}}
\newcommand{\grad}{\text{grad}}

\theoremstyle{plain}
\newtheorem{axiom}{Аксиома}
\newtheorem{lemma}{Лемма}
\newtheorem{manuallemmainner}{Лемма}
\newenvironment{manuallemma}[1]{%
  \renewcommand\themanuallemmainner{#1}%
  \manuallemmainner
}{\endmanuallemmainner}

\theoremstyle{remark}
\newtheorem*{remark}{Примечание}
\newtheorem*{solution}{Решение}
\newtheorem{corollary}{Следствие}[theorem]
\newtheorem*{examp}{Пример}
\newtheorem*{observation}{Наблюдение}

\theoremstyle{definition}
\newtheorem{task}{Задача}
\newtheorem{theorem}{Теорема}[section]
\newtheorem*{definition}{Определение}
\newtheorem*{symb}{Обозначение}
\newtheorem{manualtheoreminner}{Теорема}
\newenvironment{manualtheorem}[1]{%
  \renewcommand\themanualtheoreminner{#1}%
  \manualtheoreminner
}{\endmanualtheoreminner}
\captionsetup{justification=centering,margin=2cm}
\newenvironment{colored}[1]{\color{#1}}{}

\tikzset{->-/.style={decoration={
  markings,
  mark=at position .5 with {\arrow{>}}},postaction={decorate}}}
\makeatletter
\newcommand*{\relrelbarsep}{.386ex}
\newcommand*{\relrelbar}{%
  \mathrel{%
    \mathpalette\@relrelbar\relrelbarsep
  }%
}
\newcommand*{\@relrelbar}[2]{%
  \raise#2\hbox to 0pt{$\m@th#1\relbar$\hss}%
  \lower#2\hbox{$\m@th#1\relbar$}%
}
\providecommand*{\rightrightarrowsfill@}{%
  \arrowfill@\relrelbar\relrelbar\rightrightarrows
}
\providecommand*{\leftleftarrowsfill@}{%
  \arrowfill@\leftleftarrows\relrelbar\relrelbar
}
\providecommand*{\xrightrightarrows}[2][]{%
  \ext@arrow 0359\rightrightarrowsfill@{#1}{#2}%
}
\providecommand*{\xleftleftarrows}[2][]{%
  \ext@arrow 3095\leftleftarrowsfill@{#1}{#2}%
}
\makeatother
\author{Ilya Yaroshevskiy}
\date{\today}
\title{Лекция 3}
\hypersetup{
 pdfauthor={Ilya Yaroshevskiy},
 pdftitle={Лекция 3},
 pdfkeywords={},
 pdfsubject={},
 pdfcreator={Emacs 28.0.50 (Org mode )}, 
 pdflang={English}}
\begin{document}

\maketitle
\tableofcontents

\newcommand{\vdomino}{
\begin{tikzpicture}
\draw (0, 0) rectangle (0.15,0.3);
\end{tikzpicture}}
\newcommand{\hdomino}{
\begin{tikzpicture}
\draw (0, 0) rectangle (0.3,0.15);
\draw (0, 0.15) rectangle (0.3,0.3);
\end{tikzpicture}}

\section{Производящие функции для объектов}
\label{sec:orgb8efd65}
\begin{itemize}
\item Оюъединение \\
\(A, B\ A \cap B = \emptyset\ C = A \cup B\) \\
\(A(t)\ B(t)\)
\[ C(t) = A(t) + B(t)\]
\[ c_n = a_n + b_n \]
\item Пара \\
\(C = A \times B\ \text{Pair}(A, B)\)
\[C(t) = A(t) \cdot B(t)\] 
\[ c_n = \sum_{i = 0}^na_nb_n \]
\item Последовательности \\
\(C = \text{Seq }A = A^0 \cup A^1 \cup A^2 \cup A^3 \cup \dots\ a_0 = 0\)
\[ C(t) = 1 + A(t) + A(t)\cdot A(t) + A(t)^3 + \dots \]
\[ C(t) = \frac{1}{1 - A(t)} \]
\item Множества \\
\(\varepsilon\) вес \(0\) \\
\(\text{Set }A = \bigtimes_{a \in A} (\varepsilon \cup a)\) \\
\[ C(t) = \prod_{a \in A}(1 + t^{\omega(a)}) = \prod_{k = 0}^\infty (1 + t^k)^{a_k} \]
\begin{examp}
\(\text{Set}\left\{\vdomino, \hdomino\right\}\ a_1 = 1,a_2 = 1\) \\
\[ C(t) = (1 + t)(1 + t^2) = t^3 + t^2 + t + 1 \]
\end{examp}
\item Мультимножества \\
\[\text{MSet} A = \bigtimes_{a \in A}(\varepsilon \cup a \cup a^2 \cup \dots) = \prod_{a \in A}\text{Seq}\{a\}\]
\[ C(t) = \prod_{a \in A}\frac{1}{1 - t^{\omega{a}}} = \prod_{k = 1}^\infty\left(\frac{1}{1 - t^k}\right)^{a_k} = \prod_{k = 1}^\infty(1 - t^k)^{-a_k}\]
\begin{examp}
\(\text{MSet}\{\vdomino, \hdomino\}\)
\[ C(t) = \frac{1}{(1 - t)(1 - t^2)} = \frac{1}{(1 - t^2)(1 + t)} \]
\[ c_n = dn + e + f\cdot(-1)^n \]
\end{examp}
\end{itemize}
\begin{examp}
\(\text{Seq}_{=k}(A) = A^k\) --- ровно 3 элемента \\
\(\text{Seq}_{\ge k}(A) = A^k \times \text{Seq}(A)\frac{A(t)^k}{1 - A(t)}\) \\
\(\text{Seq}_{\le k}(A) = \frac{1}{1 - A(t)} - \frac{A(t)^{k + 1}}{1 - A(t)} = \frac{1 - A(t)^{k + 1}}{1 - A(t)}\)
\end{examp}
\end{document}
