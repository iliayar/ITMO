% Created 2021-03-16 Tue 23:19
% Intended LaTeX compiler: pdflatex

  \documentclass[oneside]{book}
  \usepackage[T1, T2A]{fontenc}
\usepackage[lutf8]{luainputenc}
\usepackage[english, russian]{babel}
\usepackage{minted}
\usepackage{graphicx}
\usepackage{longtable}
\usepackage{hyperref}
\usepackage{xcolor}
\usepackage{natbib}
\usepackage{amssymb}
\usepackage{stmaryrd}
\usepackage{amsmath}
\usepackage{caption}
\usepackage{mathtools}
\usepackage{amsthm}
\usepackage{tikz}
\usepackage{grffile}
\usepackage{extarrows}
\usepackage{wrapfig}
\usepackage{rotating}
\usepackage{placeins}
\usepackage[normalem]{ulem}
\usepackage{amsmath}
\usepackage{textcomp}
\usepackage{capt-of}
  
  \addto\captionsrussian{\renewcommand{\chaptername}{Лекция}}
  
   \usepackage{hyperref}
   \hypersetup{
       colorlinks=true,
       linkcolor=blue,
       filecolor=orange,
       citecolor=black,      
       urlcolor=cyan,
       }

  \usetikzlibrary{decorations.markings}
  \usetikzlibrary{cd}
  \usetikzlibrary{patterns}
  \usetikzlibrary{automata, arrows}

  \newcommand\addtag{\refstepcounter{equation}\tag{\theequation}}
  \newcommand{\eqrefoffset}[1]{\addtocounter{equation}{-#1}(\arabic{equation}\addtocounter{equation}{#1})}


  \newcommand{\R}{\mathbb{R}}
  \renewcommand{\C}{\mathbb{C}}
  \newcommand{\N}{\mathbb{N}}
  \newcommand{\rank}{\text{rank}}
  \newcommand{\const}{\text{const}}
  \newcommand{\grad}{\text{grad}}

  \theoremstyle{plain}
  \newtheorem{axiom}{Аксиома}
  \newtheorem{lemma}{Лемма}
  \newtheorem{manuallemmainner}{Лемма}
  \newenvironment{manuallemma}[1]{%
    \renewcommand\themanuallemmainner{#1}%
    \manuallemmainner
  }{\endmanuallemmainner}

  \theoremstyle{remark}
  \newtheorem*{remark}{Примечание}
  \newtheorem*{solution}{Решение}
  \newtheorem{corollary}{Следствие}[theorem]
  \newtheorem*{examp}{Пример}
  \newtheorem*{observation}{Наблюдение}

  \theoremstyle{definition}
  \newtheorem{task}{Задача}
  \newtheorem{theorem}{Теорема}[section]
  \newtheorem*{definition}{Определение}
  \newtheorem*{symb}{Обозначение}
  \newtheorem{manualtheoreminner}{Теорема}
  \newenvironment{manualtheorem}[1]{%
    \renewcommand\themanualtheoreminner{#1}%
    \manualtheoreminner
  }{\endmanualtheoreminner}
  \captionsetup{justification=centering,margin=2cm}
  \newenvironment{colored}[1]{\color{#1}}{}

  \tikzset{->-/.style={decoration={
    markings,
    mark=at position .5 with {\arrow{>}}},postaction={decorate}}}
  \makeatletter
  \newcommand*{\relrelbarsep}{.386ex}
  \newcommand*{\relrelbar}{%
    \mathrel{%
      \mathpalette\@relrelbar\relrelbarsep
    }%
  }
  \newcommand*{\@relrelbar}[2]{%
    \raise#2\hbox to 0pt{$\m@th#1\relbar$\hss}%
    \lower#2\hbox{$\m@th#1\relbar$}%
  }
  \providecommand*{\rightrightarrowsfill@}{%
    \arrowfill@\relrelbar\relrelbar\rightrightarrows
  }
  \providecommand*{\leftleftarrowsfill@}{%
    \arrowfill@\leftleftarrows\relrelbar\relrelbar
  }
  \providecommand*{\xrightrightarrows}[2][]{%
    \ext@arrow 0359\rightrightarrowsfill@{#1}{#2}%
  }
  \providecommand*{\xleftleftarrows}[2][]{%
    \ext@arrow 3095\leftleftarrowsfill@{#1}{#2}%
  }
  \makeatother
\author{Ilya Yaroshevskiy}
\date{\today}
\title{Лекции по Дискретной математике 4 семестр}
\hypersetup{
 pdfauthor={Ilya Yaroshevskiy},
 pdftitle={Лекции по Дискретной математике 4 семестр},
 pdfkeywords={},
 pdfsubject={},
 pdfcreator={Emacs 28.0.50 (Org mode )}, 
 pdflang={English}}
\begin{document}

\maketitle
\tableofcontents


\chapter{}
\label{sec:org35205b9}
\chapter{}
\label{sec:org5e1ce4d}
\section{Производящие функции}
\label{sec:orgd8b4274}
\begin{definition}
\textbf{Полином} --- степенныой ряд, у которого начиная с некоторого места
 \(n\) все коэффиценты 0.
\end{definition}
\begin{symb}
\(\deg p = n\)
\end{symb}
\begin{definition}
\(\frac{P(t)}{Q(t)}\) --- \textbf{дробно рациональная функция}
\end{definition}
\subsection{Рекурентные соотношения}
\label{sec:orgf00c3a6}
\begin{definition}
\[ m: a_0, a_1, \dots, a_{m - 1} \]
\(k \le m, n \ge m\) \[ a_n = c_1a_{n-1} + \dots + a_ka_{n - k} \], где \(c_1, \dots, c_k\) --- коэффиценты рекурентности
\end{definition}
\begin{examp}
\-
\begin{itemize}
\item \(m = 2,\ k = 2\)
\item \(f_0 = f_1 = 1\)
\item \(c_1 = c_2 = 1\)
\end{itemize}
f\textsubscript{n} = f\textsubscript{n - 1} + f\textsubscript{n - 2} --- числа Фибоначи
\end{examp}


\begin{definition}
\textbf{Квазиполином}
\[ f(n) = \sum_{i = 1}^k p_i(n)r_i^n \], где \(p_i\) --- полином, \(r_i\) --- числа
\end{definition}
\begin{theorem}
\begin{itemize}
\item \(a_0, a_1, \dots, a_n, \dots\)
\end{itemize}
\uline{Тогда} эквивалентны:
\begin{enumerate}
\item \(A(t) = \frac{P(t)}{Q(t)}\), \(P, Q\) --- полиномы, \(q_0 \neq 0\)
\item для \(n \ge m\) \(a_n\) задается линейным рекурентным соотношением: \(a_n = c_1a_{n - 1} + \dots + c_ka_{n - k}\), причем:
\begin{itemize}
\item \(Q(t) = 1 - c_1t - c_2t^2 - \dots - c_kt^k\)
\item \(\deg P \le m - 1\)
\end{itemize}
\item \(a_n\) --- квазиполином \[ a_n = \sum_{i = 1}^k p_i(n)r_i^n \label{kvazi_1}\addtag \]
причем:
\begin{itemize}
\item \(r_i\) --- обратные величины корням \(Q(t)\)
\item \(k\) --- число различных его корней
\item \(\deg p_i = (\text{кратность корня}(r_i^{-1})) - 1\) \\
(\ref{kvazi_1} кроме \(\le m\) первых членов)
\end{itemize}
\end{enumerate}
\end{theorem}
\chapter{}
\label{sec:orgc5250f3}
\newcommand{\vdomino}{
\begin{tikzpicture}
\draw (0, 0) rectangle (0.15,0.3);
\end{tikzpicture}}
\newcommand{\hdomino}{
\begin{tikzpicture}
\draw (0, 0) rectangle (0.3,0.15);
\draw (0, 0.15) rectangle (0.3,0.3);
\end{tikzpicture}}

\section{Производящие функции для объектов}
\label{sec:orgd50eb2a}
\begin{itemize}
\item Оюъединение \\
\(A, B\ A \cap B = \emptyset\ C = A \cup B\) \\
\(A(t)\ B(t)\)
\[ C(t) = A(t) + B(t)\]
\[ c_n = a_n + b_n \]
\item Пара \\
\(C = A \times B\ \text{Pair}(A, B)\)
\[C(t) = A(t) \cdot B(t)\] 
\[ c_n = \sum_{i = 0}^na_nb_n \]
\item Последовательности \\
\(C = \text{Seq }A = A^0 \cup A^1 \cup A^2 \cup A^3 \cup \dots\ a_0 = 0\)
\[ C(t) = 1 + A(t) + A(t)\cdot A(t) + A(t)^3 + \dots \]
\[ C(t) = \frac{1}{1 - A(t)} \]
\item Множества \\
\(\varepsilon\) вес \(0\) \\
\(\text{Set }A = \bigtimes_{a \in A} (\varepsilon \cup a)\) \\
\[ C(t) = \prod_{a \in A}(1 + t^{\omega(a)}) = \prod_{k = 0}^\infty (1 + t^k)^{a_k} \]
\begin{examp}
\(\text{Set}\left\{\vdomino, \hdomino\right\}\ a_1 = 1,a_2 = 1\) \\
\[ C(t) = (1 + t)(1 + t^2) = t^3 + t^2 + t + 1 \]
\end{examp}
\item Мультимножества \\
\[\text{MSet} A = \bigtimes_{a \in A}(\varepsilon \cup a \cup a^2 \cup \dots) = \prod_{a \in A}\text{Seq}\{a\}\]
\[ C(t) = \prod_{a \in A}\frac{1}{1 - t^{\omega{a}}} = \prod_{k = 1}^\infty\left(\frac{1}{1 - t^k}\right)^{a_k} = \prod_{k = 1}^\infty(1 - t^k)^{-a_k}\]
\begin{examp}
\(\text{MSet}\{\vdomino, \hdomino\}\)
\[ C(t) = \frac{1}{(1 - t)(1 - t^2)} = \frac{1}{(1 - t^2)(1 + t)} \]
\[ c_n = dn + e + f\cdot(-1)^n \]
\end{examp}
\end{itemize}
\begin{examp}
\(\text{Seq}_{=k}(A) = A^k\) --- ровно 3 элемента \\
\(\text{Seq}_{\ge k}(A) = A^k \times \text{Seq}(A)\frac{A(t)^k}{1 - A(t)}\) \\
\(\text{Seq}_{\le k}(A) = \frac{1}{1 - A(t)} - \frac{A(t)^{k + 1}}{1 - A(t)} = \frac{1 - A(t)^{k + 1}}{1 - A(t)}\)
\end{examp}
\chapter{}
\label{sec:org7861030}
\usetikzlibrary{automata}



\section{Производящие функции для регулярных языков}
\label{sec:org478a7b7}
\(L\) --- регулярный язык
\[ | L \cap \Sigma^n | = a_n \] 
\[ L(t) = a_0 + a_1 t + \dots \]
\begin{remark}
\(L\) --- регулярная спецификация \\
\(\psi\) --- регулярное выражение:
\begin{enumerate}
\item \(L(\psi) = L\)
\item \(\forall x \in \L\ \exists !\) способ \(x\) удовлетворяющий \(\psi\)
\end{enumerate}
\end{remark}
\begin{lemma}
\(\Sigma\) --- конечный алфавит, \(L \subset \Sigma^*\) \\
\(L\) --- регулярная спецификация \(\Leftrightarrow\) \(L\) получаетя из \(\Sigma\):
\begin{enumerate}
\item Дизъюнктное объединение \(+\)
\item Прямое произведение \(\times\)
\item Последовательность Seq
\end{enumerate}
\end{lemma}
\begin{proof}
Общее рассжудение: по индукции рассмотрим для каждой операции во что
она перейдет, надо показать что единственность вывода сохраняется \color{red} Не работает \color{black}
\end{proof}
\begin{examp}
\[ ab^*|a^*b \]
\[ \color{blue}a \times \text{Seq }b\color{black}|\color{blue}\text{Seq } a \times b\color{black} \]
объединение дизъюнктное? \(\Rightarrow\) не регелярная спецификация
\end{examp}
\begin{examp}
\[ (ab^*)^* \]
\[ \color{blue} \text{Seq}(a \times \text{Seq } b) \color{black} \]
\end{examp}
\begin{theorem}
Если у \(L\) есть регулярная спецификация, то \(L\) --- дробно рациональная
\end{theorem}
\begin{theorem}[Производящая функция регулярного языка]
\(L\) --- регулярный язык над \(\Sigma\), ДКА \(A\): \\
\begin{itemize}
\item Состояния \(Q,\ |Q| = n\)
\item \(s \in Q\) --- стартовое сотояние
\item \(T \subset Q\) --- терминальные
\end{itemize}
\[ u = (\overbrace{0, 0, \dots, \underbrace{1}_s, 0, \dots, 0}^n) \]
\[ v = (\overbrace{0, \underbrace{1}_{\in T}, 0, \underbrace{1}_{\in T}, \dots, \underbrace{1}_{\in T}, 0}^n) \]

\[ D = (d_{ij})^T,\ d_{ij} = |\{c | i \xrightarrow[]{c} j\}| \]
\[ L(t) = \vec{u}(I - tD)^{-1}\vec{v} \]
\end{theorem}
\begin{examp}
Язык из слов, которые содержат abb как подстроку \\
\begin{center}
\begin{tikzpicture}
\node[state, initial] at (0, 0) (A) [circle] {\(0\)};
\node[state] at (2, 0) (B) [circle] {\(1\)};
\node[state] at (4, 0) (C) [circle] {\(2\)};
\node[state,accepting] at (6, 0) (D) [circle] {\(3\)};
\draw[->] (A) edge node[above] {a} (B);
\draw[->] (A) edge[loop above] node[above] {b} (A);
\draw[->] (B) edge[loop above] node[above] {a} (B);
\draw[->] (B) edge node[above] {b} (C);
\draw[->] (C) edge[loop above] node[above] {a} (B);
\draw[->] (C) edge node[above] {b} (D);
\draw[->] (D) edge[loop above] node[above] {a} (D);
\draw[->] (D) edge[loop right] node[below] {b} (D);
\end{tikzpicture}
\end{center}
\[ \begin{pmatrix} L_0 \\ L_1 \\ L_2 \\ L_3\end{pmatrix}  = \begin{pmatrix}0 \\ 0 \\ 0 \\ 1\end{pmatrix} + t \begin{pmatrix}1 & 1 & 0 & 0 \\ 0 & 1 & 1 & 0 \\ 0 & 1 & 0 & 1 \\ 0 & 0 & 0 & 2\end{pmatrix}\begin{pmatrix}L_0 \\ L_1 \\ L_2 \\ L_3\end{pmatrix}\]
\[ L_0 = \frac{t^3}{(1 - t)(1 - 2t)(1 - t - t^2)} \]
\end{examp}
\section{Автомат КМП и автокор. многочлен}
\label{sec:org9310257}
Конструкция Гуибаса-Одлызко
\[ p = \fbox{p_1, p_2, \dots, p_k} \]
\[ c_i = [p[i+1\dots k] = p[1\dots k-i]] \]
\[ c(t) = c_0 + c_1 t + c_2 t^2 + \dots + c_{k - 1}t^{k - 1} \]
\begin{examp}
\(p = \text{aabbaa}\) \\
\(c = (1, 0, 0, 0, 1, 1)\) \\
\(c(t) = 1 + t^4 + t^5\)
\end{examp}
\begin{theorem}
\-
\begin{itemize}
\item \(\Sigma,\ |\Sigma| = m\)
\end{itemize}
\(S_n\) --- количество слов длины \(n\), не содержащих \(p\)
\[ S(t) = s_0 + s_1t + s_2t^2 + \dots \]
\[ S(t) = \frac{c(t)}{t^k + (1 - mt)c(t)} \]
\end{theorem}
\begin{examp}
\(p = \text{abb}\)
\[ c(t) = 1 \]
\[ \frac{1}{t^3 + (1 - 2t)\cdot 1} = \frac{1}{1 - 2t + t^3} \]
\end{examp}

\subsection{Пентагональная формула Эйлера}
\label{sec:org26cb94f}
\[ p_0\ p_1\ p_2\ \dots\ p_n\ \dots \]
\(p_n\) --- количество разбиений \(n\) на слагаемые из \(\N\). Порядок не важен
\begin{itemize}
\item \(U = \{0\},\ u_1= 1,\ U(t) = t\)
\item \(N = \text{Seq}^+U=\)положительно целые числа
\item \(P = \text{MSet }N\)
\[ P(t) = \prod_{k = 1}^\infty \frac{1}{1 - t^k} \]
\end{itemize}
\[ Q(t) = \prod_{k = 1}^\infty (1 - t^k) \]
\[ R(t) = \prod_{k = 1}^\infty(1 + t^k)\ [t^n]R \to r_n \]
\(r_n\) --- количество разбиений на различные слагаемые
\[ [t^n]Q = \sum_{\substack{\text{разбиение } n \text{ на } \\ \text{различные слагаемые}}} (-1)^\text{число слагаемых} \]
\[ q_n = e_n - o_n \]
\(e_n\) --- число разбиений на четное число различных слагаемы, \(o_n\) --- число разбиений на нечетное число различных слагаемы, 
\begin{theorem}
\[ Q(t) = 1 + \sum_{k = 1}^\infty (-1)^k(t^{\frac{3k^2 - k}{2}} + t^{\frac{3k^2 + k}{2}}) \]
\end{theorem}
\begin{lemma}
\[ n \neq \frac{ek^2 \pm k}{2}, \text{то } e_n = o_n \]
\[ n = \frac{ek^2 \pm k}{2}, \text{то } e_n = o_n + (-1)^k \]
\end{lemma}
\end{document}
