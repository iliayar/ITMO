% Created 2021-03-10 Wed 17:03
% Intended LaTeX compiler: pdflatex
\documentclass[english]{article}
\usepackage[T1, T2A]{fontenc}
\usepackage[lutf8]{luainputenc}
\usepackage[english, russian]{babel}
\usepackage{minted}
\usepackage{graphicx}
\usepackage{longtable}
\usepackage{hyperref}
\usepackage{xcolor}
\usepackage{natbib}
\usepackage{amssymb}
\usepackage{stmaryrd}
\usepackage{amsmath}
\usepackage{caption}
\usepackage{mathtools}
\usepackage{amsthm}
\usepackage{tikz}
\usepackage{grffile}
\usepackage{extarrows}
\usepackage{wrapfig}
\usepackage{rotating}
\usepackage{placeins}
\usepackage[normalem]{ulem}
\usepackage{amsmath}
\usepackage{textcomp}
\usepackage{capt-of}

\usepackage{geometry}
\geometry{a4paper,left=2.5cm,top=2cm,right=2.5cm,bottom=2cm,marginparsep=7pt, marginparwidth=.6in}

 \usepackage{hyperref}
 \hypersetup{
     colorlinks=true,
     linkcolor=blue,
     filecolor=orange,
     citecolor=black,      
     urlcolor=cyan,
     }

\usetikzlibrary{decorations.markings}
\usetikzlibrary{cd}
\usetikzlibrary{patterns}

\newcommand\addtag{\refstepcounter{equation}\tag{\theequation}}
\newcommand{\eqrefoffset}[1]{\addtocounter{equation}{-#1}(\arabic{equation}\addtocounter{equation}{#1})}


\newcommand{\R}{\mathbb{R}}
\renewcommand{\C}{\mathbb{C}}
\newcommand{\N}{\mathbb{N}}
\newcommand{\rank}{\text{rank}}
\newcommand{\const}{\text{const}}
\newcommand{\grad}{\text{grad}}

\theoremstyle{plain}
\newtheorem{axiom}{Аксиома}
\newtheorem{lemma}{Лемма}
\newtheorem{manuallemmainner}{Лемма}
\newenvironment{manuallemma}[1]{%
  \renewcommand\themanuallemmainner{#1}%
  \manuallemmainner
}{\endmanuallemmainner}

\theoremstyle{remark}
\newtheorem*{remark}{Примечание}
\newtheorem*{solution}{Решение}
\newtheorem{corollary}{Следствие}[theorem]
\newtheorem*{examp}{Пример}
\newtheorem*{observation}{Наблюдение}

\theoremstyle{definition}
\newtheorem{task}{Задача}
\newtheorem{theorem}{Теорема}[section]
\newtheorem*{definition}{Определение}
\newtheorem*{symb}{Обозначение}
\newtheorem{manualtheoreminner}{Теорема}
\newenvironment{manualtheorem}[1]{%
  \renewcommand\themanualtheoreminner{#1}%
  \manualtheoreminner
}{\endmanualtheoreminner}
\captionsetup{justification=centering,margin=2cm}
\newenvironment{colored}[1]{\color{#1}}{}

\tikzset{->-/.style={decoration={
  markings,
  mark=at position .5 with {\arrow{>}}},postaction={decorate}}}
\makeatletter
\newcommand*{\relrelbarsep}{.386ex}
\newcommand*{\relrelbar}{%
  \mathrel{%
    \mathpalette\@relrelbar\relrelbarsep
  }%
}
\newcommand*{\@relrelbar}[2]{%
  \raise#2\hbox to 0pt{$\m@th#1\relbar$\hss}%
  \lower#2\hbox{$\m@th#1\relbar$}%
}
\providecommand*{\rightrightarrowsfill@}{%
  \arrowfill@\relrelbar\relrelbar\rightrightarrows
}
\providecommand*{\leftleftarrowsfill@}{%
  \arrowfill@\leftleftarrows\relrelbar\relrelbar
}
\providecommand*{\xrightrightarrows}[2][]{%
  \ext@arrow 0359\rightrightarrowsfill@{#1}{#2}%
}
\providecommand*{\xleftleftarrows}[2][]{%
  \ext@arrow 3095\leftleftarrowsfill@{#1}{#2}%
}
\makeatother
\author{Ilya Yaroshevskiy}
\date{\today}
\title{Лекция 4}
\hypersetup{
 pdfauthor={Ilya Yaroshevskiy},
 pdftitle={Лекция 4},
 pdfkeywords={},
 pdfsubject={},
 pdfcreator={Emacs 28.0.50 (Org mode )}, 
 pdflang={English}}
\begin{document}

\maketitle
\tableofcontents

\usetikzlibrary{automata}



\section{Производящие функции для регулярных языков}
\label{sec:orgc1a1652}
\(L\) --- регулярный язык
\[ | L \cap \Sigma^n | = a_n \] 
\[ L(t) = a_0 + a_1 t + \dots \]
\begin{remark}
\(L\) --- регулярная спецификация \\
\(\psi\) --- регулярное выражение:
\begin{enumerate}
\item \(L(\psi) = L\)
\item \(\forall x \in \L\ \exists !\) способ \(x\) удовлетворяющий \(\psi\)
\end{enumerate}
\end{remark}
\begin{lemma}
\(\Sigma\) --- конечный алфавит, \(L \subset \Sigma^*\) \\
\(L\) --- регулярная спецификация \(\Leftrightarrow\) \(L\) получаетя из \(\Sigma\):
\begin{enumerate}
\item Дизъюнктное объединение \(+\)
\item Прямое произведение \(\times\)
\item Последовательность Seq
\end{enumerate}
\end{lemma}
\begin{proof}
Общее рассжудение: по индукции рассмотрим для каждой операции во что
она перейдет, надо показать что единственность вывода сохраняется \color{red} Не работает \color{black}
\end{proof}
\begin{examp}
\[ ab^*|a^*b \]
\[ \color{blue}a \times \text{Seq }b\color{black}|\color{blue}\text{Seq } a \times b\color{black} \]
объединение дизъюнктное? \(\Rightarrow\) не регелярная спецификация
\end{examp}
\begin{examp}
\[ (ab^*)^* \]
\[ \color{blue} \text{Seq}(a \times \text{Seq } b) \color{black} \]
\end{examp}
\begin{theorem}
Если у \(L\) есть регулярная спецификация, то \(L\) --- дробно рациональная
\end{theorem}
\begin{theorem}[Производящая функция регулярного языка]
\(L\) --- регулярный язык над \(\Sigma\), ДКА \(A\): \\
\begin{itemize}
\item Состояния \(Q,\ |Q| = n\)
\item \(s \in Q\) --- стартовое сотояние
\item \(T \subset Q\) --- терминальные
\end{itemize}
\[ u = (\overbrace{0, 0, \dots, \underbrace{1}_s, 0, \dots, 0}^n) \]
\[ v = (\overbrace{0, \underbrace{1}_{\in T}, 0, \underbrace{1}_{\in T}, \dots, \underbrace{1}_{\in T}, 0}^n) \]

\[ D = (d_{ij})^T,\ d_{ij} = |\{c | i \xrightarrow[]{c} j\}| \]
\[ L(t) = \vec{u}(I - tD)^{-1}\vec{v} \]
\end{theorem}
\begin{examp}
Язык из слов, которые содержат abb как подстроку \\
\begin{center}
\begin{tikzpicture}
\node[state, initial] at (0, 0) (A) [circle] {\(0\)};
\node[state] at (2, 0) (B) [circle] {\(1\)};
\node[state] at (4, 0) (C) [circle] {\(2\)};
\node[state,accepting] at (6, 0) (D) [circle] {\(3\)};
\draw[->] (A) edge node[above] {a} (B);
\draw[->] (A) edge[loop above] node[above] {b} (A);
\draw[->] (B) edge[loop above] node[above] {a} (B);
\draw[->] (B) edge node[above] {b} (C);
\draw[->] (C) edge[loop above] node[above] {a} (B);
\draw[->] (C) edge node[above] {b} (D);
\draw[->] (D) edge[loop above] node[above] {a} (D);
\draw[->] (D) edge[loop right] node[below] {b} (D);
\end{tikzpicture}
\end{center}
\[ \begin{pmatrix} L_0 \\ L_1 \\ L_2 \\ L_3\end{pmatrix}  = \begin{pmatrix}0 \\ 0 \\ 0 \\ 1\end{pmatrix} + t \begin{pmatrix}1 & 1 & 0 & 0 \\ 0 & 1 & 1 & 0 \\ 0 & 1 & 0 & 1 \\ 0 & 0 & 0 & 2\end{pmatrix}\begin{pmatrix}L_0 \\ L_1 \\ L_2 \\ L_3\end{pmatrix}\]
\[ L_0 = \frac{t^3}{(1 - t)(1 - 2t)(1 - t - t^2)} \]
\end{examp}
\section{Автомат КМП и автокор. многочлен}
\label{sec:org2147644}
Конструкция Гуибаса-Одлызко
\[ p = \fbox{p_1, p_2, \dots, p_k} \]
\[ c_i = [p[i+1\dots k] = p[1\dots k-i]] \]
\[ c(t) = c_0 + c_1 t + c_2 t^2 + \dots + c_{k - 1}t^{k - 1} \]
\begin{examp}
\(p = \text{aabbaa}\) \\
\(c = (1, 0, 0, 0, 1, 1)\) \\
\(c(t) = 1 + t^4 + t^5\)
\end{examp}
\begin{theorem}
\-
\begin{itemize}
\item \(\Sigma,\ |\Sigma| = m\)
\end{itemize}
\(S_n\) --- количество слов длины \(n\), не содержащих \(p\)
\[ S(t) = s_0 + s_1t + s_2t^2 + \dots \]
\[ S(t) = \frac{c(t)}{t^k + (1 - mt)c(t)} \]
\end{theorem}
\begin{examp}
\(p = \text{abb}\)
\[ c(t) = 1 \]
\[ \frac{1}{t^3 + (1 - 2t)\cdot 1} = \frac{1}{1 - 2t + t^3} \]
\end{examp}

\subsection{Пентагональная формула Эйлера}
\label{sec:orgbcfa2d4}
\[ p_0\ p_1\ p_2\ \dots\ p_n\ \dots \]
\(p_n\) --- количество разбиений \(n\) на слагаемые из \(\N\). Порядок не важен
\begin{itemize}
\item \(U = \{0\},\ u_1= 1,\ U(t) = t\)
\item \(N = \text{Seq}^+U=\)положительно целые числа
\item \(P = \text{MSet }N\)
\[ P(t) = \prod_{k = 1}^\infty \frac{1}{1 - t^k} \]
\end{itemize}
\[ Q(t) = \prod_{k = 1}^\infty (1 - t^k) \]
\[ R(t) = \prod_{k = 1}^\infty(1 + t^k)\ [t^n]R \to r_n \]
\(r_n\) --- количество разбиений на различные слагаемые
\[ [t^n]Q = \sum_{\substack{\text{разбиение } n \text{ на } \\ \text{различные слагаемые}}} (-1)^\text{число слагаемых} \]
\[ q_n = e_n - o_n \]
\(e_n\) --- число разбиений на четное число различных слагаемы, \(o_n\) --- число разбиений на нечетное число различных слагаемы, 
\begin{theorem}
\[ Q(t) = 1 + \sum_{k = 1}^\infty (-1)^k(t^{\frac{3k^2 - k}{2}} + t^{\frac{3k^2 + k}{2}}) \]
\end{theorem}
\begin{lemma}
\[ n \neq \frac{ek^2 \pm k}{2}, \text{то } e_n = o_n \]
\[ n = \frac{ek^2 \pm k}{2}, \text{то } e_n = o_n + (-1)^k \]
\end{lemma}
\end{document}
