% Created 2021-04-06 Tue 20:33
% Intended LaTeX compiler: pdflatex

\documentclass[english]{article}
\usepackage[T1, T2A]{fontenc}
\usepackage[lutf8]{luainputenc}
\usepackage[english, russian]{babel}
\usepackage{minted}
\usepackage{graphicx}
\usepackage{longtable}
\usepackage{hyperref}
\usepackage{xcolor}
\usepackage{natbib}
\usepackage{amssymb}
\usepackage{stmaryrd}
\usepackage{amsmath}
\usepackage{caption}
\usepackage{mathtools}
\usepackage{amsthm}
\usepackage{tikz}
\usepackage{grffile}
\usepackage{extarrows}
\usepackage{wrapfig}
\usepackage{rotating}
\usepackage{placeins}
\usepackage[normalem]{ulem}
\usepackage{amsmath}
\usepackage{textcomp}
\usepackage{capt-of}

\usepackage{geometry}
\geometry{a4paper,left=2.5cm,top=2cm,right=2.5cm,bottom=2cm,marginparsep=7pt, marginparwidth=.6in}
 \usepackage{hyperref}
 \hypersetup{
     colorlinks=true,
     linkcolor=blue,
     filecolor=orange,
     citecolor=black,      
     urlcolor=cyan,
     }

\usetikzlibrary{decorations.markings}
\usetikzlibrary{cd}
\usetikzlibrary{patterns}
\usetikzlibrary{automata, arrows}

\newcommand\addtag{\refstepcounter{equation}\tag{\theequation}}
\newcommand{\eqrefoffset}[1]{\addtocounter{equation}{-#1}(\arabic{equation}\addtocounter{equation}{#1})}


\newcommand{\R}{\mathbb{R}}
\renewcommand{\C}{\mathbb{C}}
\newcommand{\N}{\mathbb{N}}
\newcommand{\A}{\mathfrak{A}}
\newcommand{\rank}{\mathop{\rm rank}\nolimits}
\newcommand{\const}{\var{const}}
\newcommand{\grad}{\mathop{\rm grad}\nolimits}

\newcommand{\todo}{{\color{red}\fbox{\text{Доделать}}}}
\newcommand{\fixme}{{\color{red}\fbox{\text{Исправить}}}}

\newcounter{propertycnt}
\setcounter{propertycnt}{1}
\newcommand{\beginproperty}{\setcounter{propertycnt}{1}}

\theoremstyle{plain}
\newtheorem{propertyinner}{Свойство}
\newenvironment{property}{
  \renewcommand\thepropertyinner{\arabic{propertycnt}}
  \propertyinner
}{\endpropertyinner\stepcounter{propertycnt}}
\newtheorem{axiom}{Аксиома}
\newtheorem{lemma}{Лемма}
\newtheorem{manuallemmainner}{Лемма}
\newenvironment{manuallemma}[1]{%
  \renewcommand\themanuallemmainner{#1}%
  \manuallemmainner
}{\endmanuallemmainner}

\theoremstyle{remark}
\newtheorem*{remark}{Примечание}
\newtheorem*{solution}{Решение}
\newtheorem{corollary}{Следствие}[theorem]
\newtheorem*{examp}{Пример}
\newtheorem*{observation}{Наблюдение}

\theoremstyle{definition}
\newtheorem{task}{Задача}
\newtheorem{theorem}{Теорема}[section]
\newtheorem*{definition}{Определение}
\newtheorem*{symb}{Обозначение}
\newtheorem{manualtheoreminner}{Теорема}
\newenvironment{manualtheorem}[1]{%
  \renewcommand\themanualtheoreminner{#1}%
  \manualtheoreminner
}{\endmanualtheoreminner}
\captionsetup{justification=centering,margin=2cm}
\newenvironment{colored}[1]{\color{#1}}{}

\tikzset{->-/.style={decoration={
  markings,
  mark=at position .5 with {\arrow{>}}},postaction={decorate}}}
\makeatletter
\newcommand*{\relrelbarsep}{.386ex}
\newcommand*{\relrelbar}{%
  \mathrel{%
    \mathpalette\@relrelbar\relrelbarsep
  }%
}
\newcommand*{\@relrelbar}[2]{%
  \raise#2\hbox to 0pt{$\m@th#1\relbar$\hss}%
  \lower#2\hbox{$\m@th#1\relbar$}%
}
\providecommand*{\rightrightarrowsfill@}{%
  \arrowfill@\relrelbar\relrelbar\rightrightarrows
}
\providecommand*{\leftleftarrowsfill@}{%
  \arrowfill@\leftleftarrows\relrelbar\relrelbar
}
\providecommand*{\xrightrightarrows}[2][]{%
  \ext@arrow 0359\rightrightarrowsfill@{#1}{#2}%
}
\providecommand*{\xleftleftarrows}[2][]{%
  \ext@arrow 3095\leftleftarrowsfill@{#1}{#2}%
}
\makeatother
\author{Ilya Yaroshevskiy}
\date{\today}
\title{Лекция 7}
\hypersetup{
 pdfauthor={Ilya Yaroshevskiy},
 pdftitle={Лекция 7},
 pdfkeywords={},
 pdfsubject={},
 pdfcreator={Emacs 28.0.50 (Org mode 9.4.4)}, 
 pdflang={English}}
\begin{document}

\maketitle
\tableofcontents

\newcommand{\stirlF}[2]{\left[\begin{matrix} #1 \\ #2 \end{matrix}\right]}
\newcommand{\stirlS}[2]{\left\{\begin{matrix} #1 \\ #2 \end{matrix}\right\}}
\newcommand{\Set}{\text{Set}\,}
\newcommand{\Seq}{\text{Seq}\,}
\newcommand{\Cyc}{\text{Cyc}\,}


Рассмотрим деревья:
\[ T = t \times \Text{Seq}\,T \]
, где \(t\) --- корень
\[ A(t) = t\cdot \phi(A(t)) \]
\[ \phi(s) = \frac{1}{1 - s} \]
Решить это уравнение в общем виде можно с помощью формулы Лагранжа
\begin{theorem}[формула обращения Лагранжа]
\[ a_n = \frac{1}{n}\cdot[s^{n - 1}](\phi(s))^n \], где \([s^{n}]A(s)\) --- коэффицент при \(s^n\) в \(A(s)\)
\[ A(t) = \sum_{n = 0}^\infty a_n t^n \]
\end{theorem}
\begin{examp}
Применим ее для деревьев
\[ a_n = \frac{1}{n}\cdot[s^{n - 1}]\left(\frac{1}{1 - s}\right)^n \]
\[ \left(\frac{1}{1 - s}\right)^n = (1 + s + s^2 + s^3 + \dots + s^k + \dots)^n \]
\[ (1 - s)^{-n} = 1 - \binom{-n}{1} s + \binom{-n}{2}s^2 - \binom{-n}{3}s^3 \]
\[ \binom{-n}{n - 1} = \frac{-n\cdot(-n - 1)\cdot(-n - 2)\cdot\dots\cdot(-n - (n - 1) + 1)}{1\cdot 2 \cdot 3 \cdot \dots \cdot (n - 1)} \]
\[ \frac{1}{n}(-1)^{n - 1}\binom{-n}{n - 1} = \frac{(2n - 2)!}{(n - 1)!n!} = \binom{2n - 2}{n - 1} \]
\end{examp}
\begin{examp}
\[ \phi(s) = e^s \]
\[ \frac{a_n}{n!} = \frac{1}{n}\cdot[s^{n- 1}] e^{ns} \]
\[ e^{ns} = 1 + \frac{1}{1!}(ns) + \frac{1}{2!}(ns)^2 + \frac{1}{3!}(ns)^3 + \dots \]
\[ [s^{n - 1}]e^{ns} = \frac{n^{n - 1}}{(n - 1)!} \]
\end{examp}
\section{Производящая функция от нескольких перменных}
\label{sec:orgb9b0e6d}
\(\binom{n}{k}\) образуют таблицу:
\begin{center}
\begin{tabular}{l|rrrll}
\(n\textbackslash k\) &  &  &  &  & \\
\hline
 & 1 &  &  &  & \\
 & 1 & 1 &  &  & \\
 & 1 & 2 & 1 &  & \\
 & 1 & 3 & 3 & 1 & \\
 & 1 & 4 & 6 & 4 & 1\\
\end{tabular}
\end{center}
\[ A_k(t) = \sum_{n = 0}^\infty \binom{n}{k}t^n \]
\[ B_n(t) = \sum_{n = 0}^\infty \binom{n}{l}t^k \]
\[ C(u ,z) = \sum_{n, k}\binom{n}{k}z^nu^k = \frac{1}{1 - z - uz}\]
Посмотрим на \(C(u, z)\) так: \(n\) --- вес, \(k\) --- стоимость. Будем считать, что \(z\) --- не берем объект, \(uz\) --- берем объект
\[ \text{Seq}\,\{z, uz\} = [], [z], [uz], [z, z], [z, uz], [uz, z], [uz, uz], \dots \]
\[ A(u, z) = z + uz \]
\subsection{Числа Стирлинга I рода}
\label{sec:orged2ba7e}
\fixme \\
Помеченные перстановки, \(\text{Set}\,\text{Cyc}\,Z\)
\[ \bigcup_{k = 0}^\infty \text{Set}_{=k}\text{Cyc}\, Z \]
\[ \text{Set}\,\text{Cyc}\,Z = e^{\ln\frac{1}{1 - Z}} = \frac{1}{1 - Z} \]
\[ \bigcup_{k = 0}^\infty \text{Set}_{= k}\left(u \times \text{Cyc}\,Z\right) \mapsto \sum_{n ,k}\stirlF{n}{k}\frac{1}{n!}z^nu^k\]
\[ \Set_{=k}(A) = \frac{\Seq_{=k}(A)}}{k!} = \frac{A(Z)^k}{k!}  \]
\[ u \times \Cyc Z = \sum_{n = 0}^\infty \frac{\left(u \ln \left(\frac{1}{1 - Z}\right)\right)^k}{k!} = e^{u\ln\frac{1}{1 - Z}} = (1 - Z)^{-u}\]
\[ (1 - Z)^{-u} = \sum_{n,k}\frac{\stirlF{n}{k}}{n!} Z^n u^k \]
\subsection{Числа Стирлинга II рода}
\label{sec:org168bf92}
\[ \stirlS{n}{k}\quad \Set\Set_{>0} Z \]
\[ \bigcup_{k = 0}^\infty\Set_{=k}(u\times \Set_{>0}Z) \]
\[ \sum_{k = 0}^\infty \frac{\left(u(e^Z - 1)^k\right)}{k!} = e^{ue^Z - u} = \sum_{n,k} \frac{\stirlS{n}{k}}{n!}z^nu^k\]
\subsection{Средняя стоимоть}
\label{sec:org0c6b9d9}
\begin{itemize}
\item \(A\quad a_{n,k} = [z^nu^k]A(u, z)\) --- количество объектов веса \(n\) стоимости \(k\)
\end{itemize}
\[ w_n = \frac{\sum_{k = 0}^\infty k a_{n,k}}{\sum_{k = 0}^\infty a_{n, k}} = \frac{\left.[z^n]\left(\frac{\partial}{\partial u}A(u ,z)\right)\right|_{u = 1}}{[z^n]A(1, z)}\]
\subsubsection{Разбиение на слагаемые, порядок важен}
\label{sec:org6ea0464}
Аналогично рассотовке перегорожок, \(\Seq\Seq_{>0}Z\)
\[ \Seq \left(u\times\Seq_{>0}Z\right) \]
\[ \frac{1}{1 - z} - 1= \frac{z}{1 - z} \]
\[ A(u, z) =  \frac{1}{1 - \frac{uz}{1 - z}} = \frac{1 - z}{1 - z - uz} \]
\[ \frac{\partial A(u,z)}{\partial u}\bigg|_{u = 1} = \frac{z(1 - z)}{(1 - z- uz)^2} \bigg|_{u = 1} = \frac{z(1 - z)}{(1 - 2z)^2}\]
\begin{description}
\item[{Числитель}] \[ [z^n]\frac{z(1 - z)}{(1 - 2z)^2} = \frac{2^n(n + 1)}{4} \]
\item[{Знаменатель}] \[ [z^n]\frac{1 - z}{1 - 2z} = 2^n - 2^{n - 1} = 2^{n - 1} \]
\end{description}
Среднее число слагаемых: \[ \frac{2^n(n + 1)}{2^{n - 1}\cdot4} = \frac{n + 1}{2} \]
\subsubsection{Среднее число циклов в перестановке}
\label{sec:org66efff8}
\[ A(u ,z) = (1 - z)^{- u} \]
\[ \frac{\partial}{\partial u}A(u, z) = \frac{\partial}{\partial u} e^{u \ln\frac{1}{1 - z}} = \ln \frac{1}{1 - z}\cdot e^{u\ln\frac{1}{1 -z }} \]
Подставляем \(u = 1\):
\begin{description}
\item[{Числитель}] \[ [z^n]\frac{\ln \left(\frac{1}{1 - z}\right)}{1 - z} = B(z)\]
\item[{Знаменатель}] \[ (1 - z)^{-u}\bigg_{u = 1} = \frac{1}{1 - z} \]
\[ [z^n]\frac{1}{1 - z} = 1 \]
\end{description}
\[ (z + \frac{1}{2}z^2 + \frac{1}{3}z^3 + \dots + \frac{1}{k}z^k + \dots)\cdot(1 + z + z^2 + \dots) \]
\[ [z^n]B(z) = \sum_{k = 1}^n \frac{1}{k} = H_n \sim \log n \]
\end{document}
