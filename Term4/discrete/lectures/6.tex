% Created 2021-04-06 Tue 20:34
% Intended LaTeX compiler: pdflatex

\documentclass[english]{article}
\usepackage[T1, T2A]{fontenc}
\usepackage[lutf8]{luainputenc}
\usepackage[english, russian]{babel}
\usepackage{minted}
\usepackage{graphicx}
\usepackage{longtable}
\usepackage{hyperref}
\usepackage{xcolor}
\usepackage{natbib}
\usepackage{amssymb}
\usepackage{stmaryrd}
\usepackage{amsmath}
\usepackage{caption}
\usepackage{mathtools}
\usepackage{amsthm}
\usepackage{tikz}
\usepackage{grffile}
\usepackage{extarrows}
\usepackage{wrapfig}
\usepackage{rotating}
\usepackage{placeins}
\usepackage[normalem]{ulem}
\usepackage{amsmath}
\usepackage{textcomp}
\usepackage{capt-of}

\usepackage{geometry}
\geometry{a4paper,left=2.5cm,top=2cm,right=2.5cm,bottom=2cm,marginparsep=7pt, marginparwidth=.6in}
 \usepackage{hyperref}
 \hypersetup{
     colorlinks=true,
     linkcolor=blue,
     filecolor=orange,
     citecolor=black,      
     urlcolor=cyan,
     }

\usetikzlibrary{decorations.markings}
\usetikzlibrary{cd}
\usetikzlibrary{patterns}
\usetikzlibrary{automata, arrows}

\newcommand\addtag{\refstepcounter{equation}\tag{\theequation}}
\newcommand{\eqrefoffset}[1]{\addtocounter{equation}{-#1}(\arabic{equation}\addtocounter{equation}{#1})}


\newcommand{\R}{\mathbb{R}}
\renewcommand{\C}{\mathbb{C}}
\newcommand{\N}{\mathbb{N}}
\newcommand{\A}{\mathfrak{A}}
\newcommand{\rank}{\mathop{\rm rank}\nolimits}
\newcommand{\const}{\var{const}}
\newcommand{\grad}{\mathop{\rm grad}\nolimits}

\newcommand{\todo}{{\color{red}\fbox{\text{Доделать}}}}
\newcommand{\fixme}{{\color{red}\fbox{\text{Исправить}}}}

\newcounter{propertycnt}
\setcounter{propertycnt}{1}
\newcommand{\beginproperty}{\setcounter{propertycnt}{1}}

\theoremstyle{plain}
\newtheorem{propertyinner}{Свойство}
\newenvironment{property}{
  \renewcommand\thepropertyinner{\arabic{propertycnt}}
  \propertyinner
}{\endpropertyinner\stepcounter{propertycnt}}
\newtheorem{axiom}{Аксиома}
\newtheorem{lemma}{Лемма}
\newtheorem{manuallemmainner}{Лемма}
\newenvironment{manuallemma}[1]{%
  \renewcommand\themanuallemmainner{#1}%
  \manuallemmainner
}{\endmanuallemmainner}

\theoremstyle{remark}
\newtheorem*{remark}{Примечание}
\newtheorem*{solution}{Решение}
\newtheorem{corollary}{Следствие}[theorem]
\newtheorem*{examp}{Пример}
\newtheorem*{observation}{Наблюдение}

\theoremstyle{definition}
\newtheorem{task}{Задача}
\newtheorem{theorem}{Теорема}[section]
\newtheorem*{definition}{Определение}
\newtheorem*{symb}{Обозначение}
\newtheorem{manualtheoreminner}{Теорема}
\newenvironment{manualtheorem}[1]{%
  \renewcommand\themanualtheoreminner{#1}%
  \manualtheoreminner
}{\endmanualtheoreminner}
\captionsetup{justification=centering,margin=2cm}
\newenvironment{colored}[1]{\color{#1}}{}

\tikzset{->-/.style={decoration={
  markings,
  mark=at position .5 with {\arrow{>}}},postaction={decorate}}}
\makeatletter
\newcommand*{\relrelbarsep}{.386ex}
\newcommand*{\relrelbar}{%
  \mathrel{%
    \mathpalette\@relrelbar\relrelbarsep
  }%
}
\newcommand*{\@relrelbar}[2]{%
  \raise#2\hbox to 0pt{$\m@th#1\relbar$\hss}%
  \lower#2\hbox{$\m@th#1\relbar$}%
}
\providecommand*{\rightrightarrowsfill@}{%
  \arrowfill@\relrelbar\relrelbar\rightrightarrows
}
\providecommand*{\leftleftarrowsfill@}{%
  \arrowfill@\leftleftarrows\relrelbar\relrelbar
}
\providecommand*{\xrightrightarrows}[2][]{%
  \ext@arrow 0359\rightrightarrowsfill@{#1}{#2}%
}
\providecommand*{\xleftleftarrows}[2][]{%
  \ext@arrow 3095\leftleftarrowsfill@{#1}{#2}%
}
\makeatother
\author{Ilya Yaroshevskiy}
\date{\today}
\title{Лекция 6}
\hypersetup{
 pdfauthor={Ilya Yaroshevskiy},
 pdftitle={Лекция 6},
 pdfkeywords={},
 pdfsubject={},
 pdfcreator={Emacs 28.0.50 (Org mode 9.4.4)}, 
 pdflang={English}}
\begin{document}

\maketitle
\tableofcontents

\newcommand{\Seq}{\text{Seq}\,}
\newcommand{\Set}{\text{Set}\,}
\newcommand{\Cyc}{\text{Cyc}\,}

\section{Помеченные КО и экспоненциальные производящие функции}
\label{sec:org1bb7779}
\[ a_0\ a_1\ a_2\ \dots\ a_n\ \dots\quad A(t) = a_0 + a_1 t + a_2 t^2 + \dots + a_nt^n + \dots \]
\begin{definition}
Экспоненциальная производящая функция:
\[ a(t) = \frac{a_0}{0!} + \frac{a_1}{1!}\cdot t + \frac{a_2}{2!}\cdot t^2 + \dots + \frac{a_n}{n!}\cdot t^n + \dots \]
\end{definition}
\begin{symb}
Мы будет обозначать ЭПФ так-же большой буквой
\end{symb}
\begin{examp}
\(1, 1, 1,1, ,1,1 ,1, 1, 1\)
\begin{description}
\item[{ОПФ}] \(\frac{1}{1 - t}\)
\item[{ЭПФ}] \(1 + 1\cdot t + \frac{1}{2!}\cdot t^2 + \frac{1}{3!}\cdot t^3 + \dots = \sum\limits_{n = 0}^{ + \infty} \frac{1}{n!}\cdot t^n = e^t = \exp(t)\)
\end{description}
\end{examp}
\begin{examp}
\(1, 1, 2, 6, 24, \dots, n!, \dots\quad a_n = n!\)
\begin{description}
\item[{ОПФ}] \(1 + t + 2\cdot t^2 + 6\cdot t^3 + \dots + n!\cdot t^n + \dots\)
\item[{ЭПФ}] \(\sum \limits_{n = 0}^{ + \infty} \frac{n!}{n!}\cdot t^n = \frac{1}{1 - t}\)
\end{description}
\end{examp}

\[ A(t) = \frac{a_0}{0!} + \frac{a_1}{1!}\cdot t + \frac{a_2}{2!}\cdot t^2 + \dots + \frac{a_n}{n!}\cdot t^n + \dots \]
\[ B(t) = \frac{b_0}{0!} + \frac{b_1}{1!}\cdot t + \frac{b_2}{2!}\cdot t^2 + \dots + \frac{b_n}{n!}\cdot t^n + \dots \]

\begin{property}
\[ C(t) = A(t) \pm B(t)\quad c_n = a_n \pm b_n \]
\end{property}
\begin{property}
\[ C(t) = a(t)\cdot B(t) \]
\[ \frac{C_n}{n!} = \sum\limits_{k = 0}^n \frac{a_k}{k!}\cdot\frac{b_{n - k}}{(n - k)!} \]
\[ c_n = \sum\limits_{k = 0}^n \binom{n}{k}a_kb_{n - k} \]
\end{property}
\begin{property}
\[ C(t) = \frac{A(t)}{B(t)} \]
\[ a_n = \sum^n_{k = 0}\binom{n}{k}b_kc_{n - k} = \sum\limits_{k = 1}^n \binom{n}{k} b_k c_{n - k} + b_0 c_n \]
\[ c_n = \frac{a - \sum_{k = 1}^n\binom{n}{k} b_k c_{n - k}}{b_0} \]
\end{property}
\textbf{Далее все производящие функции --- экспоненциальные, а объекты помеченые}
\subsection{Помеченные объекты}
\label{sec:org57b1f70}
\begin{examp}
Перестановк. \(P_n = n!\) --- количество перестановок из \(n\) элементов
\end{examp}
\begin{examp}
Пустые графы. \(E_n = 1\) --- количество графов с \(n\) вершинами \\
ЭПФ: \(\exp(t)\)
\end{examp}
\begin{examp}
Циклы. \(C_n = (n - 1)!\) --- количество циклов из \(n\) вершин. Направление обхода фиксировано. \\
ЭПФ: \(\sum\limits_{n = 1}^{ + \infty} \frac{n!}{n}\cdot \frac{1}{n!}\cdot t^n = \sum\limits_{n = 1}^{ + \infty} \frac{t^n}{n} = \ln \frac{1}{1 - t}\)
\end{examp}
\subsection{Операции}
\label{sec:org11b51f0}
\subsubsection{Дизъюнктное объединение (сумма)}
\label{sec:org6dbabd1}
\begin{itemize}
\item \(A\)
\item \(B\)
\item \(A \cap B = \emptyset\)
\item \(C = A \cup B\)
\end{itemize}
\[ c_n = a_n + b_n\quad C(t) = A(t) + B(t) \]
\subsubsection{Пара (произведение)}
\label{sec:org8d97a25}
\begin{itemize}
\item \(A\)
\item \(B\)
\item \(C = A \times B\)
\end{itemize}
\[ C = \{\langle \underbrace{a}_{k\text{ атомов}}, \underbrace{b}_{n - k\text{ атомов}} \rangle\} \]
Получим последовательность \(c_1 c_2 \dots c_n\). Перенумеруем элементы: \\
Первые \(k\) в \(d_1d_2 \dots d_k\), где \(d_i = |\{c_j | 1 \le j \le k,\ c_k \le c_i\}|\). \\
А остальные \(c_{k + 1}\dots c_n\) в \(e_1 \dots e_{n - k}\), где \(e_i = |\{c_j | k + 1\le j \le n,\ c_j \le c_{i + k}\}|\). \\
Пусть \(d_i = a_i\), а \(e_i = b_i\)
\[ c_n = \sum_{k = 0}^{ n } \binom{n}{k} a_k b_{n - k} \quad C(t) = A(t) \cdot B(t)\]
\begin{examp}
Пары перестановок. \(C(t) = \frac{1}{(1 - t)^2}\). Тогда \(c_n = (n + 1)n!\)
\end{examp}
\subsubsection{Последовательность}
\label{sec:org7361d1b}
\[ C = \Seq A = \emptyset + A \times \Seq A \]
\[ C(t) = 1 + A(t)\cdot C(t) \]
\[ C(t) = \frac{1}{1 - A(t)} \]
\begin{examp}
\-
\begin{itemize}
\item \(U = \{\circ\}\)
\item \(U(t) = t\)
\item \(\Seq U = P\)
\end{itemize}
\[P(t) = \frac{1}{1 - t}\]
\end{examp}
\subsubsection{Множества (Set)}
\label{sec:org119bc2f}
\begin{itemize}
\item \(\Set_k A\) ---  множества, содержащие \(k\) обхектов
\end{itemize}
\[ B_k = \Seq_k A = \underbrace{A \times A \times \dots \times A}_k\quad B_k(t) = A(t)^k \]
\[ \Set_k A = \Seq_k A /_\sim\]
\([x_1 x_2 \dots x_k] \sim [y_1 y_2 \dots y_k]\). \(\exists\) перестановка \(\pi: x_i = y_{\pi[i]}\)
\[ C_k(t) = \frac{1}{k!}\quad B_k(t) = \frac{A(t)^k}{k!} \]
\[ \Set A = \bigcup_{k = 0}^{\infty} \Set_k A = \sum_{k = 0}^\infty \frac{A(t)^k}{k!} = e^{A(t)} \]
\begin{examp}
\-
\begin{itemize}
\item \(U = \{\circ\}\)
\item \(U(t) = t\)
\end{itemize}
\[ \Set U = E\quad E(t) = e^t \]
, где \(E\) --- пустые графы
\end{examp}
\begin{examp}
Циклы.
\begin{itemize}
\item \(U = \{\circ\}\)
\item \(U(t) = t\)
\item \(B = \Set \Cyc U\)
\end{itemize}
\[ B(t) = e^{C(t)} = e^{\ln \frac{1}{1 - t}} = \frac{1}{1 - t} \]
Набор помеченных циклов являеся престановкой
\end{examp}
\subsubsection{Циклы}
\label{sec:org02fe53c}
\begin{itemize}
\item \(\Cyc_k A\) --- количество циклов длины \(k\)
\end{itemize}
\[ C = \Cyc_k A = \Seq_k A /_\sim \], где классы эквивалентности с точностью до циклических сдвигов. \\
\([x_1\dots x_k] \sim [y_1 \dots y_k]\). \(\exists i:\ x_j = y_{(i + j)\mod k + 1}\)
\[ \Cyc U = \ln\frac{1}{1 - t} \]
\[ C_k(t) = \frac{1}{k}A(t)^k \]
\[ C(t) = \sum_{k = 1}^\infty \frac{1}{k} A(t)^k = \ln\frac{1}{1 - A(t)} \]
\[ \Set\Cyc U = P \]
\[ \Set \Cyc A \simeq \Seq A \]


\subsection{Обобщение}
\label{sec:orge59a004}

\begin{theorem}[о подстановке]
\-
\begin{itemize}
\item \(A\) --- помеченные КО --- \(A(t)\)
\item \(B\) --- помеченные КО --- \(B(t)\)
\end{itemize}
\(C = A[B]\) --- вместо каждого атома \(A\) подставляем КО \(B\), перенумеруем получившиеся атомы произвольным образом
\[ C(t) = A(B(t)) \]
\end{theorem}
\begin{examp}
\(A\times A\) --- пара атомов. Их две \(B(t) = t^2 = 2 \cdot \frac{1}{2!} \cdot t^2\). Подставляем \(B(A(t)) = A(t)^2\)
\end{examp}
\end{document}
