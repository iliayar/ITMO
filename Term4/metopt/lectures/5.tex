% Created 2021-03-03 Wed 11:08
% Intended LaTeX compiler: pdflatex
\documentclass[english]{article}
\usepackage[T1, T2A]{fontenc}
\usepackage[lutf8]{luainputenc}
\usepackage[english, russian]{babel}
\usepackage{minted}
\usepackage{graphicx}
\usepackage{longtable}
\usepackage{hyperref}
\usepackage{xcolor}
\usepackage{natbib}
\usepackage{amssymb}
\usepackage{amsmath}
\usepackage{caption}
\usepackage{mathtools}
\usepackage{amsthm}
\usepackage{tikz}
\usepackage{grffile}
\usepackage{extarrows}
\usepackage{wrapfig}
\usepackage{rotating}
\usepackage{placeins}
\usepackage[normalem]{ulem}
\usepackage{amsmath}
\usepackage{textcomp}
\usepackage{capt-of}

\usepackage{geometry}
\geometry{a4paper,left=2.5cm,top=2cm,right=2.5cm,bottom=2cm,marginparsep=7pt, marginparwidth=.6in}

 \usepackage{hyperref}
 \hypersetup{
     colorlinks=true,
     linkcolor=blue,
     filecolor=orange,
     citecolor=black,      
     urlcolor=cyan,
     }

\usetikzlibrary{decorations.markings}
\usetikzlibrary{cd}
\usetikzlibrary{patterns}

\newcommand\addtag{\refstepcounter{equation}\tag{\theequation}}
\newcommand{\eqrefoffset}[1]{\addtocounter{equation}{-#1}(\arabic{equation}\addtocounter{equation}{#1})}


\newcommand{\R}{\mathbb{R}}
\renewcommand{\C}{\mathbb{C}}
\newcommand{\N}{\mathbb{N}}
\newcommand{\rank}{\text{rank}}
\newcommand{\const}{\text{const}}
\newcommand{\grad}{\text{grad}}

\theoremstyle{plain}
\newtheorem{axiom}{Аксиома}
\newtheorem{lemma}{Лемма}
\newtheorem{manuallemmainner}{Лемма}
\newenvironment{manuallemma}[1]{%
  \renewcommand\themanuallemmainner{#1}%
  \manuallemmainner
}{\endmanuallemmainner}

\theoremstyle{remark}
\newtheorem*{remark}{Примечание}
\newtheorem*{solution}{Решение}
\newtheorem{corollary}{Следствие}[theorem]
\newtheorem*{examp}{Пример}
\newtheorem*{observation}{Наблюдение}

\theoremstyle{definition}
\newtheorem{task}{Задача}
\newtheorem{theorem}{Теорема}[section]
\newtheorem*{definition}{Определение}
\newtheorem*{symb}{Обозначение}
\newtheorem{manualtheoreminner}{Теорема}
\newenvironment{manualtheorem}[1]{%
  \renewcommand\themanualtheoreminner{#1}%
  \manualtheoreminner
}{\endmanualtheoreminner}
\captionsetup{justification=centering,margin=2cm}
\newenvironment{colored}[1]{\color{#1}}{}

\tikzset{->-/.style={decoration={
  markings,
  mark=at position .5 with {\arrow{>}}},postaction={decorate}}}
\makeatletter
\newcommand*{\relrelbarsep}{.386ex}
\newcommand*{\relrelbar}{%
  \mathrel{%
    \mathpalette\@relrelbar\relrelbarsep
  }%
}
\newcommand*{\@relrelbar}[2]{%
  \raise#2\hbox to 0pt{$\m@th#1\relbar$\hss}%
  \lower#2\hbox{$\m@th#1\relbar$}%
}
\providecommand*{\rightrightarrowsfill@}{%
  \arrowfill@\relrelbar\relrelbar\rightrightarrows
}
\providecommand*{\leftleftarrowsfill@}{%
  \arrowfill@\leftleftarrows\relrelbar\relrelbar
}
\providecommand*{\xrightrightarrows}[2][]{%
  \ext@arrow 0359\rightrightarrowsfill@{#1}{#2}%
}
\providecommand*{\xleftleftarrows}[2][]{%
  \ext@arrow 3095\leftleftarrowsfill@{#1}{#2}%
}
\makeatother
\author{Ilya Yaroshevskiy}
\date{\today}
\title{Лекция 5}
\hypersetup{
 pdfauthor={Ilya Yaroshevskiy},
 pdftitle={Лекция 5},
 pdfkeywords={},
 pdfsubject={},
 pdfcreator={Emacs 28.0.50 (Org mode )}, 
 pdflang={English}}
\begin{document}

\maketitle
\tableofcontents



\section{Метод Ньютона(продолжение). Вывод через ряд Тейлора}
\label{sec:org6986af2}
\begin{itemize}
\item \(x_k\) --- текущая оценка решения \(x^*\)
\end{itemize}
\[ f(x_k + p) = f(x_k) + pf'(x_k) + \frac{1}{2!}p^2f''(x_k) + \dots \]
\[ f(x^*) = \min_x f(x) = \min_p f(x_k + p) = \min_p [ f(x_k) + pf'(x_k) + \frac{1}{2}p^2 f''(x_k)  + \dots ] \approx \]
\[ \approx \min_p[f(x_k) + pf'(x_k) + \frac{1}{2}p^2 f''(x_k)] \]
\[ f'(x_k) + pf''(x_k) = 0 \]
\[ p = -\frac{f'(x_k)}{f''(x_k)} \]
\(p\) --- аппроксимация шага: от \(x_k \to x^*\). \(x^* \approx x_k + p\)
\[ x_{k + 1} = x_k + p = x_k - \frac{f'(x_k)}{f''(x_k)} \label{nuton_1_5}\addtag \]
Главное преимущество метода Ньютона:
\begin{itemize}
\item высокая(квадратичная) скорость сходимости
\begin{itemize}
\item если \(x_k\) достаточно близка \(x^*\) и если \(f''(x^*) > 0\), то:
\[ |x_{k+1} - x^*| \le \beta|x_k - x^*|^2\quad, \beta = \const > 0 \]
\end{itemize}
\end{itemize}

\noindentНеудачи в методе Ньютона:
\begin{enumerate}
\item \(f(x)\) плохо аппроксимируется первыми тремя членами в ряде Тейлора. \(x_{k+1}\) может быть хуже \(x_k\)
\item \(p=-\frac{f'(x_k)}{f''(x_k)}\) определено только тогда, когда \(f''(x_k)\neq0\) \\
\(f''(x_k) > 0\) --- условие минимума квадратичной аппроксимации \\
Если \(f''(x_k) < 0\) --- алгоритм сходится к максимуму
\item Кроме \(f(x)\) нужно вычислять \(f'(x)\) и \(f''(x)\), что в реальных задачах затруднительно
\end{enumerate}
\subsection{Аппроксимация производных}
\label{sec:org0d138ce}
Правая разностная схема:
\[ f'(x_k) \approx \frac{f(x_k + h) - f(x_k)}{h}\quad ,h \sim \varepsilon \]
Центральная разностная схема:
\[ f'(x_k) \approx \frac{f(x_k + h) - f(x_k - h)}{2h} \]
порядок точности --- \(O(h^2)\)
\subsection{Метод Ньютона(продолжение)}
\label{sec:org0398faf}
Если \(f(x)\) --- квадратичная функция, то \(f'(x)\) --- линейная \\
В \ref{nuton_1_5} точное равенство, и следовательно метод Ньютона
сходится за один шаг, при любом выборе \(x\) \\
Пусть \(x^* \in [a, b]\) и \(f(x)\) --- трижды непрерывно дифференцируемая и выпуклая на \([a, b]\) функция. \\
\(\{x_k\}\) будет сходится к пределе \(x^*\) монотонно, если:
\[ 0<\frac{x^* - x_{k+1}}{x^* - x_k} < 1 \]
\[ f'(x^*) = 0 = f'(x_k) + f''(x_k(x^* - x_k)) + \frac{f'''(x_k)}{2}(x^* - x_k)^2 \]
\[ \frac{x^* - x_{k + 1}}{x^* - x_k} = \frac{x^* - x_k + \frac{f'(x_k)}{f''(x_k)}}{x^* - x_k} = 1 - \frac{2}{2 + \frac{f'''(x)(x^* - x_k)^2}{f'(x_k)}}\]
Итерацияонная поледовательность \(\{x_k\}\) монотонна, если \(\frac{f'''(x)}{f'(x_k)} > 0\),
то есть достаточное условие \color{red}\dots{}\color{black}

\begin{examp}
\[ f(x) = x\cdot\arctg(x) - \frac{1}{2}\ln(1 + x^2) \]
, пусть \(|f'(x)| \le 10^{-7}\)
\begin{minted}[frame=lines,linenos=true,mathescape]{python}
from sympy import *
\end{minted}

\[ f'''(x) = -\frac{2x}{(1 + x^2)^2} \]
\[ f'(x)\cdotf''(x) < 0 \]
Выбор начального приближение \(x_0 = 1\)
\[ x_{k + 1} = x_k - \frac{f'(x_k)}{f''(x_k)} \]
\begin{center}
\begin{tabular}{r|rrl|}
\(k\) & \(x_k\) & \(f'(x_k)\) & \(f''(x_k)\)\\
\hline
0 & 1 & 0.785 & \(\frac{1}{2}\)\\
1 & -0.57 & -0.518 & 0.754\\
2 & 0.117 & 0.116 & \dots{}\\
3 & \dots{} & \dots{} & \dots{}\\
4 & \(9\cdot 10^{-8}\) & \(9\cdot10^{-8}\) & \dots{}\\
\end{tabular}
\end{center}
Выолнилось условие \(|f'(x_k)| \le 10^{-7}\) --- окончание итерационного процесса. \(x \approx 9\cdot 10^{-8} \approx 0\)
\end{examp}
\subsection{Модификации метода Ньютона}
\label{sec:org24151ce}
\subsubsection{Метод Ньютона-Рафсона}
\label{sec:org7d0ce42}
\[ x_{k + 1} = x_k - \tau_k\frac{f'(x_k)}{f''(x_k)}\quad, 0 < \tau_k \le 1 \]
\(\tau_k = \tau = \const\) (\(\tau = 1\) --- метод Ньютона)
\[ \varphi(\tau) = f(x_k - \tau\frac{f'(x_k)}{f''(x_k)}) \to \min \]
\[ \tau_k = \frac{(f'(x_k))^2}{(f'(x_k))^2 + (f'(\tilde{x}_k))^2} \]
, где \(\tilde{x} = x_k - \frac{f'(x_k)}{f''(x_k)}\)
\subsubsection{Метод Маркрафта}
\label{sec:org99f723a}
\[ x_{k + 1} = x_k - \frac{f'(x_k)}{f''(x_k) + \mu_k}\quad,\mu_k > 0 \]
\(\mu_0\) рекомендуется выбирать на порядок больше значения второй произвдной в \(x_0\) \\
\(\mu_{k + 1}\): \(\mu_{k + 1} = \frac{\mu_k}{2}\), если \(f(x_{k + 1}) < f(x_k)\), иначе \(\mu_{k + 1} = 2\cdot\mu_k\)
\subsection{Метод минимизации многомодальных функций}
\label{sec:orgd7c2796}
\subsubsection{Метод ломанных}
\label{sec:org5f3942a}
Условие Липшица: \(f(x),\ x\in [a, b]\) будет удовлетворять условию, если:
\[ |f(x_1) - f(x_2)| \le L|x_1 - x_2|\quad,\forall x_1, x_2 \in [a, b] \]
\end{document}
