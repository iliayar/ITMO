% Created 2021-04-21 Wed 11:14
% Intended LaTeX compiler: pdflatex

\documentclass[english]{article}
\usepackage[T1, T2A]{fontenc}
\usepackage[lutf8]{luainputenc}
\usepackage[english, russian]{babel}
\usepackage{minted}
\usepackage{graphicx}
\usepackage{longtable}
\usepackage{hyperref}
\usepackage{xcolor}
\usepackage{natbib}
\usepackage{amssymb}
\usepackage{stmaryrd}
\usepackage{amsmath}
\usepackage{caption}
\usepackage{mathtools}
\usepackage{amsthm}
\usepackage{tikz}
\usepackage{grffile}
\usepackage{extarrows}
\usepackage{wrapfig}
\usepackage{algorithm}
\usepackage{algorithmic}
\usepackage{lipsum}
\usepackage{rotating}
\usepackage{placeins}
\usepackage[normalem]{ulem}
\usepackage{amsmath}
\usepackage{textcomp}
\usepackage{capt-of}

\usepackage{geometry}
\geometry{a4paper,left=2.5cm,top=2cm,right=2.5cm,bottom=2cm,marginparsep=7pt, marginparwidth=.6in}
 \usepackage{hyperref}
 \hypersetup{
     colorlinks=true,
     linkcolor=blue,
     filecolor=orange,
     citecolor=black,      
     urlcolor=cyan,
     }

\usetikzlibrary{decorations.markings}
\usetikzlibrary{cd}
\usetikzlibrary{patterns}
\usetikzlibrary{automata, arrows}

\newcommand\addtag{\refstepcounter{equation}\tag{\theequation}}
\newcommand{\eqrefoffset}[1]{\addtocounter{equation}{-#1}(\arabic{equation}\addtocounter{equation}{#1})}
\newcommand{\llb}{\llbracket}
\newcommand{\rrb}{\rrbracket}


\newcommand{\R}{\mathbb{R}}
\renewcommand{\C}{\mathbb{C}}
\newcommand{\N}{\mathbb{N}}
\newcommand{\A}{\mathfrak{A}}
\newcommand{\B}{\mathfrak{B}}
\newcommand{\rank}{\mathop{\rm rank}\nolimits}
\newcommand{\const}{\var{const}}
\newcommand{\grad}{\mathop{\rm grad}\nolimits}

\newcommand{\todo}{{\color{red}\fbox{\text{Доделать}}}}
\newcommand{\fixme}{{\color{red}\fbox{\text{Исправить}}}}

\newcounter{propertycnt}
\setcounter{propertycnt}{1}
\newcommand{\beginproperty}{\setcounter{propertycnt}{1}}

\theoremstyle{plain}
\newtheorem{propertyinner}{Свойство}
\newenvironment{property}{
  \renewcommand\thepropertyinner{\arabic{propertycnt}}
  \propertyinner
}{\endpropertyinner\stepcounter{propertycnt}}
\newtheorem{axiom}{Аксиома}
\newtheorem{lemma}{Лемма}
\newtheorem{manuallemmainner}{Лемма}
\newenvironment{manuallemma}[1]{%
  \renewcommand\themanuallemmainner{#1}%
  \manuallemmainner
}{\endmanuallemmainner}

\theoremstyle{remark}
\newtheorem*{remark}{Примечание}
\newtheorem*{solution}{Решение}
\newtheorem{corollary}{Следствие}[theorem]
\newtheorem*{examp}{Пример}
\newtheorem*{observation}{Наблюдение}

\theoremstyle{definition}
\newtheorem{task}{Задача}
\newtheorem{theorem}{Теорема}[section]
\newtheorem*{definition}{Определение}
\newtheorem*{symb}{Обозначение}
\newtheorem{manualtheoreminner}{Теорема}
\newenvironment{manualtheorem}[1]{%
  \renewcommand\themanualtheoreminner{#1}%
  \manualtheoreminner
}{\endmanualtheoreminner}
\captionsetup{justification=centering,margin=2cm}
\newenvironment{colored}[1]{\color{#1}}{}

\tikzset{->-/.style={decoration={
  markings,
  mark=at position .5 with {\arrow{>}}},postaction={decorate}}}
\makeatletter
\newcommand*{\relrelbarsep}{.386ex}
\newcommand*{\relrelbar}{%
  \mathrel{%
    \mathpalette\@relrelbar\relrelbarsep
  }%
}
\newcommand*{\@relrelbar}[2]{%
  \raise#2\hbox to 0pt{$\m@th#1\relbar$\hss}%
  \lower#2\hbox{$\m@th#1\relbar$}%
}
\providecommand*{\rightrightarrowsfill@}{%
  \arrowfill@\relrelbar\relrelbar\rightrightarrows
}
\providecommand*{\leftleftarrowsfill@}{%
  \arrowfill@\leftleftarrows\relrelbar\relrelbar
}
\providecommand*{\xrightrightarrows}[2][]{%
  \ext@arrow 0359\rightrightarrowsfill@{#1}{#2}%
}
\providecommand*{\xleftleftarrows}[2][]{%
  \ext@arrow 3095\leftleftarrowsfill@{#1}{#2}%
}
\makeatother

\newenvironment{rualgo}[1][]
  {\begin{algorithm}[#1]
     \selectlanguage{russian}%
     \floatname{algorithm}{Алгоритм}%
     \renewcommand{\algorithmicif}{{\color{red}\textbf{если}}}%
     \renewcommand{\algorithmicthen}{{\color{red}\textbf{тогда}}}%
     \renewcommand{\algorithmicelse}{{\color{red}\textbf{иначе}}}%
     \renewcommand{\algorithmicend}{{\color{red}\textbf{конец}}}%
     \renewcommand{\algorithmicfor}{{\color{red}\textbf{для}}}%
     \renewcommand{\algorithmicto}{{\color{red}\textbf{до}}}%
     \renewcommand{\algorithmicdo}{{\color{red}\textbf{делать}}}%
     \renewcommand{\algorithmicwhile}{{\color{red}\textbf{пока}}}%
     \renewcommand{\algorithmicrepeat}{{\color{red}\textbf{повторять}}}%
     \renewcommand{\algorithmicuntil}{{\color{red}\textbf{до тех пор пока}}}%
     \renewcommand{\algorithmicloop}{{\color{red}\textbf{повторять}}}%
     \renewcommand{\algorithmicnot}{{\color{blue}\textbf{не}}}%
     \renewcommand{\algorithmicand}{{\color{blue}\textbf{и}}}%
     \renewcommand{\algorithmicor}{{\color{blue}\textbf{или}}}%
     \renewcommand{\algorithmicrequire}{{\color{blue}\textbf{Ввод}}}%
     \renewcommand{\algorithmicensure}{{\color{blue}\textbf{Вывод}}}%
     \renewcommand{\algorithmicreturn}{{\color{red}\textbf{Вернуть}}}%
     \renewcommand{\algorithmicrtrue}{{\color{blue}\textbf{истинна}}}%
     \renewcommand{\algorithmicrfalse}{{\color{blue}\textbf{ложь}}}%
     % Set other language requirements
  }
  {\end{algorithm}}
\author{Ilya Yaroshevskiy}
\date{\today}
\title{Лекция 13}
\hypersetup{
 pdfauthor={Ilya Yaroshevskiy},
 pdftitle={Лекция 13},
 pdfkeywords={},
 pdfsubject={},
 pdfcreator={Emacs 28.0.50 (Org mode 9.4.4)}, 
 pdflang={English}}
\begin{document}

\maketitle
\tableofcontents


\section{Число обусловленности}
\label{sec:org7534b8b}
\newcommand{\cond}{\mathop{\rm cond}\nolimits}

\begin{examp}
\[ Ax = b \]
\[ A = \begin{pmatrix}
9.7 & 6.6 \\
4.1 & 2. 8
\end{pmatrix}\quad b = \begin{pmatrix}
9.7 \\
4.1
\end{pmatrix}\quad  x = \begin{pmatrix}
1 \\
0
\end{pmatrix}\]
\[ \Vert b \Vert = 13.8 \quad \Vert x \Vert = 1 \]
\[ \cond(A) = 2249.4 \]
\[ b' = \begin{pmatrix}
9.70 \\
4.11
\end{pmatrix} \quad x' = \begin{pmatrix}
0.34 \\
0.97
\end{pmatrix}\]
\[ \Delta b = b - b' \quad \Vert \Delta b \Vert = 0.01 \]
\[ \Delta x = x - x' \quad \Vert \Delta x \Vert = 1.63 \]
\[ \frac{\Vert \Delta b \Vert}{\Vert b \Vert} =0.00072464 \]
\[ \frac{\Vert \Delta x \Vert}{\Vert x \Vert} = 1.63 \]
\end{examp}
\subsection{Нормы и анализ ошибок}
\label{sec:orgfac5e00}
\[ \Vert A \Vert = \sum_{i = 1}^n \sum_{j = 1}^n |a_{ij}| \]
\[ \Vert A x \Vert \le \Vert A \Vert \cdot \Vert x \Vert \]
\[ \tilde{x}: \Vert A x \Vert = \Vert A \Vert \cdot \Vert \tilde{x} \Vert \]
\[ \Vert A \Ver = M = \max_{x \neq 0} \frac{\Vert A x \Vert}{\Vert x \Vert} \]
\[ \Vert a \Vert = \maxx_j \Vert a_j \Vert \]
Результат Уилкинсона
\[ x^*: (A + E) x^* = b \], где элементы \(E\) имеют уровень ошибок  округления
\todo
\[ \cond(A) = \Vert A \Vert \cdot \Vert A^{-1} \Vert \]
\[ \frac{\Vert x - x^* \Vert}{\Vert x^* \Vert} \le c \cdot \cond(A)\cdot \varepsilon_{??} \]
\begin{itemize}
\item \(a_j\) --- столбцы \(A\)
\item \(\tilde{a}_j\) --- столбцы \(A^{-1}\)
\end{itemize}
\[ \cond(A) = \max_j \Vert a_j \Vert \cdot \max_j \Vert \tilde{a}_j \Vert \]
\subsection{Оценивание числа обусловленности}
\label{sec:org97e4b90}
\[ \cond(A) = \max \frac{\max\frac{\Vert A x \Vert}{\Vert x \Vert}}{\min\frac{\Vert A^{-1} \Vert}{\Vert x \Vert}} \]
\[ \cond(A) = \Vert A \Vert \cdot \Vet A^{-1} \Vert \]
1-норма:
\begin{itemize}
\item \(a_j\) --- столбец
\end{itemize}
\[ \Vert A \Vert = \max_j \Vert a_j \Vert \]
\todo
\section{Дополнительно о градиентных методах}
\label{sec:org016f34b}
\(\{x_k\}\): \(x^k = x^{k - 1} + \alpha_k u^k\quad k \in \N\) \\
\(u^k \in E_n\). \((\nabla f(x), u) < 0\) --- условие спуска \\
Как находить \(\alpha_k\)
\[ f(x^{k - 1} + \alpha_k u^k) \le (1 - \lambda_k)f(x^{k - 1}) + \lambda_k \min_{\alpha \in E}f(x^{k - 1}+ \alpha u^k)\quad \lambda_k \in [0, 1] \]
\[ f(x^{k - 1} + \alpha_k u^k) \le f(x^{k - 1}) \] --- если это выполнено, то \(\{x^k\}\) --- релаксационная
\todo
\subsection{Градиентный спуск}
\label{sec:orgada7a47}
\todo
\begin{theorem}
Пусть \(f(x)\) ограничена снизу и дифференцируема в \(E_n\), а ее градиент удовлетворяет условию Липница, т.е. \(\forall x, y \in E_n\)
\[ |\nabla f(x) - \nabla f(y)| \le L | x - y | \]
\todo
\end{theorem}
\end{document}
