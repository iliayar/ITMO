% Created 2021-04-07 Wed 11:48
% Intended LaTeX compiler: pdflatex

  \documentclass[english]{article}
  \usepackage[T1, T2A]{fontenc}
\usepackage[lutf8]{luainputenc}
\usepackage[english, russian]{babel}
\usepackage{minted}
\usepackage{graphicx}
\usepackage{longtable}
\usepackage{hyperref}
\usepackage{xcolor}
\usepackage{natbib}
\usepackage{amssymb}
\usepackage{stmaryrd}
\usepackage{amsmath}
\usepackage{caption}
\usepackage{mathtools}
\usepackage{amsthm}
\usepackage{tikz}
\usepackage{grffile}
\usepackage{extarrows}
\usepackage{wrapfig}
\usepackage{algorithm}
\usepackage{algorithmic}
\usepackage{lipsum}
\usepackage{rotating}
\usepackage{placeins}
\usepackage[normalem]{ulem}
\usepackage{amsmath}
\usepackage{textcomp}
\usepackage{capt-of}
  
  \usepackage{geometry}
  \geometry{a4paper,left=2.5cm,top=2cm,right=2.5cm,bottom=2cm,marginparsep=7pt, marginparwidth=.6in}
   \usepackage{hyperref}
 \hypersetup{
     colorlinks=true,
     linkcolor=blue,
     filecolor=orange,
     citecolor=black,      
     urlcolor=cyan,
     }

\usetikzlibrary{decorations.markings}
\usetikzlibrary{cd}
\usetikzlibrary{patterns}
\usetikzlibrary{automata, arrows}

\newcommand\addtag{\refstepcounter{equation}\tag{\theequation}}
\newcommand{\eqrefoffset}[1]{\addtocounter{equation}{-#1}(\arabic{equation}\addtocounter{equation}{#1})}


\newcommand{\R}{\mathbb{R}}
\renewcommand{\C}{\mathbb{C}}
\newcommand{\N}{\mathbb{N}}
\newcommand{\A}{\mathfrak{A}}
\newcommand{\rank}{\mathop{\rm rank}\nolimits}
\newcommand{\const}{\var{const}}
\newcommand{\grad}{\mathop{\rm grad}\nolimits}

\newcommand{\todo}{{\color{red}\fbox{\text{Доделать}}}}
\newcommand{\fixme}{{\color{red}\fbox{\text{Исправить}}}}

\newcounter{propertycnt}
\setcounter{propertycnt}{1}
\newcommand{\beginproperty}{\setcounter{propertycnt}{1}}

\theoremstyle{plain}
\newtheorem{propertyinner}{Свойство}
\newenvironment{property}{
  \renewcommand\thepropertyinner{\arabic{propertycnt}}
  \propertyinner
}{\endpropertyinner\stepcounter{propertycnt}}
\newtheorem{axiom}{Аксиома}
\newtheorem{lemma}{Лемма}
\newtheorem{manuallemmainner}{Лемма}
\newenvironment{manuallemma}[1]{%
  \renewcommand\themanuallemmainner{#1}%
  \manuallemmainner
}{\endmanuallemmainner}

\theoremstyle{remark}
\newtheorem*{remark}{Примечание}
\newtheorem*{solution}{Решение}
\newtheorem{corollary}{Следствие}[theorem]
\newtheorem*{examp}{Пример}
\newtheorem*{observation}{Наблюдение}

\theoremstyle{definition}
\newtheorem{task}{Задача}
\newtheorem{theorem}{Теорема}[section]
\newtheorem*{definition}{Определение}
\newtheorem*{symb}{Обозначение}
\newtheorem{manualtheoreminner}{Теорема}
\newenvironment{manualtheorem}[1]{%
  \renewcommand\themanualtheoreminner{#1}%
  \manualtheoreminner
}{\endmanualtheoreminner}
\captionsetup{justification=centering,margin=2cm}
\newenvironment{colored}[1]{\color{#1}}{}

\tikzset{->-/.style={decoration={
  markings,
  mark=at position .5 with {\arrow{>}}},postaction={decorate}}}
\makeatletter
\newcommand*{\relrelbarsep}{.386ex}
\newcommand*{\relrelbar}{%
  \mathrel{%
    \mathpalette\@relrelbar\relrelbarsep
  }%
}
\newcommand*{\@relrelbar}[2]{%
  \raise#2\hbox to 0pt{$\m@th#1\relbar$\hss}%
  \lower#2\hbox{$\m@th#1\relbar$}%
}
\providecommand*{\rightrightarrowsfill@}{%
  \arrowfill@\relrelbar\relrelbar\rightrightarrows
}
\providecommand*{\leftleftarrowsfill@}{%
  \arrowfill@\leftleftarrows\relrelbar\relrelbar
}
\providecommand*{\xrightrightarrows}[2][]{%
  \ext@arrow 0359\rightrightarrowsfill@{#1}{#2}%
}
\providecommand*{\xleftleftarrows}[2][]{%
  \ext@arrow 3095\leftleftarrowsfill@{#1}{#2}%
}
\makeatother

\newenvironment{rualgo}[1][]
  {\begin{algorithm}[#1]
     \selectlanguage{russian}%
     \floatname{algorithm}{Алгоритм}%
     \renewcommand{\algorithmicif}{{\color{red}\textbf{если}}}%
     \renewcommand{\algorithmicthen}{{\color{red}\textbf{тогда}}}%
     \renewcommand{\algorithmicelse}{{\color{red}\textbf{иначе}}}%
     \renewcommand{\algorithmicend}{{\color{red}\textbf{конец}}}%
     \renewcommand{\algorithmicfor}{{\color{red}\textbf{для}}}%
     \renewcommand{\algorithmicto}{{\color{red}\textbf{до}}}%
     \renewcommand{\algorithmicdo}{{\color{red}\textbf{делать}}}%
     \renewcommand{\algorithmicwhile}{{\color{red}\textbf{пока}}}%
     \renewcommand{\algorithmicrepeat}{{\color{red}\textbf{повторять}}}%
     \renewcommand{\algorithmicuntil}{{\color{red}\textbf{до тех пор пока}}}%
     \renewcommand{\algorithmicloop}{{\color{red}\textbf{повторять}}}%
     \renewcommand{\algorithmicnot}{{\color{blue}\textbf{не}}}%
     \renewcommand{\algorithmicand}{{\color{blue}\textbf{и}}}%
     \renewcommand{\algorithmicor}{{\color{blue}\textbf{или}}}%
     \renewcommand{\algorithmicrequire}{{\color{blue}\textbf{Предусловие}}}%
     \renewcommand{\algorithmicrensure}{{\color{blue}\textbf{Постусловие}}}%
     \renewcommand{\algorithmicrtrue}{{\color{blue}\textbf{истинна}}}%
     \renewcommand{\algorithmicrfalse}{{\color{blue}\textbf{ложь}}}%
     % Set other language requirements
  }
  {\end{algorithm}}
\author{Ilya Yaroshevskiy}
\date{\today}
\title{Лекция 10}
\hypersetup{
 pdfauthor={Ilya Yaroshevskiy},
 pdftitle={Лекция 10},
 pdfkeywords={},
 pdfsubject={},
 pdfcreator={Emacs 28.0.50 (Org mode 9.4.4)}, 
 pdflang={English}}
\begin{document}

\maketitle
\tableofcontents



\section{Разреженный формат}
\label{sec:org54428ea}
\subsection{Строчно столбцовый формат}
\label{sec:org5422d77}
\begin{enumerate}
\item Вещественный массив \(di[n]\) --- диагональные элементы
\item Вещественный массив \(al, au\) --- по строкам и стобцам соответсвенно
\item Целочисленный массив \(ja\) --- содержит номера столбцов (строк)
хранимых внедиагональных элементов нижнего(верхнего) треугольника
матрицы. \(j \le m\), где \(m\) --- размерность массивов \(ja, al, au\), \(ja[j]\) --- номер столбца для \(al[j]\), или номер строки для \(au[j]\)
\item Целочисленный массив \(ia\), \(ia[k]\) --- равен индексу(в нумерации с 1) с которого начинается \(k\)-той строки(столбца)
\end{enumerate}
Размерность \(ja, al, au\): \(ia[n + 1] - 1\). \(ia[i + 1] - ia[i]\) --- количество хранимых внедиагональных элементов \(i\)-той строки(столбца) нижнего(верхнего) треугольника. \(ia\) и \(ja\) --- портрет матрицы.

\begin{examp}
\-
\[ \begin{matrix}
a_{11} & & & & & & & & \\
0 & a_{22} & a_{23} & a_{24} & & &  \text{\huge0}& &  \\
0 & a_{32} & a_{33} & 0 & a_{35} & a_{36} & & &  \\
0 & a_{42} & 0 & a_{44} & a_{45} & 0 & a_{47} & &  \\
& & a_{53} & a_{54} & a_{55} & a_{56} & 0 & a_{58} & a_{59} \\
& & a_{63} & 0 & a_{65} & a_{66} & 0 & a_{68} & 0 \\
& & & a_{74} & 0 & 0 & a_{77} & 0 & a_{79} \\
& \text{\huge0}& &  & a_{85} & a_{86} & 0 & a_{88} & 0 \\
& & &  & a_{95} & 0 & a_{97} & 0 & a_{99}
\end{matrix} \]

\[ di = [a_{11}, a_{22}, a_{33}, a_{44}, a_{55}, a_{66}, a_{77}, a_{88}, a_{99}] \]
\[ ia = [1, 1, 1, 2, 3, 5, 7, 8, 10, 12] \]
\[ ja = [2, 2, 3, 4, 3, 5, 4, 5, 6, 5, 7] \]
\[ al = [a_{32}, a_{42}, a_{53}, a_{54}, a_{63}, a_{65}, a_{74}, a_{85}, a_{86}, a_{95}, a_{97}] \]
\[ au = [a_{23}, a_{24}, a_{35}, a_{45}, a_{36}, a_{56}, a_{47}, a_{58}, a_{68}, a_{59}, a_{79}] \]
Для шестой строчки: \(ia[6] = 5\) --- начало шестой строчки в массиве \(ja\) и \(al\). \(ia[6 + 1] - ia[6] = 7 - 5 = 2\) --- количество элементов.
\begin{enumerate}
\item \(ja[ia[6]] = ja[5] = 3\)
\item \(ja[ia[6] + 1] = ja[6] = 5\)
\end{enumerate}
\end{examp}
\subsection{Решение СЛАУ. Метод Гаусса}
\label{sec:org5e10bce}
\[ \begin{cases}
a_{11}x_1 + a_{12} x_2 + \dots + a_{1n}x_n = b_1 \\
a_{21}x_1 + a_{22} x_2 + \dots + a_{2n}x_n = b_2 \\
\vdots \\
a_{n1}x_1 + a_{n2} x_2 + \dots + a_{nn}x_n = b_n
\end{cases} \]

\[ Ax = b \]
\begin{itemize}
\item \(A = (a_{ij})_{i,j = 1}^n\) --- вещественные числа
\item \(b = (b_1, b_2, \dots, b_n)^T\)
\item \(x = (x_1, x_2, \dots, x_n)^T\)
\end{itemize}
\(-\frac{a_{21}}{a_{11}}, -\frac{a_{31}}{a_{11}}, \dots, -\frac{a_{n1}}{a_{11}}\)
Верхний индекс обозначает этап.
\[ \begin{cases}
a_{11}x_1 + a_{12} x_2 + \dots + a_{1n}x_n = b_1 \\
a_{22}^{(1)}x_2 + a_{23}^{(1)}x_3 + \dots + a_{2n}^{(1)}x_n = b_2^{(1)} \\
a_{32}^{(1)}x_2 + a_{33}^{(1)}x_3 + \dots + a_{3n}^{(1)}x_n = b_3^{(1)} \\
\vdots \\
a_{n2}^{(1)}x_2 + a_{n3}^{(1)}x_3 + \dots + a_{nn}^{(1)}x_n = b_n^{(1)}
\end{cases} \]
\[ a_{ij}^{(1)} = a_{ij} - \frac{a_{i1}}{a_{11}}a_{1j} \quad b_i^{(1)} = b_i - \frac{a_{i1}}{a_{11}}b_1 \quad i,j = \overline{2, n}\]
\begin{remark}
\(a_{11} \neq 0\)
\end{remark}

\[ \begin{cases}
a_{11}x_1 + a_{12} x_2 + \dots + a_{1n}x_n = b_1 \\
a_{22}^{(1)}x_2 + a_{23}^{(1)}x_3 + \dots + a_{2n}^{(1)}x_n = b_2^{(1)} \\
a_{33}^{(2)}x_3 + a_{34}^{(2)}x_4 + \dots + a_{3n}^{(2)}x_n = b_3^{(2)} \\
\vdots \\
a_{n3}^{(2)}x_3 + a_{n4}^{(2)}x_4 + \dots + a_{nn}^{(2)}x_n = b_n^{(2)}
\end{cases} \]
\[ a_{ij}^{(2)} = a_{ij}^{(1)} - \frac{a_{i2}^{(1)}}{a_{22}^{(1)}}a_{2j}^{(1)} \quad b_i^{(2)} = b_i^{(1)} - \frac{a_{i2}^{(1)}}{a_{22}^{(1)}}b_2^{(1)} \]

\(n - 1\) этап:

\[ \begin{cases}
a_{11}x_1 + a_{12} x_2 + \dots + a_{1n}x_n = b_1 \\
a_{22}^{(1)}x_2 + a_{23}^{(1)}x_3 + \dots + a_{2n}^{(1)}x_n = b_2^{(1)} \\
a_{33}^{(2)}x_3 + a_{34}^{(2)}x_4 + \dots + a_{3n}^{(2)}x_n = b_3^{(2)} \\
\vdots \\
a_{nn}^{(n - 1)}x_n = b_n^{(n - 1)}
\end{cases} \]

\[ a_{ij}^{(k)} = a_{ij}^{(k - 1)} - \frac{a_{ik}^{(k - 1)}}{a_{kk}^{(k - 1)}}a_{kj}^{(k - 1)} \quad b_i^{(k)} = b_i^{(k - 1)} - \frac{a_{ik}^{(k - 1)}}{a_{kk}^{(k - 1)}}b_k^{(k - 1)} \quad \ k \in \overline{1, n};\ i,j \in \overline{k + 1, n}\]
\subsection{Обратный ход Гаусса}
\label{sec:org12b964f}
\[ x_n = \frac{b_n^{(n - 1)}}{a_{nn}^{(n - 1)}} \]
\[ \vdots \]
\[ x_2 = \frac{b_2^{(1)} - a_{23}^{(1)} x_3 - \dots - a_{2n}^{(1)}x_n}{a_{22}^{(1)}} \]
\[ x_1 = \frac{b_1 - a_{12} x_2 - \dots - a_{1n} x_n}{a_{11}} \]
\[ x_k = \frac{b_k^{(k - 1)} - \sum_{i = k + 1}^n a_{ki}^{(k - 1)} x_i}{a_{kk}^{(k - 1)}} \quad k \in \overline{n, 1}\]
\todo
Модификация
\begin{rualgo}[H]
\caption{Модификация алгоритма Гаусса}
\begin{algorithmic}[1]
\STATE \(m:\ m \ge k,\ |a_{mk}| = \max_{i \ge k}\{|a_{ik}|\}\)
\IF{\(a_{mk} = 0\)}
\STATE Нет однозначного решения. Завершить алгоритм
\PROCEDURE
\ELSE
\FOR{\(j = k,\dots,n\)}
\STATE Поменять местами \(b_x\) и \(b_m\)
\STATE Поменять местами \(a_{kj}\) и \(a_{mj}\)
\ENDFOR
\ENDIF
\end{algorithmic}
\end{rualgo}
\end{document}
