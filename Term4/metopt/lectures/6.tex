% Created 2021-03-10 Wed 11:34
% Intended LaTeX compiler: pdflatex
\documentclass[english]{article}
\usepackage[T1, T2A]{fontenc}
\usepackage[lutf8]{luainputenc}
\usepackage[english, russian]{babel}
\usepackage{minted}
\usepackage{graphicx}
\usepackage{longtable}
\usepackage{hyperref}
\usepackage{xcolor}
\usepackage{natbib}
\usepackage{amssymb}
\usepackage{stmaryrd}
\usepackage{amsmath}
\usepackage{caption}
\usepackage{mathtools}
\usepackage{amsthm}
\usepackage{tikz}
\usepackage{grffile}
\usepackage{extarrows}
\usepackage{wrapfig}
\usepackage{rotating}
\usepackage{placeins}
\usepackage[normalem]{ulem}
\usepackage{amsmath}
\usepackage{textcomp}
\usepackage{capt-of}

\usepackage{geometry}
\geometry{a4paper,left=2.5cm,top=2cm,right=2.5cm,bottom=2cm,marginparsep=7pt, marginparwidth=.6in}

 \usepackage{hyperref}
 \hypersetup{
     colorlinks=true,
     linkcolor=blue,
     filecolor=orange,
     citecolor=black,      
     urlcolor=cyan,
     }

\usetikzlibrary{decorations.markings}
\usetikzlibrary{cd}
\usetikzlibrary{patterns}

\newcommand\addtag{\refstepcounter{equation}\tag{\theequation}}
\newcommand{\eqrefoffset}[1]{\addtocounter{equation}{-#1}(\arabic{equation}\addtocounter{equation}{#1})}


\newcommand{\R}{\mathbb{R}}
\renewcommand{\C}{\mathbb{C}}
\newcommand{\N}{\mathbb{N}}
\newcommand{\rank}{\text{rank}}
\newcommand{\const}{\text{const}}
\newcommand{\grad}{\text{grad}}

\theoremstyle{plain}
\newtheorem{axiom}{Аксиома}
\newtheorem{lemma}{Лемма}
\newtheorem{manuallemmainner}{Лемма}
\newenvironment{manuallemma}[1]{%
  \renewcommand\themanuallemmainner{#1}%
  \manuallemmainner
}{\endmanuallemmainner}

\theoremstyle{remark}
\newtheorem*{remark}{Примечание}
\newtheorem*{solution}{Решение}
\newtheorem{corollary}{Следствие}[theorem]
\newtheorem*{examp}{Пример}
\newtheorem*{observation}{Наблюдение}

\theoremstyle{definition}
\newtheorem{task}{Задача}
\newtheorem{theorem}{Теорема}[section]
\newtheorem*{definition}{Определение}
\newtheorem*{symb}{Обозначение}
\newtheorem{manualtheoreminner}{Теорема}
\newenvironment{manualtheorem}[1]{%
  \renewcommand\themanualtheoreminner{#1}%
  \manualtheoreminner
}{\endmanualtheoreminner}
\captionsetup{justification=centering,margin=2cm}
\newenvironment{colored}[1]{\color{#1}}{}

\tikzset{->-/.style={decoration={
  markings,
  mark=at position .5 with {\arrow{>}}},postaction={decorate}}}
\makeatletter
\newcommand*{\relrelbarsep}{.386ex}
\newcommand*{\relrelbar}{%
  \mathrel{%
    \mathpalette\@relrelbar\relrelbarsep
  }%
}
\newcommand*{\@relrelbar}[2]{%
  \raise#2\hbox to 0pt{$\m@th#1\relbar$\hss}%
  \lower#2\hbox{$\m@th#1\relbar$}%
}
\providecommand*{\rightrightarrowsfill@}{%
  \arrowfill@\relrelbar\relrelbar\rightrightarrows
}
\providecommand*{\leftleftarrowsfill@}{%
  \arrowfill@\leftleftarrows\relrelbar\relrelbar
}
\providecommand*{\xrightrightarrows}[2][]{%
  \ext@arrow 0359\rightrightarrowsfill@{#1}{#2}%
}
\providecommand*{\xleftleftarrows}[2][]{%
  \ext@arrow 3095\leftleftarrowsfill@{#1}{#2}%
}
\makeatother
\author{Ilya Yaroshevskiy}
\date{\today}
\title{Лекция 6}
\hypersetup{
 pdfauthor={Ilya Yaroshevskiy},
 pdftitle={Лекция 6},
 pdfkeywords={},
 pdfsubject={},
 pdfcreator={Emacs 28.0.50 (Org mode )}, 
 pdflang={English}}
\begin{document}

\maketitle
\tableofcontents


\section{Постановка задачи}
\label{sec:org78379d9}
\begin{enumerate}
\item \(x^* = (x_1, x_2, \dots, x_n)^T,\ x_i \in U \subset E_n\), где \(U\) --- множество допустимых значений, \(E_n\) --- эвклидово
пространство размера \(n\). \(f(x^*) = \min_{x \in U} f(x)\). Если
ствится задача найти максимум, то млжно перейти к поиску минимума: \(f(x^*) = \max_{x\in U}f(x) = -\min_{x \in U}(-f(x))\)
\item \(f(x^*) = \text{extr}_{x \in U}f(x)\)
\item Если \(U\) задается ограничением на вектор \(x\), то задача поиска
условного экстремума. Если \(U = E_n\) --- не имеет ограничений,
то задача поиска безусловного экстремума
\item Решение задачи поиска экстремума --- пара \((x^*, f(x^*))\)
\end{enumerate}

\noindent\rule{\textwidth}{0.5pt}
Если \(\forall x \in U\ f(x^*) \le f(x)\) --- то \(x^*\) --- глобальный минимум. Локальный минимум \(x^* \in U\): если \(\exists \varepsilon > 0\), что \(\forall x \in U\) и \(\Vert x - x^* \Vert < \varepsilon\), то \(f(x^*) \le f(x)\)
\begin{definition}
\textbf{Поверхностью уровня} функции \(f(x)\) называется множество точек, в которых функция принимает постоянные значения, т.е. \(f(x) = \const\)
\end{definition}
\begin{definition}
\textbf{Градиентом} \(\nabla\) f(x) непрерывно жифференцируемой функции \(f(x)\) в x:
\[ \nabla f(x) = \left(\begin{array}{c} \frac{\partial f(x)}{\partial x_1} \\ \frac{\partial f(x)}{\partial x_2} \\ \vdots \\ \frac{\partial f(x)}{\partial x_n}\end{array}\right) \]
Градиент направлен по нормали к поверхности уровня, т.е. перпендикулярно к касательной плоскости в точке \(x\), проведенной в сторону наибольшего возрастания функции
\end{definition}
\begin{definition}
\textbf{Матрицей Гессе} \(H(x)\) дважды непрерывно дифференцируемой в точке
\(x\) функции \(f(x)\) называется матрица частных производных второго
порядка, вычисленных в данной точке.
\[ H(x) = \left( \begin{array}{cccc} \frac{\partial^2 f(x)}{\partial x_1^2} & \frac{\partial^2 f(x)}{\partial x_1x_2} & \dots & \frac{\partial^2 f(x)}{\partial x_1x_n} \\ \vdots & \vdots & \ddots & \vdots \\ \frac{\partial^2 f(x)}{\partial x_nx_1} & \frac{\partial^2 f(x)}{\partial x_nx_2} & \dots & \frac{\partial^2 f(x)}{\partial x_n^2} \end{array}\right) \]
\end{definition}
\begin{enumerate}
\item \(H(x)\) --- симметричная, размер \(n \time n\)
\item Антиградиент: вектор, равный по модулю вектору градиента, но противоположный по направлению. Указывает в сторну наибольшего убывания функции \(f(x)\)
\item \[ \nabla f(x) = f(x + \Delta x) - f(x) = \nabla f(x)^T\Delta x + \frac{1}{2} \Delta x^T H(x)\Delta x + o(\Vert \Delta x \Vert^2) \]
\(o(\Vert \Delta x \Vert^2)\) --- сумма всех членов разложения, имеющих порядок выше второго, \(\Delta x^T H(x) \Delta x\) --- квадратичная форма
\end{enumerate}
\subsection{Свойства квадратичных форм}
\label{sec:org2ed1c46}
Квадратичная форма \(\Delta x^T H(x) \Delta x\) (и соответсвующая матрица \(H(x)\)) называется:
\begin{itemize}
\item положительно опрделенной \(H(x) > 0\), если \(\forall \Delta x \neq 0\ \Delta x^T H(x) \Delta x > 0\)
\item отрицательно определенной \(H(x) < 0\), если \(\forall \Delta x \neq 0\ \Delta x^T H(x) \Delta x < 0\)
\item положительно полуопределенной \(H(x) \ge 0\), если \(\forall \Delta x \neq 0\ \Delta x^T H(x) \Delta x \ge 0\)
и имеется \(\Delta x \neq 0: \Delta x^T H(x) \Delta x = 0\)
\item отрицательно полуопределенной \(H(x) \le 0\), если \(\forall \Delta x \neq 0\ \Delta x^T H(x) \Delta x \le 0\)
и имеется \(\Delta x \neq 0: \Delta x^T H(x) \Delta x = 0\)
\item неопределнной, если \(\exists \Delta x, \Delta \tilde{x}: \Delta x^T H(x) \Delta x > 0,\ \Delta \tilde{x}^T H(\tilde{x}) \Delta \tilde{x} < 0\)
\item тождественно равной нулю \(H(x) \equiv 0\), если \(\forall \Delta x\ \Delta x^T H(x) \Delta x = 0\)
\end{itemize}
\subsection{Свойства выпуклых множеств и выпуклых функций}
\label{sec:org44d345d}
\begin{definition}
Пусть \(x, y \in E_n\). Множество точек вида \(\{z\} \subset E_n: z = \alpha x + (1 - \alpha)y\), \(\alpha \in [0, 1]\), \(z\) --- отрезок, соединяющий \(x\) и \(y\).
\end{definition}
\begin{examp}
\(E_n: n \le 3\): \(z\) --- отрезок(обычный)
\end{examp}
\begin{definition}
\(U \subset E_n\) выпуклое, если вместе с точками \(x\) и (y \(\in\) U) оно содержит и весь отрезок \(z = \alpha x + (1 - \alpha)y, \alpha \in [0, 1]\)
\end{definition}
\begin{definition}
Функция \(f(x)\), заданая на выпуклом \(U \subset E_n\) называется:
\begin{itemize}
\item выпуклой, если \(\forall x, y \in U\) и \(\forall \alpha [0, 1]\) выполняется \(f(\alpha x + (1 - \alpha)y) \le \alpha f(x) + (1- \alpha)f(y)\)
\item строго выпуклой, если \(\forall \alpha \in (0, 1)\) выполняется \(f(\alpha x + (1 - \alpha)y) < \alpha f(x) + (1-\alpha)f(y)\)
\item сильно выпуклой с константой \(l > 0\), если \(\forall x, y \in U\) и \(\forall \alpha \in [0, 1]\) выполняется \(f(\alpha x + (1 - \alpha)y) \le \alpha f(x) + (1- \alpha) f(y) - \frac{l}{2}\alpha(1 - \alpha)\Vert x - y \Vert^2\)
\end{itemize}
\end{definition}
\emph{Свойства:}
\begin{enumerate}
\item Функция \(f(x)\) \uline{выпуклая}, если ее грфик целиком лежит не выше отрезка, соединяющего две ее произвольные точки \\
Функция \(f(x)\) \uline{строго выпуклая}, если ее график лежит целиком ниже отрезка, соединяющего две ее произвольные, но не совпадающие точки
\item Если функция \(f(x)\) \uline{сильно выпуклая}, то она одноверменно строго выпуклая и выпуклая \\
Если функция \(f(x)\) \uline{строго выпуклая}, то она одновременно выпуклая
\item Выпуклость функции можно определить по матрице Гессе \(H(x)\)
\begin{itemize}
\item Если \(H(x) \ge 0\ \forall x \in E_n\), то \(f(x)\) выпуклая
\item Если \(H(x) > 0\ \forall x\in E_n\), то \(f(x)\) строго выпуклая
\item Если \(H(x)\ \ge lE\ \forall x \in E_n\), где \(E\) --- единичная матрица, то \(f(x)\) сильно выпуклая
\end{itemize}
\end{enumerate}
\emph{Свойства выпуклых функций:}
\begin{enumerate}
\item Если f(x) выпуклая функция на выпуклом множестве \(U\), то всякая точка локального минимума есть точка глобального минимума на \(U\)
\item Если выпуклая функция достигает своего минимума в двух различных точках, то она достигает миниума во всех точках отрезка, соединяющих это точки.
\item Если \(f(x)\) строго выпуклая функция множества \(U\), то она может достигать своего глобального минимума на \(U\) не более чем в одной точке
\end{enumerate}
\subsection{Необходимое и достаточное условие безусловного экстремума}
\label{sec:org4c6103d}
\begin{theorem}[Необходимое условие экстремума первого порядка]
Пусть \(x^* \in E_n\) --- локальный минимум или максимум \(f(x)\) на \(E_n\) и \(f(x)\) --- дифференцируема в точке \(x^*\) \\
\uline{Тогда} \(\nabla f(x)\) в точке \(x^*\) равен нулю \(\nabla f(x^*) = 0\), т.е.
\[ \frac{\partial f(x^*)}{\partial x_i} = 0,\ i = \overline{1, n} \]
\end{theorem}
\begin{definition}
Точки \(x^*: \nabla f(x^*) = 0\) --- \textbf{стационарные}
\end{definition}
\begin{theorem}[Необходимое условие экстремума второго порядка]
Пусть \(x^* \in E_n\) --- точка локального минимума или максимума \(f(x)\) на \(E_n\) и \(f(x)\) --- дважды дииференцируемая в точке. \\
\uline{Тогда} \(H(x^*)\) --- является положительно или отрицательно(если максимум) полуопределенной, т.е. \(H(x^*) \ge 0\) или \(H(x^*) \le 0\)(если максимум)
\end{theorem}
\begin{theorem}[Достаточное условие экстремума]
Пусть \(f(x)\) в \(x^* \in E_n\) дважды дифференцируема, ее \(\nabla f(x) = 0\), а \(H(x^*) > 0\) или \(H(x^*) < 0\)(для максимума). \\
\uline{Тогда} \(x^*\) --- точка локального минимума(максимума) \(f(x)\) на \(E_n\)
\end{theorem}
\subsubsection{Проверка выполнения условий}
\label{sec:orgfc78a8e}
\begin{itemize}
\item вычисление угловых миноров \(H(x)\)
\item вычисление главных миноров \(H(x)\) \\
\end{itemize}


\begin{enumerate}
\item Ислледование положительной или отрицательной определнности угловых и главных миноров
\item Анализ собственных значений матрицы \(H(x)\)
\end{enumerate}
\end{document}
