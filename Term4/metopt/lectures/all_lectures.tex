% Created 2021-03-17 Wed 11:34
% Intended LaTeX compiler: pdflatex

  \documentclass[oneside]{book}
  \usepackage[T1, T2A]{fontenc}
\usepackage[lutf8]{luainputenc}
\usepackage[english, russian]{babel}
\usepackage{minted}
\usepackage{graphicx}
\usepackage{longtable}
\usepackage{hyperref}
\usepackage{xcolor}
\usepackage{natbib}
\usepackage{amssymb}
\usepackage{stmaryrd}
\usepackage{amsmath}
\usepackage{caption}
\usepackage{mathtools}
\usepackage{amsthm}
\usepackage{tikz}
\usepackage{grffile}
\usepackage{extarrows}
\usepackage{wrapfig}
\usepackage{rotating}
\usepackage{placeins}
\usepackage[normalem]{ulem}
\usepackage{amsmath}
\usepackage{textcomp}
\usepackage{capt-of}
  
  \addto\captionsrussian{\renewcommand{\chaptername}{Лекция}}
  
   \usepackage{hyperref}
   \hypersetup{
       colorlinks=true,
       linkcolor=blue,
       filecolor=orange,
       citecolor=black,      
       urlcolor=cyan,
       }

  \usetikzlibrary{decorations.markings}
  \usetikzlibrary{cd}
  \usetikzlibrary{patterns}
  \usetikzlibrary{automata, arrows}

  \newcommand\addtag{\refstepcounter{equation}\tag{\theequation}}
  \newcommand{\eqrefoffset}[1]{\addtocounter{equation}{-#1}(\arabic{equation}\addtocounter{equation}{#1})}


  \newcommand{\R}{\mathbb{R}}
  \renewcommand{\C}{\mathbb{C}}
  \newcommand{\N}{\mathbb{N}}
  \newcommand{\rank}{\text{rank}}
  \newcommand{\const}{\text{const}}
  \newcommand{\grad}{\text{grad}}

  \theoremstyle{plain}
  \newtheorem{axiom}{Аксиома}
  \newtheorem{lemma}{Лемма}
  \newtheorem{manuallemmainner}{Лемма}
  \newenvironment{manuallemma}[1]{%
    \renewcommand\themanuallemmainner{#1}%
    \manuallemmainner
  }{\endmanuallemmainner}

  \theoremstyle{remark}
  \newtheorem*{remark}{Примечание}
  \newtheorem*{solution}{Решение}
  \newtheorem{corollary}{Следствие}[theorem]
  \newtheorem*{examp}{Пример}
  \newtheorem*{observation}{Наблюдение}

  \theoremstyle{definition}
  \newtheorem{task}{Задача}
  \newtheorem{theorem}{Теорема}[section]
  \newtheorem*{definition}{Определение}
  \newtheorem*{symb}{Обозначение}
  \newtheorem{manualtheoreminner}{Теорема}
  \newenvironment{manualtheorem}[1]{%
    \renewcommand\themanualtheoreminner{#1}%
    \manualtheoreminner
  }{\endmanualtheoreminner}
  \captionsetup{justification=centering,margin=2cm}
  \newenvironment{colored}[1]{\color{#1}}{}

  \tikzset{->-/.style={decoration={
    markings,
    mark=at position .5 with {\arrow{>}}},postaction={decorate}}}
  \makeatletter
  \newcommand*{\relrelbarsep}{.386ex}
  \newcommand*{\relrelbar}{%
    \mathrel{%
      \mathpalette\@relrelbar\relrelbarsep
    }%
  }
  \newcommand*{\@relrelbar}[2]{%
    \raise#2\hbox to 0pt{$\m@th#1\relbar$\hss}%
    \lower#2\hbox{$\m@th#1\relbar$}%
  }
  \providecommand*{\rightrightarrowsfill@}{%
    \arrowfill@\relrelbar\relrelbar\rightrightarrows
  }
  \providecommand*{\leftleftarrowsfill@}{%
    \arrowfill@\leftleftarrows\relrelbar\relrelbar
  }
  \providecommand*{\xrightrightarrows}[2][]{%
    \ext@arrow 0359\rightrightarrowsfill@{#1}{#2}%
  }
  \providecommand*{\xleftleftarrows}[2][]{%
    \ext@arrow 3095\leftleftarrowsfill@{#1}{#2}%
  }
  \makeatother
\author{Ilya Yaroshevskiy}
\date{\today}
\title{Лекции по Методам оптимизации 4 семестр}
\hypersetup{
 pdfauthor={Ilya Yaroshevskiy},
 pdftitle={Лекции по Методам оптимизации 4 семестр},
 pdfkeywords={},
 pdfsubject={},
 pdfcreator={Emacs 28.0.50 (Org mode )}, 
 pdflang={English}}
\begin{document}

\maketitle
\tableofcontents


\chapter{}
\label{sec:org70d6c51}
\chapter{Лекции 1 и 2}
\label{sec:orgeaefb0f}
\section{Теория погрешности}
\label{sec:orged29555}
\begin{defintion}
Отклонение от теоретического решения
\end{defintion}
Виды погрешности:
\begin{enumerate}
\item Неустранимая погрешность
\begin{examp}
Физические величины, другие константы
\end{examp}
\item Устранимая погрешнеость
Связана с методом решения
\begin{enumerate}
\item Погрешность модели \\
Связана с матиматической формулировкой задачи. Она плохо отображает реальную модель
\item Остаточная погрешность(Погрешноть аппроксимации)
\item Погрешность округления
\item Накапливаемая погрешность \\
Нецелые числа
\end{enumerate}
\end{enumerate}

\noindent\rule{\textwidth}{0.5pt}

\begin{itemize}
\item \(X^*\) --- точное решение \\
\item \(X\) --- Приближенное решение
\item \(X^* - X\) --- погрешность
\item \(\Delta X = |X^* - X|\) --- абсолютная погрешность \\
\(\Delta_X \ge |X^* - X|\) --- предельная абсолютная погрешность, т.е. \[ X - \Delta_X \le X^* \le X + \Delta_X \]
\item \(\delta X = \left|\frac{X^* - X}{|X|}\right|\) --- относительная погрешость \\
\(\delta_X \ge \left|\frac{X^* - X}{|X|}\right|\) --- предельная относительная погрешность
\end{itemize}

\subsection{Значащие цифры}
\label{sec:org63865ca}
\begin{definition}
Все цифры в изображении отличные от нуля, и нули если они содержатся
между значащими цифрами, или расположены в конце числа и указывают на
сохранение разряды точности.  Нули стоящие левее, отличной от нуля
цифры, не являются значащимицифрами Между ненулевыми, или указывающие
на точность
\end{definition}

\begin{examp}
\(\underbrace{0.00}_\text{незнач.}2080\)
\end{examp}
\begin{examp}
\(689000 = 0.689 \cdot 10^6\) --- 3 значащие цифры
\(689000 = 0.689000 \cdot 10^6\) --- 6 значащих цифр
\end{examp}

\subsection{Верные цифры}
\label{sec:orga9c7c5c}
\begin{defintion}
Если, значащая цифра приближенного значения, находящаяся в разряде, в
котором выполняется условие --- абсолютное значение погрешности не
превосходит половину уиницы этого разряда \(\Delta \le 0.5\cdot 10^k\),
где \(k\) --- номер разряда, то она называется верной
\end{defintion}

\begin{examp}
\(a = 3.635\) \\
\(\Delta a = 0.003\) \\
\begin{description}
\item[{(3) \(k = 0\)}] \(\frac{1}{2} \cdot 10^0 = \frac{1}{2} \ge \Delta a\)
\item[{(6) \(k = -1\)}] \(\frac{1}{2} \cdot 10^{-1} = 0.05 \ge \Delta a\)
\item[{(3) \(k = -2\)}] \(\frac{1}{2} \cdot 10^{-2} = 0.005 \ge \Delta a\)
\item[{(5) \(k = -3\)}] \(\frac{1}{2} \cdot 10^{-3} = 0.0005 < \Delta a\) \(\Rightarrow\) 5 --- сомнительная цифра
\end{description}
\end{examp}

\subsection{Распространение погрешности}
\label{sec:org02728e5}
\begin{examp}
\(\left(\frac{\sqrt{2} - 1}{\sqrt{2} + 1}\right)^3 = (\sqrt{2} - 1)^6 = (3 - 2\sqrt{2})^3 = 99 - 70\sqrt{2}\) \\
\begin{description}
\item[{\(\sqrt{2}\)}] \[ \frac{7}{5} = 1.4 \]
\[ \frac{17}{12} = 1.41666 \]
\[ \frac{707}{500} = 1.414 \]
\[ \sqrt{2} = 1.4142145624 \]
\end{description}
\end{examp}

\[ \Delta_{x \pm y} = \Delta_x \pm \Delta_y \]
\[ \Delta_{(x\cdot y)} \approx |Y|\Delta_X + |X|\Delta_Y \]
\[ \Delta_{(\frac{x}{y})} \approx \left|\frac{1}{Y}\right|\Delta_X + \left|\frac{X}{Y^2}|\Delta_Y \]
\[ |\Delta u| = |f(x_1 + \Delta x_1, \dots, x_n + \Delta x_n) - f(x_1, \dots, x_n)| \]
\[ |\Delta u| \approx |df(x_1, \dots, x_n)| = \left|\sum_{i=1}^n \frac{\partial u}{\partial x_i}\Delta x_i \right| \le \sum_{i = 1}^n\left|\frac{\partial u}{\partial x_i}\right|\cdot|\Delta x_i| \addtag\label{star_1} \]
\[ |\delta u| = \frac{\ref{star_1}}{|u|} = \sum_{i = 1}^n\left|\frac{\partial u}{\partial x_i}\cdot\frac{1}{u}\right|\cdot|\Delta x_i| = \sum_{i = 1}^n \left|\frac{\partial \ln u}{\partial x_i}\right|\cdot|\Delta x_i| \]
\[ \delta_u = \sum_{i = 1}^n \left|\frac{\partial \ln u}{\partial x_i}\right| \cdot |\Delta x_i| \]
\[ \delta_{(X \pm Y)} = \left|\frac{X}{X \pm Y}\right| \delta_X + \left|\frac{Y}{X \pm Y}\right|\delta_Y \]
\[ \delta_{(X\cdot Y)} = \delta_X + \delta_Y \]
\[ \delta_{(\frac{X}{Y})} = \delta_X + \delta_Y \]

\begin{examp}
\(x = \frac{7}{5}\)
\begin{itemize}
\item \(f_1 = \left(\frac{\sqrt{2} - 1}{\sqrt{2} + 1}\right)\) \\
\[ \delta_{f_1} = 3\left|\frac{1}{x - 1} - \frac{1}{x + 1}\right|\cdot|\Delta X| = 6.25|\Delta X| \]
\item \(f_2 = (\sqrt{2} - 1)^6\) \\
\[ \delta_{f_2} = 6\left|\frac{1}{x - 1}\right|\cdot|\Delta X| = 15|\Delta X| \]
\item \(f_3 = (3 - 2\sqrt{2})^3\)
\[ \delta_{f_3} = 6\left|\frac{1}{3 - 2x}\right|\cdot|\Delta X| = 30|\Delta X| \]
\item \(f_4 = 99 - 70\sqrt{2}\)
\[ \delta_{f_4} = \left|\frac{90}{99 - 70x}\right|\cdot|\Delta X| = 70|\Delta X| \]
\end{itemize}
\end{examp}

\begin{examp}
\[ y^2 - 140y + 1 = 0 \]
Вычислить корни.
\begin{itemize}
\item \(y = 70 - \sqrt{4899}\) \\
\(\sqrt{4899} = 69.992\dots\) \\
\(\sqrt{4899} \approx 69.99\) \\
\(y \approx 70 - 69.99 = 0.01\)
\[ y = \frac{1}{70 + \sqrt{4899}} \]
\(\sqrt{4899} = 69.99;\ 70 + 69.99 = 139.99\) \\
\(y = \frac{1}{140} = 0.00714285 \approx 0.007143\)
\end{itemize}
\end{examp}


\section{Одномерная минимизация функций}
\label{sec:org002272c}
\subsection{Унимодальные функции}
\label{sec:orgc78efea}
\(f(x) \to \min,\ x \in U\) \\
\(f(x) \to \max \Rightarrow -f(x) \to min\) \\
\(x^* \in U\) --- точка минимума: \(f(x*) \le f(x)\ \forall x \in U\) \\
\(U^*\) --- множество точек минимума \\
\(\tilde{x} \in U: \exists V(\tilde{x})\ \forall x \in V\ f(\tilde{x}) \le f(x)\) --- локальный минимум
\begin{definition}
\(f(x)\) --- \textbf{унимодальная функция} на \([a, b]\), если:
\begin{enumerate}
\item \(f(x)\) --- непрерывна на \([a, b]\)
\item \(\exists \alpha, \beta: a \le \alpha\le\beta\le b\)
\begin{enumerate}
\item Если \(a < \alpha\), то \([a, \alpha]\) \(f(x)\) --- монотонно убывает
\item Если \(\beta < b\), то на \([\beta, b]\) \(f(x)\) --- монотонна возрастает
\item \(\forall x\in[\alpha, \beta]\) \(f(x) = f^* = \min_{[a, b]}f(x)\)
\end{enumerate}
\end{enumerate}
\end{definition}
\begin{remark}
Свойства:
\begin{enumerate}
\item Любая из точек локального минимума является глобальным минимумом на этом же отрезке
\item Функця унимодальная на \([a, b]\)  унимодальна на \([c, d] \subset [a, b]\)
\item \(f(x)\) унимодальна на \([a, b]\) \(a \le x_1 < x_2 \le b\)
\begin{enumerate}
\item если \(f(x_1)\le f(x_2)\), то \(x^* \in [a, x_2]\)
\item если \(f(x_1) > f(x_2)\), то \(x^* \in [x_1, b]\)
\end{enumerate}
\end{enumerate}
\end{remark}
\begin{definition}
\(f(x)\) \textbf{выпукла} на \([a, b]\), если:
\begin{itemize}
\item \(\forall x', x'' \in [a, b]\) и \(\alpha \in [0, 1]\): \\
\[ f(\alpha x' + (1 - \alpha)x'') \le \alpha f(x') + (1 - \alpha)f(x'') \]
\end{itemize}
\end{definition}
\begin{remark}
Свойства:
\begin{enumerate}
\item Если \(f(x)\) на \([a, b]\) \([x', x''] \subset [a, b]\)
\item Всякая выпуклая и непрерывная функция на \([a, b]\) является унимодальной на этом отрезке. Обратное не верно
\end{enumerate}
\end{remark}
\begin{definition}
\(x: f'(x) = 0\) --- \textbf{стационарная точка}
\end{definition}


\subsection{Прямые методы}
\label{sec:orge9f8a82}
Не требуют вычисление производной. Могут использовать только известные значения. 
\begin{enumerate}
\item Метод дихотомии
\label{sec:orgb93046f}
\[ x_1 = \frac{b + a - \delta}{2}\quad x_2 = \frac{b + a + \delta}{2} \addtag\label{delta_1} \]
\[ \tau = \frac{b - x_1}{b - a} = \frac{x_2 - a}{b - a} \rightarrow \frac{1}{2} \]
\[ X^* [a_i, b_i]\quad \frac{b_i - a_i}{2} \le \varepsilon \]

\begin{enumerate}
\item \(x_1\) и \(x_2\); вычислить \(f(x_1)\) и \(f(x_2)\)
\item \(f(x_1)\) и \(f(x_2)\)
\begin{itemize}
\item Если \(f(x_1) \le f(x_2) \rightarrow [a, x_2]\), т.е. \(b = x_2\)
\item Иначе \([x_1, b] \rightarrow [x_1, b]\), т.е. \(a = x_1\)
\end{itemize}
\item \(\varepsilon_n = \frac{b - a}{2}\) (\(n\) --- номер итерации)
\begin{itemize}
\item Если \(\varepsilon_n > \varepsilon\) --- переход к следующей итерации(шаг 1)
\item Если \(\varepsilon_n \le \varepsilon\), заврешить поиск(шаг 4)
\end{itemize}
\item \(x^* \approx \overline{x} = \frac{a + b}{2} \quad f^* \approx f(\overline{x})\)
\end{enumerate}

\(\ref{delta_1}\quad \delta \in (0, 2\varepsilon)\) \\
Число итерций \(n \ge \log_2\frac{b - a - \delta}{2\varepsilon - \delta}\)
\end{enumerate}
\chapter{}
\label{sec:orgfcb15ea}
\section{Одномерный поиск}
\label{sec:org3223d69}
\subsection{Метод золотого сечения}
\label{sec:org7464b3f}
\begin{remark}
Возьмем отрезок \([0, 1]\)
\begin{itemize}
\item \(x_2 = \tau \Rightarrow x_1 = 1 - \tau\)
\item \(x_1 \Rightarrow x'_2 = 1 - \tau \in [0, \tau]\)
\end{itemize}
\[ \frac{1}{\tau} = \frac{tau}{1 - \tau} \Rightarrow \tau^2 = 1 - \tau\]
\[ \tau = \frac{\sqrt{5} - 1}{2} \approx 0.61803 \]
\begin{itemize}
\item \(x_1 = 1 - \tau = \frac{3 - \sqrt{5}}{2}\)
\item \(x_2 = \tau = \frac{\sqrt{5} - 1}{2}\)
\end{itemize}
\end{remark}
\begin{enumerate}
\item \label{x_1_3} \(x_1 = a + \frac{3 - \sqrt{5}}{2}(b - a)\)
\item \label{x_2_3} \(x_2 = a + \frac{\sqrt{5} - 1}{2}(b - a)\)
\end{enumerate}
\[ \Delta_n = \tau^n(b - a) \]
\[ \varepsilon_n = \frac{\Delta_n}{2} = \frac{1}{2}\left(\frac{\sqrt{5} - 1}{2}\right)^n(b - a) \]
\(\varepsilon\) --- задано. Окончание: \(\varepsilon_n \le \varepsilon\) \\
На \(n\text{-ой}\) итерации: \(x^* = \frac{a_{(n)} + b_{(n)}}{2}\) \\
\[ n \ge \frac{\ln\left(\frac{2\varepsilon}{b - a}\right)}{\ln \tau} \approx 2.1 \ln\left(\frac{b - a}{2\varepsilon}\right) \]

\textbf{Алгоритм}.
\begin{enumerate}
\item \(x_1,\ x_2\) по формулам \ref{x_1_3} и \ref{x_2_3}
\[ \tau = \frac{\sqrt{5} - 1}{2}\ \varepsilon_n = \frac{b - a}{2} \]
\item \(\varepsilon_n > \varepsilon\) --- шаг 3, иначе 4
\item Если \(f(x_1) \le f(x_2)\), то:
\begin{itemize}
\item запоминаем \(f(x_1)\)
\item \(b = x_1\)
\item \(x_2 = x_1\)
\item \(x_1 = a + \tau(b - a)\)
\end{itemize}
Иначе:
\begin{itemize}
\item запоминаем \(f(x_2)\)
\item \(a = x_1\)
\item \(x_1 = x_2\)
\item \(x_2 = b - \tau(b - a)\)
\end{itemize}
\(\varepsilon_n = \tau\varepsilon_n\), переход к шагу 2
\item \(x^* = \bar{x} = \frac{a_{(n)} + b_{(n)}}{2}\) \\
\(f^* \approx f(\bar{x})\)
\end{enumerate}
\subsection{Метод Фибоначчи}
\label{sec:org51c9cef}
\[ F_{n + 2} = F_{n + 1} + F_n\quad, n = 1,\ F_1 = F_2 = 1 \]
\[ F_n = \left(\left(\frac{1 + \sqrt{5}}{2}\right)^n - \left(\frac{1 - \sqrt{5}}{2}\right)^n\right)\cdot\frac{1}{\sqrt{5}} \]
\[ F_n \approx \left(\frac{1 + \sqrt{5}}{2}\right)^n \cdot \frac{1}{\sqrt{5}} \quad n \to \infty \]
Итерация 0:
\begin{itemize}
\item \(x_1 = a + \frac{F_n}{F_{n + 2}} (b - a)\)
\item \(x_2 = a + \frac{F_{n + 1}}{F_{n + 2}}(b - a) = a + b - x_1\)
\end{itemize}
Итерация \(k\):
\begin{itemize}
\item \[ x_1 = a_{(k)} + \frac{F_{n - k + 1}}{F_{n - k + 3}}(b_k - a_k) = a_k + \frac{F_{n -k + 1}}{F_{n + 2}}(b_0 - a_0) \]
\item \[ x_2 = a_{(k)} + \frac{F_{n - k + 2}}{F_{n - k + 3}}(b_k - a_k) = a_k + \frac{F_{n -k + 2}}{F_{n + 2}}(b_0 - a_0) \]
\end{itemize}
Итерация \(n\):
\begin{itemize}
\item \(x_1 = a_n + \frac{F_1}{F_{n + 1}}(b_0 - a_0)\)
\item \(x_2 = a_n + \frac{F_2}{F_{n + 2}}(b_0 - a_0)\)
\end{itemize}
\[ \frac{b_0 - a_0}{2} = \frac{b_0 - a_0}{F_{n + 2}} < \varepsilon \]
Как выбирать \(n\):
\[ \frac{b_0 - a_0}{\varepsilon} < F_{n + 2} \]
Когда \(n\) большое \(\Rightarrow\) \(\frac{F_n}{F_{n + 2}}\) --- бесконечная десятичная дробь
\subsection{Метод парабол}
\label{sec:org9554841}
\begin{itemize}
\item \(x_1, x_2, x_3 \in [a, b]\)
\item \(x_1 < x_2 < x_3\)
\item \(f(x_1) \ge f(x_2) \le f(x_3)\)
\end{itemize}
\[ q(x) = a_0 + a_1(x - x_1) + a_2(x - x_1)(x - x_2) \]
\begin{itemize}
\item \(q(x_1) = f(x_1) = f_1\)
\item \(q(x_2) = f(x_2) = f_2\)
\item \(q(x_3) = f(x_3) = f_3\)
\end{itemize}


\begin{itemize}
\item \(a_0 = f_1\)
\item \(a_1 = \frac{f_2 - f_1}{x_2 - x_1}\)
\item \(a_2 = \frac{1}{x_3 - x_2}\left(\frac{f_3 - f_1}{x_3 - x_1} - \frac{f_2 - f_1}{x_2 - x_1}\right)\)
\end{itemize}
\[ \bar{x} = \frac{1}{2}\left(x_1 + x_2 - \frac{a_1}{a_2}\right)\text{ --- минимум параболы } q(x) \]
\chapter{}
\label{sec:orgab93cc5}
\[ \frac{l_\text{з.с}^i}{l_\text{дих.}^i} \approx (0.87\dots)^n \] \[
\frac{l_\text{з.с}^i}{l_\text{фиб.}^i} \approx 1.17 \]

\section{Одномерная оптимизация}
\label{sec:org805c4f2}
\subsection{Определение интервала неопределенности}
\label{sec:orgfc9cd3d}
\(x_0\)
\begin{enumerate}
\item Если \(f(x_0) > f(x_0 + \delta)\), то:
\begin{itemize}
\item \(k = 1\)
\item \(x_1 = x_0 + \delta\)
\item \(h = \delta\)
\end{itemize}
иначе если \(f(x_0) > f(x_0) - \delta\), то:
\begin{itemize}
\item \(x_1 = x_0 - \delta\)
\item \(h = -\delta\)
\end{itemize}
\item Удваиваем \(h\):
\begin{itemize}
\item \(h = 2h\)
\item \(x_{k + 1} = x_k + h\)
\end{itemize}
\item Если \(f(x_k) > f(x_{k + 1})\), то:
\begin{itemize}
\item \(k = k + 1\)
\item переходим к шагу 2
\end{itemize}
Иначе:
\begin{itemize}
\item прекращаем поиск \([x_{k - 1}, x_{k + 1}]\)
\end{itemize}
\end{enumerate}
\section{Методы с использованием производной}
\label{sec:org1a9575c}

\begin{itemize}
\item \(f(x)\) --- дифференцируемая или дважды дифференцируемая выпуклая функция
\item вычисление производных в заданых точках
\end{itemize}

\(f'(x) = 0\) --- необходимое и достаточное условие глобального
минимума.  Если \(x^* \in [a, b]\ f'(x) \approx 0\) или \(f'(x) \le
\varepsilon\) --- условие остановки вычислений
\subsection{Метод средней точки}
\label{sec:orgeabf81f}
\(f'(x)\quad \bar{x} = \frac{a + b}{2}\) \\
\begin{itemize}
\item Если \(f'(\bar{x}) > 0\), то \(\bar{x}\in\) монотонно возрастающая
\(f(x)\), минимум на \([a, \bar{x}]\)
\item Если \(f'(x) < 0\) минимум на \([\bar{x}, b]\)
\item Если \(f'(x) = 0\) то \(x^* = x\)
\end{itemize}

\textbf{Алгоритм}
\begin{enumerate}
\item \(\bar{x} = \frac{a + b}{2}\), вычислим \(f'(\bar{x})\) \(\rightarrow\) шаг
2
\item Если \(|f'(x)| \le \varepsilon\), то \(x^* = \bar{x}\) и \(f(x^*) =
   f(\bar{x})\) \(\rightarrow\) завершить
\item Сравнить \(f'(\bar{x})\) с нулем:
\begin{itemize}
\item Если \(f'(x) > 0\), то \([a, \bar{x}], b = \bar{x}\)
\item Иначе \([\bar{x}, b], a=\bar{x}\)
\end{itemize}
\(\rightarrow\) шаг 1
\end{enumerate}
\[ \Delta_n = \frac{b - a}{2^n} \]
\subsection{Метод хорд(метод секущей)}
\label{sec:orgd6350c3}
Если на концах \([a, b]\) \(f'(x)\): \(f'(a)\cdot f'(b) < 0\) и непрерывна,
то на \((a, b)\) \(\exists x\ f'(x) = 0\) \\
\(f(x)\) --- минимум на \([a, b]\), если \(f'(x) = 0\), \(x\in(a, b)\) \\
\(F(x) = f'(x) = 0\) на \([a, b]\) \\
\(F(a)\cdot F(b) < 0\), \(\bar{x}\) --- точка пересечения \(F(x)\) с осью \(Ox\) на \([a, b]\)
\[ \bar{x} = a - \frac{f'(a)}{f'(a) - f'(b)}(a - b) \label{hord_1_4}\addtag\]
\(x^* \in [a, \tilde{x}]\) либо \([\tilde{x}, b]\)

\textbf{Алгоритм}
\begin{enumerate}
\item \(\tilde{x}\) --- вычислим по \ref{hord_1_4} \\
вычислим \(f'(\tilde{x})\) \(\to\) шаг 2
\item Если \(|f'(\tilde{x})|\le\varepsilon\), то:
\begin{itemize}
\item \(x^* = \tilde{x}\)
\item \(f^* = f(\tilde{x})\)
\item завершить
\end{itemize}
Иначе:
\begin{itemize}
\item \(\to\) шаг 3
\end{itemize}
\item Переход к новому отрезку. Если \(f'(\tilde{x}) > 0\), то:
\begin{itemize}
\item \([a, \tilde{x}]\)
\item \(b = \tilde{x}\)
\item \(f'(b) = f'(\tilde{x})\)
\end{itemize}
Иначе:
\begin{itemize}
\item \([\tilde{x}, b]\)
\item \(a = \tilde{x}\)
\item \(f'(a) = f'(\tilde{x})\)
\end{itemize}
\(\to\) шаг 1
\end{enumerate}
\textbf{Исключение}.
\begin{enumerate}
\item \(f'(a)\cdot f'(b) > 0\), \(f(x)\) --- возрастает
\begin{itemize}
\item \(x^* = a\)
\item \(x^* = b\)
\end{itemize}
\item \(f'(a)\cdot f'(b)\), \textbf{одно из}:
\begin{itemize}
\item \(x^* = a\)
\item \(x^* = b\)
\end{itemize}
\end{enumerate}
\subsection{Метод Ньютона(метод касательной)}
\label{sec:orgf2fcff4}
Если выпуклая на \([a, b]\) функция \(f(x)\) --- дважды непрерывно
дифференцируема, то \(x^* \in [a, b]:\ f'(x) = 0\) \\
Пусть \(x_0 \in [a, b]\) --- начальное приближение к \(x^*\)
\[ F(x) = f'(x)\text{ --- линеаризуем в корестнтсти } x_0 \]
(x\textsubscript{0}, f'(x\textsubscript{0})), то есть:
\[ F(x) \approx F(x_0) + F'(x_0)(x - x_0) \]
\(x_1\) ---
\begin{itemize}
\item следующее приближение к \(x^*\)
\item пересечение касательной с \(Ox\)
\end{itemize}

При \(x = x_1\):
\[ F(x_0) + F'(x_0)(x_1 - x_0) = 0 \]
\[ x_1 = x_0 = \frac{F(x_0)}{F'(x_0)} \]
\(\{x_k\},\ k = 1, 2, \dots\) --- итерационная последовательность \\
\(F(x)\) в точке \(x = x_k\) имеет вид:
\[ y = F(x_k) + F'(x_k)(x - x_k) \]
\(x = x_{k + 1}\ y = 0\):
\[ x_{k+1} = x_k - \frac{F(x_k)}{F'(x_k)} \]
\[ f'(x) = 0 \Rightarrpw x_{k + 1} = x_k - \frac{f'(x_k)}{f''(x_k)}\quad,k=1,2,\dots \]
Итерационный процесс: \(|f'(x_k)| \le \varepsilon\):
\begin{itemize}
\item \(x^* \approx x\)
\item \(x^* \approx f(x_k)\)
\end{itemize}
\chapter{}
\label{sec:orgf8c3755}
\section{Метод Ньютона(продолжение). Вывод через ряд Тейлора}
\label{sec:orgb928a71}
\begin{itemize}
\item \(x_k\) --- текущая оценка решения \(x^*\)
\end{itemize}
\[ f(x_k + p) = f(x_k) + pf'(x_k) + \frac{1}{2!}p^2f''(x_k) + \dots \]
\[ f(x^*) = \min_x f(x) = \min_p f(x_k + p) = \min_p [ f(x_k) + pf'(x_k) + \frac{1}{2}p^2 f''(x_k)  + \dots ] \approx \]
\[ \approx \min_p[f(x_k) + pf'(x_k) + \frac{1}{2}p^2 f''(x_k)] \]
\[ f'(x_k) + pf''(x_k) = 0 \]
\[ p = -\frac{f'(x_k)}{f''(x_k)} \]
\(p\) --- аппроксимация шага: от \(x_k \to x^*\). \(x^* \approx x_k + p\)
\[ x_{k + 1} = x_k + p = x_k - \frac{f'(x_k)}{f''(x_k)} \label{nuton_1_5}\addtag \]
Главное преимущество метода Ньютона:
\begin{itemize}
\item высокая(квадратичная) скорость сходимости
\begin{itemize}
\item если \(x_k\) достаточно близка \(x^*\) и если \(f''(x^*) > 0\), то:
\[ |x_{k+1} - x^*| \le \beta|x_k - x^*|^2\quad, \beta = \const > 0 \]
\end{itemize}
\end{itemize}

\noindentНеудачи в методе Ньютона:
\begin{enumerate}
\item \(f(x)\) плохо аппроксимируется первыми тремя членами в ряде Тейлора. \(x_{k+1}\) может быть хуже \(x_k\)
\item \(p=-\frac{f'(x_k)}{f''(x_k)}\) определено только тогда, когда \(f''(x_k)\neq0\) \\
\(f''(x_k) > 0\) --- условие минимума квадратичной аппроксимации \\
Если \(f''(x_k) < 0\) --- алгоритм сходится к максимуму
\item Кроме \(f(x)\) нужно вычислять \(f'(x)\) и \(f''(x)\), что в реальных задачах затруднительно
\end{enumerate}
\subsection{Аппроксимация производных}
\label{sec:org7837d5e}
Правая разностная схема:
\[ f'(x_k) \approx \frac{f(x_k + h) - f(x_k)}{h}\quad ,h \sim \varepsilon \]
Центральная разностная схема:
\[ f'(x_k) \approx \frac{f(x_k + h) - f(x_k - h)}{2h} \]
порядок точности --- \(O(h^2)\)
\subsection{Метод Ньютона(продолжение)}
\label{sec:orgcade1bd}
Если \(f(x)\) --- квадратичная функция, то \(f'(x)\) --- линейная \\
В \ref{nuton_1_5} точное равенство, и следовательно метод Ньютона
сходится за один шаг, при любом выборе \(x\) \\
Пусть \(x^* \in [a, b]\) и \(f(x)\) --- трижды непрерывно дифференцируемая и выпуклая на \([a, b]\) функция. \\
\(\{x_k\}\) будет сходится к пределе \(x^*\) монотонно, если:
\[ 0<\frac{x^* - x_{k+1}}{x^* - x_k} < 1 \]
\[ f'(x^*) = 0 = f'(x_k) + f''(x_k(x^* - x_k)) + \frac{f'''(x_k)}{2}(x^* - x_k)^2 \]
\[ \frac{x^* - x_{k + 1}}{x^* - x_k} = \frac{x^* - x_k + \frac{f'(x_k)}{f''(x_k)}}{x^* - x_k} = 1 - \frac{2}{2 + \frac{f'''(x)(x^* - x_k)^2}{f'(x_k)}}\]
Итерацияонная поледовательность \(\{x_k\}\) монотонна, если \(\frac{f'''(x)}{f'(x_k)} > 0\),
то есть достаточное условие \color{red}\dots{}\color{black}

\begin{examp}
\[ f(x) = x\cdot\arctg(x) - \frac{1}{2}\ln(1 + x^2) \]
, пусть \(|f'(x)| \le 10^{-7}\)
\begin{minted}[frame=lines,linenos=true,mathescape]{python}
from sympy import *
\end{minted}

\[ f'''(x) = -\frac{2x}{(1 + x^2)^2} \]
\[ f'(x)\cdotf''(x) < 0 \]
Выбор начального приближение \(x_0 = 1\)
\[ x_{k + 1} = x_k - \frac{f'(x_k)}{f''(x_k)} \]
\begin{center}
\begin{tabular}{r|rrl|}
\(k\) & \(x_k\) & \(f'(x_k)\) & \(f''(x_k)\)\\
\hline
0 & 1 & 0.785 & \(\frac{1}{2}\)\\
1 & -0.57 & -0.518 & 0.754\\
2 & 0.117 & 0.116 & \dots{}\\
3 & \dots{} & \dots{} & \dots{}\\
4 & \(9\cdot 10^{-8}\) & \(9\cdot10^{-8}\) & \dots{}\\
\end{tabular}
\end{center}
Выолнилось условие \(|f'(x_k)| \le 10^{-7}\) --- окончание итерационного процесса. \(x \approx 9\cdot 10^{-8} \approx 0\)
\end{examp}
\subsection{Модификации метода Ньютона}
\label{sec:org503f458}
\begin{enumerate}
\item Метод Ньютона-Рафсона
\label{sec:org09c3ed4}
\[ x_{k + 1} = x_k - \tau_k\frac{f'(x_k)}{f''(x_k)}\quad, 0 < \tau_k \le 1 \]
\(\tau_k = \tau = \const\) (\(\tau = 1\) --- метод Ньютона)
\[ \varphi(\tau) = f(x_k - \tau\frac{f'(x_k)}{f''(x_k)}) \to \min \]
\[ \tau_k = \frac{(f'(x_k))^2}{(f'(x_k))^2 + (f'(\tilde{x}_k))^2} \]
, где \(\tilde{x} = x_k - \frac{f'(x_k)}{f''(x_k)}\)
\item Метод Маркрафта
\label{sec:org9eda35b}
\[ x_{k + 1} = x_k - \frac{f'(x_k)}{f''(x_k) + \mu_k}\quad,\mu_k > 0 \]
\(\mu_0\) рекомендуется выбирать на порядок больше значения второй произвдной в \(x_0\) \\
\(\mu_{k + 1}\): \(\mu_{k + 1} = \frac{\mu_k}{2}\), если \(f(x_{k + 1}) < f(x_k)\), иначе \(\mu_{k + 1} = 2\cdot\mu_k\)
\end{enumerate}
\subsection{{\bfseries\sffamily TODO} Метод минимизации многомодальных функций}
\label{sec:org6613d7d}
\begin{enumerate}
\item Метод ломанных
\label{sec:orgecf7fad}
Условие Липшица: \(f(x),\ x\in [a, b]\) будет удовлетворять условию, если:
\[ |f(x_1) - f(x_2)| \le L|x_1 - x_2|\quad,\forall x_1, x_2 \in [a, b] \]
\end{enumerate}
\chapter{}
\label{sec:orgb8818ff}
\section{Постановка задачи}
\label{sec:org702a7f0}
\begin{enumerate}
\item \(x^* = (x_1, x_2, \dots, x_n)^T,\ x_i \in U \subset E_n\), где \(U\) --- множество допустимых значений, \(E_n\) --- эвклидово
пространство размера \(n\). \(f(x^*) = \min_{x \in U} f(x)\). Если
ствится задача найти максимум, то млжно перейти к поиску минимума: \(f(x^*) = \max_{x\in U}f(x) = -\min_{x \in U}(-f(x))\)
\item \(f(x^*) = \text{extr}_{x \in U}f(x)\)
\item Если \(U\) задается ограничением на вектор \(x\), то задача поиска
условного экстремума. Если \(U = E_n\) --- не имеет ограничений,
то задача поиска безусловного экстремума
\item Решение задачи поиска экстремума --- пара \((x^*, f(x^*))\)
\end{enumerate}

\noindent\rule{\textwidth}{0.5pt}
Если \(\forall x \in U\ f(x^*) \le f(x)\) --- то \(x^*\) --- глобальный минимум. Локальный минимум \(x^* \in U\): если \(\exists \varepsilon > 0\), что \(\forall x \in U\) и \(\Vert x - x^* \Vert < \varepsilon\), то \(f(x^*) \le f(x)\)
\begin{definition}
\textbf{Поверхностью уровня} функции \(f(x)\) называется множество точек, в которых функция принимает постоянные значения, т.е. \(f(x) = \const\)
\end{definition}
\begin{definition}
\textbf{Градиентом} \(\nabla\) f(x) непрерывно жифференцируемой функции \(f(x)\) в x:
\[ \nabla f(x) = \left(\begin{array}{c} \frac{\partial f(x)}{\partial x_1} \\ \frac{\partial f(x)}{\partial x_2} \\ \vdots \\ \frac{\partial f(x)}{\partial x_n}\end{array}\right) \]
Градиент направлен по нормали к поверхности уровня, т.е. перпендикулярно к касательной плоскости в точке \(x\), проведенной в сторону наибольшего возрастания функции
\end{definition}
\begin{definition}
\textbf{Матрицей Гессе} \(H(x)\) дважды непрерывно дифференцируемой в точке
\(x\) функции \(f(x)\) называется матрица частных производных второго
порядка, вычисленных в данной точке.
\[ H(x) = \left( \begin{array}{cccc} \frac{\partial^2 f(x)}{\partial x_1^2} & \frac{\partial^2 f(x)}{\partial x_1x_2} & \dots & \frac{\partial^2 f(x)}{\partial x_1x_n} \\ \vdots & \vdots & \ddots & \vdots \\ \frac{\partial^2 f(x)}{\partial x_nx_1} & \frac{\partial^2 f(x)}{\partial x_nx_2} & \dots & \frac{\partial^2 f(x)}{\partial x_n^2} \end{array}\right) \]
\end{definition}
\begin{enumerate}
\item \(H(x)\) --- симметричная, размер \(n \time n\)
\item Антиградиент: вектор, равный по модулю вектору градиента, но противоположный по направлению. Указывает в сторну наибольшего убывания функции \(f(x)\)
\item \[ \nabla f(x) = f(x + \Delta x) - f(x) = \nabla f(x)^T\Delta x + \frac{1}{2} \Delta x^T H(x)\Delta x + o(\Vert \Delta x \Vert^2) \]
\(o(\Vert \Delta x \Vert^2)\) --- сумма всех членов разложения, имеющих порядок выше второго, \(\Delta x^T H(x) \Delta x\) --- квадратичная форма
\end{enumerate}
\subsection{Свойства квадратичных форм}
\label{sec:orgc6a8a56}
Квадратичная форма \(\Delta x^T H(x) \Delta x\) (и соответсвующая матрица \(H(x)\)) называется:
\begin{itemize}
\item положительно опрделенной \(H(x) > 0\), если \(\forall \Delta x \neq 0\ \Delta x^T H(x) \Delta x > 0\)
\item отрицательно определенной \(H(x) < 0\), если \(\forall \Delta x \neq 0\ \Delta x^T H(x) \Delta x < 0\)
\item положительно полуопределенной \(H(x) \ge 0\), если \(\forall \Delta x \neq 0\ \Delta x^T H(x) \Delta x \ge 0\)
и имеется \(\Delta x \neq 0: \Delta x^T H(x) \Delta x = 0\)
\item отрицательно полуопределенной \(H(x) \le 0\), если \(\forall \Delta x \neq 0\ \Delta x^T H(x) \Delta x \le 0\)
и имеется \(\Delta x \neq 0: \Delta x^T H(x) \Delta x = 0\)
\item неопределнной, если \(\exists \Delta x, \Delta \tilde{x}: \Delta x^T H(x) \Delta x > 0,\ \Delta \tilde{x}^T H(\tilde{x}) \Delta \tilde{x} < 0\)
\item тождественно равной нулю \(H(x) \equiv 0\), если \(\forall \Delta x\ \Delta x^T H(x) \Delta x = 0\)
\end{itemize}
\subsection{Свойства выпуклых множеств и выпуклых функций}
\label{sec:orgb168881}
\begin{definition}
Пусть \(x, y \in E_n\). Множество точек вида \(\{z\} \subset E_n: z = \alpha x + (1 - \alpha)y\), \(\alpha \in [0, 1]\), \(z\) --- отрезок, соединяющий \(x\) и \(y\).
\end{definition}
\begin{examp}
\(E_n: n \le 3\): \(z\) --- отрезок(обычный)
\end{examp}
\begin{definition}
\(U \subset E_n\) выпуклое, если вместе с точками \(x\) и (y \(\in\) U) оно содержит и весь отрезок \(z = \alpha x + (1 - \alpha)y, \alpha \in [0, 1]\)
\end{definition}
\begin{definition}
Функция \(f(x)\), заданая на выпуклом \(U \subset E_n\) называется:
\begin{itemize}
\item выпуклой, если \(\forall x, y \in U\) и \(\forall \alpha [0, 1]\) выполняется \(f(\alpha x + (1 - \alpha)y) \le \alpha f(x) + (1- \alpha)f(y)\)
\item строго выпуклой, если \(\forall \alpha \in (0, 1)\) выполняется \(f(\alpha x + (1 - \alpha)y) < \alpha f(x) + (1-\alpha)f(y)\)
\item сильно выпуклой с константой \(l > 0\), если \(\forall x, y \in U\) и \(\forall \alpha \in [0, 1]\) выполняется \(f(\alpha x + (1 - \alpha)y) \le \alpha f(x) + (1- \alpha) f(y) - \frac{l}{2}\alpha(1 - \alpha)\Vert x - y \Vert^2\)
\end{itemize}
\end{definition}
\emph{Свойства:}
\begin{enumerate}
\item Функция \(f(x)\) \uline{выпуклая}, если ее грфик целиком лежит не выше отрезка, соединяющего две ее произвольные точки \\
Функция \(f(x)\) \uline{строго выпуклая}, если ее график лежит целиком ниже отрезка, соединяющего две ее произвольные, но не совпадающие точки
\item Если функция \(f(x)\) \uline{сильно выпуклая}, то она одноверменно строго выпуклая и выпуклая \\
Если функция \(f(x)\) \uline{строго выпуклая}, то она одновременно выпуклая
\item Выпуклость функции можно определить по матрице Гессе \(H(x)\)
\begin{itemize}
\item Если \(H(x) \ge 0\ \forall x \in E_n\), то \(f(x)\) выпуклая
\item Если \(H(x) > 0\ \forall x\in E_n\), то \(f(x)\) строго выпуклая
\item Если \(H(x)\ \ge lE\ \forall x \in E_n\), где \(E\) --- единичная матрица, то \(f(x)\) сильно выпуклая
\end{itemize}
\end{enumerate}
\emph{Свойства выпуклых функций:}
\begin{enumerate}
\item Если f(x) выпуклая функция на выпуклом множестве \(U\), то всякая точка локального минимума есть точка глобального минимума на \(U\)
\item Если выпуклая функция достигает своего минимума в двух различных точках, то она достигает миниума во всех точках отрезка, соединяющих это точки.
\item Если \(f(x)\) строго выпуклая функция множества \(U\), то она может достигать своего глобального минимума на \(U\) не более чем в одной точке
\end{enumerate}
\subsection{Необходимое и достаточное условие безусловного экстремума}
\label{sec:org959927c}
\begin{theorem}[Необходимое условие экстремума первого порядка]
Пусть \(x^* \in E_n\) --- локальный минимум или максимум \(f(x)\) на \(E_n\) и \(f(x)\) --- дифференцируема в точке \(x^*\) \\
\uline{Тогда} \(\nabla f(x)\) в точке \(x^*\) равен нулю \(\nabla f(x^*) = 0\), т.е.
\[ \frac{\partial f(x^*)}{\partial x_i} = 0,\ i = \overline{1, n} \]
\end{theorem}
\begin{definition}
Точки \(x^*: \nabla f(x^*) = 0\) --- \textbf{стационарные}
\end{definition}
\begin{theorem}[Необходимое условие экстремума второго порядка]
Пусть \(x^* \in E_n\) --- точка локального минимума или максимума \(f(x)\) на \(E_n\) и \(f(x)\) --- дважды дииференцируемая в точке. \\
\uline{Тогда} \(H(x^*)\) --- является положительно или отрицательно(если максимум) полуопределенной, т.е. \(H(x^*) \ge 0\) или \(H(x^*) \le 0\)(если максимум)
\end{theorem}
\begin{theorem}[Достаточное условие экстремума]
Пусть \(f(x)\) в \(x^* \in E_n\) дважды дифференцируема, ее \(\nabla f(x) = 0\), а \(H(x^*) > 0\) или \(H(x^*) < 0\)(для максимума). \\
\uline{Тогда} \(x^*\) --- точка локального минимума(максимума) \(f(x)\) на \(E_n\)
\end{theorem}
\begin{enumerate}
\item Проверка выполнения условий
\label{sec:orgf05c286}
\begin{itemize}
\item вычисление угловых миноров \(H(x)\)
\item вычисление главных миноров \(H(x)\) \\
\end{itemize}


\begin{enumerate}
\item Ислледование положительной или отрицательной определнности угловых и главных миноров
\item Анализ собственных значений матрицы \(H(x)\)
\end{enumerate}
\end{enumerate}
\chapter{}
\label{sec:org95ed7a4}
\newcommand{\diff}[2]{\frac{\partial #1}{\partial #2}}


\section{Критерии Сильвестра}
\label{sec:org08785f5}
\subsection{Достаточный условия}
\label{sec:org49b552b}
\begin{enumerate}
\item \(H(x^*) > 0\) и \(x^*\) --- локальный минимум \(\Leftrightarrow\) \(\Delta_1 > 0, \Delta_2 > 0, \dots , \Delta_n > 0\)
\item \(H(x^*) < 0\) и \(x^*\) --- локальный максимум \(\Leftrightarrow\) \(\Delta_1 < 0, \Delta_2 > 0, \dots , (-1)^n\Delta_n > 0\)
\end{enumerate}
, где \(\Delta_i\) --- угловой минор
\subsection{Необходимые условия}
\label{sec:org2606857}
\begin{enumerate}
\item \(H(x^*) \ge 0\) и \(x^*\) --- может быть локальный минимум \(\Leftrightarrow\) \(\Delta_1 \ge 0, \Delta_2 \ge 0, \dots, \Delta_n \ge 0\)
\item \(H(x^*) \le 0\) и \(x^*\) --- может быть локальный максимум \(\Leftrightarrow\) \(\Delta_1 \le 0, \Delta_2 \ge 0, \dots, (-1)^n\Delta_n \ge 0\)
\end{enumerate}
, где \(\Delta_i\) --- главный минор

\section{Собственные значения}
\label{sec:org0ae451c}
\begin{definition}
\textbf{Собственные значения} \(\lambda_i\ (i = 1..n)\) \(H(x^*)_{n\times n}\) находятся как корни характеристического уравнения \(|H(x^*) - \lambda E| = 0\). Если \(H(x)\) --- вещественная, симметричная матрица, то \(\lambda_i\) --- вещественные
\end{definition}
\section{Общие прицнипы многмерной оптимизации}
\label{sec:org8356b55}
\subsection{Выпуклые квадратичные функции}
\label{sec:org2dfca3e}
\[ f(x) = \frac{1}{2}ax^2 + bx + c \]
\begin{definition}
Функция вида
\[ f(x) = \sum^n{i = 1}\sum^n_{j = 1}a_{ij}x_ix_j + \sum^n_{j = 1}b_j x_j + c \addtag\label{7_1_quad} \]
Называется \textbf{квадратичной функией \(n\) перменных}
\end{definition}
Положим \(a_{ij} = a_{ij} + a_{ji} {\color{red}??}\) \(\Rightarrow\) симметрия. матрица \(A\)
\[ f(x) = \frac{1}{2}(Ax, x) + (b, x) + c \]
, где \(b = (b_1, \dots b_n)^T \in E_n\) --- вектор коэффицентов, \(x = (x_1, \dots, x_n)^T\). \(x, y\) --- скалярное произведение
Свойства квадратичных функций:
\begin{enumerate}
\item \(\nabla f(x) = Ax + b\)
\[ \diff{f}{x_k} = \diff{}{x_k}\left(\frac{1}{2}\sum^n_{i = 1}\sum^n_{j = 1}a_{ij}x_ix_j + \sum^n_{j = 1}b_j x_j + c\right) = \]
\[ \frac{1}{2}\sum^n_{i = 1}(a_{ik} + a_{ki})x_i + b_k = \sum^n_{i = 1} a_{ki}x_i + b_k \]
\item \(H(x) = A\), где \(H(x)\) --- Гессиан\(\color{red}???\) 
\[ \diff{^2 f}{x_l \partial x_k} = \diff{}{x_k}\left(\diff{f}{x_k}\right) = \diff{}{x_l}\left(\sum^n_{i = 1} a_{ki} x_i + b_k\right) \]
\item Квадратичная функция \(f(x)\) с положительно определенной матрицей \(A\) сильно выпукла
\[ A = \begin{vmatrix} \lambda_1 & 0 & \dots & 0 \\ 0 & \lambda_2 & \dots & 0 \\ \vdots & \vdots & \ddots & \vdots \\ 0 & 0 & \dots & \lambda_n \end{vmatrix} \]
\[ A - lE = \begin{vmatrix} \lambda_1 - l & 0 & \dots & 0 \\ 0 & \lambda_2 - l & \dots & 0 \\ \vdots & \vdots & \ddots & \vdots \\ 0 & 0 & \dots & \lambda_n - l \end{vmatrix} \]
В этом базисе все угловые миноры матрцы \(A\) и матрицы \(A - lE\) --- положительны при достаточно малом \(l: 0 < l < \lambda_\min \Rightarrow f(x)\) --- сильно выпукла
\end{enumerate}
\subsection{Принципы многмерной оптимизации}
\label{sec:org91a6768}
\[ f(x) \to \min,\ x \in E_n \]
\[ x^{k + 1} = \Phi(x^k, x^{k + 1}, \dots x)^0,\ x^0 \in E_n \addtag\label{5_3_iter} \]
--- итериционная процедура(общего вида)
\begin{description}
\item[{\(\{x^k\}\):}] \[ \lim_{k \to \infty} f(x^k) = f^* = \min_{E_n} f(x), \text{ если } U^* \neq \emptyset \]
\[ \lim_{k \to \infty} f(x^k) = f^* = \inf_{E_n} f(x), \text{ если } U^* = \emptyset \]
\end{description}
, где \(U^*\) -- множестве точек глобального минимума функции \(f(x)\) \\
\(\{x^k\} +\) условие \ref{5_3_iter} = минимизирующая последовательность для \(f(x)\) \\
Если для \(U^* \neq \emptyset\) выполняется условие
\[ \lim_{k \to \infty} \rho(x^k, U^*) = 0 \], то \(x^k\) сходится к множеству \(U^*\). Если \(U^*\) содежит единственную точку \(x^*\), то для \(\{x^k\}\) сходящейся к \(U^*\) будет справедливо \(\lim_{k \to \infty} x^k = x^*\)
\begin{definition}
\(\rho(x, U) = \inf_{y \in U}\rho(x, y)\) --- растояние от точки \(x\) до множества \(U\)
\end{definition}
\begin{remark}
Минимизирующая последовательность \(\{x^k\}\) может и не сходится к точке минимума
\end{remark}
\begin{theorem}[Вейерштрасса]
Если \(f(x)\) непрерывна в \(E_n\) и множество \(U^\alpha = {x: f(x) \le \alpha}\) для некоторого \(\alpha\) непусто и ограничено, то \(f(x)\) достигает глобального минимума в \(E_n\)
\end{theorem}
\begin{enumerate}
\item Скорость сходимости(минизирующих последовательностей)
\label{sec:org0fcfbbd}
\begin{definition}
\(\{x^k\}\) сходится к точке \(x^*\) \textbf{линейно} (со скоростью геометрической последовательности), если \(\exists q \in (0, 1):\)
\[ \rho(x^k, x^*) \le q \rho(x^{k - 1}, x^*) \addtag\label{5_5_linear}\]
\[ \rho(x^k, x^*) \le q^k \rho(x^0, x^*) \]
\end{definition}
\begin{definition}
Сходимость называется \textbf{сверхлинейной} если
\[ \rho(x^k, x^*) \le q_k \rho(x^{k - 1}, x^*) \], и \(q_k \xrightarrow[k \to \infty]{} +0\)
\end{definition}
\begin{definition}
\textbf{Квадратичная сходимость}:
\[ \rho(x^k, x^*) \le \left[ c \rho(x^{k - 1}, x^*)\right]^2,\ c > 0 \]
\end{definition}
\item Критерии окончания итерационного процесса
\label{sec:org6abef15}
\[ \rho(x^{k + 1}, x^*) < \varepsilon_1 \]
\[ |f(x^{k + 1}) - f(x^k)| < \varepsilon_2 \addtag\label{5_6_eps2}\]
\[ \Vert \nabla f(x^k) \Vert < \varepsilon_3 \]
, где \(\varepsilon_i\) --- заранее заданные точности
\[ x^{k + 1} = x^k + \alpha_k p^k,\ k=0, 1, \dots \addtag\label{5_7_iter}\]
, где \(p^k\) --- направление поиска из \(x^k\) в \(x^{k + 1}\), \(\alpha_k\) --- величина шага
\[ f(x^{k + 1}) < f(x^k) \] --- условие выбора \(\alpha_k\)
\begin{definition}
В итерационном процессе \ref{5_7_iter} производится \textbf{исчерпывающий спуск}, если величина шага \(\alpha_k\) находится из решения одномерной задачи минизации:
\[ \Phi_k(\alpha) \to \min_\alpha,\ \Phi_k(\alpha) = f(x^k + \alpha p^k) \addtag\label{5_8_cond}\]
\end{definition}
\begin{theorem}
Если функция \(f(x)\) дифференцируема в пространстве \(E_n\), то в итерационном процессе \ref{5_7_iter} c выбором шага с ичерпывающим спуском для любого \(k \ge 1\):
\[ (\nabla f(x^{k + 1}), p^k) = 0 \addtag\label{5_9_orto}\]
--- это значит что эти два вектора ортогональны
\end{theorem}
\noindentдля \(\Phi_k(\alpha)\) необходимое условие минимума функции:
\[ \frac{d\Phi_k(\alpha)}{d \alpha} = \sum^n_{j = 1} \diff{f(x^{k + 1})}{x_j} \cdot \frac{d x_j^{k + 1}}{d \alpha} = 0 \]
учитывая \(x_j^{k + 1} = x_j^k + \alpha p_j^k \Rightarrow \frac{dx^k_j}{d\alpha} = p_j^k\)

\begin{theorem}
Для квадратичной функции \(f(x) = \frac{1}{2}(Ax, x) + (b ,x) + c\) величина \(\alpha_k\) исчерпывающего спуска в итерационном процессе
\[ x^{k + 1} = x^k + \alpha_k p^k, \k = 0, 1, \dots \]
равна
\[ \alpha_k = -\frac{(\nabla f(x^k), p^k)}{(A p^k, p^k)} = - \frac{(Ax^k + b, p^k)}{(Ap^k, p^k)} \addtag\label{5_10_alpha}\]
\end{theorem}
\end{enumerate}
\end{document}
