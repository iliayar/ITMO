% Created 2021-04-13 Tue 14:19
% Intended LaTeX compiler: pdflatex

\documentclass[english]{article}
\usepackage[T1, T2A]{fontenc}
\usepackage[lutf8]{luainputenc}
\usepackage[english, russian]{babel}
\usepackage{minted}
\usepackage{graphicx}
\usepackage{longtable}
\usepackage{hyperref}
\usepackage{xcolor}
\usepackage{natbib}
\usepackage{amssymb}
\usepackage{stmaryrd}
\usepackage{amsmath}
\usepackage{caption}
\usepackage{mathtools}
\usepackage{amsthm}
\usepackage{tikz}
\usepackage{grffile}
\usepackage{extarrows}
\usepackage{wrapfig}
\usepackage{algorithm}
\usepackage{algorithmic}
\usepackage{lipsum}
\usepackage{rotating}
\usepackage{placeins}
\usepackage[normalem]{ulem}
\usepackage{amsmath}
\usepackage{textcomp}
\usepackage{capt-of}

\usepackage{geometry}
\geometry{a4paper,left=2.5cm,top=2cm,right=2.5cm,bottom=2cm,marginparsep=7pt, marginparwidth=.6in}
 \usepackage{hyperref}
 \hypersetup{
     colorlinks=true,
     linkcolor=blue,
     filecolor=orange,
     citecolor=black,      
     urlcolor=cyan,
     }

\usetikzlibrary{decorations.markings}
\usetikzlibrary{cd}
\usetikzlibrary{patterns}
\usetikzlibrary{automata, arrows}

\newcommand\addtag{\refstepcounter{equation}\tag{\theequation}}
\newcommand{\eqrefoffset}[1]{\addtocounter{equation}{-#1}(\arabic{equation}\addtocounter{equation}{#1})}
\newcommand{\llb}{\llbracket}
\newcommand{\rrb}{\rrbracket}


\newcommand{\R}{\mathbb{R}}
\renewcommand{\C}{\mathbb{C}}
\newcommand{\N}{\mathbb{N}}
\newcommand{\A}{\mathfrak{A}}
\newcommand{\B}{\mathfrak{B}}
\newcommand{\rank}{\mathop{\rm rank}\nolimits}
\newcommand{\const}{\var{const}}
\newcommand{\grad}{\mathop{\rm grad}\nolimits}

\newcommand{\todo}{{\color{red}\fbox{\text{Доделать}}}}
\newcommand{\fixme}{{\color{red}\fbox{\text{Исправить}}}}

\newcounter{propertycnt}
\setcounter{propertycnt}{1}
\newcommand{\beginproperty}{\setcounter{propertycnt}{1}}

\theoremstyle{plain}
\newtheorem{propertyinner}{Свойство}
\newenvironment{property}{
  \renewcommand\thepropertyinner{\arabic{propertycnt}}
  \propertyinner
}{\endpropertyinner\stepcounter{propertycnt}}
\newtheorem{axiom}{Аксиома}
\newtheorem{lemma}{Лемма}
\newtheorem{manuallemmainner}{Лемма}
\newenvironment{manuallemma}[1]{%
  \renewcommand\themanuallemmainner{#1}%
  \manuallemmainner
}{\endmanuallemmainner}

\theoremstyle{remark}
\newtheorem*{remark}{Примечание}
\newtheorem*{solution}{Решение}
\newtheorem{corollary}{Следствие}[theorem]
\newtheorem*{examp}{Пример}
\newtheorem*{observation}{Наблюдение}

\theoremstyle{definition}
\newtheorem{task}{Задача}
\newtheorem{theorem}{Теорема}[section]
\newtheorem*{definition}{Определение}
\newtheorem*{symb}{Обозначение}
\newtheorem{manualtheoreminner}{Теорема}
\newenvironment{manualtheorem}[1]{%
  \renewcommand\themanualtheoreminner{#1}%
  \manualtheoreminner
}{\endmanualtheoreminner}
\captionsetup{justification=centering,margin=2cm}
\newenvironment{colored}[1]{\color{#1}}{}

\tikzset{->-/.style={decoration={
  markings,
  mark=at position .5 with {\arrow{>}}},postaction={decorate}}}
\makeatletter
\newcommand*{\relrelbarsep}{.386ex}
\newcommand*{\relrelbar}{%
  \mathrel{%
    \mathpalette\@relrelbar\relrelbarsep
  }%
}
\newcommand*{\@relrelbar}[2]{%
  \raise#2\hbox to 0pt{$\m@th#1\relbar$\hss}%
  \lower#2\hbox{$\m@th#1\relbar$}%
}
\providecommand*{\rightrightarrowsfill@}{%
  \arrowfill@\relrelbar\relrelbar\rightrightarrows
}
\providecommand*{\leftleftarrowsfill@}{%
  \arrowfill@\leftleftarrows\relrelbar\relrelbar
}
\providecommand*{\xrightrightarrows}[2][]{%
  \ext@arrow 0359\rightrightarrowsfill@{#1}{#2}%
}
\providecommand*{\xleftleftarrows}[2][]{%
  \ext@arrow 3095\leftleftarrowsfill@{#1}{#2}%
}
\makeatother

\newenvironment{rualgo}[1][]
  {\begin{algorithm}[#1]
     \selectlanguage{russian}%
     \floatname{algorithm}{Алгоритм}%
     \renewcommand{\algorithmicif}{{\color{red}\textbf{если}}}%
     \renewcommand{\algorithmicthen}{{\color{red}\textbf{тогда}}}%
     \renewcommand{\algorithmicelse}{{\color{red}\textbf{иначе}}}%
     \renewcommand{\algorithmicend}{{\color{red}\textbf{конец}}}%
     \renewcommand{\algorithmicfor}{{\color{red}\textbf{для}}}%
     \renewcommand{\algorithmicto}{{\color{red}\textbf{до}}}%
     \renewcommand{\algorithmicdo}{{\color{red}\textbf{делать}}}%
     \renewcommand{\algorithmicwhile}{{\color{red}\textbf{пока}}}%
     \renewcommand{\algorithmicrepeat}{{\color{red}\textbf{повторять}}}%
     \renewcommand{\algorithmicuntil}{{\color{red}\textbf{до тех пор пока}}}%
     \renewcommand{\algorithmicloop}{{\color{red}\textbf{повторять}}}%
     \renewcommand{\algorithmicnot}{{\color{blue}\textbf{не}}}%
     \renewcommand{\algorithmicand}{{\color{blue}\textbf{и}}}%
     \renewcommand{\algorithmicor}{{\color{blue}\textbf{или}}}%
     \renewcommand{\algorithmicrequire}{{\color{blue}\textbf{Предусловие}}}%
     \renewcommand{\algorithmicrensure}{{\color{blue}\textbf{Постусловие}}}%
     \renewcommand{\algorithmicrtrue}{{\color{blue}\textbf{истинна}}}%
     \renewcommand{\algorithmicrfalse}{{\color{blue}\textbf{ложь}}}%
     % Set other language requirements
  }
  {\end{algorithm}}
\author{Ilya Yaroshevskiy}
\date{\today}
\title{Лекция 7}
\hypersetup{
 pdfauthor={Ilya Yaroshevskiy},
 pdftitle={Лекция 7},
 pdfkeywords={},
 pdfsubject={},
 pdfcreator={Emacs 28.0.50 (Org mode 9.4.4)}, 
 pdflang={English}}
\begin{document}

\maketitle
\tableofcontents

\newcommand{\diff}[2]{\frac{\partial #1}{\partial #2}}


\section{Критерии Сильвестра}
\label{sec:org1b9587a}
\subsection{Достаточный условия}
\label{sec:org374fce3}
\begin{enumerate}
\item \(H(x^*) > 0\) и \(x^*\) --- локальный минимум \(\Leftrightarrow\) \(\Delta_1 > 0, \Delta_2 > 0, \dots , \Delta_n > 0\)
\item \(H(x^*) < 0\) и \(x^*\) --- локальный максимум \(\Leftrightarrow\) \(\Delta_1 < 0, \Delta_2 > 0, \dots , (-1)^n\Delta_n > 0\)
\end{enumerate}
, где \(\Delta_i\) --- угловой минор
\subsection{Необходимые условия}
\label{sec:org1465308}
\begin{enumerate}
\item \(H(x^*) \ge 0\) и \(x^*\) --- может быть локальный минимум \(\Leftrightarrow\) \(\Delta_1 \ge 0, \Delta_2 \ge 0, \dots, \Delta_n \ge 0\)
\item \(H(x^*) \le 0\) и \(x^*\) --- может быть локальный максимум \(\Leftrightarrow\) \(\Delta_1 \le 0, \Delta_2 \ge 0, \dots, (-1)^n\Delta_n \ge 0\)
\end{enumerate}
, где \(\Delta_i\) --- главный минор

\section{Собственные значения}
\label{sec:orgf333c37}
\begin{definition}
\textbf{Собственные значения} \(\lambda_i\ (i = 1..n)\) \(H(x^*)_{n\times n}\) находятся как корни характеристического уравнения \(|H(x^*) - \lambda E| = 0\). Если \(H(x)\) --- вещественная, симметричная матрица, то \(\lambda_i\) --- вещественные
\end{definition}
\section{Общие прицнипы многмерной оптимизации}
\label{sec:org11fb971}
\subsection{Выпуклые квадратичные функции}
\label{sec:org0b9b4a5}
\[ f(x) = \frac{1}{2}ax^2 + bx + c \]
\begin{definition}
Функция вида
\[ f(x) = \sum^n_{i = 1}\sum^n_{j = 1}a_{ij}x_ix_j + \sum^n_{j = 1}b_j x_j + c \addtag\label{7_1_quad} \]
Называется \textbf{квадратичной функией \(n\) перменных}
\end{definition}
Положим \(a_{ij} = a_{ij} + a_{ji} {\color{red}??}\) \(\Rightarrow\) симметрия. матрица \(A\)
\[ f(x) = \frac{1}{2}(Ax, x) + (b, x) + c \]
, где \(b = (b_1, \dots b_n)^T \in E_n\) --- вектор коэффицентов, \(x = (x_1, \dots, x_n)^T\). \(x, y\) --- скалярное произведение
Свойства квадратичных функций:
\begin{enumerate}
\item \(\nabla f(x) = Ax + b\)
\[ \diff{f}{x_k} = \diff{}{x_k}\left(\frac{1}{2}\sum^n_{i = 1}\sum^n_{j = 1}a_{ij}x_ix_j + \sum^n_{j = 1}b_j x_j + c\right) = \]
\[ \frac{1}{2}\sum^n_{i = 1}(a_{ik} + a_{ki})x_i + b_k = \sum^n_{i = 1} a_{ki}x_i + b_k \]
\item \(H(x) = A\), где \(H(x)\) --- Гессиан\(\color{red}???\) 
\[ \diff{^2 f}{x_l \partial x_k} = \diff{}{x_k}\left(\diff{f}{x_k}\right) = \diff{}{x_l}\left(\sum^n_{i = 1} a_{ki} x_i + b_k\right) \]
\item Квадратичная функция \(f(x)\) с положительно определенной матрицей \(A\) сильно выпукла
\[ A = \begin{vmatrix} \lambda_1 & 0 & \dots & 0 \\ 0 & \lambda_2 & \dots & 0 \\ \vdots & \vdots & \ddots & \vdots \\ 0 & 0 & \dots & \lambda_n \end{vmatrix} \]
\[ A - lE = \begin{vmatrix} \lambda_1 - l & 0 & \dots & 0 \\ 0 & \lambda_2 - l & \dots & 0 \\ \vdots & \vdots & \ddots & \vdots \\ 0 & 0 & \dots & \lambda_n - l \end{vmatrix} \]
В этом базисе все угловые миноры матрцы \(A\) и матрицы \(A - lE\) --- положительны при достаточно малом \(l: 0 < l < \lambda_\min \Rightarrow f(x)\) --- сильно выпукла
\end{enumerate}
\subsection{Принципы многмерной оптимизации}
\label{sec:orgcedb79d}
\[ f(x) \to \min,\ x \in E_n \]
\[ x^{k + 1} = \Phi(x^k, x^{k + 1}, \dots x)^0,\ x^0 \in E_n \addtag\label{5_3_iter} \]
--- итериционная процедура(общего вида)
\begin{description}
\item[{\(\{x^k\}\):}] \[ \lim_{k \to \infty} f(x^k) = f^* = \min_{E_n} f(x), \text{ если } U^* \neq \emptyset \]
\[ \lim_{k \to \infty} f(x^k) = f^* = \inf_{E_n} f(x), \text{ если } U^* = \emptyset \]
\end{description}
, где \(U^*\) -- множестве точек глобального минимума функции \(f(x)\) \\
\(\{x^k\} +\) условие \ref{5_3_iter} = минимизирующая последовательность для \(f(x)\) \\
Если для \(U^* \neq \emptyset\) выполняется условие
\[ \lim_{k \to \infty} \rho(x^k, U^*) = 0 \], то \(x^k\) сходится к множеству \(U^*\). Если \(U^*\) содежит единственную точку \(x^*\), то для \(\{x^k\}\) сходящейся к \(U^*\) будет справедливо \(\lim_{k \to \infty} x^k = x^*\)
\begin{definition}
\(\rho(x, U) = \inf_{y \in U}\rho(x, y)\) --- растояние от точки \(x\) до множества \(U\)
\end{definition}
\begin{remark}
Минимизирующая последовательность \(\{x^k\}\) может и не сходится к точке минимума
\end{remark}
\begin{theorem}[Вейерштрасса]
Если \(f(x)\) непрерывна в \(E_n\) и множество \(U^\alpha = {x: f(x) \le \alpha}\) для некоторого \(\alpha\) непусто и ограничено, то \(f(x)\) достигает глобального минимума в \(E_n\)
\end{theorem}
\subsubsection{Скорость сходимости(минизирующих последовательностей)}
\label{sec:orgac729ca}
\begin{definition}
\(\{x^k\}\) сходится к точке \(x^*\) \textbf{линейно} (со скоростью геометрической последовательности), если \(\exists q \in (0, 1):\)
\[ \rho(x^k, x^*) \le q \rho(x^{k - 1}, x^*) \addtag\label{5_5_linear}\]
\[ \rho(x^k, x^*) \le q^k \rho(x^0, x^*) \]
\end{definition}
\begin{definition}
Сходимость называется \textbf{сверхлинейной} если
\[ \rho(x^k, x^*) \le q_k \rho(x^{k - 1}, x^*) \], и \(q_k \xrightarrow[k \to \infty]{} +0\)
\end{definition}
\begin{definition}
\textbf{Квадратичная сходимость}:
\[ \rho(x^k, x^*) \le \left[ c \rho(x^{k - 1}, x^*)\right]^2,\ c > 0 \]
\end{definition}
\subsubsection{Критерии окончания итерационного процесса}
\label{sec:orgc02dbb4}
\[ \rho(x^{k + 1}, x^*) < \varepsilon_1 \]
\[ |f(x^{k + 1}) - f(x^k)| < \varepsilon_2 \addtag\label{5_6_eps2}\]
\[ \Vert \nabla f(x^k) \Vert < \varepsilon_3 \]
, где \(\varepsilon_i\) --- заранее заданные точности
\[ x^{k + 1} = x^k + \alpha_k p^k,\ k=0, 1, \dots \addtag\label{5_7_iter}\]
, где \(p^k\) --- направление поиска из \(x^k\) в \(x^{k + 1}\), \(\alpha_k\) --- величина шага
\[ f(x^{k + 1}) < f(x^k) \] --- условие выбора \(\alpha_k\)
\begin{definition}
В итерационном процессе \ref{5_7_iter} производится \textbf{исчерпывающий спуск}, если величина шага \(\alpha_k\) находится из решения одномерной задачи минизации:
\[ \Phi_k(\alpha) \to \min_\alpha,\ \Phi_k(\alpha) = f(x^k + \alpha p^k) \addtag\label{5_8_cond}\]
\end{definition}
\begin{theorem}
Если функция \(f(x)\) дифференцируема в пространстве \(E_n\), то в итерационном процессе \ref{5_7_iter} c выбором шага с ичерпывающим спуском для любого \(k \ge 1\):
\[ (\nabla f(x^{k + 1}), p^k) = 0 \addtag\label{5_9_orto}\]
--- это значит что эти два вектора ортогональны
\end{theorem}
\noindentдля \(\Phi_k(\alpha)\) необходимое условие минимума функции:
\[ \frac{d\Phi_k(\alpha)}{d \alpha} = \sum^n_{j = 1} \diff{f(x^{k + 1})}{x_j} \cdot \frac{d x_j^{k + 1}}{d \alpha} = 0 \]
учитывая \(x_j^{k + 1} = x_j^k + \alpha p_j^k \Rightarrow \frac{dx^k_j}{d\alpha} = p_j^k\)

\begin{theorem}
Для квадратичной функции \(f(x) = \frac{1}{2}(Ax, x) + (b ,x) + c\) величина \(\alpha_k\) исчерпывающего спуска в итерационном процессе
\[ x^{k + 1} = x^k + \alpha_k p^k, \k = 0, 1, \dots \]
равна
\[ \alpha_k = -\frac{(\nabla f(x^k), p^k)}{(A p^k, p^k)} = - \frac{(Ax^k + b, p^k)}{(Ap^k, p^k)} \addtag\label{5_10_alpha}\]
\end{theorem}
\end{document}
