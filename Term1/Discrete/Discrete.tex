\documentclass[12pt, a4paper]{article}

\usepackage{mathtools}
\usepackage{xltxtra}
\usepackage{libertine}
\usepackage{amsmath}
\usepackage{amsthm}
\usepackage{amsfonts}
\usepackage{amssymb}
\usepackage{enumitem}
\usepackage[left=2.3cm, right=2.3cm, top=2.7cm, bottom=2.7cm, bindingoffset=0cm]{geometry}
\usepackage{fancyhdr}

\pagestyle{fancy}
\lfoot{M3137y2019}
\rhead{\thepage}

\DeclareMathOperator*{\xor}{\oplus}
\DeclareMathOperator*{\R}{\mathbb{R}}
\DeclareMathOperator*{\Q}{\mathbb{Q}}
\DeclareMathOperator*{\C}{\mathbb{C}}
\DeclareMathOperator*{\Z}{\mathbb{Z}}
\DeclareMathOperator*{\N}{\mathbb{N}}

\DeclarePairedDelimiter{\ceil}{\lceil}{\rceil}

\setmainfont{Linux Libertine}

\theoremstyle{plain}
\newtheorem{theorem}{Теорема}
\newtheorem{axiom}{Аксиома}
\newtheorem{lemma}{Лемма}

\theoremstyle{remark}
\newtheorem*{remark}{Примечание}
\newtheorem*{consequence}{Следствие}
\newtheorem*{example}{Пример}

\theoremstyle{definition}
\newtheorem*{definition}{Определение}
% \documentclass[12pt, a4paper]{article}
\usepackage{langcode}
\usepackage{xunicode}
\usepackage{xltxtra}
\usepackage{pdfpages}
\usepackage[utf8]{inputenc}
\usepackage[russian, english]{babel}


\begin{document}
    \title{Дискретная Математика}
    \author{Ilya Yaroshevkiy}
    \date{Last changes: \today}
    \maketitle
\section{Отношения}
    \begin{definition}
        \textbf{Декартово произведение} - множество всех упорядоченных пар $(a_i,b_i)$ множеств $A$ и $B$. Обозначается: $A \times B$. Формально:
        $$A \times B = \{(x,y) \vert x \in A \land y \in B\}$$
    \end{definition}
    \begin{example}
        \begin{align*}
            A = \{1,2\}, B = \{a,b,c\} \\
            A \times B = \{(1,a),(1,b),(1,c),(2,a),(2,b),(2,c)\}        
        \end{align*}
    \end{example}
    \begin{definition}
        \textbf{Бинарное отношение} - всякое подмножество декратова произведения $A \times B$
    \end{definition}
    \begin{remark}
        По аналогии так-же применимо и к $n$-арным отношениям.
    \end{remark}
    \subsection{Виды отношений}
    \begin{enumerate}
        \item Симметрия

    \end{enumerate}

\end{document}